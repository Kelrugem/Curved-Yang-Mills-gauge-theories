\documentclass[a4paper,oneside,11pt,bibliography=totoc]{scrartcl}
%Use bibliography=totoc or bibliography=totocnumbered to show the List of References in the table of contents, numbered if using the latter
\usepackage{CJKutf8}
\usepackage[utf8]{inputenc}
\usepackage[T1]{fontenc}
%\usepackage{exscale}
%\newcommand\hmmax{0}
%\newcommand\bmmax{1}
%\newcommand{\Plus}{\mathord{\tikz\draw[line width=0.3ex, x=1ex, y=1ex] (0.75,0) -- (0.75,1.5)(0,0.75) -- (1.5,0.75);}} 
\def\RPlus{\ensuremath{\mathbin{\rule[.13em]{.66em}{.22em}\hspace{-.44em}\rule[-.08em]{.22em}{.66em}\,}}} %Fat plus symbol
\usepackage[english]{babel}
\usepackage{graphicx} %extended \includegraphics options
%\usepackage{CJKutf8} %Kanji signs
\usepackage{dsfont}%even more mathematical symbols
%\usepackage{bm} %use \bm to get bold math symbols
\usepackage{amssymb} %even more mathematical symbols
\usepackage{amsthm} %theorem style usually needed for AMS (american mathematical society)
\usepackage{amsmath} %important stuff like equation structures and so on
\usepackage{amsfonts} %new mathematical symbols like fractals
%\usepackage{amscd}
%\usepackage{amstext}
%\usepackage{mathabx}
\usepackage{booktabs} %enhanced table stuff
%\usepackage{amsbsy} %bold mathematical symbols, also loaded in amsmath
\usepackage{fancybox} %for theorem boxes
%\usepackage[hypcap=false]{caption} %new captions for floating options
%\captionsetup{font={small,sf},labelfont=bf,textfont=rm} 
%\usepackage{float} %provides H for images-here
%\usepackage{subfigure} %citing subfigures in a figure
\usepackage{lmodern} %latin modern font
\usepackage{fixcmex} %Fixing the issue with scaling brackets etc. in lmodern
%\renewcommand\familydefault{lmr}% uses lm only for text
%\usepackage[numbib,nottoc]{tocbibind} %Options for including the bibliography to content list etc.; nottoc prevents listing of Contents in the list of contents, numbib also numbers the section of the biblio
%\usepackage[pdfstartview={FitH},linkbordercolor={0 1 0}]{hyperref}
\usepackage[pdfstartview={FitH},linkbordercolor={0 1 0},colorlinks]{hyperref}
\usepackage{esint} %für \oiint und andere Integralformen
%\usepackage{fancyhdr} %Different handling for footers and headers
\usepackage[headsepline=.5pt,footsepline=.5pt,automark]{scrlayer-scrpage}
\usepackage{pdfpages} %For \includepdf etc
%\usepackage{trfsigns} % Fourier and Laplacesymbols for engineering und für \e für \exp
\newcommand{\e}{\ensuremath{\mathrm{e\;\!}}}
%\usepackage{wasysym} %Smiley
%\usepackage{siunitx} %SI Einheiten
%\usepackage{eqnarray} %Gleichungen in Tabellenform
\usepackage{mathtools} %for mathclap and coloneqq
\usepackage{relsize} %Integralgröße
\usepackage{scrhack} %Warnung bei book bezüglich toc
\usepackage{mleftright} %Distanz zu \left \right weg
\usepackage{tikz} %Diagramme
\usepackage{stmaryrd} % fuer llbracket usw.
\usepackage[many]{tcolorbox} %Coloured boxes around theorems etc
\tcbuselibrary{theorems} % siehe oben
%\usepackage{float} %For H descriptor in figure, but also for allowing figure in minipages
%\usepackage{xfrac}    %for \sfrac quotient of sets

\usetikzlibrary{cd}
%For small mathcal letters:
\makeatletter
\DeclareFontEncoding{LS1}{}{}
\DeclareFontSubstitution{LS1}{stix}{m}{n}
\DeclareMathAlphabet{\mathKel}{LS1}{stixscr}{m}{n}
\DeclareMathAlphabet{\mathcal}{LS1}{stixscr}{m}{n}
%Big tilde
\def\oversortoftilde#1{\mathop{\vbox{\m@th\ialign{##\crcr\noalign{\kern3\p@}%
      \sortoftildefill\crcr\noalign{\kern3\p@\nointerlineskip}%
      $\hfil\displaystyle{#1}\hfil$\crcr}}}\limits}

\def\sortoftildefill{$\m@th \setbox\z@\hbox{$\braceld$}%
  \braceld\leaders\vrule \@height\ht\z@ \@depth\z@\hfill\braceru$}
\makeatother
\DeclareMathOperator{\sAut}{\mathKel{A\mkern-5.5mu u\mkern-4mu t\mkern-1.5mu}}
\DeclareMathOperator{\sAd}{\mathKel{A\mkern-5.5mu d}}
\DeclareMathOperator{\saut}{\mathKel{a\mkern-4.5mu u\mkern-4mu t\mkern-1.5mu}}
\DeclareMathOperator{\sEnd}{\mathKel{E\mkern-4mu n\mkern-4.5mu d\mkern-1mu}}

%\tikzset{every node/.style={align=center}}
%\usepackage{pst-node} %Diagramme
%\usepackage{auto-pst-pdf}
%\usepackage{stackengine} %Kommawedge
%\usepackage{mathabx}
%\usepackage{here}
%\usepackage[style=authortitle-icomp]{biblatex}
%\usepackage[babel,german=guillemets]{csquotes}

\setcounter{tocdepth}{4}
\setcounter{tocdepth}{5}
\setcounter{secnumdepth}{4}
\setcounter{secnumdepth}{5}
\setlength{\jot}{12pt} % Zeilenabstand in der Align-Umgebung
\allowdisplaybreaks % mit \displaybreak einen Seitenumbruch in align-Umgebung erzeugen

%\pagestyle{headings}
\pagestyle{scrheadings}
%\pagestyle{fancy} %eigener Seitenstil
%\fancyhf{}
%\lhead{\leftmark} %\nouppercase, um auch Kleinbuchstaben zu bekommen
%\renewcommand{\sectionmark}[1]{\markright{#1}{}}
%  \markboth{\thesection{} #1}} 
%\fancyhead[L]{Titel} %Kopfzeile links
\clearpairofpagestyles
\ihead{\headmark} %Kopfzeile links, automark has to be loaded in the package scrlayer-scrpage
\ohead{Simon-Raphael Fischer} %Kopfzeile rechts
%\renewcommand{\headrulewidth}{0.4pt} %obere Trennlinie
%\fancyfoot[C]{\thepage} %Seitennummer
\cfoot{\pagemark}
%\renewcommand{\footrulewidth}{0.4pt} %untere Trennlinie

\renewcommand{\theequation}{\arabic{equation}}


%\renewcommand*{\chapterpagestyle}{scrheadings}


%Folgender Befehl dafür da, dass sections auf ungeraden Seiten anfangen
%\let\Section\section
%\renewcommand\section[2][]{%
  %\cleardoublepage
  %\def\myTEMP{#1}\ifx\myTEMP\empty\Section{#2}\else\Section[#1]{#2}\fi}
	
	
%\usepackage[automark]{scrlayer-scrpage}
%\ohead{\pagemark}
%\ihead{\leftmark}


\renewcommand{\listoffigures}{\begingroup
\tocsection
\tocfile{\listfigurename}{lof}
\endgroup} 

\renewcommand{\listoftables}{\begingroup
\tocsection
\tocfile{\listtablename}{lot}
\endgroup} 

\oddsidemargin -0.0 cm
\evensidemargin -0.5 cm
\topmargin -1.5 cm
\headheight 1.2 cm
\headsep 1.3 cm
\topskip 0.5 cm
\textheight 23.0 cm
\textwidth 16.0 cm
\parindent 0.0 cm
\renewcommand{\baselinestretch}{1.2}
%\renewcommand{\arraystretch}{1.6}
\setcounter{totalnumber}{2}

\def\tp{^{^{^{\leftrightarrow}}}\!\!\!\!\!\!\!}

%%%%%%%%%%%%%%%%%%%%%%%%%%%% Definitionen %%%%%%%%%%%%%%%%%%%%%%%%%%%%

%\renewcommand{\chapterheadstartvskip}{\vspace{0cm}} %für bookklasse

\def\be{\begin{equation}}
\def\ee{\end{equation}}
\def\bs{\begin{subequations}}
\def\es{\end{subequations}}
\def\ba#1\ea{\begin{align}#1\end{align}}
\def\bes{\begin{equation*}}
\def\ees{\end{equation*}}
\def\bas#1\eas{\begin{align*}#1\end{align*}}


%um die Theoreme etc. in das Inhaltsverzeichnis zu packen
%\let\amsthmhead\thmhead
%\let\amsswappedhead\swappedhead
%\makeatletter
%\renewcommand*\thmhead[3]{\amsthmhead{#1}{#2}{#3}%
%%\@ifnotempty{#2}{\addcontentsline{toc}{subsection}{#2 ~ #3}}{}}
%\renewcommand*\swappedhead[3]{\amsswappedhead{#1}{#2}{#3}%
%%\@ifnotempty{#2}{\addcontentsline{toc}{subsection}{#2 ~ #3}}{}}
%\makeatother

\renewcommand{\qedsymbol}{$\blacksquare$}

\theoremstyle{plain}
\newtheorem{theorem}{Theorem}[section]
%\newtcbtheorem[number within=section]{theorem}{Theorem}
%{colback=green!5,colframe=green!35!black,fonttitle=\bfseries}{th}
\newtheorem{corollary}[theorem]{Corollary}
\newtheorem{lemma}[theorem]{Lemma}
\newtcbtheorem
  [use counter*=theorem,number within=section]% init options
  {lemmata}% name
  {Lemma}% title
  {%
		fontupper=\itshape,
		breakable,
		enhanced,
    colback=gray!25,
    colframe=gray!0!black,
    fonttitle=\bfseries,
  }% options
  {lem}% prefix
\newtheorem{proposition}[theorem]{Proposition}
\newtcbtheorem
  [use counter*=theorem,number within=section]% init options
  {propositions}% name
  {Proposition}% title
  {%
		fontupper=\itshape,
		breakable,
		enhanced,
    colback=gray!25,
    colframe=gray!0!black,
    fonttitle=\bfseries,
  }% options
  {prop}% prefix
\newtcbtheorem
  [use counter*=theorem,number within=section]% init options
  {theorems}% name
  {Theorem}% title
  {%
		fontupper=\itshape,
		breakable,
		enhanced,
    colback=gray!25,
    colframe=gray!0!black,
    fonttitle=\bfseries,
  }% options
  {thm}% prefix
\newtcbtheorem
  [use counter*=theorem,number within=section]% init options
  {corollaries}% name
  {Corollary}% title
  {%
		fontupper=\itshape,
		breakable,
		enhanced,
    colback=gray!25,
    colframe=gray!0!black,
    fonttitle=\bfseries,
  }% options
  {cor}% prefix
\newtcbtheorem
  [use counter*=theorem,number within=section]% init options
  {conjectures}% name
  {Conjecture}% title
  {%
		fontupper=\itshape,
		breakable,
		enhanced,
    colback=gray!25,
    colframe=gray!0!black,
    fonttitle=\bfseries,
  }% options
  {con}% prefix
\newtheorem{conjecture}[theorem]{Conjecture}


\theoremstyle{remark}
\newtcbtheorem
  [use counter*=theorem,number within=section]% init options
  {remarks}% name
  {Remark}% title
  {%
		breakable,
		enhanced,
    colback=gray!5,
    colframe=gray!50!black,
    fonttitle=\bfseries,
  }% options
  {rem}% prefix
\newtheorem*{note}{Note}
\newtheorem{remark}[theorem]{Remarks}
\newtheorem{motivation}[theorem]{Motivation} 


\theoremstyle{definition}
\newtheorem{definition}[theorem]{Definition}
\newtcbtheorem
  [use counter*=theorem,number within=section]% init options
  {definitions}% name
  {Definition}% title
  {%
		breakable,
		enhanced,
    colback=gray!25,
    colframe=gray!0!black,
    fonttitle=\bfseries,
  }% options
  {def}% prefix
\newtcbtheorem
  [use counter*=theorem,number within=section]% init options
  {examples}% name
  {Example}% title
  {%colback=black!5,colframe=red!35!black
		breakable,
		enhanced,
    colback=gray!5,
    colframe=gray!50!black,
    fonttitle=\bfseries,
  }% options
  {ex}% prefix
\newtcbtheorem
  [use counter*=theorem,number within=section]% init options
  {situations}% name
  {Situation}% title
  {%colback=black!5,colframe=red!35!black
		breakable,
		enhanced,
    colback=gray!5,
    colframe=gray!50!black,
    fonttitle=\bfseries,
  }% options
  {sit}% prefix
\newtcbtheorem
  [use counter*=theorem,number within=section]% init options
  {fieldredefinitions}% name
  {Theorem}% title
  {%colback=black!5,colframe=red!35!black
		breakable,
		enhanced,
    colback=gray!25,
    colframe=gray!0!black,
    fonttitle=\bfseries,
  }% options
  {fieldredef}% prefix
\newtheorem{example}[theorem]{Example}


% hier Namen etc. einsetzen
%\newcommand{\fullname}{Simon-Raphael Fischer}
%\newcommand{\email}{sfischer@ncts.tw}
%\newcommand{\titel}{Examples and No-Gos of curves Yang-Mill-Higgs gauge theories}
%\newcommand{\titel}{Titel der Arbeit}
%\newcommand{\jahr}{2020}
%\newcommand{\matnr}{11424951}
%\newcommand{\gutachterA}{Prof. Dr. Mark John David Hamilton}
%\newcommand{\betreuer}{Professor Dr. Anna Dall'Acqua}
% hier richtige Fakultät auswählen
%\newcommand{\fakultaet}{Fakultät noch ergänzen}
%\newcommand{\fakultaet}{Mathematik und\\Wirtschaftswissenschaften}
%\newcommand{\fakultaet}{Naturwissenschaften}
%\newcommand{\fakultaet}{Medizin}
% nun noch unten das Institut einsetzen
%\newcommand{\institut}{Institut noch ergänzen}

% Für \widecheck: (umgekehrter Zirkumflex)

\DeclareFontFamily{U}{mathx}{\hyphenchar\font45}
\DeclareFontShape{U}{mathx}{m}{n}{
      <5> <6> <7> <8> <9> <10>
      <10.95> <12> <14.4> <17.28> <20.74> <24.88>
      mathx10
      }{}
\DeclareSymbolFont{mathx}{U}{mathx}{m}{n}
\DeclareFontSubstitution{U}{mathx}{m}{n}
\DeclareMathAccent{\widecheck}{0}{mathx}{"71}
\DeclareMathAccent{\wideparen}{0}{mathx}{"75}

%\newtheorem{cor}{Corollary}
%\newtheorem{ad}{Theorem}
%\newtheorem{theorem}{Theorem}
%\newtheorem{theorem}{Theorem}
%\newtheorem{theorem}{Theorem}

%\includeonly{PhysicalBasics/f(R)gravity} %nur das kompilieren

%\let\endtitlepage\relax %No page break after titlepage

\begin{document}
\pagenumbering{Roman}
\renewcommand{\thefootnote}{\fnsymbol{footnote}}

\begin{titlepage}
%\thispagestyle{empty}

\author{Simon-Raphael Fischer\footnote{Email: \href{mailto:sfischer@ncts.tw}{sfischer@ncts.tw}; ORCiD: \href{https://orcid.org/0000-0002-5859-2825}{0000-0002-5859-2825}} }
\title{Integrating curved Yang-Mills gauge theories} 
\subtitle{Gauge theories related to principal bundles equipped with Lie group bundle actions}
\date{\today} 
\maketitle
\thispagestyle{empty}

\begin{center}
National Center for Theoretical Sciences (\begin{CJK*}{UTF8}{bsmi}國家理論科學研究中心\end{CJK*}),
Mathematics Division,
\\
National Taiwan University
(\begin{CJK*}{UTF8}{bsmi}國立台灣大學)\end{CJK*}
\\
No.\ 1, Sec.\ 4, Roosevelt Rd., Taipei City 106, Room 407, Cosmology Building, Taiwan
\ \\
\ \\
%\ \\
\textbf{Abstract}\footnote[2]{Abbreviations used in this paper: \textbf{LGB(s)} for Lie group bundle(s), \textbf{LAB(s)} for Lie algebra bundle(s).}
%\footnote[2]{Abbreviations used in this paper: \textbf{(C)YMH GT} for (curved) Yang-Mills-Higgs gauge theory.}
\begin{abstract}
  \small{
	We construct a gauge theory based on principal bundles $\mathcal{P}$ equipped with a right $\mathcal{G}$-action, where $\mathcal{G}$ is a Lie group bundle instead of a Lie group. Due to the fact that a $\mathcal{G}$-action acts fibre by fibre, pushforwards of tangent vectors via a right-translation act now only on the vertical structure of $\mathcal{P}$. Thus, we generalize pushforwards using sections of $\mathcal{G}$, and in order to provide a definition independent of the choice of section we fix a connection on $\mathcal{G}$, which will modify the pushforward by subtracting the fundamental vector field generated by a generalized Darboux derivative of the chosen section. A horizontal distribution on $\mathcal{P}$ invariant under such a modified pushforward leads to a parallel transport on $\mathcal{P}$ which is a homomorphism w.r.t.\ the $\mathcal{G}$-action and the parallel transport on $\mathcal{G}$. For achieving gauge invariance we impose conditions on the connection 1-form $\mu$ on $\mathcal{G}$: $\mu$ has to be a multiplicative form, \textit{i.e.}\ closed w.r.t.\ a certain simplicial differential $\delta$ on $\mathcal{G}$, and the curvature $R_\delta$ of $\mu$ has to be $\delta$-exact with primitive $\zeta$; $\mu$ will be the generalization of the Maurer-Cartan form of the classical gauge theory, while the $\delta$-exactness of $R_\delta$ will generalize the role of the Maurer-Cartan equation. For allowing curved connections on $\mathcal{G}$ we will need to generalize the typical definition of the curvature/field strength $F$ on $\mathcal{P}$ by adding $\zeta$ to $F$.
	
This leads to a generalized gauge theory with many similar, but generalized, statements, including Bianchi identity, gauge transformations and Darboux derivatives. An example for a gauge theory with a curved Maurer-Cartan form $\mu$ will be provided by the inner group bundle of the Hopf fibration $\mathds{S}^7 \to \mathds{S}^4$. We conclude that this gauge theory is an integral of the infinitesimal gauge theory developed by Alexei Kotov and Thomas Strobl.
}
 \end{abstract}
\end{center}

\textit{2010 MSC:} Primary 53D17; Secondary 81T13, 22E99.

\textit{Keywords:} \texttt{Mathematical Gauge Theory}, Differential Geometry, High Energy Physics - Theory, Mathematical Physics

\end{titlepage}

%%%%%%%%%%%%%%%%%%%%%%%%%%%% Inhaltsverzeichnis %%%%%%%%%%%%%%%%%%%%%%%%%%%%



\tableofcontents

\pagenumbering{arabic}

%\thispagestyle{empty}\hspace{1em}\newpage
% \thispagestyle{empty}\hspace{1em}\newpage


\renewcommand{\thefootnote}{\arabic{footnote}}
%
\setlength{\parindent}{12 pt}
%%%%%%%%%%%%%%%%%%%%%%%%%%%% Hier beginnt der Hauptteil %%%%%%%%%%%%%%%%%%%%%%%%%%%%

%\cleardoublepage
% für die leere(n) Seite(n)


\section{Introduction and summary}

This paper's research concerns curved Yang-Mills-Higgs gauge theories, originally introduced by Alexei Kotov and Thomas Strobl in \cite{CurvedYMH}, where essentially the structural Lie algebra together with its action on the manifold $N$ of values of the Higgs fields is replaced by a general Lie algebroid $E \to N$:

\begin{definitions*}{Lie algebroid,  \cite[reduced definition of \S 16.1, page 113]{DaSilva}}
%\leavevmode\newline
Let $E \to N$ be a real vector bundle. Then $E$ is a smooth \textbf{Lie algebroid} if there is a bundle map $\rho: E \to \mathrm{T}N$, called the \textbf{anchor}, and a Lie algebra structure on $\Gamma(E)$ with Lie bracket $\left[ \cdot, \cdot \right]_E$ satisfying\footnote{With $\Gamma(E)$ I denote the space of sections of a vector bundle E, and with $\mathrm{T}N$ the tangent bundle of $N$.}
\begin{align*}
  \left[\mu, f \nu\right]_E = f \left[\mu, \nu\right]_E + \mathcal{L}_{\rho(\mu)}(f) ~ \nu
\end{align*}
for all $f \in C^\infty(N)$ and $\mu, \nu \in \Gamma(E)$, where $\mathcal{L}_{\rho(\mu)}(f)$ is the action of the vector field $\rho(\mu)$ on the function $f$ by derivation.
\end{definitions*}

The idea of replacing Lie algebras with Lie algebroids was proposed by Thomas Strobl in \cite{OriginofCYMH}, with further understanding of the involved gauge transformations in \cite{mayer2009lie}; eventually, this type of infinitesimal gauge theory got summarised and finalised in \cite{CurvedYMH}. My Ph.D.\ thesis was devoted to this type of infinitesimal gauge theory, attempting to find new (physical) examples and understanding the geometry of this infinitesimal gauge theory; see \cite{My1stpaper} and \cite{MyThesis}.

A short summary of this type of generalized infinitesimal gauge theory follows; we have the following ingredients:
\begin{itemize}
	\item $M$ a spacetime;
	\item $N$ a smooth manifold, serves as set for the values of the Higgs field $\Xi: M \to N$;
	\item $E \to N$ a Lie algebroid with anchor $\rho$, replacing the structural Lie algebra $\mathfrak{g}$ and its action $\gamma: \mathfrak{g} \to \mathfrak{X}(N)$ of the classical formulation, where $\mathfrak{X}$ denotes the set of vector fields in this work;
	\item a vector bundle connection $\nabla$ on $E$;
	\item a fibre metric $\kappa$ on $E$, as a substitute of the ad-invariant scalar product on $\mathfrak{g}$;
	\item a Riemannian metric $g$ on $N$, replacing the scalar product on the vector space in which the Higgs field usually has values in and which is invariant under the action of $\gamma$, used for the kinetic term of $\Xi$ which is minimally coupled to the field of gauge bosons $A \in \Omega^1(M; \Xi^*E)$ (1-form on $M$ with values in $\Xi^*E$);
	\item a 2-form on $N$ with values in $E$, $\zeta \in \Omega^2(N;E)$, an additional contribution to the field strength of $A$.
\end{itemize}

Infinitesimal gauge invariance of the Yang-Mills type functional leads to two \textbf{infinitesimal compatibility conditions} and two metric compatibilities to be satisfied between these structures; we will present those compatibility conditions later in this introduction. If the connection on $E$ is flat, the compatibilities imply that the Lie algebroid is locally an action Lie algebroid:

\begin{definitions*}{Action Lie algebroids, \cite[\S 16.2, Example 5; page 114]{DaSilva}}
Let $\mleft(\mathfrak{g}, \mleft[\cdot, \cdot \mright]_{\mathfrak{g}}\mright)$ be a Lie algebra equipped with a Lie algebra action $\gamma: \mathfrak{g} \to \mathfrak{X}(N)$ on a smooth manifold $N$. A \textbf{transformation Lie algebroid} or \textbf{action Lie algebroid} is defined as the bundle $E \coloneqq N \times \mathfrak{g}$ over $N$ with anchor
\bas
\rho(p, v) &\coloneqq \gamma(v)|_p
\eas
for $(p, v) \in E$, and Lie bracket
\bas
	\mleft.\mleft[\mu, \nu\mright]_E\mright|_p
	&\coloneqq 
	\mleft[\mu_p, \nu_p\mright]_{\mathfrak{g}}
		+ \mleft.\mleft(\mathcal{L}_{\gamma(\mu(p))}(\nu^a) - \mathcal{L}_{\gamma(\nu(p))}(\mu^a) \mright)\mright|_p ~ e_a
\eas
	for all $p \in N$ and $\mu, \nu \in \Gamma(E)$, where one views a section $\mu \in \Gamma(E)$ as a map $\mu: N \to \mathfrak{g}$, and $\mleft( e_a \mright)_a$ is some arbitrary frame of constant sections.
\end{definitions*}

As it is shown in the mentioned references for this type of infinitesimal gauge theory, one gets back to the standard Yang-Mills-Higgs gauge theory if additionally $\zeta \equiv 0$. Thus, the theory represents a covariantized version of gauge theory equipped with an additional 2-form $\zeta$. If $\nabla$ is flat we say in general that we have a \textbf{pre-classical} gauge theory, and if additionally $\zeta \equiv 0$ we have a \textbf{classical} gauge theory.

The 2-form $\zeta$ is needed to allow non-flat $\nabla$, because otherwise only flat $\nabla$ could satisfy the compatibility conditions. But $\zeta$ is not just an auxiliary map, in \cite{MyThesis} I have shown that there is also a class of \textbf{field redefinitions} for the classical formulation of gauge theory. This field redefinition breaks the gauge invariance; so, in order to keep gauge invariance, one needs to add $\zeta$ to the field strength, and at the same time one achieves a richer framework for gauge theories. It is then natural to study whether there is an infinitesimal gauge theory where $\zeta$ is non-zero and cannot be transformed to zero by the mentioned field redefinition.

In \cite{MyThesis} I first focused on Lie algebroids where the anchor map is an isomorphism and, thus, the Lie algebroid is just the tangent bundle of $M$. Locally, such examples can be excluded: The curvature can be always transformed to zero by the field redefinitions. However, globally, this is not always true: I have shown that the tangent bundle of the seven-dimensional sphere $\mathds{S}^7$ is an example of a curved gauge theory which cannot be transformed to a flat pre-classical gauge theory, otherwise it would be a Lie group which is not possible.

However, I also studied the other edge case of having Lie algebra bundles (LABs) instead of Lie algebroids: \cite{MyThesis} (also in \cite{My1stpaper}) points out that locally, \textit{i.e.}~over a contractible base manifold $N$, there is always a field redefinition transforming the initial gauge theory to a pre-classical one. Hence, we arrive at a similar situation as for tangent bundles. But globally there are examples given by the adjoint bundle of the Hopf fibration $\mathds{S}^7 \to \mathds{S}^4$, where $\mathds{S}^n$ ($n \in \mathbb{N}$) denotes the $n$-dimensional sphere.

This is now the starting point of this paper. We want to understand why the adjoint bundle of $\mathds{S}^7 \to \mathds{S}^4$ works as a curved example, and where it differs from the classical formalism of gauge theory. To truly understand this, and also to possibly get new information about the general case regarding Lie algebroids, we will integrate curved Yang-Mills-Higgs gauge theories in the case of LABs in this paper. Using LABs means that there is no coupling to a Higgs field in the usual sense and therefore we will now just speak instead of a curved Yang-Mills gauge theory. In the following we will start outlining the paper's main results; we will focus on an easy presentation in the introduction, especially we will not always restate our assumptions, and the notation will be simplified w.r.t.\ the actual formulation.

The gauge theory presented here will be based on principal bundles $\mathcal{P} \stackrel{\pi}{\to} M$, $M$ a smooth manifold, related to a right action of a Lie group bundle (LGB) $\mathcal{G} \stackrel{\pi_{\mathcal{G}}}{\to} M$. The most important distinction to classical gauge theory is that the action $\Phi$ is a map
\bas
\mathcal{P} * \mathcal{G} \coloneqq \pi^*\mathcal{G} &\to \mathcal{P},\\
(p, g) &\mapsto p \cdot g,
\eas
where $\pi^*\mathcal{G}$ is the pullback LGB of $\mathcal{G}$ along $\pi$. Observe that $\mathcal{G}_x$ only acts on $\mathcal{P}_x$ as a Lie group, thus an element $g \in \mathcal{G}_x$ cannot act on all of $\mathcal{P}$, where notations like $\mathcal{G}_x$ denote the fibre of $\mathcal{G}$ over an $x \in M$. This leads to certain difficulties for defining gauge theory which we will resolve in this paper.

However, this paper aims to be accessible for all people knowing the basics of "classical" gauge theory. Thus, we start with an extensive introduction to LGBs, their actions and their infinitesimal analogues, the LABs, in Section \ref{LGBSection} to \ref{LGBActionIISection}. These sections introduce and generalize all the notions known in the "classical" Lie group situations in the context of gauge theory, most of which is straight-forward to generalize. We will especially follow \cite{mackenzieGeneralTheory}, but since this and other references are usually introducing these basic notions on (Lie) groupoids, we decided to reintroduce all the needed notions and proofs so that it should not be required for the reader to follow and learn about groupoids because we just need the much simpler situation regarding LGBs. The experienced reader may start directly with Section \ref{ConnCurvOnPrincLGBBundle}, but there is one partially new result in these preliminary sections: For gauge theory we will need to understand the differential of the $\mathcal{G}$-action $\Phi$ which we will derive in Thm.\ \ref{thm:DiffOfLGBAction} and which will form \textit{the} fundament for many calculations. For this we will need notations like $\mathrm{T}M$ which will denote a tangent bundle with fibre $\mathrm{T}_xM$ at $x \in M$; the fibres of other bundles with similar notation are denoted in the same fashion.

\begin{theorems*}{Differential of LGB action $\Phi$, simplified situation}
We have
\bas
\mathrm{D}_{(p, g)}\Phi(X, Y)
&=
\mathrm{D}_pr_\sigma(X)
	+ \mleft.{\oversortoftilde{\mleft( \mu_{\mathcal{G}}\mright)_g \bigl(Y - \mathrm{D}_{x}\sigma (\omega)\bigr)}}\mright|_{p \cdot g}
\eas
for all $(p, g) \in \mathcal{P} * \mathcal{G}$ and $(X, Y) \in \mathrm{T}_{(p, g)}(\mathcal{P}*\mathcal{G})$, where $x \coloneqq \pi(p)$, $\sigma$ is any (local) section of $\mathcal{G}$ with $\sigma_{x} = g$ and $r_\sigma$ denotes the right-translation of $\sigma$ in $\mathcal{P}$, $\mu_{\mathcal{G}}$ is the fibre-wise defined Maurer-Cartan form of the fibres, $\oversortoftilde{\mleft( \mu_{\mathcal{G}} \mright)_g(\dotsc)}$ denotes its generated fundamental vector field on $\mathcal{P}_x$, and $\omega$ is an element of $\mathrm{T}_xM$ given by
\bas
\omega
&\coloneqq
\mathrm{D}_p \pi(X)
= 
\mathrm{D}_g\pi_{\mathcal{G}}(Y).
\eas
\end{theorems*}

In Section \ref{ConnCurvOnPrincLGBBundle} we will introduce connections and generalized curvatures on principal LGB-bundles: Subsection \ref{PrincBundlLGBBased} starts with introducing principal $\mathcal{G}$-bundles $\mathcal{P}$. Such principal bundles were already introduced in \cite[\S 5.7, page 144f.]{GroupoidBasedPrincipalBundles}, but also here we rephrased it in such a way that these bundles' definition has a more familiar shape for readers not so proficient with the study of groupoids. The main difference to Lie group based principal bundles is the previously-mentioned LGB action $\Phi$. Since these bundles were not studied a lot before we introduce basic notions like morphisms and easy examples, for example LGBs $\mathcal{G}$ themselves are principal $\mathcal{G}$-bundles, so that the gauge theory presented in this paper can also be understood as a gauge theory based not only on "classical" principal bundles but also on LGBs which were excluded in general in the classical formalism. Furthermore we generalize certain statements known about classical principal bundles, \textit{e.g.}\ we observe that a section of $\mathcal{P}$ does not trivialize $\mathcal{P}$ in general, however it introduces an isomorphism to $\mathcal{G}$, as will be pointed out in Lemma \ref{lem:SectionsNowInduceIsomToLGBsNotNecTriv}.

\begin{lemmata*}{Local sections of principal bundles induce isomorphisms to the structural LGB, simplified language}
Let $s: U \to \mathcal{P}$ be a smooth local section of $\mathcal{P}$ over an open subset $U$ of $M$. Then the orbit map through $s$,
\bas
\mathcal{G}|_U &\to \mathcal{P}|_U,\\
g &\mapsto s_{\pi_{\mathcal{G}}(g)} \cdot g,
\eas
is a base-preserving principal $\mathcal{G}$-bundle isomorphism.
\end{lemmata*}

In Subsection \ref{ConnectionSubsection} we finally turn to the notion of connections on $\mathcal{P}$, described as horizontal subbundle $\mathrm{H}\mathcal{P}$ of the tangent bundle $\mathrm{T}\mathcal{P}$, complementary to the vertical bundle $\mathrm{V}\mathcal{P}$, and equivalently we want to describe $\mathrm{H}\mathcal{P}$ as a connection 1-form $A$. As one can already see in the last theorem about the derivative of the action $\Phi$, due to the fact that right-translations $r_g$ of an element $g \in \mathcal{G}_x$ now only acts on $\mathcal{P}_x$, its derivative $\mathrm{D}r_g$ only acts on the tangent bundle of $\mathcal{P}_x$, the vertical tangent vectors of $\mathcal{P}$ (over $x$). Hence, in order to define the pushforward of non-vertical vectors we make use of auxiliary sections. However, this leads to the problem that there are many different sections with the same value over a fixed base point $x \in M$, all of whose pushforwards of non-vertical vectors are in general different. To get rid of the ambiguity in the choice of section we are going to modify and generalize the pushforward of tangent vectors by right-translations. To do so, we fix a horizontal distribution $\mathrm{H}\mathcal{G}$ on $\mathcal{G}$, without further assumptions than the bare-bones for horizontal distributions. Then we start to construct the horizontal distribution $\mathrm{H}\mathcal{P}$ initially by its associated parallel transport, shortly denoted now by $\mathrm{PT}^{\mathcal{P}}$; similarly we denote the parallel transport associated to $\mathrm{H}\mathcal{G}$ by $\mathrm{PT}^{\mathcal{G}}$. We want a connection on $\mathcal{P}$ characterised by
\bas
\mathrm{PT}^{\mathcal{P}}(p \cdot g)
&=
\mathrm{PT}^{\mathcal{P}}(p) \cdot \mathrm{PT}^{\mathcal{G}}(g)
\eas
for all $(p, g) \in \mathcal{P}*\mathcal{G}$. However, in order to define the connection 1-form it is useful to find a definition of such $\mathrm{H}\mathcal{P}$ via a symmetry w.r.t.\ a certain map acting on $\mathrm{T}\mathcal{P}$. As already high-lighted, this will be a modification of the pushforward via right-translations. Differentiating the condition about parallel transports w.r.t.\ the curve parameter achieves this modification, summarised in Prop.\ \ref{prop:IsomorphismRightPushAndDarboux}; we explain the second summand of the following proposition afterwards.

\begin{propositions*}{The modified right-pushforward}
For $g \in \mathcal{G}_x$ ($x \in M$) define the map
\bas
\mleft.\mathrm{T}\mathcal{P}\mright|_{\mathcal{P}_x} &\to \mleft.\mathrm{T}\mathcal{P}\mright|_{\mathcal{P}_x},\\
X 
&\mapsto 
\mathcal{r}_{g*}(X) 
\coloneqq
\mathrm{D}_pr_\sigma\mleft( 
	X 
\mright)
	- \mleft.{\oversortoftilde{
		\mleft. \mleft( \pi^!\Delta\sigma \mright) \mright|_p(X)
	}}\mright|_{p \cdot g},
\eas
where $p \in \mathcal{P}_x$, $X \in \mathrm{T}_p \mathcal{P}$, $\pi^!$ denotes the pull-back of forms with $\pi$, and $\sigma$ is any (local) section of $\mathcal{G}$ with $\sigma_x = g$. Then $\mathcal{r}_{g*}$ is independent of the choice of the local section $\sigma$, and it is a vector bundle automorphism over the right-translation $r_g$. 

Furthermore, if we have a horizontal distribution $\mathrm{H}\mathcal{P}$ on $\mathcal{P}$, then $\mathrm{H}_p\mathcal{P}$ is isomorphic via $\mathcal{r}_{g*}$ to a complement of $\mathrm{V}_{p \cdot g} \mathcal{P}$ in $\mathrm{T}_{p \cdot g}\mathcal{P}$ (this complement is not necessarily $\mathrm{H}_{p \cdot g}\mathcal{P}$).
\end{propositions*}

This can be generalized to sections $\sigma$ of $\mathcal{G}$ in a straight-forward manner, giving rise to an automorphism $\mathcal{r}_\sigma$ of $\mathrm{T}\mathcal{P}$ over the right-translation $r_\sigma$.
$\Delta\sigma$ is the generalised version of the Darboux derivative, often simply denoted as $\sigma^{-1} \mathrm{d}\sigma$ in the classical formalism, but in some works like \cite[\S 5.1, page 182ff.]{mackenzieGeneralTheory}\ also already in the classical formalism written as $\Delta\sigma$. As expected, this derivative plays an important role in gauge theory, which is why we will also discuss and introduce the Darboux derivative; usually it only appears in the gauge transformations, but in our case it will already appear now. In Def.\ \ref{def:TotMCFormOnLGB} we will first introduce the \textbf{total Maurer-Cartan form $\mu_{\mathcal{G}}^{\mathrm{tot}}$} as the connection 1-form associated to $\mathcal{G}$ with values in the LAB $\mathcal{g}$ of $\mathcal{G}$; the labelling comes from that this form will play a similar role like the Maurer-Cartan form in the typical formalism for gauge theory, and in contrast to $\mu_{\mathcal{G}}$ it acts on the whole of $\mathrm{T}\mathcal{G}$ instead of just the vertical subbundle.

\begin{definitions*}{Total Maurer-Cartan form}
Let us denote with $\pi^{\mathrm{vert}}$ the projection onto the vertical bundle $\mathcal{G}$, corresponding to its horizontal bundle $\mathrm{H}\mathcal{G}$. Then we define the \textbf{total Maurer-Cartan form $\mu_{\mathcal{G}}^{\mathrm{tot}} \in \Omega^1\mleft(\mathcal{G}; \pi_{\mathcal{G}}^*\mathcal{g}\mright)$ of $\mathcal{G}$} as the connection 1-form corresponding to $\mathrm{H}\mathcal{G}$, \textit{i.e.}\
\bas
\mleft( \mu_{\mathcal{G}}^{\mathrm{tot}} \mright)_g(Y)
&\coloneqq
\mleft.\mleft(\mu_\mathcal{G} \circ \pi^{\mathrm{vert}}\mright)\mright|_g(Y)
=
\mleft( \mathrm{D}_g L_{g^{-1}} \mright)\mleft(\pi^{\mathrm{vert}}(Y)\mright)
\eas
for all $g \in \mathcal{G}$ and $Y \in \mathrm{T}_g\mathcal{G}$.
\end{definitions*}

As expected, the Darboux derivative is the form-pullback of the total Maurer-Cartan form, as such we will introduce this in Def.\ \ref{def:DarbouxDerivativeOnLGBs}.

\begin{definitions*}{Darboux derivative}
For (local) $\sigma \in \Gamma(\mathcal{G})$ we define the \textbf{Darboux derivative $\Delta \sigma \in \Omega^1(M; \mathcal{g})$}
\bas
\Delta \sigma
&=
\sigma^! \mu_{\mathcal{G}}^{\mathrm{tot}}.
\eas
\end{definitions*}

Using the Darboux derivative, we defined the modified pushforward via right-translations (also called modified right-pushforward) which eventually leads to the definition of an Ehresmann connection on the principal $\mathcal{G}$-bundle $\mathcal{P}$, as pinpointed in Def.\ \ref{def:FinallyTheConnection}.

\begin{definitions*}{Ehresmann connection on principal LGB-bundles}
We call $\mathrm{H}\mathcal{P}$ an \textbf{Ehresmann connection} or a \textbf{connection on $\mathcal{P}$} if it is \textbf{right-invariant (w.r.t.\ modified right-pushforward)}, \textit{i.e.}\
\bas
\mathcal{r}_{g*}\mleft( \mathrm{H}_p\mathcal{P} \mright)
&=
\mathrm{H}_{p\cdot g}\mathcal{P}
\eas
for all $p \in \mathcal{P}_x$ and $g \in \mathcal{G}_x$ ($x \coloneqq \pi(p)$).
\end{definitions*}

$\mathrm{H}\mathcal{P}$ will now always be an Ehresmann connection in the following.
The basic idea now is to replace the typical right-pushforward with the modified one in all the involved definitions, starting with the needed pull-back of forms, which will be simply given by Def.\ \ref{def:PullbackOfFormsViaModRight}.

\begin{definitions*}{The pullback of forms via modified right-pushforward, the section formulation}
For $\omega \in \Omega^k(\mathcal{P}; \pi^*\mathcal{g})$ ($k \in \mathbb{N}_0$) we define the \textbf{pullback via the modified right-pushforward $\mathcal{r}^!_{\sigma}\omega$ with a (local) section $\sigma \in \Gamma(\mathcal{G})$} as an element of $\Omega^k(\mathcal{P}; \pi^*\mathcal{g})$ by 
\bas
\mleft.\mleft(\mathcal{r}_\sigma^!\omega\mright)\mright|_p \mleft( Y_1, \dotsc, Y_k\mright)
&\coloneqq
\omega_{p \cdot \sigma_x} \bigl( \mathcal{r}_{\sigma*}(Y_1), \dotsc, \mathcal{r}_{\sigma*}(Y_k)\bigr)
\eas
for all $p \in \mathcal{P}_x$ ($x \in M$) and $Y_1, \dotsc, Y_k \in \mathrm{T}_p\mathcal{P}$.
\end{definitions*}

It is now straight-forward to define the connection 1-form $A$ on $\mathcal{P}$, which we will do in Def.\ \ref{def:GaugeBosonsOnLGBPrincies}.

\begin{definitions*}{Connection 1-forms on principal LGB-bundles, simplified notation}
A \textbf{connection 1-form} or \textbf{gauge field} on $\mathcal{P}$ is a 1-form $A \in \Omega^1(\mathcal{P}; \pi^*\mathcal{g})$ satisfying:
\begin{itemize}
	\item \textbf{($\mathcal{G}$-equivariance, but w.r.t.\ modified right-pushforward)}
		\bas 
			\mathcal{r}_\sigma^! A
			&=
			\mathrm{Ad}_{\sigma^{-1}} \circ A
		\eas
	for all (local) $\sigma \in \Gamma(\mathcal{G})$, where $\mathrm{Ad}$ is the adjoint representation of $\mathcal{G}$ on $\mathcal{g}$.
	\item \textbf{(Identity on $\mathrm{V}\mathcal{P}$)}
	\bas
	A\mleft(\widetilde{\nu}\mright)
	&=
	\pi^*\nu
	\eas
	for all (local) $\nu \in \Gamma(\mathcal{g})$, where notations like $\pi^*$ denote the pull-back of sections.
\end{itemize}
\end{definitions*}

Of course we will achieve a 1:1 correspondence to Ehresmann connections in Thm.\ \ref{thm:OurConnectionHasAUniqueoneForm}.

\begin{theorems*}{1:1 correspondence of Ehresmann connections and connection 1-forms}
There is a 1:1 correspondence between Ehresmann connections and connection 1-forms on $\mathcal{P}$:
\begin{itemize}
	\item Let $\mathrm{H}\mathcal{P}$ be an Ehresmann connection on $\mathcal{P}$. Then $\mathrm{H}\mathcal{P}$ defines a connection 1-form $A \in \Omega^1(\mathcal{P}; \pi^*\mathcal{g})$ by
	\bas
	A_p\bigl( \widetilde{v}_p + X_p \bigr)
	&=
	(p, v)
	\eas
	for all $p \in \mathcal{P}_x$ ($x \in M$), $v \in \mathcal{g}_x$ and $X \in \mathrm{H}_p\mathcal{P}$.
	\item Let $A \in \Omega^1(\mathcal{P}; \pi^*\mathcal{g})$ be a connection 1-form on $\mathcal{P}$. Then $A$ defines an Ehresmann connection $\mathrm{H}\mathcal{P}$ on $\mathcal{P}$ via its kernel $\mathrm{Ker}(A)$, that is,
	\bas
	\mathrm{H}_p\mathcal{P}
	&=
	\mathrm{Ker}(A_p)
	\eas
	for all $p \in \mathcal{P}$.
\end{itemize}
\end{theorems*}

Not only these results are straight-forward once one has the definition of the modified right-pushforward, also other results will be simple to guess. For example in Subsection \ref{GaugeTrafoForA} we turn to the gauge transformations which will have a familiar shape but with the generalized Darboux derivative appearing, see Thm.\ \ref{thm:GaugeTrafoOfGaugeBoson}.

\begin{theorems*}{Gauge transformations of connection 1-forms, simplified notation}
Let $H$ be a (base-preserving) automorphism of $\mathcal{P}$. We then have that $H^!A$ is a connection 1-form on $\mathcal{P}$ and
\bas
H^!A
&=
{\mathrm{Ad}_{\mleft(\sigma^H\mright)^{-1}}} \circ A 
	+ \mleft(\pi^*\Delta\mright)\sigma^H,
\eas
where $\sigma^H \in \Gamma(\pi^*\mathcal{G})$ is uniquely defined by
\bas
H(p)
=
p \cdot \sigma_p^H
\eas
for all $p \in \mathcal{P}$, and $\pi^*\Delta$ is the Darboux derivative on $\pi^*\mathcal{G}$ naturally inherited by the pullback of the connection on $\mathcal{G}$.
\end{theorems*}

The 1:1 correspondence between $H$ and $\sigma^H$ arising here leads to similar statements as in the classical gauge theory which we will also shortly point out. We will show in Thm.\ \ref{thm:LocalGaugeTrafoChangeGauge} that this theorem about gauge transformations implies a pullback version, that is, one looks at the pullback $A_{s_i} \coloneqq s_i^!A$ of $A$ w.r.t.\ a section (also called gauge) $s_i$ of $\mathcal{P}$ defined over an open subset $U_i \subset M$, and then the change of $A$ by changing the gauge $s_i$ to another gauge follows by the previous theorem.

\begin{theorems*}{Gauge transformations as a change of gauge in the local gauge field}
Let $U_i$ and $U_j$ be two open subsets of $M$ so that $U_i \cap U_j \neq \emptyset$, two gauges $s_i \in \Gamma\mleft(\mathcal{P}|_{U_i}\mright)$ and $s_j \in \Gamma\mleft(\mathcal{P}|_{U_j}\mright)$, and the unique $\sigma_{ji} \in \Gamma\mleft( \mleft.\mathcal{G}\mright|_{U_i \cap U_j} \mright)$ with $s_i = s_j \cdot \sigma_{ji}$ on $U_i \cap U_j$. Then we have over $U_i \cap U_j$ that
\bas
A_{s_i}
&=
\mathrm{Ad}_{\sigma_{ji}^{-1}}\circ A_{s_j}
	+ \Delta\sigma_{ji}.
\eas
\end{theorems*}

This finished the discussion about $A$, but of course we also needs its field strength $F$ and a certain shape of gauge transformation for $F$. Up until now $\mathrm{H}\mathcal{G}$ was an arbitrary horizontal distribution, but we will need to fix certain conditions on it to assure gauge invariance later. We will discuss this in Subsection \ref{CurvatureSubsection}. On one hand, viewing $\mathcal{G}$ as the principal $\mathcal{G}$-bundle $\mathcal{P}$ itself whose horizontal distribution $\mathrm{H}\mathcal{P}$ aligns with $\mathrm{H}\mathcal{G}$, it is natural to guess that $\mathrm{H}\mathcal{G}$ should be an Ehresmann connection itself. Ehresmann connections on LGBs were already discussed in works like \cite{LAURENTGENGOUXStienonXuMultiplicativeForms}; and there is also a rather recent preprint related to a similar subject, see \cite{FernandesMarcutMultiplicativeForms}. One of the results of these works is that an Ehresmann connection on LGBs is characterized by the fact that the total Maurer-Cartan form $\mu_{\mathcal{G}}^{\mathrm{tot}}$ has to be a multiplicative form:

\begin{definitions*}{Multiplicative forms, simplified situation, \newline \cite[\S 2.1, special situation of Def.\ 2.1]{crainic2015multiplicative}}
We call an $\omega \in \Omega^1\mleft( \mathcal{G}; \pi_{\mathcal{G}}^*\mathcal{g} \mright)$ a \textbf{multiplicative form} if
\bas
\omega_{gq}\mleft( \mathrm{D}_{(g, q)}\Phi(X, Y)  \mright)
&=
\mathrm{Ad}_{q^{-1}}\bigl( \omega_{g}(X) \bigr)
	+ \omega_{q}(Y)
\eas
for all $(g, q) \in \mathcal{G}*\mathcal{G}$ and $(X, Y) \in \mathrm{T}_{(g, q)}(\mathcal{G}*\mathcal{G})$, where $\Phi: \mathcal{G} * \mathcal{G} \to \mathcal{G}$ is the multiplication in $\mathcal{G}$.
\end{definitions*}

\begin{theorems*}{Connection 1-forms on LGBs are multiplicative, \newline \cite[\S 4.4, implication of Lemma 4.14]{LAURENTGENGOUXStienonXuMultiplicativeForms}}
$\mathrm{H}\mathcal{G}$ is an (Ehresmann) connection on $\mathcal{G}$ as principal bundle if and only if $\mu_{\mathcal{G}}^{\mathrm{tot}}$ is multiplicative.
\end{theorems*}

Assuming that $\mathrm{H}\mathcal{G}$ is an Ehresmann connection will also allow us to formulate certain technical identities for the Darboux derivative like a Leibniz rule. In order to calculate the gauge transformation for the curvature/field strength $F$ of $A$ we need to understand the curvature of the total Maurer-Cartan form. Classically, this is described by the Maurer-Cartan equation, however, we generalize this condition, allowing non-flat connections on $\mathcal{G}$. For this we will need a connection on the LAB $\mathcal{g}$ of $\mathcal{G}$, naturally induced by $\mathrm{H}\mathcal{G}$; we will construct this connection in Prop.\ \ref{prop:FinallyTheNablaInduction} and Def.\ \ref{def:ConnectionOnLAB}, and our construction aligns with \cite[\S 4.5, Prop.\ 4.22]{LAURENTGENGOUXStienonXuMultiplicativeForms} even though we argue the following differently since we have shown the following statement actually for all horizontal distributions $\mathrm{H}\mathcal{G}$ instead of just Ehresmann connections.

\begin{propositions*}{LGB connection induces LAB connection, simplified notation}
The map $\nabla^{\mathcal{G}}: \Gamma(\mathcal{g}) \to \Omega^1(M; \mathcal{g})$, $\nu \mapsto \nabla^{\mathcal{G}}\nu$ denoted as an element of $\Omega^1(M; \mathcal{g})$ by $X \mapsto \nabla^{\mathcal{G}}_X \nu$, defined by
\bas
\mleft.\nabla^{\mathcal{G}}_X \nu\mright|_x
&\coloneqq
\mleft.\frac{\mathrm{d}}{\mathrm{d}t}\mright|_{t=0} \Bigl( \mleft(\Delta \e^{t \nu}\mright)_x (X) \Bigr)
\eas
for all $x \in M$, $X \in \mathrm{T}_xM$ and $\nu \in \Gamma(\mathcal{g})$, is a vector bundle connection on $\mathcal{g}$, where $t \in \mathbb{R}$. We will call $\nabla^{\mathcal{G}}$ the \textbf{$\mathcal{G}$-connection (on its LAB $\mathcal{g}$)}.
\end{propositions*}

In fact, $\nabla^{\mathcal{G}}$ will play the role of $\nabla$ from the beginning of this introduction. Furthermore, this connection will allow us to define the generalized Maurer-Cartan equation in Thm.\ \ref{thm:GenMCEq} and Def.\ \ref{def:NowReallyYangMillsConnectio}.

\begin{definitions*}{Yang-Mills connection, simplified notation}
We say that $\mathrm{H}\mathcal{G}$, and $\mu_{\mathcal{G}}^{\mathrm{tot}}$, is a \textbf{Yang-Mills connection (w.r.t.\ a $\zeta \in \Omega^2(M; \mathcal{g})$)}, if it satisfies the \textbf{compatibility conditions}:
\begin{enumerate}
	\item $\mu_{\mathcal{G}}^{\mathrm{tot}}$ is multiplicative (\textit{i.e.}\ $\mathrm{H}\mathcal{G}$ is an Ehresmann connection),
	\item $\mu_{\mathcal{G}}^{\mathrm{tot}}$ satisfies the \textbf{generalized Maurer-Cartan equation}, that is,
	\bas
	\mleft.\mleft(\mathrm{d}^{\pi_{\mathcal{G}}^*\nabla^{\mathcal{G}}} \mu_{\mathcal{G}}^{\mathrm{tot}}
	+ \frac{1}{2} \mleft[ \mu_{\mathcal{G}}^{\mathrm{tot}} \stackrel{\wedge}{,} \mu_{\mathcal{G}}^{\mathrm{tot}} \mright]_{\pi_{\mathcal{G}}^*\mathcal{g}} \mright)\mright|_g
&=
\mathrm{Ad}_{g^{-1}} \circ \mleft.\pi_{\mathcal{G}}^!\zeta\mright|_g
	- \mleft.\pi_{\mathcal{G}}^!\zeta\mright|_g
	\eas
	for all $g \in \mathcal{G}_x$ ($x \in M$), where $\mleft[\cdot, \cdot\mright]_{\pi_{\mathcal{G}}^*\mathcal{g}}$ is the Lie bracket of the pullback-LAB $\pi_{\mathcal{G}}^*\mathcal{g}$, and the $\stackrel{\wedge}{,}$ denotes the typical graded extension of tensors, that is, the second summand on the left hand side is an element of $\Omega^2(\mathcal{G}; \pi_{\mathcal{G}}^*\mathcal{g})$ given here by
	\bas
	\mleft(\frac{1}{2} \mleft[ \mu_{\mathcal{G}}^{\mathrm{tot}} \stackrel{\wedge}{,} \mu_{\mathcal{G}}^{\mathrm{tot}} \mright]_{\pi_{\mathcal{G}}^*\mathcal{g}} \mright) (X, Y)
	&=
	\mleft[ \mu_{\mathcal{G}}^{\mathrm{tot}}(X), \mu_{\mathcal{G}}^{\mathrm{tot}}(Y) \mright]_{\pi_{\mathcal{G}}^*\mathcal{g}}
	\eas
	for all $X, Y \in \mathfrak{X}(\mathcal{G})$.
\end{enumerate}
\end{definitions*}

Making a pull-back of the generalized Maurer-Cartan equation w.r.t.\ a section $\sigma \in \Gamma(\mathcal{G})$ is straight-forward and will be provided in Cor.\ \ref{cor:PullbackOfMCSupperEquation}.

\begin{corollaries*}{Pullback of generalized Maurer-Cartan equation}
Let $\mathrm{H}\mathcal{G}$ be a Yang-Mills connection w.r.t.\ a $\zeta \in \Omega^2\mleft( M; \mathcal{g} \mright)$. Then
\bas
\mathrm{d}^{\nabla^{\mathcal{G}}} \Delta \sigma 
	+ \frac{1}{2} \mleft[ \Delta \sigma \stackrel{\wedge}{,} \Delta\sigma \mright]_{\mathcal{g}} 
	+ \zeta 
&=
\mathrm{Ad}_{\sigma^{-1}} \circ \zeta
\eas
for all $\sigma \in \Gamma(\mathcal{G})$.
\end{corollaries*}

We will now argue that those compatibility conditions are the integrals of the infinitesimal compatibility conditions proposed by Alexei Kotov and Thomas Strobl, mentioned earlier (without details). In \cite{LAURENTGENGOUXStienonXuMultiplicativeForms} it was already observed that an Ehresmann connection on an LGB naturally implies that $\nabla^{\mathcal{G}}$ is a Lie bracket derivation.

\begin{lemmata*}{Ehresmann connections induce Lie bracket derivations, \newline \cite[\S 4.5, Prop.\ 4.21]{LAURENTGENGOUXStienonXuMultiplicativeForms}}
If $\mathrm{H}\mathcal{G}$ is an (Ehresmann) connection on $\mathcal{G}$ as principal bundle, then
\bas
\nabla^{\mathcal{G}}\mleft( \mleft[ \mu, \nu \mright]_{\mathcal{g}} \mright)
&=
\mleft[ \nabla^{\mathcal{G}} \mu, \nu \mright]_{\mathcal{g}}
	+ \mleft[ \mu, \nabla^{\mathcal{G}} \nu \mright]_{\mathcal{g}}
\eas 
for all $\mu, \nu \in \Gamma(\mathcal{g})$.
\end{lemmata*}

That $\nabla^{\mathcal{G}}$ is a Lie bracket derivation is one of the infinitesimal compatibility conditions.
Furthermore, we will show in Thm.\ \ref{thm:GenMCEq} that the generalized Maurer-Cartan equation is actually equivalent to its infinitesimal version.

\begin{theorems*}{Generalized Maurer-Cartan equation, changed formulation}
Let $\mathrm{H}\mathcal{G}$ be an Ehresmann connection on $\mathcal{G}$, and $\zeta \in \Omega^2\mleft( M; \mathcal{g} \mright)$. Then $\mathrm{H}\mathcal{G}$ satisfies the generalized Maurer-Cartan equation w.r.t.\ $\zeta$ if and only if 
\bas
R_{\nabla^{\mathcal{G}}}(X, Y)\mu
&=
\mleft[ \zeta(X, Y), \mu \mright]_{\mathcal{g}}
\eas
for all $X, Y \in \mathfrak{X}(M)$ and $\nu \in \Gamma(\mathcal{g})$, where $R_{\nabla^{\mathcal{G}}}$ is the curvature of $\nabla^{\mathcal{G}}$.
\end{theorems*}

The condition about the curvature is in fact the second infinitesimal compatibility condition proposed by Alexei Kotov and Thomas Strobl, so that we can conclude that Yang-Mills connections on LGBs serve as an integral of the infinitesimal compatibility conditions which we will point out in Def.\ \ref{def:YangMillsConnection} and Rem.\ \ref{rem:YangMillsEqualsInfYM}.

\begin{theorems*}{Yang-Mills connections satisfy the infinitesimal compatibility conditions}
Every Yang-Mills connection (w.r.t.\ $\zeta \in \Omega^2(M; \mathcal{g})$) $\mathrm{H}\mathcal{G}$ is an \textbf{infinitesimal Yang-Mills connection (on $\mathcal{G}$)}, that is, it satisfies the \textbf{infinitesimal compatibility conditions}
\bas
\nabla^{\mathcal{G}}\mleft( \mleft[ \mu, \nu \mright]_{\mathcal{g}} \mright)
&=
\mleft[ \nabla^{\mathcal{G}} \mu, \nu \mright]_{\mathcal{g}}
	+ \mleft[ \mu, \nabla^{\mathcal{G}} \nu \mright]_{\mathcal{g}},
\\
R_{\nabla^{\mathcal{G}}}(X, Y)\mu
&=
\mleft[ \zeta(X, Y), \mu \mright]_{\mathcal{g}}
\eas
for all $\mu, \nu \in \Gamma(\mathcal{g})$ and $X, Y \in \mathfrak{X}(M)$.
\end{theorems*}

We will denote such connections $\nabla^{\mathcal{G}}$ by $\nabla^{\mathrm{YM}}$. Additionally we will mention in Rem.\ \ref{rem:SimplicialDifferentialStuff} that multiplicativity of the total Maurer-Cartan form is in fact a closedness condition w.r.t.\ a certain simplicial differential $\delta$, introduced in \cite[beginning of \S 1.2]{crainic2003differentiable}. The generalized Maurer-Cartan equation is then an exactness-condition of the "classical" curvature for the total Maurer-Cartan form, its $\delta$-primitive given by $\zeta$. This is in alignment with the infinitesimal compatibility conditions which are also equivalent to statements about closedness and exactness w.r.t.\ the Chevalley-Eilenberg complex.

This will finish the discussion about the connection on $\mathcal{G}$, and it will be important for defining the field strength which will be discussed afterwards, starting with Def.\ \ref{def:NewFieldStrength}. For this we will always assume now that $\mathrm{H}\mathcal{G}$ is a Yang-Mills connection w.r.t.\ a $\zeta \in \Omega^2(M; \mathcal{g})$ (while $\mathrm{H}\mathcal{P}$ is still an Ehresmann connection).

\begin{definitions*}{(Generalized) Field strength}
We define the \textbf{(generalized) curvature} or \textbf{(generalized) field strength $F$ (of $A$)} as an element of $\Omega^2\mleft( \mathcal{P}; \pi^*\mathcal{g} \mright)$ by
\bas
F
&\coloneqq
\mathrm{d}^{\pi^*\nabla^{\mathrm{YM}}} A \circ \mleft( \pi^{\mathrm{H}\mathcal{P}}, \pi^{\mathrm{H}\mathcal{P}} \mright)
	+ \pi^!\zeta,
\eas
where $\mathrm{d}^{\pi^*\nabla^{\mathrm{YM}}}$ is the exterior covariant derivative related to the pullback connection $\pi^*\nabla^{\mathrm{YM}}$, and $\pi^{\mathrm{H}\mathcal{P}}: \mathrm{T}\mathcal{P} \to \mathrm{H}\mathcal{P}$ is the canonical projection onto the associated Ehresmann connection $\mathrm{H}\mathcal{P}$ on $\mathcal{P}$; that is,
\bas
F(X, Y)
&=
\mathrm{d}^{\pi^*\nabla^{\mathrm{YM}}} A \mleft( \pi^{\mathrm{H}\mathcal{P}}(X), \pi^{\mathrm{H}\mathcal{P}}(Y) \mright)
	+ \mleft(\pi^*\zeta\mright) \bigl( \mathrm{D}\pi(X), \mathrm{D}\pi(Y) \bigr)
\eas
for all $X, Y \in \mathfrak{X}(\mathcal{P})$.
\end{definitions*}

With a lengthy calculation evolving around that $\mathrm{H}\mathcal{G}$ is a Yang-Mills connection, especially using the generalized Maurer-Cartan equation, we will show in Prop.\ \ref{prop:NewFieldStrengthWithCoolProps} that $F$ transforms in a suitable way when making a pull-back w.r.t.\ the modified right-pushforward, which is the last step needed for defining the physical theory and its associated Lagrangian.

\begin{propositions*}{Properties of the generalized field strength, simplified notation}
We have the following properties of the field strength:
\begin{itemize}
	\item \textbf{(Form of type $\mathrm{Ad}$)}
	\bas
	\mathcal{r}_{\sigma}^!F
	&=
	\mathrm{Ad}_{\sigma^{-1}} \circ F
	\eas
	for all (local) $\sigma \in \Gamma(\mathcal{G})$.
	\item \textbf{(Horizontal form)}
	\newline
	For $X, Y \in \mathrm{T}_p\mathcal{P}$ ($p \in \mathcal{P}$) we have
	\bas
	F(X, Y)
	&=
	0
	\eas
	if either of $X$ and $Y$ is vertical.
\end{itemize}
\end{propositions*}

We will also derive a structure equation in Thm.\ \ref{thm:StructureEq}.

\begin{theorems*}{Structure equation of the generalized field strength}
We have the \textbf{structure equation}
\bas
F
=
\mathrm{d}^{\pi^*\nabla^{\mathrm{YM}}} A
	+ \frac{1}{2} \mleft[ A \stackrel{\wedge}{,} A \mright]_{\pi^*\mathcal{g}}
	+ \pi^!\zeta.
\eas
\end{theorems*}

We will argue  that $\zeta$ of course also affects the Bianchi identity, which we have in fact already derived in earlier works.

\begin{theorems*}{Generalized Bianchi identity, \newline \cite[\S 7, Thm.\ 7.3]{My1stpaper} \& \cite[\S 5, Thm.\ 5.1.42]{MyThesis}}
We have the \textbf{(generalized) Bianchi identity}
\bas
\mathrm{d}^{\pi^*\nabla^{\mathrm{YM}}}F
	+ \mleft[ A \stackrel{\wedge}{,} F \mright]_{\pi^*\mathcal{g}}
&=
\pi^! \mathrm{d}^{\nabla^{\mathrm{YM}}} \zeta.
\eas
\end{theorems*}

Similarly to the discussion about connections, Subsection \ref{CurvatureSubsection} will be concluded with a discussion about the gauge transformation of $F$ in Thm.\ \ref{thm:GaugeTrafoOfCurv}; as for $A$ the gauge transformations of $F$ are given by form-pullbacks with a (base-preserving) principal bundle automorphism $H$ with associated unique $\sigma^H \in \Gamma(\pi^*\mathcal{G})$ given by $H(p) = p \cdot \sigma^H_p$ for all $p \in \mathcal{P}$.

\begin{theorems*}{Gauge transformation of the generalized field strength, simplified notation}
We have that $H^!F$ is the field strength related to $H^!A$ and
\bas
H^!F
&=
{\mathrm{Ad}_{\mleft(\sigma^H\mright)^{-1}}} \circ F.
\eas
\end{theorems*}

Again as for $A$, we are interested about what this formula for the gauge transformation implies for pull-backs $F_s \coloneqq s^!F$ w.r.t.\ a gauge $s$ of $\mathcal{P}$ when changing the choice of $s$. This is straight-forward to calculate and will be stated in Thm.\ \ref{thm:LocalGaugeTrafoChangeGaugeFieldStrength}.

\begin{theorems*}{Gauge transformations again as a change of gauge}
Let $U_i$ and $U_j$ be two open subsets of $M$ so that $U_i \cap U_j \neq \emptyset$, two gauges $s_i \in \Gamma\mleft(\mathcal{P}|_{U_i}\mright)$ and $s_j \in \Gamma\mleft(\mathcal{P}|_{U_j}\mright)$, and the unique $\sigma_{ji} \in \Gamma\mleft( \mleft.\mathcal{G}\mright|_{U_i \cap U_j} \mright)$ with $s_i = s_j \cdot \sigma_{ji}$ on $U_i \cap U_j$.

Then we have for the fields strength of $A$ over $U_i \cap U_j$ that
\bas
F_{s_i}
&=
\mathrm{Ad}_{\sigma_{ji}^{-1}}\circ F_{s_j}.
\eas
\end{theorems*}

Eventually in Section \ref{CYMSection} we will start to introduce the physical theory, that is, first of all defining the Lagrangian of a gauge theory based on principal LGB-bundles in Subsection \ref{CYMDefGaugeInv}. Due to that we will have pointed out that this theory integrates the infinitesimal curved Yang-Mills-Higgs gauge theory developed by Alexei Kotov and Thomas Strobl (in the case of LABs), we are going to label this theory \textbf{curved Yang-Mills gauge theory} in Def.\ \ref{def:CYMGTFinally}. To do so, $M$ is now a spacetime, so that we can define the Hodge star operator $*$, and we assume that we have an $\mathrm{Ad}$-invariant fibre metric $\kappa$ on $\mathcal{g}$.

\begin{definitions*}{Curved Yang-Mills gauge theory}
Let $\mleft( U_i \mright)_i$ be an open covering of $M$ so that there are subordinate gauges $s_i \in \Gamma\mleft(\mathcal{P}|_{U_i}\mright)$. Then the top-degree form $\mathfrak{L}_{\mathrm{CYM}}[A] \in \Omega^{\mathrm{dim}(M)}(M; \mathbb{R})$, defined locally by
\bas
\mleft.\bigl(\mathfrak{L}_{\mathrm{CYM}}[A]\bigr)\mright|_{U_i}
&\coloneqq 
- \frac{1}{2} \kappa \mleft( F_{s_i} \stackrel{\wedge}{,} *F_{s_i} \mright),
\eas
is called the \textbf{curved Yang-Mills Lagrangian}.
\end{definitions*}

We will have argued in Cor.\ \ref{cor:ContractionOfLocalFIeldWellDefined} that this Lagrangian is well-defined; a trivial consequence of the previously-highlighted gauge transformations. Similarly, we will achieve gauge invariance in Thm.\ \ref{thm:GaugeInvarianceOfLagrangian}.

\begin{theorems*}{Gauge invariance of the curved Yang-Mills Lagrangian}
We have
\bas
\mathfrak{L}_{\mathrm{CYM}}\mleft[ H^!A \mright]
&=
\mathfrak{L}_{\mathrm{CYM}}[A]
\eas
for all principal bundle automorphisms $H$.
\end{theorems*}

As the final part of statements we will discuss new examples of gauge theory in Subsection \ref{CYMExamples}, after pointing out that this gauge theory generalizes the classical formalism. We will also discuss this matter in the context of my previously-mentioned Ph.D.\ thesis, \cite{MyThesis}, and the field redefinitions. Especially, the paper concludes with the reason why this project started: The inner group bundle of the Hopf fibration $\mathds{S}^7 \to \mathds{S}^4$ is an example of curved Yang-Mills gauge theory, so that $\nabla^{\mathrm{YM}}$ is not curved; we will present this in Ex.\ \ref{ex:HopfEx}. We will point out that this is indeed due to the fact that involved LGB is non-trivial, so that we also conjecture that trivial LGBs imply the existence of a field redefinition flattening $\nabla^{\mathrm{YM}}$.

\begin{examples*}{Hopf fibration $\mathds{S}^7 \to \mathds{S}^4$ giving rise to a curved Yang-Mills gauge theory}
Let $P$ be the Hopf bundle
\begin{center}
	\begin{tikzcd}
		\mathrm{SU}(2) \cong \mathds{S}^3 \arrow{r}	& \mathds{S}^7 \arrow{d} \\
			& \mathds{S}^4
	\end{tikzcd}
\end{center}
We define the principal bundle $\mathcal{P}$ and LGB $\mathcal{G}$ as the inner group bundle of $P$
\bas
\mathcal{P}
\coloneqq
\mathcal{G}
&\coloneqq
c_{\mathrm{SU}(2)}(P)
\coloneqq 
(P\times G) \Big/ G,
\eas
where $(P\times G) \Big/ G$ is the associated Lie group bundle, where the action of $G$ on $G$ is given by its conjugation, $c_g(q) = gqg^{-1}$ for all $g, q \in G$. This principal $c_{\mathrm{SU}(2)}(P)$-bundle admits the structure as curved Yang-Mills gauge theory, which is neither classical nor pre-classical. Furthermore, all possible (curved) Yang-Mills gauge theory structures lead to a curved $\nabla^{\mathrm{YM}}$ and non-zero $\zeta$, which also implies that this gauge theory cannot be described with the typical formalism of gauge theory.
\end{examples*}

 We will finish this paper with a short discussion about future prospects in Section \ref{conclusions}, highlighting further constructions and conjecturing how to integrate the full formalism of the gauge theory developed by Alexei Kotov and Thomas Strobl.

\subsection{Basic notations and remarks}\label{BasicNotations}

\begin{itemize}
	\item The appendix serves for providing extra information or background knowledge which did not fit in the flow of this paper's text. We will sometimes refer to the appendix, but the experienced reader may be able to ignore the appendix. 
	\item In the main text we usually repeat and reintroduce needed objects for the statements (like "\textbf{Let $M$ be a manifold [...]}" in every statement) (almost) allowing to just read the statements without having read the text introducing it, while the appendix is written as a continuous text which has to be read as a whole in order to understand the essential statements.
	\item As usual, there will be definitions of certain objects depending on other elements, and for keeping notations simple we will not always explicitly denote all dependencies. It will be clear by context on which it is based on, that is for example, if we define an object $A$ using the notion of Lie algebra actions $\gamma$ and we write "Let $X$ be an object $A$", then it will be clear by context which Lie algebra action is going to be used, for example given in a previous sentence writing "Let $\gamma$ be a Lie algebra action".
	\item Throughout this work we always use Einstein's sum convention if suitable.
	\item If not mentioned explicitly, we always assume finite dimensions and morphisms denote base-preserving ones.
	\item With $f^*F$ we denote the pullback/pull-back of the fibre bundles $F \to M$ under a smooth map $f: N \to M$. Similarly we denote the pullbacks of sections of a fibre bundle.
	%We will also have sections $F$ as an element of $\Gamma\left( \left(\bigotimes_{m=1}^{l} E_m^*\right) \otimes E_{l+1} \right)$, where $E_1, \dots E_{l+1}$ ($l \in \mathbb{N}$) are real vector bundles of finite rank over $N$. Those pull-back as section, denoted by $\Phi^*F$, we will view as an element of $\Gamma\left( \mleft(\bigotimes_{m=1}^{l} \mleft(\Phi^*E_m\mright)^*\mright) \otimes \Phi^*E_{l+1} \right)$, and it is essentially given by
%\bas
	%(\Phi^*F)(\Phi^*\nu_1, \dotsc , \Phi^*\nu_l)
	%&=
	%\Phi^*\mleft( F\mleft( \nu_1, \dotsc, \nu_l \mright) \mright)
%\eas
%for all $\nu_1 \in \Gamma(E_1), \dotsc, \nu_l \in \Gamma(E_l)$, using that pullbacks of sections generate the sections of a pullback bundle. 
	\item For $V \to M$ a vector bundle over $M$ do not confuse the pull-back of sections with the pull-back of forms $\omega \in \Omega^l(M; V)$ ($l \in \mathbb{N}_0$), here denoted by $f^!\omega$, which is an element of $\Gamma\left( \mleft(\bigwedge_{m=1}^{l} \mathrm{T}^*M \mright) \otimes f^*V \right) \cong \Omega^l(M; f^*V)$, and not of $\Gamma\left( \mleft(\bigotimes_{m=1}^{l} \mleft(f^*\mathrm{T}N\mright)^*\mright) \otimes f^*V \right)$ like $f^*\omega$. 
	%$f^!\omega$ is defined by
%\ba
%\mleft.\mleft(f^!\omega\mright)(Y_1, \dots, Y_l)\mright|_p
%&\coloneqq
%\omega_{f(p)}\mleft(\mathrm{D}_pf\mleft(\mleft.Y_1\mright|_p\mright), \dots, \mathrm{D}_pf\mleft(\mleft.Y_l\mright|_p\mright)\mright)
%\ea
%for all $p \in M$ and $Y_1, \dots, Y_l \in \mathfrak{X}(M)$.
	\item Let $F \stackrel{\pi_F}{\to} M$ and $G \stackrel{\pi_G}{\to} N$ be two fibre bundles over smooth manifolds $M$ and $N$, respectively, and let $\phi: N \to M$ be a smooth map. Furthermore, let us assume we have a morphism $\Phi: G \to F$ of fibre bundles over $\phi$, that is, $\Phi$ is a smooth map such that the following diagram commutes
	\begin{center}
		\begin{tikzcd}
		G \arrow{d}{\pi_G} \arrow{r}{\Phi} & F \arrow{d}{\pi_F}\\
		N \arrow{r}{\phi} & M
		\end{tikzcd}
	\end{center}
especially, $\pi_F \circ \Phi = \phi \circ \pi_G$.
We make often use of that such morphisms have a 1:1 correspondence to \textbf{base-preserving} fibre bundle morphisms $\widetilde{\Phi}: G \to \phi^*F$, \textit{i.e.}\ $\widetilde{\Phi}$ is a smooth map with $\phi^*\pi_F \circ \widetilde{\Phi} = \pi_G$. For $p \in N$ the morphism $\widetilde{\Phi}$ has the form
\bas
\widetilde{\Phi}_p
&=
(p, \Phi_p),
\eas
that is,
\bas
\widetilde{\Phi}_p(g)
&=
\bigl(p, \Phi_p(g) \bigr)
\eas
for all $g \in G_p$, which is well-defined since $\Phi_p(g) \in F_{\phi(p)}$. The map $\widetilde{\Phi} \mapsto \Phi \coloneqq \mathrm{pr}_2 \circ \widetilde{\Phi}$ is then a bijective map between base-preserving morphisms $G \to \phi^*F$ and morphisms $G \to F$ over $\phi$, where $\mathrm{pr}_2$ is the projection onto the second component. 

In total, $\widetilde{\Phi}$ is a base-preserving morphism if and only if $\Phi$ is a morphism over $\phi$; in fact, one defines pullback bundles in such a way that this equivalence holds. Observe that $\widetilde{\Phi}$ is an isomorphism (diffeomorphism) if and only if $\Phi$ is a fibre-wise isomorphism (diffeomorphism).

One can extend all of this similarly for more specific types of morphisms like vector bundle-morphisms.

Very often we will not mention this 1:1 correspondence explicitly, it should be clear by context. Hence, we will also denote $\widetilde{\Phi}$ by $\Phi$. In fact, we usually calculate with $\widetilde{\Phi}$, while $\Phi$ and its diagram may only arise to give an illustration about the geometry. However, sometimes we may need to be careful and then we will explicitly mention 

\item When we differentiate maps $\gamma$ depending on just one parameter $t \in \mathbb{R}$, then we may shortly write
\bas
\frac{\mathrm{d}}{\mathrm{d}t} \gamma(t)
&\coloneqq
\mleft.\frac{\mathrm{d}}{\mathrm{d}t} [t \mapsto \gamma]\mright|_t.
\eas

\item We will often have connections on a bundle, that is, a splitting into a horizontal and vertical bundle of its tangent bundle. When we write "Let $\pi$ be the projection onto the vertical/horizontal bundle", then this will be always w.r.t.\ to the fixed connection even though we are not explicitly mention it again.

\item We also need to extend contractions of tensors to graded extensions:
\begin{definitions}{Graded extension of products, \newline \cite[generalization of Definition 5.5.3; page 275]{Hamilton}}{GradingOfProducts}
Let $l \in \mathbb{N}$ and $E_1, \dots E_{l+1} \to M$ be vector bundles over a smooth manifold $M$, and $F \in \Gamma\left( \left(\bigotimes_{m=1}^{l} E_m^*\right) \otimes E_{l+1} \right)$. Then we define the \textbf{graded extension of $F$} as
	\bas
\Omega^{k_1}(M; E_1) \times \dots \times \Omega^{k_l}(M; E_l)
&\to \Omega^{k}(M; E_{l+1}), \\
(A_1, \dots, A_l)
&\mapsto
F\mleft(A_1\stackrel{\wedge}{,} \dotsc \stackrel{\wedge}{,} A_l\mright),
\eas
where $k := k_1+\dots k_l$ and $k_i \in \mathbb{N}_0$ for all $i\in \{1, \dots, l\}$. $F\mleft(A_1\stackrel{\wedge}{,} \dotsc \stackrel{\wedge}{,} A_l\mright)$ is defined as an element of $\Omega^{k}(M; E_{l+1})$ by
\bas
&F\mleft(A_1\stackrel{\wedge}{,} \dotsc \stackrel{\wedge}{,} A_l\mright)\mleft(Y_1, \dots, Y_{k}\mright)
\coloneqq \\
&\frac{1}{k_1! \cdot \dots \cdot k_l!} \sum_{\sigma \in S_{k}} \mathrm{sgn}(\sigma) ~ F\left( A_1\left( Y_{\sigma(1)}, \dots, Y_{\sigma(k_1)} \right), \dots, A_l\left( Y_{\sigma(k-k_l+1)}, \dots, Y_{\sigma(k)} \right) \right)
\eas
for all $Y_1, \dots, Y_{k} \in \mathfrak{X}(M)$, where $S_{k}$ is the group of permutations of $\{1, \dots, k\}$ and $\mathrm{sgn}(\sigma)$ the signature of a given permutation $\sigma$. 

$\stackrel{\wedge}{,}$ may be written just as a comma when a zero-form is involved.
%
%Locally, with respect to given frames $\mleft( e^{(i)}_{a_i} \mright)_{a_i}$ of $E_i$, this definition has the form
%\ba\label{CoordExprOfGradedExtension}
%F\mleft(A_1\stackrel{\wedge}{,} \dotsc \stackrel{\wedge}{,} A_l\mright)
%&=
%F\mleft(e^{(1)}_{a_1}, \dotsc, e^{(l)}_{a_l}\mright) \otimes A_1^{a_1} \wedge \dotsc \wedge A_l^{a_l}
%\ea
%for all $A_i = A_i^{a_i} \otimes e^{(i)}_{a_i}$, where $A_i^{a_i}$ are $k_i$-forms on $M$.
\end{definitions}

In case of antisymmetric tensors we of course get:

\begin{propositions}{Graded extensions of antisymmetric tensors}{GradedExtensionPlusAntiSymm}
Let $E_1, E_2 \to N$ be real vector bundles of finite rank over a smooth manifold $N$, $F \in \Omega^2(E_1; E_2)$. Then
\ba
F \mleft( A \stackrel{\wedge}{,} B \mright)
&=
-\mleft( -1 \mright)^{km}
F \mleft( B \stackrel{\wedge}{,} A \mright)
\ea 
for all $A \in \Omega^k(N; E_1)$ and $B \in \Omega^m(N; E_1)$ ($k,m \in \mathbb{N}_0$). Similarly extended to all $F \in \Omega^l(E_1; E_2)$.
\end{propositions}

\begin{remark}
\leavevmode\newline
This is a generalization of similar relations just using the Lie algebra bracket $\mleft[ \cdot, \cdot\mright]_{\mathfrak{g}}$ of a Lie algebra $\mathfrak{g}$, see \cite[\S 5, first statement of Exercise 5.15.14; page 316]{Hamilton}.
\end{remark}

\begin{proof}
\leavevmode\newline
Trivial, for example by using local coordinates.
\end{proof}

\item References are not only given in the text, the references of referenced statements and
definitions are most of the time given in the title of those statements. The title also mentions
whether the statement as written in this paper is a variation or generalization; if it is a
strong generalization, then the reference will be mentioned in a remark after the statement
or its proof.

\end{itemize}

\subsection{Assumed background knowledge}

It is highly recommended to have basic knowledge about differential geometry and gauge theory as presented in \cite[especially Chapter 1 to 5]{Hamilton}, and we will follow the style and labeling as in \cite{Hamilton} when we generalize certain notions; however, sometimes we will still give explicit references to help with more technical details. It can be useful to have knowledge about Lie algebra and Lie group bundles, and even Lie algebroids and Lie groupoids, but we will introduce the basic notions of LGBs and LABs such that it is not necessarily needed to have knowledge about these upfront.

We also often give references about Lie group bundles (LGBs), but the given references are most of the time about Lie groupoids. If the reader has no knowledge about Lie groupoids, then it is important to know that LGBs are a special example of Lie groupoids; Lie groupoids carry "two projections", called \textbf{source} and \textbf{target}. An LGB is a special example of a Lie groupoid whose source equals the target.\footnote{But not every Lie groupoid with equal source and target is an LGB, they're in general bundles of Lie groups which is not completely the same; this nuance will not be important here.} If you look into such a reference, then the source and target are often denoted by $\alpha$ and $\beta$, or by $s$ and $t$; simply put both to be the same and identify these with our bundle projection which we often denote by $\pi$ or $\pi_{\mathcal{G}}$, $\mathcal{G}$ the corresponding LGB. In that way it should be possible to read the references without the need to know Lie groupoids. However, we try to re-prove the needed statements such that these types of references could be avoided by the reader.

%\twocolumn
%\section{Basic definitions}\label{BasicDefinitions}
%
%In the following, we denote with $V^*$ the dual of a vector bundle $V \to N$ over a smooth manifold $N$, and $\Phi^*V$ denotes the pull-back of $V$ by $\Phi: M \to N$, a smooth map from a smooth manifold $M$ to $N$. We have a similar notation for the pull-back of sections, especially we will have sections $F$ as an element of $\Gamma\left( \left(\bigotimes_{m=1}^{l} E_m^*\right) \otimes E_{l+1} \right)$, where $E_1, \dots, E_{l+1} \to N$ ($l \in \mathbb{N}$) are real vector bundles of finite rank over a smooth manifold $N$, and $\Gamma(\cdot)$ denotes the space of smooth sections. Then we view the pull-back $\Phi^*F$ as an element of $\Gamma\left( \mleft(\bigotimes_{m=1}^{l} \mleft(\Phi^*E_m\mright)^*\mright) \otimes \Phi^*E_{l+1} \right)$, and it is essentially given by
%\bas
	%(\Phi^*F)(\Phi^*\nu_1, \dotsc , \Phi^*\nu_l)
	%&=
	%\Phi^*\mleft( F\mleft( \nu_1, \dotsc, \nu_l \mright) \mright)
%\eas
%for all $\nu_1 \in \Gamma(E_1), \dotsc, \nu_l \in \Gamma(E_l)$. In general we also make use of that sections of $\Phi^*E$ can be viewed as sections of $E$ along $\Phi$, where $E \stackrel{\pi}{\to} N$ is any vector bundle over $N$. Let $\mu \in \Gamma(\Phi^*E)$, then it has the form $\mu_p = (p, u_p)$ for all $p \in M$, where $u_p \in E_{\Phi(p)}$, the fibre of $E$ at $\Phi(p)$; and a section $\nu$ of $E$ along $\Phi$ is a smooth map $M \to E$ such that $\pi \circ \nu = \Phi$. Then on one hand $\mathrm{pr}_2 \circ \mu$ is a section along $\Phi$, where $\mathrm{pr}_2$ is the projection onto the second component, and on the other hand $M \ni p \mapsto (p, \nu_p)$ defines an element of $\Gamma(\Phi^*E)$. With that one can show that there is a 1:1 correspondence of $\Gamma(\Phi^*E)$ with sections along $\Phi$. Similarly, vector bundle morphisms $L: G \to E$ over $\Phi$ have 1:1 correspondences to base-preserving vector bundle morpishms $G \to \Phi^*E$, where $G \to M$ is a vector bundle over $M$. We do not necessarily mention it when we make use of such trivial identifications, it should be clear by the context. For example $\mathrm{D}\Phi$ denotes the total differential of $\Phi$ (also called tangent map). It can be viewed as a vector bundle morphism $\mathrm{T}M \to \mathrm{T}N$ over $\Phi$, and we often view it as an element of $\Omega^1(M; \Phi^*\mathrm{T}N)$ by $\mathfrak{X}(M) \ni Y \mapsto \mathrm{D}\Phi(Y)$, where $\mathrm{D}\Phi(Y) \in \Gamma(\Phi^*\mathrm{T}N), M \ni p \mapsto \mathrm{D}_p\Phi(Y_p)$.
%
%Additionally, with $\Omega^k(N; E)$ ($k \in \mathbb{N}_0$) we denote $k$-forms on $N$ with values in a vector bundle $E \to N$, and we always use the Einstein's sum convention. If one has a connection $\nabla$ on a vector bundle $V \to N$, then one has the notion of the exterior covariant derivative on $\Omega^p(M;E)$, denoted by $\mathrm{d}^\nabla$. In the case of a trivial vector bundle $V=N \times W \to N$, where $W$ is some vector space, we will often use the \textbf{canonical flat connection} for $\nabla$, defined by $\nabla \nu = 0$, where $\nu$ is a constant section of $N \times W$, see \textit{e.g.}~\cite[Example 5.1.7; page 260f.]{hamilton}\ for a geometric interpretation as horizontal distribution. The canonical flat connection is clearly uniquely defined (if a trivialization is given) because constant sections generate all sections and due to the Leibniz rule and linearity of $\nabla$. Let $\mleft( e_a \mright)_a$ be a constant global frame of $N \times W$, thence,
%\bas
%\mathrm{d}^\nabla \omega
%&=
%\mathrm{d} \omega^a \otimes e_a
%\eas
%for all $\omega \in \Omega^p(M; W)$, where we write $\omega= \omega^a \otimes e_a$. Hence, we define
%\ba
%\mathrm{d}\omega
%&\coloneqq
%\mathrm{d}^\nabla \omega,
%\ea
%when $\nabla$ is the canonical flat connection. $\mathrm{d}$ is clearly a differential.
%
%As usual, there will be definitions of certain objects depending on other elements, and for keeping notations simple we will not always explicitly denote all dependencies. It will be clear by context on which it is based on, that is, when we define an object $A$ using the notion of Lie algebra actions $\gamma$ and we write "Let $A$ be [as defined before]", then it will be clear by context which Lie algebra action is going to be used, for example given in a previous sentence writing "Let $\gamma$ be a Lie algebra action".
%%, and recall the following wedge product\footnote{As also defined in \cite[\S 5, third part of Exercise 5.15.12; page 316]{hamilton}.} of forms with values in a vector bundle $E$ and values in its space of endomorphisms $\mathrm{End}(E)$,
%%\bas
%%\wedge: \Omega^k(N; \mathrm{End}(E)) \times \Omega^l(N; E)
%%&\mapsto
%%\Omega^{k+l}(N; E) \\
%%(T, \omega) &\mapsto T \wedge \omega
%%\eas
%%for all $k, l \in \mathbb{N}_0$, given by
%%\ba\label{DefVonWedgedemitEnd}
%%\mleft( T \wedge \omega \mright) \mleft( Y_1, \dotsc, Y_{k+l} \mright)
%%&\coloneqq
%%\frac{1}{k! l!} \sum_{\sigma \in S_{k+l}} \mathrm{sgn}(\sigma) ~
	%%T \mleft( Y_{\sigma(1)}, \dotsc, Y_{\sigma(k)} \mright)
		%%\mleft( \omega\mleft( Y_{\sigma(k+1)}, \dotsc, Y_{\sigma(k+l)} \mright) \mright),
%%\ea
%%where $S_{k+l}$ is the group of permutations $\{1, \dotsc, k+l\}$. This is then locally given by, with respect to a frame $\mleft( e_a \mright)_a$ of $E$,
%%\bas
%%T \wedge \omega &= T(e_a) \wedge w^a,
%%\eas
%%where $T$ acts as an endomorphism on $e_a$, \textit{i.e.}~$T(e_a) \in \Omega^k(N; E)$, and $\omega = \omega^a \otimes e_a$. Also recall that there is the canonical extension of $\nabla$ on $\mathrm{End}(E)$ by forcing the Leibniz rule. We still denote this connection by $\nabla$, too.
%
%We also need the following definitions.
%
%\begin{definitions}{Graded extension of products, \newline \cite[generalization of Definition 5.5.3; page 275]{hamilton}}{GradingOfProducts}
%Let $l \in \mathbb{N}$ and $E_1, \dots E_{l+1} \to N$ be vector bundles over a smooth manifold $N$, and $F \in \Gamma\left( \left(\bigotimes_{m=1}^{l} E_m^*\right) \otimes E_{l+1} \right)$. Then we define the \textbf{graded extension of $F$} as
	%\bas
%\Omega^{k_1}(N; E_1) \times \dots \times \Omega^{k_l}(N; E_l)
%&\to \Omega^{k}(N; E_{l+1}), \\
%(A_1, \dots, A_l)
%&\mapsto
%F\mleft(A_1\stackrel{\wedge}{,} \dotsc \stackrel{\wedge}{,} A_l\mright),
%\eas
%where $k := k_1+\dots k_l$ and $k_i \in \mathbb{N}_0$ for all $i\in \{1, \dots, l\}$. $F\mleft(A_1\stackrel{\wedge}{,} \dotsc \stackrel{\wedge}{,} A_l\mright)$ is defined as an element of $\Omega^{k}(N; E_{l+1})$ by
%\bas
%&F\mleft(A_1\stackrel{\wedge}{,} \dotsc \stackrel{\wedge}{,} A_l\mright)\mleft(Y_1, \dots, Y_{k}\mright)
%\coloneqq \\
%&\frac{1}{k_1! \cdot \dots \cdot k_l!} \sum_{\sigma \in S_{k}} \mathrm{sgn}(\sigma) ~ F\left( A_1\left( Y_{\sigma(1)}, \dots, Y_{\sigma(k_1)} \right), \dots, A_l\left( Y_{\sigma(k-k_l+1)}, \dots, Y_{\sigma(k)} \right) \right)
%\eas
%for all $Y_1, \dots, Y_{k} \in \mathfrak{X}(N)$, where $S_{k}$ is the group of permutations of $\{1, \dots, k\}$ and $\mathrm{sgn}(\sigma)$ the signature of a given permutation $\sigma$. 
%
%$\stackrel{\wedge}{,}$ may be written just as a comma when a zero-form is involved.
%
%Locally, with respect to given frames $\mleft( e^{(i)}_{a_i} \mright)_{a_i}$ of $E_i$, this definition has the form
%\ba\label{CoordExprOfGradedExtension}
%F\mleft(A_1\stackrel{\wedge}{,} \dotsc \stackrel{\wedge}{,} A_l\mright)
%&=
%F\mleft(e^{(1)}_{a_1}, \dotsc, e^{(l)}_{a_l}\mright) \otimes A_1^{a_1} \wedge \dotsc \wedge A_l^{a_l}
%\ea
%for all $A_i = A_i^{a_i} \otimes e^{(i)}_{a_i}$, where $A_i^{a_i}$ are $k_i$-forms on $N$.
%\end{definitions}
%
%\begin{remark}
%\leavevmode\newline
%Assume $F \in \Gamma\left( \mleft(\bigwedge_{m=1}^{l} \mathrm{T}^*N \mright) \otimes E \right) \cong \Omega^l(N; E)$ for some vector bundle $E$, \textit{i.e.}~$F$ is an $l$-form on $N$ with values in $E$. The pull-back $\Phi^*F$ by $\Phi$ can be then viewed as an element of $\Gamma\left( \bigwedge_{m=1}^{l} \mleft(\Phi^*\mathrm{T}N\mright)^* \otimes \Phi^*E \right)$.
%
%Do not confuse this pull-back with the pull-back of forms, here denoted by $\Phi^!F$, which is an element of $\Gamma\left( \mleft(\bigwedge_{m=1}^{l} \mathrm{T}^*M \mright) \otimes \Phi^*E \right) \cong \Omega^l(M; \Phi^*E)$ defined by
%\ba
%\mleft.\mleft(\Phi^!F\mright)(Y_1, \dots, Y_l)\mright|_p
%&\coloneqq
%F_{\Phi(p)}\mleft(\mathrm{D}_p\Phi\mleft(\mleft.Y_1\mright|_p\mright), \dots, \mathrm{D}_p\Phi\mleft(\mleft.Y_l\mright|_p\mright)\mright)
%\ea
%for all $p \in M$ and $Y_1, \dots, Y_l \in \mathfrak{X}(M)$. Then
%\ba\label{EqPullBackFormelFuerVerschiedeneDefinitionen}
%\Phi^!F 
%&=
%\frac{1}{l!}~
%\mleft(\Phi^*F\mright) ( \underbrace{\mathrm{D}\Phi \stackrel{\wedge}{,} \dotsc \stackrel{\wedge}{,} \mathrm{D}\Phi}_{l \text{ times}} )
%\ea
%by using the anti-symmetry of $F$ and Def.~\ref{def:GradingOfProducts}, \textit{i.e.}
%\bas
%&\mleft.\frac{1}{l!}~
%\Big(\mleft(\Phi^*F\mright) ( \mathrm{D}\Phi \stackrel{\wedge}{,} \dotsc \stackrel{\wedge}{,} \mathrm{D}\Phi ) \Big) (Y_1, \dots, Y_l)\mright|_p \\
%&\hspace{1cm}
%=
%\frac{1}{l!}~
%\sum_{\sigma \in S_{l}} \mathrm{sgn}(\sigma) ~ \underbrace{(\Phi^*F)\mleft(\mathrm{D}\Phi\mleft(Y_{\sigma(1)}\mright), \dots, \mathrm{D}\Phi\mleft(Y_{\sigma(l)}\mright)\mright)}_{\mathclap{= \mathrm{sgn}(\sigma) ~ (\Phi^*F)\mleft(\mathrm{D}\Phi\mleft(Y_{1}\mright), \dots, \mathrm{D}\Phi\mleft(Y_{l}\mright)\mright)}}\Big|_p \\
%&\hspace{1cm}
%=
%\frac{1}{l!}~ \underbrace{\mleft( \sum_{\sigma \in S_{l}} 1 \mright)}_{= l!} ~
%F_{\Phi(p)}\mleft(\mathrm{D}_p\Phi\mleft(\mleft.Y_{1}\mright|_p\mright), \dots, \mathrm{D}_p\Phi\mleft(\mleft.Y_{l}\mright|_{p}\mright)\mright) \\
%&\hspace{1cm}
%= \mleft.\mleft(\Phi^!F\mright)(Y_1, \dots, Y_l)\mright|_p
%\eas
%for all $p \in M$ and $Y_1, \dots, Y_l \in \mathfrak{X}(M)$.
%\end{remark}
%
%In case of antisymmetric tensors we of course preserve that.
%
%\begin{propositions}{Graded extensions of antisymmetric tensors}{GradedExtensionPlusAntiSymm}
%Let $E_1, E_2 \to N$ be real vector bundles of finite rank over a smooth manifold $N$, $F \in \Omega^2(E_1; E_2)$. Then
%\ba
%F \mleft( A \stackrel{\wedge}{,} B \mright)
%&=
%-\mleft( -1 \mright)^{km}
%F \mleft( B \stackrel{\wedge}{,} A \mright)
%\ea 
%for all $A \in \Omega^k(N; E_1)$ and $B \in \Omega^m(N; E_2)$ ($k,m \in \mathbb{N}_0$). Similarly extended to all $F \in \Omega^l(E_1; E_2)$.
%\end{propositions}
%
%\begin{remark}
%\leavevmode\newline
%This is a generalization of similar relations just using the Lie algebra bracket $\mleft[ \cdot, \cdot\mright]_{\mathfrak{g}}$ of a Lie algebra $\mathfrak{g}$, see \cite[\S 5, first statement of Exercise 5.15.14; page 316]{hamilton}.
%\end{remark}
%
%\begin{proof}
%\leavevmode\newline
%Trivial by using Eq.~\eqref{CoordExprOfGradedExtension}.
%\end{proof}
%
%We also need to know what a Lie algebroid is, a generalization of both, tangent bundles and Lie algebras; this concept will just be defined, refer to the references for thorough discussions of these definitions, especially \cite{mackenzieGeneralTheory} and \cite[\S VII; page 113ff.]{DaSilva}.
%
%\begin{definitions}{Lie algebroid, \newline \cite[\S 3.3, first part of Definition 3.3.1; page 100]{mackenzieGeneralTheory}}{test}
%%\leavevmode\newline
%Let $E \to N$ be a real vector bundle of finite rank. Then $E$ is a smooth Lie algebroid if there is a bundle map $\rho: E \to \mathrm{T}N$, called the \textbf{anchor}, and a Lie algebra structure on $\Gamma(E)$ with Lie bracket $\mleft[ \cdot, \cdot \mright]_E$ satisfying
%\ba
  %\mleft[\mu, f \nu\mright]_E = f \mleft[\mu, \nu\mright]_E + \mathcal{L}_{\rho(\mu)}(f) ~ \nu
%\label{eq:E-Leibniz}
%\ea
%for all $f \in C^\infty(N)$ and $\mu, \nu \in \Gamma(E)$, where $\mathcal{L}_{\rho(\mu)}(f)$ is the action of the vector field $\rho(\mu)$ on the function $f$ by derivation. We will sometimes denote a Lie algebroid by $\mleft( E, \rho, \mleft[ \cdot, \cdot \mright]_E \mright)$.
%%We will sometimes denote a Lie algebroid by $\mleft( E, \rho, \mleft[ \cdot, \cdot \mright]_E \mright)$.
%\end{definitions}
%
%%Tangent bundles are a canonical example of Lie algebroids, their anchor is the identity with which we also equip them; another canonical example with zero anchor are the Lie algebra bundles:
%Tangent bundles and bundles of Lie algebras are canonical examples of Lie algebroids, their anchor is the identity and zero, respectively. The important example for us is a mixture of those examples:
%
%\begin{propositions}{Action Lie algebroids, \cite[\S 16.2, Example 5; page 114]{DaSilva}}{ActionLieoidsAreOids}
%Let $\mleft(\mathfrak{g}, \mleft[\cdot, \cdot \mright]_{\mathfrak{g}}\mright)$ be some Lie algebra equipped with a Lie algebra action $\gamma: \mathfrak{g} \to \mathfrak{X}(N)$ on a smooth manifold $N$. Then there is a unique Lie algebroid structure on $E = N \times \mathfrak{g}$ such that we have
%\ba
%\rho(\nu)
%&=
%\gamma(\nu),
%\\
%\mleft[\mu, \nu\mright]_E
%&=
%\mleft[\mu, \nu\mright]_{\mathfrak{g}}
%\ea
%for all constant sections $\mu, \nu \in \Gamma(E)$. We call this structure \textbf{action Lie algebroid}.
%\end{propositions}

\section{Lie group bundles (LGBs)}\label{LGBSection}

\subsection{Definition}
%\pagebreak
\begin{definitions}{Lie group bundle, \cite[\S 1.1, Def.\ 1.1.19; p. 11]{mackenzieGeneralTheory}}{LieGroupBundle}
Let $G, \mathcal{G}, M$ be smooth manifolds. A fibre bundle
\begin{center}
	\begin{tikzcd}
		G \arrow{r} & \mathcal{G} \arrow{d}{\pi} \\
		& M
	\end{tikzcd}
\end{center}
is called a \textbf{Lie group bundle (LGB)} if:
\begin{enumerate}
	\item $G$ and each fibre $\mathcal{G}_x \coloneqq \pi^{-1}\mleft( \{x\} \mright)$, $x\in M$, are Lie groups;
	\item there exists a bundle atlas $\mleft\{ \mleft( U_i, \phi_i \mright) \mright\}_{i \in I}$ such that the induced maps
	\bas
	\phi_{ix}
	&\coloneqq
	\mathrm{pr}_2 \circ \mleft. \phi_i\mright|_{\mathcal{G}_x}: \mathcal{G}_x \to G
	\eas
	are Lie group isomorphisms, where $I$ is an (index) set, $U_i$ are open sets covering $M$, $\phi_i: \mathcal{G}|_U \to U \times G$ subordinate trivializations, and $\mathrm{pr}_2$ the projection onto the second factor. This atlas will be called \textbf{Lie group bundle atlas} or \textbf{LGB atlas}.
\end{enumerate}

We also often say that \textbf{$\mathcal{G}$ is an LGB (over $M$)}, whose structural Lie group is either clear by context or not explicitly needed; and we may also denote LGBs by $G \to \mathcal{G} \stackrel{\pi}{\to} M$.

The global section $e$ of $\mathcal{G}$ so that $e_p$ is the neutral element of $\mathcal{G}_p$ will be labelled as the \textbf{neutral (element)/identity section (of $\mathcal{G}$)}; we may also denote it by $e^{\mathcal{G}}$.
\end{definitions}

\begin{remarks}{Principal and Lie group bundles}{LiegroupbundlesNotPrincipalBundles}
Beware, a Lie group bundle is \textbf{not} the same as a principal bundle $P \to M$ with the same fibre type $G$. First of all, the fibres of $P$ are just diffeomorphic to a Lie group, a priori they carry no Lie group structure, while the fibres of $\mathcal{G}$ carry a Lie group structure.
\newline

Second, on $P$ we have a multiplication given as an action of $G$ on $P$
\bas
P \times G \to P,
\eas
preserving the fibres $P_x$ ($x\in M$) and simply transitive on them. Restricted on $P_x$ we have
\bas
P_x \times G \to P_x.
\eas
For $\mathcal{G}$ we have canonically a multiplication over $x$ given by
\bas
\mathcal{G}_x \times \mathcal{G}_x \to \mathcal{G}_x,
\eas
also clearly simply transitive. Observe, the second factor is not "constant", \textit{i.e.}\ we do not have $\mathcal{G}_x \times G \to \mathcal{G}_x$ in general. Hence, there is in general no well-defined product $\mathcal{G} \times \mathcal{G} \to \mathcal{G}$ or $\mathcal{G} \times G \to \mathcal{G}$.
\newline

All of that is also resembled in the existence of sections. The existence of a section of $P$ has a 1:1 correspondence to trivializations of $P$, which is why $P$ in general only admits sections locally; see \textit{e.g.}\ \cite[\S 4.2, Thm.\ 4.2.19; page 219f.]{Hamilton}. $\mathcal{G}$ clearly admits always a global section, even if $\mathcal{G}$ is non-trivial; just take the section which assigns each base point the neutral element of its fibre.
\end{remarks}

If $M$ is a point we recover the notion of Lie groups, and, as usual, we have the notion of trivial LGBs:

\begin{examples}{Trivial LGB}{TrivialLGBundle}
The \textbf{trivial LGB} is given as the product manifold $M \times G \to M$ with canonical multiplication $(x, g) \cdot (x, q) \coloneqq (x, gq)$.
\end{examples}

We are also interested into LGB bundle morphisms:

\begin{definitions}{LGB morphism, \newline \cite[\S 1.2, special situation of Def.\ 1.2.1 \& 1.2.3, page 12]{mackenzieGeneralTheory}}{LGB morphism}
Let $\mathcal{H} \stackrel{\pi_{\mathcal{H}}}{\to} N$ and $\mathcal{G} \stackrel{\pi_{\mathcal{G}}}{\to} M$ be two LGBs over two smooth manifolds $N$ and $M$. An \textbf{LGB morphism $F$ over $f$} is a pair of smooth maps $F: \mathcal{H} \to \mathcal{G}$ and $f: N \to M$ such that
\ba\label{FibreRelationOverf}
\pi_{\mathcal{G}} \circ F &= f \circ \pi_{\mathcal{H}},\\
F(gq) &= F(g) ~ F(q)\label{LGBHomomorph}
\ea
for all $g, q \in \mathcal{H}$ with $\pi_{\mathcal{H}}(g) = \pi_{\mathcal{H}}(q)$. We then also say that \textbf{$F$ is an LGB morphism over $f$}. If $N = M$ and $f = \mathrm{id}_M$, then we often omit mentioning $f$ explicitly and just write that \textbf{$F$ is a (base-preserving) LGB morphism}.

We speak of an \textbf{LGB isomorphism (over $f$)} if $F$ is a diffeomorphism.
\end{definitions}

\begin{remark}\label{LGBMOrphismRemark}
\leavevmode\newline
\indent $\bullet$ The right hand side of Eq.\ \eqref{LGBHomomorph} is well-defined because of Eq.\ \eqref{FibreRelationOverf}.

$\bullet$ It is clear that condition 2 in Def.\ \ref{def:LieGroupBundle} is equivalent to say that $\mathcal{G}$ is locally isomorphic to a trivial LGB; as one may have expected already.

$\bullet$ If $F$ is a diffeomorphism, then also $f$: By Eq.\ \eqref{FibreRelationOverf} surjectivity of $f$ is clear; for $y \in M$ just take any $g \in \mathcal{G}_y$, and since $F$ is a bijective, we have a $q \in \mathcal{H}_x$ for some $x\in N$ with $F(q)=g$. By Eq.\ \eqref{FibreRelationOverf} we have $y = \pi_{\mathcal{G}}(F(q)) \stackrel{\eqref{FibreRelationOverf}}{=} f(x)$, thence, surjectivity follows. For injectivity we know by Eq.\ \eqref{LGBHomomorph} and \eqref{FibreRelationOverf} that $F\mleft(e^{\mathcal{H}}_x\mright) = e^{\mathcal{G}}_{f(x)}$, where $e^{\mathcal{H}}_x$ and $e^{\mathcal{G}}_{f(x)}$ denote the unique neutral elements of $\mathcal{H}_x$ and $\mathcal{G}_{f(x)}$, respectively. Assume that there are $x, x^\prime \in N$ with $f(x) = f(x^\prime)$, then we can derive
\bas
F\mleft(e^{\mathcal{H}}_x\mright)
&=
e^{\mathcal{G}}_{f(x)}
=
e^{\mathcal{G}}_{f(x^\prime)}
=
F\mleft(e^{\mathcal{H}}_{x^\prime}\mright).
\eas
Then we have $e^{\mathcal{H}}_x = e^{\mathcal{H}}_{x^\prime}$ due to that $F$ is bijective, and hence $x = x^\prime$. Therefore $f$ is bijective. Finally, $F^{-1}$ is by assumption also a diffeomorphism, Eq.\ \eqref{LGBHomomorph} clearly carries over, and Eq.\ \eqref{FibreRelationOverf} is w.r.t.\ $f^{-1}$, that is
\bas
\pi_{\mathcal{H}} \circ F^{-1} &= f^{-1} \circ \pi_{\mathcal{G}}.
\eas
Since $\pi_{\mathcal{H}} \circ F^{-1}$ is smooth and $\pi_{\mathcal{G}}$ is a smooth surjective submersion, it follows that $f^{-1}$ is smooth; this is a well-known fact for right-compositions with surjective submersions, see \textit{e.g.}\ \cite[\S 3.7.2, Lemma 3.7.5, page 153]{Hamilton}. We can conclude that $f$ is a diffeomorphism. Observe that we also concluded that $F^{-1}$ is an LGB isomorphism, too.
\end{remark}

Similar to the case of Lie groups, \textit{the} example of an LGB are the automorphisms of a vector bundle.

\begin{examples}{Automorphisms of a vector bundle,\newline \cite[\S 1.1, special situation of Ex.\ 1.1.12, page 8]{mackenzieGeneralTheory}}{AutOfVectorBundleAnLGB}
Let $V\to M$ be a vector bundle and $\mathrm{Aut}(V) \to M$ its bundle of fibre-wise automorphisms (not to be confused with the sections of $\mathrm{Aut}(V)$ which are the base-preserving automorphisms of $V$). Denote with $W$ the structural vector space of $V$, then $\mathrm{Aut}(V)$ is an LGB with structural Lie group $\mathrm{Aut}(W)$. It is clear that each fibre of $\mathrm{Aut}(V)$ is a Lie group, and the LGB atlas is directly inherited by a vector bundle atlas $\mleft\{ \mleft( U_i, L_i \mright) \mright\}_{i \in I}$ of $V$, where we use a similar notation as for LGB atlases, especially we have vector bundle trivializations $L_{i}: \mleft.V\mright|_{U_i} \to U_i \times W$. Then define an LGB atlas over the same open covering $\mleft( U_i \mright)_i$ by
\bas
\mleft. \mathrm{Aut}(V) \mright|_{U_i} &\to U_i \times \mathrm{Aut}(W),\\
T &\mapsto \mleft.L_{i} \circ T \circ L_{i}^{-1}\mright|_{\{x\} \times W},
\eas
where $T \in \mleft.\mathrm{Aut}(V)\mright|_x = \mathrm{Aut}(V_x)$,
and $U_i \times \mathrm{Aut}(W)$ acts canonically on $U_i \times W$ in a fibre-wise sense. Then it is trivial to check that these give local trivializations such that $\mathrm{Aut}(V)$ carries the structure as an LGB.
\end{examples}

\subsection{Associated Lie group bundles}\label{AssocLGBsSubSection}

For another important example recall that there is the notion of associated fibre bundles; following and stating the results of \cite[\S1, Construction 1.3.8, page 20]{mackenzieGeneralTheory} and \cite[\S 4.7, page 237ff.; see also Rem.\ 4.7.8, page 242f.]{Hamilton}: Let $P \stackrel{\pi_P}{\to} M$ be a principal bundle with structural Lie group $G$ over a smooth manifold $M$, $N$ another smooth manifold, and $\Psi$ a smooth left $G$-action denoted by
\bas
G \times N &\to N,\\
(g, v) &\mapsto \Psi(g, v) \coloneqq g \cdot v.
\eas
Then we have a right $G$-action on $P \times N$ given by
\bas
(P \times N) \times G &\to P \times N,\\
(p,v,g) &\mapsto \mleft( p \cdot g, g^{-1} \cdot v \mright),
\eas
and one can show that the quotient under this action, $P\times_\Psi N \coloneqq ( P \times N) \Big/ G$, yields the structure of a fibre bundle
\begin{center}
	\begin{tikzcd}
		N \arrow{r} & P\times_\Psi N \arrow{d}{\pi_{P\times_\Psi N}} \\
		& M
	\end{tikzcd}
\end{center}
such that the map to the equivalence classes $P \times N \to P \times_\Psi N$, $(p, v) \mapsto [p, v]$, is a smooth surjective submersion,
where the projection $\pi_{P\times_\Psi N}: P\times_\Psi N \to M$ is given by 
\bas
\pi_{P\times_\Psi N}\mleft( [p, v] \mright)
&\coloneqq
\pi_P(p)
\eas
for all $[p, v] \in P\times_\Psi N$. For $x \in M$, the fibre $\mleft(P\times_\Psi N\mright)_x$ is given by $\mleft( P_x \times N  \mright) \Big/ G = P_x \times_\Psi N$, and the fibre is diffeomorphic to $N$ by $N \ni v \mapsto [p, v] \in \mleft(P\times_\Psi N\mright)_x$ for a fixed $p \in P_x$. We will frequently use this diffeomorphism in the following without further notice.

A very important example are of course associated vector bundles, related to $N$ being a vetor space. We need a similar concept for Lie groups.

\begin{definitions}{Lie group representation on Lie groups, \newline \cite[special situation of the comment after Ex.\ 1.7.14, page 47]{mackenzieGeneralTheory}}{LieGroupActingOnLieGroup}
Let $G, H$ be Lie groups. Then a \textbf{Lie group representation of $G$ on $H$} is a smooth left action $\psi$ of $G$ on $H$
\bas
G \times H
&\to H,\\
(g,h)
&\mapsto
\psi_g(h)
\coloneqq
\psi(g, h)
\eas
such that
\ba
\psi_g(hq)
&=
\psi_g(h)
~ \psi_g(q)
\ea
for all $g \in G$ and $h,q \in H$.
\end{definitions}

\begin{remarks}{Note about labeling}{WhyRepresentation}
Observe that we have by the definition of group actions
\bas
\psi_{gg^\prime}
&=
\psi_g \circ \psi_{g^\prime}
\eas
for all $g, g^\prime \in G$, viewing $\psi_g$ as a map $H \to H$. Therefore we can view the action $\psi$ as a group homomorphism
\bas
G &\to \mathrm{Aut}(H),
\eas
where $\mathrm{Aut}(H)$ is the set of Lie group automorphisms. The similarity to Lie group representations on vector spaces is obvious, thence the name. In fact, these integrate Lie group representations on Lie algebras: Observe that $\mathrm{D}_{e}\psi_g$ is a Lie algebra homomorphism $\mathfrak{h} \to \mathfrak{h}$ due to that $\psi_g$ is a Lie group homomorphism, where $\mathfrak{h}$ is the Lie algebra of $H$ and $e$ is the neutral element of $H$. $\mathrm{D}_{e}\psi_g$ is an automorphism because $\psi_g$ is, and it is a homomorphism in $g$ due to that $\psi_{gg^\prime} = \psi_g \circ \psi_{g^\prime}$ implies $\mathrm{D}_e\psi_{gg^\prime} = \mathrm{D}_e\psi_g \circ \mathrm{D}_e\psi_{g^\prime}$, and thus $g \mapsto \mathrm{D}_e\psi_g$ is a $G$-representation on $\mathfrak{h}$ with values in automorphisms of Lie algebras (not just vector spaces).
\newline

This definition is of course also motivated by various references pointing out that Lie group representations define Lie group actions with extra properties; see for example \cite[\S 3, Ex.\ 3.4.2, page 143f.]{Hamilton}. In \cite[comments after Ex.\ 1.7.14, page 47]{mackenzieGeneralTheory} this definition is also called \textit{action by Lie group isomorphisms}.
\end{remarks}

With this we can discuss and define associated Lie group bundles; the following definition is clearly motivated by the definition of associated vector bundles as \textit{e.g.}\ provided in \cite[\S 4, Thm.\ 4.7.2, page 239f.]{Hamilton}.

\begin{theorems}{Associated Lie group bundle as quotient}{AssociatedGroupBundlesHaveGroupStructure}
Let $G, H$ be Lie groups, $P \stackrel{\pi_P}{\to} M$ a principal $G$-bundle over a smooth manifold $M$, and $\psi$ a $G$-representation on $H$. Then $\mathcal{H} \coloneqq P \times_\psi H$ is an LGB 
\begin{center}
	\begin{tikzcd}
		H \arrow{r} & \mathcal{H} \arrow{d}{\pi} \\
		& M
	\end{tikzcd}
\end{center}
with projection $\pi$ given by
\ba
\mathcal{H} &\to M,\nonumber\\
[p, h] &\mapsto \pi_P(p),
\ea
and fibres
\ba
\mathcal{H}_x
&=
P_x \times_\psi H
\ea
for all $x \in M$, which are isomorphic to $H$ as Lie groups. The Lie group structure on each fibre $\mathcal{H}_x$ is defined by
\ba\label{LiegroupStructureOnFibresofAssociated}
[p, h] \cdot \mleft[p, q\mright]
&\coloneqq
\mleft[ p, hq \mright]
\ea
for all $h, q \in H$ and $p \in P_x$ (that is, $\pi_P(p) = x$).
\end{theorems}

\begin{remarks}{Neutral and inverse elements}{NeutralAndInverseInAssocLGB}
The neutral element for $\mathcal{H}_x$ ($x \in M$) is given by
\bas
e_x
&=
[p, e],
\eas
where $p \in P_x$ is arbitrary and $e$ is the neutral element of $H$. This is clearly independent of the choice of $p$ due to
\bas
\mleft[ p, e \mright]
&=
\mleft[ p \cdot g, \psi_{g^{-1}}(e) \mright]
=
\mleft[ p \cdot g, e \mright]
\eas
for all $g \in G$. Thence, the fact that $e_x$ is the neutral element follows immediately.

By Def.\ \eqref{LiegroupStructureOnFibresofAssociated} the inverse of $[p, h] \in \mathcal{H}_x$ is clearly given by
\bas
\mleft( [p, h] \mright)^{-1}
&=
\mleft[ p, h^{-1} \mright].
\eas
\end{remarks}

\begin{proof}[Proof of Thm.\ \ref{thm:AssociatedGroupBundlesHaveGroupStructure}]
\leavevmode\newline
\indent $\bullet$ That $\pi$ is the well-defined projection and that the fibres are precisely $P_x \times_\psi H$ for all $x \in M$ is well-known, see our discussion before Def.\ \ref{def:LieGroupActingOnLieGroup} and the references therein; it is also very straightforward to check. We also discussed that $\mathcal{H}$ is a fibre bundle with structural fibre $H$. Hence, if one knows that the proposed group structure in Def.\ \eqref{LiegroupStructureOnFibresofAssociated} is well-defined, then the smoothness of the group structure is implied by the smoothness structures of $H$ and $\mathcal{H}$. Thence, let us check whether Def.\ \eqref{LiegroupStructureOnFibresofAssociated} is well-defined. Let $x \in M$, $p \in P_x$ and $p^\prime \coloneqq p \cdot g^\prime$ be another element of $P_x$, where $g^\prime \in G$. Also let $[p_1,h_1], [p_2, h_2] \in P_x \times_\psi H$; then we have unique elements $q_i, q_i^\prime$ of $G$ such that ($i \in \{1,2\}$)
\bas
p_i &= p \cdot q_i,&
p_i &= p^\prime \cdot q_i^\prime,
\eas
especially, it follows $q_i = g^\prime q_i^\prime$.
On the one hand, if we use $p$ as fixed element of $P_x$ to calculate the multiplication, we get
\ba\label{MultiPlicationInAssocGroup}
[p_1,h_1] \cdot [p_2,h_2]
&=
\mleft[ p, \psi_{q_1}(h_1) \mright]
\cdot \mleft[ p, \psi_{q_2}(h_2) \mright]
=
\mleft[ p, \psi_{q_1}(h_1) ~ \psi_{q_2}(h_2) \mright],
\ea
on the other hand, using Def.\ \ref{def:LieGroupActingOnLieGroup} and $p^\prime = p \cdot g^\prime$ instead of $p$,
\bas
[p_1,h_1] \cdot [p_2,h_2]
&=
\mleft[ p \cdot g^\prime, \psi_{q_1^\prime}(h_1) ~ \psi_{q_2^\prime}(h_2) \mright]
\\
&=
\Bigl[ p, \underbrace{\psi_{g^\prime} \mleft( \psi_{q_1^\prime}(h_1) ~ \psi_{q_2^\prime}(h_2) \mright)}_{= \psi_{g^\prime} \mleft( \psi_{q_1^\prime}(h_1) \mright) ~ \psi_{g^\prime} \mleft( \psi_{q_2^\prime}(h_2) \mright)} \Bigr]
\\
&=
\mleft[ p, \psi_{g^\prime q_1^\prime}(h_1) ~ \psi_{g^\prime q_2^\prime}(h_2) \mright]
\\
&=
\mleft[ p, \psi_{q_1}(h_1) ~ \psi_{q_2}(h_2) \mright],
\eas
which implies that Def.\ \eqref{LiegroupStructureOnFibresofAssociated} is well-defined, and thus defines a Lie group structure on each fibre of $\mathcal{H}$.

$\bullet$ That the fibres $\mathcal{H}_x$ are isomorphic to $H$ as Lie groups for all $x \in M$ also quickly follows. Recall by our discussion before Def.\ \ref{def:LieGroupActingOnLieGroup} that the fibres are diffeormorphic to $H$ by $H \ni h \mapsto [p, h] \in \mathcal{H}_x$ for a fixed $p \in P_x$. By Def.\ \eqref{LiegroupStructureOnFibresofAssociated} it is clear that this map is a Lie group homomorphism and hence a Lie group isomorphism.

$\bullet$ Let us now construct an LGB atlas for $\mathcal{H}$, denoting its maps by $\phi_U$ w.r.t.\ an open subset $U \subset M$. For this we will use a principal bundle atlas for $P$, that is, for some $U \subset M$ open and a local trivialization $\varphi_U: P_U \to U \times G$ of $P$ we write
\bas
\varphi_U(p)
&=
\bigl( \pi_P(p), \beta_U(p) \bigr)
\eas
for all $p \in P$, where $\beta_U: P_U \to G$ is an equivariant map, \textit{i.e.}\ $\beta_U(p \cdot g) = \beta_U(p) ~ g$ for all $g \in G$. Then define $\phi_U$ as a map by 
\bas
\mathcal{H}_U
&\to
U \times H,\\
[p, h]
&\mapsto
\mleft(
	\pi_P(p), \psi_{\beta_U(p)} (h)
\mright).
\eas
$\phi_U$ is well-defined: Let $\mleft[p^\prime, h^\prime\mright] \in \mathcal{H}_U$ with $\mleft[p^\prime, h^\prime\mright] = \mleft[p, h\mright]$. Then there is a $g \in G$ such that
\bas
\mleft(p^\prime, h^\prime\mright)
&=
\mleft( p \cdot g, \psi_{g^{-1}}(h) \mright),
\eas
hence, using the equivariance of $\beta_U$ and Def.\ \ref{def:LieGroupActingOnLieGroup},
\bas
\phi_U\mleft( \mleft[p^\prime, h^\prime\mright] \mright)
&=
\Bigl(
	\underbrace{\pi_P\mleft(p \cdot g\mright)}_{= \pi_P(p)}, \underbrace{\mleft(\psi_{\beta_U\mleft(p \cdot g\mright)} \circ \psi_{g^{-1}} \mright)}_{= \psi_{\beta_U(p)} \circ \psi_g \circ \psi_{g^{-1}} } (h)
\Bigr)
=
\mleft(
	\pi_P(p), \psi_{\beta_U(p)} (h)
\mright)
=
\phi_U\bigl( [p, h] \bigr),
\eas
which proves that $\phi_U$ is well-defined. Denote the projection onto equivalence classes $P \times H \to \mathcal{H}$ by $\varpi$, then observe
\bas
\phi_U \circ \varpi
&=
L,
\eas
where $L_U: P_U \times H \to U \times H$ is given by $L_U(p,h) \coloneqq \mleft( \pi_P(p), \psi_{\beta_U(p)} (h) \mright)$ for all $(p, h) \in P_U \times H$. $L_U$ is clearly smooth and recall that $\varpi$ is a smooth surjective submersion, therefore $\phi_U$ is smooth; this is a well-known fact for right-compositions with surjective submersions, see \textit{e.g.}\ \cite[\S 3.7.2, Lemma 3.7.5, page 153]{Hamilton}. We define a candidate of the inverse $\phi_U^{-1}: U \times H \to \mathcal{H}_U$ by
\bas
\phi_U^{-1}(x, h)
&=
\mleft[ \varphi_U^{-1}\mleft(x, e\mright), h \mright]
\eas
for all $(x, h) \in U \times H$, where $e$ is the neutral element of $G$.
By the definition of $\varphi_U$ we immediately get
\bas
(x, e)
&=
\mleft( \varphi_U \circ \varphi^{-1}_U \mright)(x, e)
=
\Bigl(
	\pi_P\mleft( \varphi_U^{-1}(x, e) \mright), \beta_U\mleft( \varphi_U^{-1}(x, e) \mright)
\Bigr),
\eas
for all $x \in U$, and, also using again the equivariance of $\beta_U$,
\bas
\varphi^{-1}_U\mleft(\pi_P(p), e\mright)
&=
\varphi^{-1}_U\Bigl(\pi_P\mleft(p \cdot \beta_U^{-1}(p) \mright), \beta_U(p)~\beta_U^{-1}(p)\Bigr)
\\
&=
\varphi^{-1}_U\Bigl(\pi_P\mleft(p \cdot \beta_U^{-1}(p) \mright), \beta_U\mleft(p \cdot \beta_U^{-1}(p)\mright)\Bigr)
\\
&=
\mleft(\varphi^{-1}_U \circ \varphi_U\mright)\mleft( p \cdot \beta_U^{-1}(p) \mright)
\\
&=
p \cdot \beta_U^{-1}(p)
\eas
for all $p \in P_U$.
Then altogether
\bas
\mleft(\phi_U \circ \phi_U^{-1}\mright)(x, h)
&=
\mleft(
	\pi_P\mleft( \varphi^{-1}_U(x, e) \mright), \psi_{\beta_U\mleft( \varphi^{-1}_U(x, e) \mright)} (h)
\mright)
=
\bigl(
	x, \psi_e(h)
\bigr)
=
(x, h),
\eas
for all $(x, h) \in U \times H$, and
\bas
\mleft(\phi_U^{-1} \circ \phi_U\mright)([p, h])
&=
\bigl[
	\underbrace{\varphi_U^{-1}\mleft( \pi_P(p), e \mright)}_{= p \cdot \beta_U^{-1}(p) },
	\psi_{\beta_U(p)}(h)
\bigr]
=
\mleft[
	p, h
\mright]
\eas
for all $[p, h] \in \mathcal{H}_U$. Thus, $\phi_U$ is bijective; additionally observe
\bas
\phi_U^{-1}(x, h)
&=
\varpi\mleft( \varphi_U^{-1}(x, e), h \mright)
\eas
such that $\phi_U^{-1}$ is clearly smooth as the composition of smooth maps, and we therefore conclude that $\phi_U$ is a diffeomorphism. Finally, derive with Def.\ \ref{def:LieGroupActingOnLieGroup} and Eq.\ \eqref{MultiPlicationInAssocGroup} that
\bas
\mleft(\mathrm{pr}_2 \circ \phi_U\mright)\bigl( [p_1, h_1] \cdot [p_2, h_2] \bigr)
&=
\mleft(\mathrm{pr}_2 \circ \phi_U\mright)\bigl( \mleft[ p, \psi_{q_1}(h_1) \cdot \psi_{q_2}(h_2) \mright] \bigr)
\\
&=
\psi_{\beta_U(p)} \bigl( \psi_{q_1}(h_1) \cdot \psi_{q_2}(h_2) \bigr)
\\
&=
\underbrace{\psi_{\beta_U(p)} \bigl( \psi_{q_1}(h_1) \bigr)}_{= \psi_{\beta_U(p) \cdot q_1} (h_1)}
	\cdot ~ \psi_{\beta_U(p)} \bigl( \psi_{q_2}(h_2) \bigr)
\\
&=
\psi_{\beta_U(p_1)} (h_1) \cdot \psi_{\beta_U(p_2)} (h_2)
\\
&=
\mleft( \mathrm{pr}_2 \circ \phi_U \mright)\bigl( [p_1, h_1] \bigr)
	\cdot \mleft( \mathrm{pr}_2 \circ \phi_U \mright)\bigl( [p_1, h_1] \bigr)
\eas
for all $[p_1, h_1], [p_2, h_2] \in \mathcal{H}_x$, where we used again the equivariance of $\beta_U$ and the same notation as introduced for Eq.\ \eqref{MultiPlicationInAssocGroup}, and $\mathrm{pr}_2$ denotes the projection onto the second factor. Thence, $\mathrm{pr}_2 \circ \phi_U$ induces Lie group isomorphisms $\mathcal{H}_x \to H$ for all $x \in U$; by Def.\ \ref{def:LieGroupBundle} we can finally conclude that $\mathcal{H}$ is an LGB.
\end{proof}

Hence, we define:

\begin{definitions}{Associated Lie group bundle, \newline labeling similar to \cite[\S 4.7, Def.\ 4.7.3, page 240]{Hamilton}}{AssociatedLGB}
Let $G, H$ be Lie groups, $P \stackrel{\pi_P}{\to} M$ a principal $G$-bundle over a smooth manifold $M$, and $\psi$ a $G$-representation on $H$. Then we call the LGB
\bas
\mathcal{H}
&\coloneqq
\mleft( P \times H \mright) \Big/ G
\coloneqq
P \times_\psi H
\eas
the \textbf{Lie group bundle (LGB) associated} to the principal bundle $P$ and the representation $\psi$ on $H$:
\begin{center}
	\begin{tikzcd}
	H \arrow{r}& P \times_\psi H \arrow{d}{\pi_{\mathcal{H}}}\\
	& M
	\end{tikzcd}
\end{center}
\end{definitions}

The special situation of $H = G$ is already an important example:

\begin{examples}{Inner group bundle, \newline \cite[\S1, paragraph after Def.\ 1.1.19, page 11; comment after Construction 1.3.8, page 20]{mackenzieGeneralTheory}}{InnerLGBs}
The \textbf{inner group bundle} or \textbf{inner LGB} of a principal bundle $P \to M$, denoted by $c_G(P)$, is defined by
\ba
c_G(P)
&\coloneqq
P \times_{c_G} G,
\ea
where $c_G: G \times G \to G$ is the left action of $G$ on itself given by the very well-known \textbf{conjugation}
\ba
c_G(g,h)
&\coloneqq
c_g (h)
=
\mleft(L_g \circ R_{g^{-1}}\mright)(h)
=
ghg^{-1}
\ea
for all $g, h \in G$, where we also denote left- and right-multiplications (with $g$) by $L_g$ and $R_g$, respectively; see \textit{e.g.}\ \cite[beginning of \S 1.5.2, page 40f.]{Hamilton}\ for its common properties. It is well-known that $c_G$ satsfies the properties of a Lie group representation of $G$ on itself in the sense of Def.\ \ref{def:LieGroupActingOnLieGroup}.

$c_G(P)$ is an LGB by Thm.\ \ref{thm:AssociatedGroupBundlesHaveGroupStructure}.
\end{examples}

\section{LGB actions, part I}

\subsection{Definition}

As for Lie groups, we are interested into their actions. The idea is the following, similar to \cite[\S 1.6, discussion around Def.\ 1.6.1, page 34]{mackenzieGeneralTheory}: We have an LGB $\mathcal{G} \to M$ over a smooth manifold $M$, and we want to construct an action of $\mathcal{G}$ on another smooth manifold $N$. Each fibre of $\mathcal{G}$ is a Lie group, and we have a notion of Lie groups acting on a manifold $N$. Therefore one could define an LGB action as a collection of Lie group actions, that is, only sections of $\mathcal{G}$ act on $N$; however, one then expects that the general outcome of a product of $\Gamma(\mathcal{G})$ on $N$ would be smooth maps from $M$ to $N$. In order to recover a typical structure of action one could instead introduce a "multiplication rule", \textit{i.e.}~each point $p \in N$ can only be multiplied with elements of a specific fibre of $\mathcal{G}$. This "multiplication rule" will be described by a smooth map $f: N \to M$ in the sense of that the fibre over $f(p)$ will act on $p$.

For this recall that there is the notion of pullbacks of fibre bundles, see \textit{e.g.}\ \cite[\S 4.1.4, page 203ff.; especially Thm.\ 4.1.17, page 204f.]{Hamilton}. That is, if we additionally have a smooth manifold $N$ and a smooth map $f: M \to N$, then we have the pullback $f^*\mathcal{G}$ of $\mathcal{G}$ as a fibre bundle defined as usual by
\ba
f^*\mathcal{G}
&\coloneqq
\left\{
	(x, g) \in N \times \mathcal{G} ~\middle|~
	f(x) = \pi(g)
\right\}.
\ea
It is an embedded submanifold of $N \times \mathcal{G}$, and the structural fibre is the same Lie group as for $\mathcal{G}$.
That is, the following diagram commutes
\begin{center}
	\begin{tikzcd}
		f^*\mathcal{G} \arrow{d}{\pi_1} \arrow{r}{\pi_2}& \mathcal{G} \arrow{d}{\pi} \\
		N \arrow{r}{f}& M
	\end{tikzcd}
\end{center}
where $\pi_1$ and $\pi_2$ are the projections onto the first and second factor, respectively, of $N \times \mathcal{G}$. Actually, $f^*\mathcal{G}$ carries a natural structure as an LGB.

\begin{corollaries}{Pullbacks of LGBs are LGBs, \newline \cite[\S 2.3, simplified situation of the discussion around Prop.\ 2.3.1, page 63ff.]{mackenzieGeneralTheory}}{PullbackLGB}
Let $M, N$ be smooth manifolds, $\mathcal{G} \stackrel{\pi}{\to} M$ an LGB over $M$ and $f: N \to M$ a smooth map. Then $f^*\mathcal{G}$ has a unique (up to isomorphisms) LGB structure such that the projection $\pi_2: f^*\mathcal{G} \to \mathcal{G}$ onto the second factor is an LGB morphism over $f$ with $\pi_2|_x: (f^*\mathcal{G})_x \to \mathcal{G}_{f(x)}$ being a Lie group isomorphism for all $x \in N$.
%
%$f^*\mathcal{G}$ is the \textbf{pullback LGB of $\mathcal{G}$ (under $f$)}.
\end{corollaries}

\begin{remark}
\leavevmode\newline
The mentioned reference, \cite[\S 2.3, discussion around Prop.\ 2.3.1, page 63ff.]{mackenzieGeneralTheory}, is rather general, formulated for Lie groupoids. If the reader is only interested into LGBs, then see \textit{e.g.}\ \cite[\S 3, Thm.\ 3.1]{PullbackLGBLAB}.
\end{remark}

\begin{proof}
\leavevmode\newline
By construction, the structural fibre of $f^*\mathcal{G}$ is the same Lie group $G$ as for $\mathcal{G}$, and for all $x \in N$ we have $\mleft(f^*\mathcal{G}\mright)_x \cong \mathcal{G}_{f(x)}$, thence, the fibres are Lie groups and the fibrewise group multiplication has the form
\ba\label{MultiplicationForPullbackLGBs}
(x, g) \cdot (x, q) &= (x, gq)
\ea
for all $x \in N$ and $g,q \in \mleft(f^*\mathcal{G}\mright)_x$.
We are left to show the existence of an LGB atlas. For this fix an LGB atlas $\{(U_i, \phi_i \}_{i \in I}$ of $\mathcal{G}$, where $I$ is an (index) set, $(U_i)_{i \in I}$ an open covering of $M$, and $\phi_i: \mathcal{G}|_{U_i} \to U_i \times G$ are LGB isomorphisms. Then $f^{-1}(U_i)$ gives rise to an open covering of $N$, and we get 
\bas
f^*\phi_i: \mleft.f^*\mathcal{G}\mright|_{f^{-1}(U_i)} &\to f^{-1}(U_i) \times G,\\
(x, g) &\mapsto \mleft( x, \phi_{i, f(x)}(g) \mright),
\eas
where $\phi_{i, f(x)}: \mathcal{G}_{f(x)} \to G$ are the Lie group isomorphisms as defined in Def.\ \ref{def:LieGroupBundle}. It is immediate by construction that this gives an LGB atlas.

That this is the unique (up to isomorphisms) LGB structure such that $\pi_2: f^*\mathcal{G} \to \mathcal{G}$ is an LGB morphism over $f$ inducing a Lie group isomorphism on each fibre simply follows by construction; observe for all $x \in N$ that $\pi_2|_x$ is clearly bijective. Furthermore, LGB morphisms need to be homomorphisms, which means here
\bas
\pi_2\bigl(
	(x, g) \cdot (x, q)
\bigr)
&\stackrel{!}{=}
\pi_2\bigl( (x, g) \bigr) \cdot \pi_2\bigl( (x, q) \bigr)
=
gq
=
\pi_2\bigl( (x, gq) \bigr)
\eas
for all $x \in N$ and $g,q \in \mleft(f^*\mathcal{G}\mright)_x$. By using the bijectivity of $\pi_2|_x$, the group structure leading to this equation is uniquely the one provided in Eq.\ \eqref{MultiplicationForPullbackLGBs}. 
%Especially, $\pi_2$ is a homomorphism with the provided structure. 
Assume we have another underlying LGB atlas in sense of Cor.\ \ref{cor:PullbackLGB} with an LGB chart $\psi_i$ on $\mleft.f^*\mathcal{G}\mright|_{f^{-1}(U_i)}$ (or just w.r.t.\ a subset of $U_i$), then
\bas
\phi_i \circ \pi_2 \circ \psi_i^{-1}
&=
\underbrace{\phi_i \circ \pi_2 \circ \mleft( f^*\phi_i \mright)^{-1}}_{= (f, \mathds{1}_G)} \circ f^*\phi_i \circ \psi_i^{-1}
=
(f, \mathds{1}_G) \circ
f^*\phi_i \circ \psi_i^{-1},
\eas
$(f, \mathds{1}_G)$ is clearly a Lie group isomorphism $f^{-1}(U_i) \times G \cong U_i \times G$ of trivial LGBs, and thus the condition about $\pi_2|_x$ being a Lie group isomorphism enforces that we can write
\bas
\mleft(f^*\phi_i \circ \psi_i^{-1}\mright)(x, g)
&=
\bigl( x, L_i(x, g) \bigr)
\eas
for all $(x, g) \in f^{-1}(U) \times G$, where $L_i(x, \cdot): G \to G$ is a Lie group automorphism. So, in total
\bas
\phi_i \circ \pi_2 \circ \psi_i^{-1}
&=
\bigl(
	f, L
\bigr),
\eas
and since $f$ and the left hand side are smooth, $L$ has to be smooth. We can conclude that $\psi_i$ gives rise to an LGB atlas compatible with the one defined by $f^*\phi_i$. This finalizes the proof.
\end{proof}

\begin{definitions}{Pullback LGB}{PullbackLGBDef}
Let $M, N$ be smooth manifolds, $\mathcal{G} \to M$ an LGB over $M$ and $f: N \to M$ a smooth map. Then we call the LGB structure on $f^*\mathcal{G}$ as given in Cor.\ \ref{cor:PullbackLGB} the \textbf{pullback LGB of $\mathcal{G}$ (under $f$)}.

With writing $f^*\mathcal{G}$ we will often refer to this structure without further mention.
\end{definitions}

Let us now define $\mathcal{G}$-actions.

\begin{definitions}{Lie group bundle actions, \newline \cite[\S 1.6, special case of Def.\ 1.6.1, page 34]{mackenzieGeneralTheory}}{LiegroupACtion}
Let $M, N$ be smooth manifolds, $\mathcal{G} \stackrel{\pi}{\to} M$ an LGB over $M$ and $f: N \to M$ a smooth map. Then a \textbf{right-action of $\mathcal{G}$ on $N$} is a smooth map 
\bas
f^*\mathcal{G} &\to N,\\
(p, g) &\mapsto p \cdot g,
\eas
satisfying the following properties:
\ba\label{InvarianceOffUnderGAction}
f(p \cdot g) &= \pi(g),\\
(p \cdot g) \cdot h &= p \cdot (gh),\label{ActionAssociative}\\
p \cdot e_{f(p)} &= p\label{ActionNeutralElement}
\ea
for all $p \in N$ and $g, h \in \mathcal{G}_{f(p)}$, where $e$ is the neutral section of $\mathcal{G}$.
\newline

We analogously define left-actions, and we often write (left or right) \textbf{$\mathcal{G}$-action on $N$}. Furthermore, in order to increase readability as long as the dependency on $f$ is not important, we introduce the notation
\ba
N * \mathcal{G}
&\coloneqq
f^*\mathcal{G},
\ea
such that the action's notation has the typical shape $N * \mathcal{G} \to N$; one may also write $N * \mathcal{G} = N \times_M \mathcal{G}$. For left actions similarly $\mathcal{G} * N \to N$; even though $\mathcal{G} * N$ is the same pullback LGB as for right-actions; we also change the order of notation in this case, that is, $(g, p) \in \mathcal{G} * N$ reads $g \in \mathcal{G}_{f(p)}$ and $p \in N$. 
\end{definitions}

\begin{remarks}{Relation to the structure of the canonical pullback Lie group bundle over $N$}{ActionAndPullbackLGBs}
Observe that by the definition of $f^*\mathcal{G}$ we can also write
\bas
f(p \cdot g) &= f(p),
\eas
so, the $\mathcal{G}$-action is defined in such a way that $f$ is invariant under it. If $f$ is the projection of $N$ as a bundle over $M$, then this means that an LGB action is fibre-preserving. Moreover, the fibre-wise group structure on $\mathcal{G}$ naturally defines a $\mathcal{G}$-action on $\mathcal{G}$; in this situation $f$ would be $\pi$ itself, see also Ex.\ \ref{ex:LGBActingOnItself} later. Furthermore, having $M = \{*\}$ recovers the notion of a Lie group action and condition \eqref{InvarianceOffUnderGAction} is then trivial.
\newline

The other conditions are the typical conditions for actions, especially such that we get a $\mathcal{G}$-action on $f^*\mathcal{G}$ by
\ba
(p, g) \cdot q
&\coloneqq
\mleft( p \cdot q, q^{-1} g \mright)
\ea
for all $p \in N$ and $g, q \in \mathcal{G}_{f(p)}$.\footnote{In alignment to Def.\ \ref{def:LiegroupACtion}, this action is a map $(f \circ \pi_1)^*\mathcal{G} \to f^*\mathcal{G}$, where $\pi_1$ is the projection onto the first factor in $f^*\mathcal{G}$.}
As usual, this gives rise to an equivalence relation, whose set of equivalence classes denoted by $f^*\mathcal{G} \Big/ \mathcal{G}$ is isomorphic to $N$ (as a set) by $[p, g] \mapsto p \cdot g$, where we denote equivalence classes of $(p, g) \in f^*\mathcal{G}$ by $[p, g]$. All of this is straight-forward to check. Finally, observe the similarity to associated fibre bundles.
\end{remarks}

\begin{remarks}{Localizing LGB actions}{LocalLGBAction}
We can actually localize the LGB action, but in general not with respect to any open neighbourhood of $N$ since that is in general not possible in a non-trivial way, \textit{i.e.}\ the action cannot be brought into the form $(N*\mathcal{G})|_U \to U$ for arbitrary $U$ because $p \cdot g$ may for example leave an neighbourhood $U$ of $p$. However, with respect to $M$ this is possible:

Fix any open neighbourhood $U$ of $M$. Then $f^{-1}(U)$ is an open neighbourhood of $N$, and we can restrict the action to $f^{-1}(U)$, resulting into a map
\bas
\mleft.(N*\mathcal{G})\mright|_{f^{-1}(U)} \to f^{-1}(U),
\eas
because of $f(p \cdot g) \stackrel{\text{Eq.\ \eqref{InvarianceOffUnderGAction}}}{=} \pi(g) = f(p)$, that is, if $(p, g) \in \mleft.(N*\mathcal{G})\mright|_{f^{-1}(U)}$, then $p \in f^{-1}(U)$ and so $p \cdot g \in f^{-1}(U)$. In fact, by the definition of $N*\mathcal{G}$ as $f^*\mathcal{G}$, this describes a $\mathcal{G}|_{U}$-action on $f^{-1}(U)$
\bas
f^{-1}(U) * \mleft.\mathcal{G}\mright|_{U} \coloneqq \mleft( f|_{f^{-1}(U)}\mright)^*(\mathcal{G}|_U) = \mleft.(N*\mathcal{G})\mright|_{f^{-1}(U)} &\to f^{-1}(U).
\eas
If $x \in M$ is a regular value of $f$, then $f^{-1}(\{x\})$ is an embedded submanifold of $N$ due to the regular value theorem (for this see \textit{e.g.}\ \cite[\S A.1, Thm.\ A.1.32, page 611]{Hamilton}). In that case one could apply the same arguments to restrict the action on $f^{-1}(\{x\})$. Since $\mathcal{G}_x$ is a Lie group one actually gets a typical Lie group action on $f^{-1}(\{x\})$. For more details about this see Ex.\ \ref{ex:TrivialLGBAction} later.
\end{remarks}

\begin{remarks}{Left- and right-actions}{LeftRightAction}
In the following we usually define everything with respect to right-actions; however, one can of course define the following notions for left actions in a similar manner. If we ever speak of a left action, then we assume precisely this. Some subtle changes like a sign change will be pointed out though. 
\end{remarks}

One can probably see that it is straightforward to extend a lot of the typical notions of Lie group actions to LGB actions; hence, we mainly focus on the definitions and properties which we need in this paper. 

\begin{definitions}{Left and right translations, \newline \cite[\S 3.2, notation similar to Def.\ 3.2.3, page 131]{Hamilton} \newline \cite[\S 1.4, special situation of Def.\ 1.4.1 and its discussion, page 22]{mackenzieGeneralTheory}}{LRTranslations}
Let $M, N$ be smooth manifolds, $\mathcal{G} \stackrel{\pi}{\to} M$ an LGB over $M$ and $f: N \to M$ a smooth map. Furthermore assume that we have a right action $N * \mathcal{G} \to N$. We define the \textbf{right translation} over $x \in M$ with $g \in \mathcal{G}_x$ as a map $r_g$ defined by
\bas
f^{-1}(\{x\}) &\to f^{-1}(\{x\}),\\
p &\mapsto p \cdot g,
\eas
and we define the \textbf{orbit map} through $p \in f^{-1}(\{x\})$ as a map $\Phi_p$ given by
\bas
\mathcal{G}_x &\to N,\\
g&\mapsto p \cdot g.
\eas
For $\sigma \in \Gamma(\mathcal{G})$ we define the \textbf{right translation} on $N$ as a map $r_\sigma$ by
\bas
N &\to N,\\
p &\mapsto p \cdot \sigma_{f(p)}.
\eas

If one has a section of $f$, \textit{i.e.}\ a smooth map $\tau: M \to N$, $x \mapsto \tau_x$, with $f \circ \tau = \mathds{1}_M$, then we can define the \textbf{orbit map} through $\tau$ as a map $\Phi_\tau$ given by
\bas
\mathcal{G} &\to N,\\
g &\mapsto \tau_{\pi(g)} \cdot g.
\eas
\end{definitions}

\begin{remarks}{Left action and translation}{LeftTranslation}
Similarly we define left translations for left actions, which we similarly denote by $l_g$ and $l_\sigma$. By Rem.\ \ref{rem:LocalLGBAction} we can define $r_\sigma$ (and $l_\sigma$) also for local sections $\sigma \in \Gamma(\mathcal{G}|_U)$ by restricting $N$ onto $f^{-1}(U)$, where $U$ is some open subset of $M$; then $r_\sigma, l_\sigma: f^{-1}(U) \to f^{-1}(U)$. In the same manner one achieves a restriction for $\Phi_\tau: \mathcal{G}|_U \to f^{-1}(U)$, if $\tau: U \to f^{-1}(U)$.

In case of $N$ being $\mathcal{G}$ itself we will introduce right (and left) translations later via capital letters.
\end{remarks}

\begin{remarks}{Group action on sections}{SectionMultiplication}
Assume that $N$ is a fibre bundle over $M$ with $f$ as its projection. Also observe that $\Gamma(\mathcal{G})$ is clearly a group and it may be possible to endow it with an infinite-dimensional Lie group structure (but this is not important here now). In the same fashion as before, we can define a $\Gamma(\mathcal{G})$-action on $\Gamma(N)$ given by
\bas
\mleft(\tau \cdot \sigma\mright)_x
&\coloneqq
\tau_x \cdot \sigma_x
\eas
for all $\tau \in \Gamma(N)$, $\sigma \in \Gamma(\mathcal{G})$, and $x \in M$. It is straightforward to show that this is well-defined. We will make use of this action without further mention.
\end{remarks}

\begin{remark}\label{SmoothnessOfACtionTranslations}
\leavevmode\newline
Similar to the arguments in \cite[\S 3.2, discussion after Def.\ 3.2.3, page 131]{Hamilton}, $\Phi_p$ is given by the composition of smooth maps
\bas
\mathcal{G}_x &\to N * \mathcal{G} \to N,\\
g &\mapsto (p, g) \mapsto p \cdot g.
\eas
The second arrow/map is smooth due to the fact that we have a smooth action; the first one is smooth because $N * \mathcal{G}$ is the pullback LGB $f^*\mathcal{G}$ and the first arrow is precisely the embedding of $\mathcal{G}_x$ into $f^*\mathcal{G}$ as a fibre over $p$; recall \textit{e.g.}\ Cor.\ \ref{cor:PullbackLGB}.

For the right-translation $r_\sigma$ we proceed in a similar fashion, namely, $r_\sigma$ is a composition of smooth maps
\bas
N &\to N*\mathcal{G} \to N,\\
p &\mapsto \mleft(p, \sigma_{f(p)}\mright) \mapsto p \cdot \sigma_{f(p)}.
\eas
The first map describes now a section of $N*\mathcal{G} = f^*\mathcal{G}$ as LGB over $N$, and thence an embedding. Thus, smoothness follows again as previously. Similarly, $\Phi_\tau$ is the composition of maps
\bas
\mathcal{G} &\to N * \mathcal{G} \to N,\\
g &\mapsto \mleft( \tau_{\pi(g)}, g \mright) \mapsto \tau_{\pi(g)} \cdot g,
\eas
and the first arrow is clearly a smooth map $\mathcal{G} \to N \times \mathcal{G}$ with values in $f^*\mathcal{G} = N * \mathcal{G}$ which is an embedded submanifold of $N \times \mathcal{G}$. Thence, smoothness follows as usual. In fact, $\Phi_\tau|_{\mathcal{G}_x} = \Phi_{\tau_x}$.

However, for $r_g$ smoothness can only be discussed if $f^{-1}(\{x\})$ is a smooth manifold. That is for example the case if $x$ is a regular value of $f$; recall the regular value theorem as cited in \cite[\S A.1, Thm.\ A.1.32, page 611]{Hamilton}. This would be the case if \textit{e.g.}\ $f$ is a submersion. If $x$ is a regular value, then $f^{-1}(\{x\})$ is an embedded submanifold of $N$, and $r_g$ is a similar composition of smooth maps as for $r_\sigma$ but restricted to $f^{-1}(\{x\})$
\bas
f^{-1}(\{x\}) &\to \mleft.\mleft(N*\mathcal{G}\mright)\mright|_{f^{-1}(\{x\})} \to f^{-1}(\{x\}),\\
p &\mapsto (p, g) \mapsto p \cdot g.
\eas
Since $f^{-1}(\{x\})$ is an embedded submanifold, $\mleft.\mleft(N*\mathcal{G}\mright)\mright|_{f^{-1}(\{x\})}$ is also a fibre bundle, see for example \cite[\S 4.1, Lemma 4.1.16, page 204]{Hamilton}, and trivially an embedded submanifold of $N*\mathcal{G}$. Altogether, the same arguments as for $r_\sigma$ apply.

Last but not least, $r_\sigma$ is clearly a diffeomorphism with inverse $r_{\sigma^{-1}}$, where $\mleft(\sigma^{-1}\mright)_x \coloneqq \sigma_x^{-1} \coloneqq (\sigma_x)^{-1}$. Similarly for $r_g$, if $f^{-1}(\{x\})$ is an embedded submanifold (otherwise $r_g$ is just a bijection). Analogously for $l_\sigma$ and $l_g$ in case of a left action.
\end{remark}

Motivated by the previous remark, it might be useful to require that $f$ is a submersion, or that $N$ is actually some bundle over $M$ and $f$ its projection. In fact, this will be the case later.

%In order to generalize principal bundles we need the following terms.
%
%\begin{definitions}{}{}
%
%\end{definitions}

\subsection{Examples of LGB actions}

If $M$ is a point or $f$ a constant map, then we recover the typical notion of a Lie group action acting on $N$.
Additionally, we have the following examples. The notation will be as in Def.\ \ref{def:LRTranslations}.

\begin{examples}{LGB acting on itself}{LGBActingOnItself}
Each LGB $\mathcal{G} \stackrel{\pi}{\to} M$ acts on itself from the left and right, having $N \coloneqq \mathcal{G}$ and $f \coloneqq \pi$,
\bas
\mathcal{G} * \mathcal{G} \coloneqq \pi^*\mathcal{G} &\to \mathcal{G},\\
(g,h) &\mapsto gh.
\eas
That this satisfies all properties for an LGB action is clear by definition of LGBs; however, let us give a note about the smoothness of this action. Recall that an LGB is locally isomorphic to a trivial LGB $U \times G$ ($U$ an open subset of $M$) with its canonical group multiplication, $(x, g) \cdot (x, q) = (x, gq)$. Hence, using that the multiplication of $G$ is smooth and using local LGB trivializations of $\mathcal{G}$ and $\pi^*\mathcal{G}$ (recall Cor.\ \ref{cor:PullbackLGB} and its proof to show that $\pi^*\mathcal{G}$ is locally diffeomorphic to the product manifold $U \times G \times G$), we achieve smoothness of the $\mathcal{G}$-action on itself because it is locally of the form
\bas
U \times G \times G &\to U \times G,\\
(x, g, q) &\mapsto (x, gq).
\eas
We will also call the $\mathcal{G}$-action on itself the \textbf{multiplication in $\mathcal{G}$}.
\end{examples}

\begin{remark}
\leavevmode\newline
Similarly one can argue that the \textbf{inverse map} $\mathcal{G} \to \mathcal{G}$, $g \mapsto g^{-1}$, is smooth, too.
\end{remark}

\begin{examples}{Trivial action, \cite[\S 1.6, special situation of Ex.\ 1.6.3, page 35]{mackenzieGeneralTheory}}{TrivialAction}
The projection $\pi_1$ onto the first factor of $f^*\mathcal{G} \stackrel{\pi_1}{\to} N$ satisfies the properties of a right $\mathcal{G}$-action on $N$, that is, the action is given by
\bas
N * \mathcal{G} &\to N,\\
(p, g) &\mapsto p \cdot g \coloneqq p.
\eas
That this action satisfies the properties of an action for all $f$ is trivial, hence we call it the \textbf{trivial action}.
\end{examples}

\begin{examples}{Actions of trivial LGBs}{TrivialLGBAction}
Assume that $\mathcal{G}$ is trivial, that is $\mathcal{G} \cong M \times G$ as LGBs, where $G$ is the structural Lie group of $\mathcal{G}$. In that case, for $p \in N$, the product $p \cdot q$ is only defined if $q \in \mathcal{G}$ is of the form $(f(p), g)$, where $g \in G$; for this recall that $(p, q)$ needs to be an element of $N*\mathcal{G}=f^*\mathcal{G}$ in order to define $(p, q) \mapsto p \cdot q$. We also have
\bas
f^*\mathcal{G}
&\cong
N \times G,
\eas
the trivial $G$-LGB over N. Hence, let us define a map by
\bas
N \times G &\to N,\\
(p, g) &\mapsto p \cdot g \coloneqq p \cdot (f(p), g),
\eas
which is clearly a smooth map since it is a composition of the $\mathcal{G}$-action on $N$ and
\bas
N \times G &\to f^*\mathcal{G},\\
(p, g) &\mapsto (p, f(p), g).
\eas
The latter is smooth because $(p, g) \mapsto (p, f(p), g)$ is a smooth map $N \times G \to N \times M \times G$ and $f^*\mathcal{G}$ is an embedded submanifold of $N \times M \times G$. Using Def.\ \ref{def:LiegroupACtion}, it is trivial to see that $p \cdot e = p$, and
\bas
p \cdot (gh)
&=
p \cdot \underbrace{(f(p), gh)}_{\mathclap{= (f(p), g) \cdot (f(p), h)}}
=
\bigl( p \cdot (f(p), g) \bigr) \cdot (f(p), h)
=
(p \cdot g) \cdot h
\eas
for all $g,h \in G$. Hence, we have a $G$-action on $N$, and by construction it is equivalent to the $\mathcal{G}$-action on $N$. Due to the discussion in Rem.\ \ref{rem:LocalLGBAction} we can therefore conclude that every $\mathcal{G}$-action is locally a typical Lie group action. If $f$ is a submersion, then the action is also a $G$-action on each fibre of $f$.

Observe, that one can therefore recover the notion of Lie group actions not only via $M = \{*\}$, the point manifold, (or equivalently via constant maps $f$) but also via trivial LGBs. Translations with constant sections of $M \times G$ w.r.t.\ the action $N*\mathcal{G}\to N$ are trivially to be seen as translations with an element of $G$ w.r.t.\ the action $N \times G \to N$.
\end{examples}

Hence, if one wants something "truly" new regarding LGB actions, then one has to look at global structures of non-trivial LGBs and their actions. In fact, the following type of example can provide such a new structure, which we will understand later once we have introduced the physical theory.

\begin{examples}{Inner group bundle acting on associated fibre bundles, \newline\cite[\S 1.6, simplified version of Ex.\ 1.6.4, page 35]{mackenzieGeneralTheory}}{AssocLGACtingOnAssocVec}
Let $P \stackrel{\pi_P}{\to} M$ be a principal $G$-bundle with structural Lie group $G$ over a smooth manifold $M$, and recall Ex.\ \ref{ex:InnerLGBs}. Furthermore, let $F$ be another smooth manifold, equipped with a smooth left $G$-action $\Psi: G \times F \to F$. In total we have two associated bundles over $M$: 
\begin{center}
	\begin{tikzcd}
		G \arrow{r} & c_G(P) \arrow{d}{\pi_{c_G(P)}} && F \arrow{r} & \mathcal{F} \coloneqq P \times_\Psi F \arrow{d}{\pi_{\mathcal{F}}} \\
		& M & && M
	\end{tikzcd}
\end{center}
the inner group bundle of $P$ and an associated $F$-bundle, respectively.

Then we have a right $c_G(P)$-action on $\mathcal{F}$ given by
\bas
\mathcal{F} * c_G(P)
\coloneqq
\pi_{\mathcal{F}}^*c_G(P) &\to \mathcal{F},\\
\bigl( \mleft[ p, v \mright], \mleft[ p, g \mright] \bigr)
&\mapsto
\mleft[ p, \Psi \mleft(g, v\mright) \mright]
=
\mleft[ p \cdot g, v \mright]
\eas
for all $p \in P_x$ ($x \in M$), $g \in G$ and $v \in F$.
\end{examples}

\begin{proof}
\leavevmode\newline
\indent $\bullet$ We first check again that the action is well-defined, that is, we are going to prove that the action is independent of the choice of fixed point in $P_x$.
Thence, let $x \in M$, $p \in P_x$ and $p^\prime \coloneqq p \cdot g^\prime$ be another element of $P_x$, where $g^\prime \in G$. Also let $[p_1,v] \in \mathcal{F}_x$ and $[p_2, g] \in c_G(P)_x$; then we have unique elements $q_i, q_i^\prime$ of $G$ such that ($i \in \{1,2\}$)
\bas
p_i &= p \cdot q_i,&
p_i &= p^\prime \cdot q_i^\prime,
\eas
especially, it follows $q_i = g^\prime q_i^\prime$.

On one hand, if we use $p$ as fixed element of $P_x$ to calculate the multiplication, we get 
\bas
[p_1, v] \cdot [p_2, g]
&=
[p, \Psi(q_1, v)] \cdot [p, c_{q_2}(g)]
=
\mleft[p \cdot c_{q_2}(g), \Psi(q_1, v) \mright]
=
\mleft[p \cdot c_{q_2}\mleft( g \mright) ~ q_1, v \mright].
\eas
On the other hand, using $p^\prime$ as a fixed element, we derive, using $q_i^\prime = \mleft( g^\prime \mright)^{-1} q_i$,
\bas
[p_1, v] \cdot [p_2, h]
&=
\mleft[p^\prime \cdot c_{q_2^\prime}\mleft( g \mright) ~ q_1^\prime, v \mright]
=
\mleft[p \cdot g^\prime q_2^\prime g \mleft( q_2^\prime \mright)^{-1} q_1^\prime, v \mright]
=
\mleft[p \cdot q_2 g q_2^{-1} q_1, v \mright]
=
\mleft[p \cdot c_{q_2}\mleft(g\mright) ~ q_1, v \mright],
\eas
which finalizes the argument needed to show that the action is well-defined.

$\bullet$ To show the smoothness of the proposed action we simply use the atlas constructed in the proof for Thm.\ \ref{thm:AssociatedGroupBundlesHaveGroupStructure}; the idea of the atlas there does not just work for $c_G(P)$ but it also serves for constructing a fibre bundle atlas for $\mathcal{F}$ following the same relevant parts of the cited proof. That is, we will use a principal bundle atlas for $P$, hence for some $U \subset M$ open and a local trivialization $\varphi_U: P_U \to U \times G$ of $P$ we write
\bas
\varphi_U(p)
&=
\bigl( \pi_P(p), \beta_U(p) \bigr)
\eas
for all $p \in P$, where $\beta_U: P_U \to G$ is an equivariant map. Then we have maps $\phi_{c_G(P)}: \mleft. c_G(P)\mright|_U \to U \times G$ and $\phi_{\mathcal{F}}: \mleft. \mathcal{F} \mright|_U \to U \times F$ given by 
\bas
\phi_{c_G(P)}\bigl([p, g]\bigr)
&=
\mleft(
	\pi_P(p), c_{\beta_U(p)} (h)
\mright),&
\phi_{\mathcal{F}}\bigl([p, v]\bigr)
&=
\mleft(
	\pi_P(p), \Psi_{\beta_U(p)} (v)
\mright),
\eas
with inverses
\bas
\phi_{c_G(P)}^{-1}(x, g)
&=
\mleft[ \varphi_U^{-1}\mleft(x, e\mright), g \mright],&
\phi_{\mathcal{F}}^{-1}(x, v)
&=
\mleft[ \varphi_U^{-1}\mleft(x, e\mright), v \mright].
\eas
Following the related calculations in the proof for Thm.\ \ref{thm:AssociatedGroupBundlesHaveGroupStructure} one shows that the proposed action looks locally like (w.r.t.\ these maps)
\bas
\bigl((x, v) , (x, g) \bigr)
&\mapsto 
\mleft( x, \Psi_{\beta_U \mleft( \varphi_U^{-1}\mleft(x, e\mright) \cdot g \mright)}(v) \mright) 
=
\bigl( x, \Psi_{g}(v) \bigr),
\eas
which is clearly a smooth map.

$\bullet$ Let us now quickly check that the conditions in Def.\ \ref{def:LiegroupACtion} are satisfied. We have
\bas
\pi_{\mathcal{F}}\bigl( [p, v] \cdot [p, g] \bigr)
&=
\pi_{\mathcal{F}}\mleft( \mleft[p , \Psi \mleft(g, v \mright) \mright] \mright)
=
\pi_P(p)
=
\pi_{c_G(P)}\bigl( [p, g] \bigr)
\eas
for all $p \in P_x$ ($x \in M$), $v \in F$ and $g \in G$; similarly, having additionally $h \in G$,
\bas
\bigl([p, v] \cdot [p, g] \bigr) \cdot [p, h]
&=
\mleft[ p \cdot g, v \mright] \cdot [p, h]
=
\mleft[ p \cdot g h, v \mright]
=
[p, v] \cdot [p, gh]
=
[p, v] \cdot \bigl([p, g] ~ [p, h] \bigr),
\eas
and
\bas
[p, v] \cdot [p, e]
&=
[p \cdot e, v]
=
[p, v].
\eas
Therefore this describes an action.
\end{proof}

\begin{remarks}{Relation to automorphisms of principal bundles and gauge transformations}{ClassGaugeTrafosAndcgPMulti}
Recall that gauge transformations are related to principal bundle automorphisms $f$ of the principal bundle $P$; see \textit{e.g.}\ \cite[\S 5.3, Def.\ 5.3.1, page 256f.]{Hamilton}\ and \cite[\S 5.4, Thm.\ 5.4.4, page 273]{Hamilton}. That is, $f$ is a diffeomorphism $P \to P$ with
\bas
\pi_P \circ f &= \mathds{1}_M,\\
f(p \cdot g) &= f(p) \cdot g
\eas
for all $p \in P$ and $g \in G$. The group of such maps will be denoted by $\sAut(P)$. One can identify such automorphisms with certain $G$-valued maps on $P$, following \cite[\S 5.3, Def.\ 5.3.2 \& Prop.\ 5.3.3, page 266f.]{Hamilton}: We define the following set of smooth maps $P \to G$ by
\bas
C^\infty(P;G)^G
&\coloneqq
\left\{
	\sigma: P \to G \text{ smooth}
	~\middle|~
	\sigma(p \cdot g) = c_{g^{-1}}\bigl( \sigma(p) \bigr) \text{ for all } p \in P, g \in G
\right\}.
\eas
It is straightforward to check that this is a group w.r.t.\ pointwise multiplication. Furthermore, there is a group isomorphism
\bas
\sAut(P) &\to C^\infty(P; G)^G,\\
f &\mapsto \sigma_f,
\eas
where $\sigma_f$ is defined by
\bas
f(p)
&=
p \cdot \sigma_f(p)
\eas
for all $p \in P$; one can prove that this is well-defined.

As argued in \cite[\S 5.3, Thm.\ 5.3.8, page 269; formulated as left action there, which is why we have an inverse here]{Hamilton}, $\sAut(P)$ acts (on the right) on associated fibre bundles $\mathcal{F} = P \times_\Psi F$ by
\bas
[p, v] \cdot f
&\coloneqq
\mleft[ f^{-1}(p), v \mright]
=
\mleft[ p \cdot \sigma_{f}(p)^{-1}, v \mright]
\eas
for all $[p, v] \in \mathcal{F}_x$ ($x \in M$) and $f \in \sAut(P)$. Therefore one could investigate whether there is also an action just with an element $g$ of $G$, basically the restriction of $\sigma_f$ onto the fibre $P_x$. However, the action given by $[p, v] \cdot g = \mleft[ p \cdot g^{-1}, v \mright]$ for $g \in G$ is in general clearly only well-defined w.r.t.\ a change of the representative of $[p, v] = \mleft[ p \cdot q, \Psi_{q^{-1}}(v) \mright]$ ($q \in G$), if $G$ is abelian. But one can resolve this by looking at it carefully: The rough idea is that $g$ basically comes from $\sigma_f(p)$ in this context, but
\bas
\sigma_f(p \cdot q)
&=
c_{q^{-1}}\bigl( \sigma_f(p) \bigr).
\eas
Roughly, while $p$ is multiplied with $g^{-1}$, $p \cdot q$ has to be multiplied with $q^{-1} g^{-1} q$. It is easy to check that this resolves that issue, and the result is precisely the action described in Ex.\ \ref{ex:AssocLGACtingOnAssocVec}. In fact, we have the following proposition:
\end{remarks}

For the following proposition observe that the (local) sections of an LGB have a group structure given by pointwise multiplication.

\begin{propositions}{Gauge transformations as sections of the inner LGB, \newline \cite[\S 1.4, (the last sentence of) Ex.\ 1.4.7, page 25]{mackenzieGeneralTheory}}{GaugeTrafoAndInnerLGB}
Let $P \stackrel{\pi_P}{\to} M$ be a principal bundle with structural Lie group $G$ over a smooth manifold $M$. Then there is a group isomorphism 
\bas
\sAut(P) &\to \Gamma\bigl( c_G(P) \bigr),\\
f &\mapsto q_f
\eas
where $q_f \in \Gamma\bigl( c_G(P) \bigr)$ is defined by
\bas
\mleft.q_f\mright|_x
&\coloneqq
\bigl[ p, \sigma_f(p) \bigr]
\eas
for all $x \in M$, where $p$ is any element of $P$ such that $\pi_P(p) = x$, and $\sigma_f$ is the element of $C^\infty(P; G)^G$ corresponding to $f$ as introduced in Rem.\ \ref{rem:ClassGaugeTrafosAndcgPMulti}.
\end{propositions}

\begin{remark}
\leavevmode\newline
As one may guess, $\Gamma\bigl( c_G(P) \bigr)$ is the analogue of $C^\infty(M; G)^G$ such that one could ask for a more direct analogue to $\sAut(P)$. Indeed, as argued in \cite[\S 1.3, Prop.\ 1.3.9, page 20]{mackenzieGeneralTheory}, $c_G(P)$ is isomorphic to $\mleft(P \times_M P\mright) \Big/ G$, where $P\times_M P \coloneqq \pi_P^*P$, and the $G$-action is the diagonal action on $P \times P$. One can prove that an isomorphism is given by 
\bas
c_G(P) &\to \mleft(P \times_M P\mright) \Big/ G,\\
[p, g] &\mapsto [p, p \cdot g].
\eas
It is also argued in \cite[\S 1.4, Ex.\ 1.4.7, page 25]{mackenzieGeneralTheory} that $\sAut(P)$ is isomorphic to $\Gamma\mleft(\mleft(P \times_M P\mright) \Big/ G\mright)$ by 
\bas
\sAut(P) &\to \Gamma\mleft(\mleft(P \times_M P\mright) \Big/ G\mright),\\
f &\mapsto L_f,
\eas
where $L_f \in \Gamma\mleft(\mleft(P \times_M P\mright) \Big/ G\mright)$ is given by
\bas
\mleft.L_f\mright|_x
&\coloneqq
[p, f(p)]
=
\mleft[ p, p \cdot \sigma_f(p) \mright]
\eas
for all $x \in M$, where $p$ is any element of $P$ such that $\pi_P(p) = x$. This is clearly well-defined, and, so, while $c_G(P)$ is the bundle-analogue of $C^\infty(P; G)^G$ one can think of $\mleft(P \times_M P\mright) \Big/ G$ as the bundle-analogue of $\sAut(P)$.

However, this description often arises if one wants to use the formalism of groupoids and algebroids, here especially using the \textbf{gauge groupoid} and \textbf{Atiyah algebroid} induced by $P$. These would allow an even more elegant version of the gauge transformations, however, we intend to write this paper in such a way that there is no need that the reader has knowledge about those bundle structures. See the cited references for more details in that regard.
\end{remark}

\begin{proof}[Proof of Prop.\ \ref{prop:GaugeTrafoAndInnerLGB}]
\leavevmode\newline
\indent $\bullet$ Let us first quickly check whether $q_f \in \Gamma\bigl( c_G(P) \bigr)$ is well-defined for all $f \in \sAut(P)$. For $p \in P_x$ ($x \in M$) we have
\bas
\mleft. q_f \mright|_x
&=
\mleft[ p, \sigma_f(p) \mright],
\eas
If $p^\prime = p \cdot g$ ($g \in G$) is another element of $P_x$, then, using $p^\prime$ to define $\mleft. q_f \mright|_x$,
\bas
\mleft. q_f \mright|_x
&=
\mleft[ p \cdot g, \sigma_f(p \cdot g) \mright]
=
\mleft[ p \cdot g, c_{g^{-1}}\bigl(\sigma_f(p)\bigr) \mright]
=
\mleft[ p, \sigma_f(p) \mright],
\eas
also using the definition of $c_G(P)$, recall Ex.\ \ref{ex:InnerLGBs}. It follows that $q_f$ is well-defined, and it is clear that $q_f$ is smooth.

$\bullet$ We want to show that $f \mapsto q_f$ is a group isomorphism by using that it is a composition of the group isomorphisms $\sAut(P) \to C^\infty(P; G)^G$ as in Rem.\ \ref{rem:ClassGaugeTrafosAndcgPMulti} and 
\ba
C^\infty(P; G)^G &\to \Gamma\bigl( c_G(P) \bigr),\nonumber\\
\sigma &\mapsto q_\sigma,\label{IsomCPGGTocGPSec}
\ea
where $q_\sigma$ is effectively the same definition as $q_f$, that is $q_\sigma|_x = [p, \sigma(p)]$ which is well-defined by the very same reasons as before. It is only left to show that $C^\infty(P; G)^G \to \Gamma\bigl( c_G(P) \bigr)$ is a group isomorphism. For injectivity let $\sigma^\prime$ be another element of $C^\infty(P; G)^G$ and assume $[p, \sigma(p)] = \mleft[ p, \sigma^\prime(p) \mright]$. Then
\bas
e_x
&=
[p, e]
=
[p, \sigma(p)] \cdot \underbrace{\mleft( \mleft[ p, \sigma^\prime(p) \mright] \mright)^{-1}}
	_{= \mleft[ p, \mleft(\sigma^\prime(p)\mright)^{-1} \mright]}
=
\mleft[ p, \sigma(p) \mleft(\sigma^\prime(p)\mright)^{-1} \mright],
\eas
such that
\bas
\sigma(p) \mleft(\sigma^\prime(p)\mright)^{-1}
&=
e,
\eas
so $\sigma = \sigma^\prime$ and hence injectivity follows. For surjectivity observe that for a section $q \in \Gamma\bigl( c_G(P) \bigr)$ we can define a map $\sigma: P \to G$ uniquely (unique due to the simply transitive action on $P$) by
\bas
q_x
&=
[p, \sigma(p)].
\eas
This map satisfies
\bas
[p, \sigma(p)]
&=
\mleft[p \cdot g, c_{g^{-1}}\bigl(\sigma(p)\bigr)\mright]
=
[p \cdot g, \sigma(p \cdot g)]
\eas
for all $g \in G$; the last equality implies $\sigma(p \cdot g) = c_{g^{-1}}\bigl(\sigma(p)\bigr)$, which is precisely what we need for $C^\infty(P; G)^G$. It is only left to show smoothness of $\sigma$. For an open neighbourhood $U \subset M$ of $x$ fix a trivialization $\varphi_U: P|_U \to U \times G$, and we denote
\bas
\varphi_U\mleft(p^\prime\mright)
&=
\mleft( \pi_P\mleft(p^\prime\mright), \beta_U\mleft(p^\prime\mright) \mright)
\eas
for all $p^\prime \in P$, where $\beta_U: P|_U \to G$ is an equivariant map, \textit{i.e.}\ $\beta_U\mleft(p^\prime \cdot g\mright) = \beta_U\mleft(p^\prime\mright) ~ g$ for all $g \in G$. As shown in the proof of Thm.\ \ref{thm:AssociatedGroupBundlesHaveGroupStructure}, we have a trivialization of $c_G(P)$ given by
\bas
c_G(P)|_U
&\to
U \times G,\\
\mleft[p^\prime, g\mright]
&\mapsto
\mleft(
	\pi_P\mleft(p^\prime\mright), c_{\beta_U\mleft(p^\prime\mright)} (g)
\mright).
\eas
Applying that trivialization to $q$ we derive that
\bas
\mleft[ p^\prime \mapsto c_{\beta_U\mleft(p^\prime\mright)}\mleft( \sigma\mleft(p^\prime\mright) \mright) \mright]
\eas
is smooth, because $q$ is smooth. Since $c_{\beta_U\mleft(p^\prime\mright)}$ is an element of $\mathrm{Aut}(G)$, we conclude that $\sigma$ is smooth. Hence, $\sigma \in C^\infty(P; G)^G$, so, Def.\ \eqref{IsomCPGGTocGPSec} is also surjective and thence bijective.

Finally let us show that Def.\ \eqref{IsomCPGGTocGPSec} is a group isomorphism. Let $\sigma, \sigma^\prime$ be elements of $C^\infty(P; G)^G$, then Def.\ \eqref{IsomCPGGTocGPSec} reads
\bas
\sigma \sigma^\prime
&\mapsto
q_{\sigma \sigma^\prime}
\eas
with
\bas
\mleft.q_{\sigma \sigma^\prime}\mright|_x
&=
\mleft[ p, \sigma(p) ~ \sigma^\prime(p) \mright]
=
[p, \sigma(p)] \cdot \mleft[p, \sigma^\prime(p) \mright]
=
\mleft.q_{\sigma}\mright|_x \cdot \mleft.q_{\sigma^\prime}\mright|_x,
\eas
such that Def.\ \eqref{IsomCPGGTocGPSec} satisfies
\bas
\sigma\sigma^\prime &\mapsto q_\sigma \cdot q_{\sigma^\prime}.
\eas
This concludes the proof.
\end{proof}
%
%Associated fibre bundles are motivated by making the invariance of gauge theory under local gauge transformations (that is, the change of gauge/local section of $P$) an inherent part of the bundle, similar to typical manifold coordinates; while the action of more global transformations "remain", similar to diffeomorphisms of a manifold. This procedure of "reducing" the action onto these is reflected in the quotient bundle $c_G(P)$.

\section{Lie algebra bundles (LABs)}

\subsection{Definition}

Lie algebras are the infinitesimal version of Lie groups, hence, we expect something similar for LGBs, the Lie algebra bundles:

\begin{definitions}{Lie algebra bundle (LAB), \cite[\S 3.3, Definition 3.3.8, page 104]{mackenzieGeneralTheory}}{LAB}
Let $\mathfrak{g}$ be a Lie algebra, and $\mathcal{g}, M$ be smooth manifolds. A vector bundle
\begin{center}
	\begin{tikzcd}
	\mathfrak{g} \arrow{r} & \mathcal{g} \arrow{d}{\pi} \\
	& M
	\end{tikzcd}
\end{center}
is called a \textbf{Lie algebra bundle (LAB)} if:
\begin{enumerate}
	\item $\mathfrak{g}$ and each fibre $\mathcal{g}_x$, $x \in M$, are Lie algebras;
	\item there exists a bundle atlas $\mleft\{ \mleft( U_i, \phi_i \mright) \mright\}_{i \in I}$ such that the induced maps
	\bas
	\phi_{ix}
	&\coloneqq
	\mathrm{pr}_2 \circ \mleft. \phi_i\mright|_{\mathcal{g}_x}: \mathcal{g}_x \to \mathfrak{g}
	\eas
	are Lie algebra isomorphisms, where $I$ is an (index) set, $U_i$ are open sets covering $M$, $\phi_i: \mathcal{g}|_U \to U \times \mathfrak{g}$ subordinate trivializations, and $\mathrm{pr}_2$ the projection onto the second factor. This atlas will be called \textbf{Lie algebra bundle atlas} or \textbf{LAB atlas}.
\end{enumerate}
We often say that \textbf{$\mathcal{g}$ is an LAB (over $M$)}, whose structural Lie algebra is either clear by context or not explicitly needed; and we may also denote LABs by $\mathfrak{g} \to \mathcal{g} \stackrel{\pi}{\to} M$.
\end{definitions}

Of course, we have the typical trivial examples:

\begin{examples}{Trivial examples}{TrivialLABs}
We recover the notion of a Lie algebra, if $M$ consist of just one point. Moreover, the \textbf{trivial LAB} is given as the product manifold $\mathcal{g} \coloneqq M \times \mathfrak{g} \to M$. We have obviously a canonical smooth field of Lie brackets on this bundle $\mleft[ \cdot, \cdot \mright]_{\mathcal{g}}: \Gamma(\mathcal{g}) \times \Gamma(\mathcal{g}) \to \Gamma(\mathcal{g})$, \textit{i.e.}\ $\mleft[ \cdot, \cdot \mright]_{\mathcal{g}} \in \Gamma\mleft(\bigwedge^2 \mathcal{g}^* \otimes \mathcal{g} \mright)$ which restricts to the Lie algebra bracket $\mleft[ \cdot, \cdot \mright]_{\mathfrak{g}}$ of $\mathfrak{g}$ on each fibre. The bracket is given by
\bas
\mleft[ (x, X), (x, Y) \mright]_{\mathcal{g}}
&\coloneqq
\mleft( x, \mleft[ X, Y \mright]_{\mathfrak{g}} \mright)
\eas
for all $(x, X), (x, Y) \in M \times \mathfrak{g}$. Smoothness is an immediate consequence.
\end{examples}

The definition of LAB morphisms is straight-forward:

\begin{definitions}{LAB morphism, \newline\cite[\S. 4.3, simplified version of Def.\ 4.3.1, page 158]{mackenzieGeneralTheory}}{LABmorphism}
Let $\mathcal{g} \stackrel{\pi_{\mathcal{g}}}{\to} M$ and $\mathcal{h} \stackrel{\pi_{\mathcal{h}}}{\to} N$ be two LGBs over two smooth manifolds $M$ and $N$. An \textbf{LAB morphism} is a pair of smooth maps $F: \mathcal{h} \to \mathcal{g}$ and $f: N \to M$ such that
\ba\label{FibreRelationOverfForLABMorph}
\pi_{\mathcal{g}} \circ F &= f \circ \pi_{\mathcal{h}},\\
F &\text{ linear},\\
F\mleft(\mleft[g, q\mright]_{\mathcal{h}_p}\mright) &= \mleft[F(g), F(q)\mright]_{\mathcal{g}_{f(p)}}\label{LABHomomorph}
\ea
for all $g, q \in \mathcal{h}_p$ ($p \in N$), where $\mleft[\cdot, \cdot\mright]_{\mathcal{h}_p}$ and $\mleft[\cdot, \cdot\mright]_{\mathcal{g}_{f(p)}}$ are Lie brackets of $\mathcal{h}_p$ and $g_{f(p)}$, respectively. We also say that \textbf{$F$ is an LAB morphism over $f$}. If $N = M$ and $f = \mathrm{id}_M$, then we often omit mentioning $f$ explicitly and just write that \textbf{$F$ is a (base-preserving) LAB morphism}.

We speak of an \textbf{LAB isomorphism (over $f$)} if $F$ is a diffeomorphism.
\end{definitions}

\begin{remarks}{Smooth field of Lie brackets}{FieldOfLieBrackets}
We have similar remarks as in Rem.\ \ref{LGBMOrphismRemark}. Additionally, we have locally a canonical smooth field of Lie brackets which restricts to a Lie bracket on each fibre because every LAB is locally isomorphic to a trivial LAB as in Ex.\ \ref{ex:TrivialLABs}. Define a field of Lie brackets $\mleft[ \cdot, \cdot \mright]_{\mathcal{g}}: \Gamma(\mathcal{g}) \times \Gamma(\mathcal{g}) \to \Gamma(\mathcal{g})$, \textit{i.e.}\ $\mleft[ \cdot, \cdot \mright]_{\mathcal{g}} \in \Gamma\mleft(\bigwedge^2 \mathcal{g}^* \otimes \mathcal{g} \mright)$, by
\ba
\mleft[ X, Y \mright]_{\mathcal{g}}
&\coloneqq
\mleft[ X, Y \mright]_{\mathcal{g}_x}
\ea
for all $X, Y \in \mathcal{g}_x$ ($x \in M$). Using a local trivialization, this bracket is locally of the form as in Rem.\ \ref{LGBMOrphismRemark} such that smoothness follows.

In fact, as also argued in \cite[\S 16.2, Example 2, page 114; but speaking in the context of Lie algebroids there, a generalization of LABs]{DaSilva}, every vector bundle equipped with a smooth field of Lie brackets is an LAB.
\end{remarks}

Endomorphisms of a vector bundle are of course another important example of LABs.

\begin{examples}{Endomorphisms of a vector bundle, \cite[\S 3.3, part of Ex.\ 3.3.4]{mackenzieGeneralTheory}}{EndVAnLAB}
Let $V \to M$ be a vector bundle, and denote with $\mathrm{End}(V) \to M$ its bundle of fibre-wise endomorphisms (its sections are the base-preserving bundle endomorphisms of $V$). This is clearly an LAB whose field of Lie brackets is given by the commutator.
\end{examples}

As we have seen for LGBs, the pullback of LABs is again an LAB.

\begin{corollaries}{Pullbacks of LABs are LABs, \cite[\S 3, Thm.\ 3.2]{PullbackLGBLAB}}{PullBackLABIsLAB}
Let $M, N$ be smooth manifolds, $\mathcal{g} \stackrel{\pi}{\to} M$ an LAB over $M$ and $f: N \to M$ a smooth map. Then the pullback vector bundle $f^*\mathcal{g}$ has a unique (up to isomorphisms) LAB structure such that the projection $\pi_2: f^*\mathcal{g} \to \mathcal{g}$ onto the second factor is an LAB morphism over $f$ with $\pi_2|_x: (f^*\mathcal{g})_x \to \mathcal{g}_{f(x)}$ being a Lie algebra isomorphism for all $x \in N$.
%
%$f^*\mathcal{g}$ is the \textbf{pullback LAB of $\mathcal{g}$ (under $f$)}.
\end{corollaries}

\begin{proof}
\leavevmode\newline
Either prove this similarly as Cor.\ \ref{cor:PullbackLGB} (by also using a similar statement already known for the pullbacks of vector bundles), or observe that the pullback $f^*\mleft(\mleft[ \cdot, \cdot \mright]_{\mathcal{g}}\mright)$ of the field of Lie brackets $\mleft[ \cdot, \cdot \mright]_{\mathcal{g}}$ on $\mathcal{g}$ as a section is clearly also a smooth field of Lie brackets on $f^*\mathcal{g}$ with same structural Lie algebra $\mathfrak{g}$.
\end{proof}

\begin{definitions}{Pullback LAB}{PullbackLABDef}
Let $M, N$ be smooth manifolds, $\mathcal{g} \to M$ an LGB over $M$ and $f: N \to M$ a smooth map. Then we call the LAB structure on $f^*\mathcal{g}$ as given in Cor.\ \ref{cor:PullBackLABIsLAB} the \textbf{pullback LAB of $\mathcal{g}$ (under $f$)}.

By writing $f^*\mathcal{g}$ we will often refer to this structure without further mention.
\end{definitions}

\subsection{From LGBs to LABs}

Let us now quickly discuss how LGBs and LABs are related; it is very similar to the relation of Lie groups and algebras, now somewhat fibre-wise. We will follow the style of \cite[\S 1.5.2, page 40ff.]{Hamilton}\ and \cite[\S 3.5, page 119ff.]{mackenzieGeneralTheory}; our approach will be using left-invariant vector fields but the mentioned latter reference actually uses right-invariant vector fields.

Let us start with introducing the basic notations.

\begin{definitions}{Left and right translation and conjugation, \newline \cite[\S 1.5, similar notation to Def.\ 1.5.3, page 40]{Hamilton}}{LeftRightTranslationConjugation}
Let $\mathcal{G} \to M$ be an LGB over a smooth manifold $M$. For $g \in \mathcal{G}_x$ ($x \in M$) we define the following maps:
\begin{itemize}
	\item \textbf{Left translation} given by
		\bas
			L_g: \mathcal{G}_x &\to \mathcal{G}_x,\\
			h &\mapsto g h.
		\eas
	\item \textbf{Right translation} given by
		\bas
			R_g: \mathcal{G}_x &\to \mathcal{G}_x,\\
			h &\mapsto h g.
		\eas
	\item \textbf{Conjugation} given by
		\bas
			c_g: \mathcal{G}_x &\to \mathcal{G}_x,\\
			h &\mapsto g h g^{-1}.
		\eas
\end{itemize}
\end{definitions}

\begin{remark}
\leavevmode\newline
By definition of $\mathcal{G}$, all these maps are smooth. Furthermore, they clearly satisfy the typical properties as known for these maps since $\mathcal{G}_x$ is a Lie group for all $x \in M$; for reference about their basic properties see for example \cite[\S 1.5, Lemma 1.5.5, page 40f.]{Hamilton}. 
%We will later introduce similar notations using sections of $\mathcal{G}$, see Def.\ \ref{def:LRTranslations}.

The left and right translations of Def.\ \ref{def:LeftRightTranslationConjugation} and \ref{def:LRTranslations} align, and thus the smoothness concerns as mentioned in the last part of Rem.\ \ref{SmoothnessOfACtionTranslations} for right translations $r_g = R_g$ ($g \in \mathcal{G}_x$, $x \in M$) do not arise. Moreover, while the conjugation $c_g$ is a Lie group automorphism of $\mathcal{G}_x$, it describes an LGB automorphism of $\mathcal{G}$ if extended to sections; following \cite[\S 1.4, Def.\ 1.4.6 and its discussion afterwards, page 24f.]{mackenzieGeneralTheory}. That is for $\sigma \in \Gamma(\mathcal{G})$ we define the conjugation $c_\sigma$ as a smooth map by
\bas
\mathcal{G} &\to \mathcal{G},\\
q &\mapsto c_\sigma(q) \coloneqq \mleft(L_\sigma \circ R_{\sigma^{-1}}\mright)(q) = \mleft( R_{\sigma^{-1}} \circ L_\sigma \mright)(q) = \sigma_{\pi(q)} \cdot q \cdot \sigma_{\pi(q)}^{-1}.
\eas
It is clear that $c_\sigma(gq) = c_\sigma(g) \cdot c_\sigma(q)$ for all $g, q \in \mathcal{G}$ with $\pi(g) = \pi(q)$, and that a smooth inverse is given by $c_{\sigma^{-1}}$; thence, $c_\sigma$ is an LGB isomorphism of $\mathcal{G}$ on itself, an automorphism, in sense of Def.\ \ref{def:LGB morphism}. It is also trivial to check that we have $c_{\sigma \cdot \tau} = c_\sigma \circ c_\tau$, where $\tau$ is another section of $\mathcal{G}$.

Analogously we defined $R_\sigma$ as $r_\sigma$ of Def.\ \ref{def:LRTranslations}; with the capital letter we put an emphasis on that the $\mathcal{G}$-action acts on $\mathcal{G}$ itself. Similarly for left translations.
\end{remark}

Since these are diffeomorphism of the fibres, it makes sense to say that a left-invariant vector field of $\mathcal{G}$ has to be a vertical vector field, that is, it is in the kernel of $\mathrm{D}\pi$, the total differential/tangent map of the projection of $\mathcal{G} \stackrel{\pi}{\to} M$. For this recall that there is the notion of a \textbf{vertical bundle} for fibre bundles $F \stackrel{\varpi}{\to} M$ (as \textit{e.g.}~introduced in \cite[\S 5.1.1, for principal bundles, but it is straightforward to extend the definitions; page 258ff.]{Hamilton}), which is defined as a subbundle $\mathrm{VF} \to F$ of the tangent bundle $\mathrm{T}F \to F$ given as the kernel of $\mathrm{D}\varpi : \mathrm{T}F \to \mathrm{T}M$. The fibres $\mathrm{V}_v F$ of $\mathrm{V}F$ at $v \in F$ are then given by 
\bas
\mathrm{V}_v F
&=
\mathrm{T}_v F_x,
\eas
where $x \coloneqq \varpi(v) \in M$ and $F_x$ is the fibre of $F$ at $x$. $F_x$ is an embedded submanifold of $F$, thence, by definition a section $X \in \Gamma(\mathrm{V}F)$ restricts to a vector field on the fibres, that is, 
\bas
X|_{F_x} \in \mathfrak{X}(F_x).
\eas

In our case $F = \mathcal{G}$, hence $F_x = \mathcal{G}_x$ is a Lie group, so, the vertical bundle just consists of the tangent bundles of Lie groups of all fibres. All of these are generated by their Lie algebra at $e_x$, the identity element of $\mathcal{G}_x$. Hence, it is natural to guess that the LAB for $\mathcal{G}$ will be $\mathrm{V}\mathcal{G}|_{e_M}$, where $e_M$ is the image of $M$ under the identity/neutral section of $\mathcal{G}$, thus, an embedding of $M$ into $\mathcal{G}$. Therefore $\mathrm{V}\mathcal{G}|_{e_M}$ is a fibre bundle by \cite[\S 4.1, Lemma 4.1.16, page 204]{Hamilton}, and clearly a vector bundle. Equivalently, since the identity section $e$ is an embedding, we think of $\mathrm{V}\mathcal{G}|_{e_M}$ as the pullback vector bundle $e^*\mathrm{V}\mathcal{G}$, which is conveniently a vector bundle over $M$.

Hence, let us now show that $\mathcal{G}$ will be related to $e^*\mathrm{V}\mathcal{G}$ similar to how a Lie group will be related to its Lie algebra.

\begin{definitions}{Left-invariant vector fields on LGBs, \newline \cite[\S 3.5, special situation of Def.\ 3.5.2, page 120]{mackenzieGeneralTheory}}{LGBLeftinvariantVectorFields}
Let $\mathcal{G} \stackrel{\pi}{\to} M$ be an LGB over a smooth manifold $M$. A vector field $X \in \mathfrak{X}(\mathcal{G})$ is a \textbf{left-invariant vector field} if
\begin{enumerate}
	\item $X$ is vertical, that is,
	\bas
		X &\in \Gamma(\mathrm{V}\mathcal{G}),
	\eas
	\item $X$ is invariant under the left-multiplication on each fibre, \textit{i.e.}\
	\bas
		\mathrm{D}_gL_q(X_g) &= X_{qg}
	\eas
	for all $q, g \in \mathcal{G}_x$, where $x \coloneqq \pi(g) = \pi(q)$.
\end{enumerate}
The set of all left-invariant vector fields on $\mathcal{G}$ will be denoted by $L(\mathcal{G})$.
\end{definitions}

\begin{remark}
\leavevmode\newline
Observe that the second point in the definition is well-defined because $X$ is a vertical vector field; that is, recall that $L_q: \mathcal{G}_x \to \mathcal{G}_x$ such that $\mathrm{D}_gL_q: \mathrm{T}_g\mathcal{G}_x \to \mathrm{T}_{qg}\mathcal{G}_x$, hence, $\mathrm{D}_gL_q: \mathrm{V}_g\mathcal{G} \to \mathrm{V}_{qg}\mathcal{G}$.
\end{remark}

\begin{remarks}{Abstract notation 1}{AbstractNotationForLeftInvarianceVf}
Since $X$ is vertical, recall that we can view the restriction of $X$ onto a fibre as a vector field on that fibre, \textit{i.e.}\
\bas
X|_{\mathcal{G}_x} &\in \mathfrak{X}\mleft( \mathcal{G}_x \mright).
\eas
$\mathcal{G}_x$ is a Lie group and left translations are diffeomorphisms on it, hence, the left-invariance can also be written as
\ba
\mathrm{D}L_q\mleft( X|_{\mathcal{G}_x} \mright) &= L_q^*\mleft(X|_{\mathcal{G}_x}\mright).
\ea
For this recall that $\mathrm{D}L_q \in \Omega^1\mleft(\mathcal{G}_x; L_q^*\mathrm{T}\mathcal{G}_x\mright)$ for the left hand side, and that $L_q^*$ is the pullback of sections on the right hand side, that is, $L_q^*\mleft(X|_{\mathcal{G}_x}\mright) \in \Gamma\mleft(L_q^*\mathrm{T}\mathcal{G}_x\mright)$.
Furthermore, $X|_{\mathcal{G}_x}$ is therefore a left-invariant vector field on $\mathcal{G}_x$ as Lie group. Which is why one may also define the left-invariance of $X$ as a vector field on $\mathcal{G}$ by saying that it has to restrict to a left-invariant vector field on each fibre in the usual sense of Lie groups.
\end{remarks}

One quickly shows that this is a Lie subalgebra of $\mathfrak{X}(\mathcal{G})$.

\begin{lemmata}{Closure of Lie bracket for left-invariant vector fields, \newline \cite[\S 3.5, special situation of Lemma 3.5.5, page 122]{mackenzieGeneralTheory}}{LeftInvVecAreClosed}
Let $\mathcal{G} \stackrel{\pi}{\to} M$ be an LGB over a smooth manifold $M$. Then $L(\mathcal{G})$ is a Lie subalgebra of $\mathfrak{X}(\mathcal{G})$.
\end{lemmata}

\begin{proof}
\leavevmode\newline
Let $X, Y \in L(\mathcal{G})$, then we need to show 
\bas
\mathrm{D}\pi\bigl( \mleft[ X, Y \mright] \bigr) &\equiv 0,
\eas
and if this holds, then we also need to derive
\bas
\mathrm{D}_gL_q\mleft( \mleft.\mleft[ X, Y \mright]\mright|_g \mright) &= \mleft.\mleft[ X, Y \mright]\mright|_{qg}.
\eas
One can either immediately show these directly by using statements like \cite[Proposition A.1.49; page 615]{Hamilton}, which essentially describes how the Lie bracket of vector fields react under push-forwards. Or use the knowledge about Lie groups, recall Rem.\ \ref{rem:AbstractNotationForLeftInvarianceVf}: Each fibre $\mathcal{G}_x$ is an embedded submanifold of $\mathcal{G}$ and both, $X|_{\mathcal{G}_x}$ and $Y|_{\mathcal{G}_x}$, are vector fields of this submanifold. Thus, $\mleft.\bigl[ X|_{\mathcal{G}_x}, Y|_{\mathcal{G}_x} \bigr]\mright|_p$ has values in $\mathrm{T}_p\mathcal{G}_x$ for all $p \in \mathcal{G}_x$. Especially,
\bas
\mleft.\bigl[ X|_{\mathcal{G}_x}, Y|_{\mathcal{G}_x} \bigr]\mright|_{\mathcal{G}_x}
&\in \mathfrak{X}(\mathcal{G}_x).
\eas
Because of this and since $X|_{\mathcal{G}_x}$ and $Y|_{\mathcal{G}_x}$ are left-invariant vector fields of $\mathcal{G}_x$ (a Lie group), left-invariance of $\mleft.\bigl[ X|_{\mathcal{G}_x}, Y|_{\mathcal{G}_x} \bigr]\mright|_{\mathcal{G}_x}$ follows, and thus the statement.
\end{proof}

Of course, elements of $L(\mathcal{G})$ are determined by their values at $e_M$, as already suggested previously. Let us show this now; starting with a small auxiliary result.

\begin{corollaries}{$L(\mathcal{G})$ a $C^\infty(M)$-module, \newline\cite[\S 3.5, comment before Lemma 3.5.5, page 122]{mackenzieGeneralTheory}}{LeftInvVefCMModule}
Let $\mathcal{G} \stackrel{\pi}{\to} M$ be an LGB over a smooth manifold $M$. Then $L(\mathcal{G})$ is a $C^\infty(M)$-module under the multiplication
\bas
fX
&\coloneqq
\pi^*f ~ X
\eas
for all $f \in C^\infty(M)$ and $X \in L(\mathcal{G})$.
\end{corollaries}

\begin{proof}
\leavevmode\newline
Obviously, $fX \in \Gamma(\mathrm{V}\mathcal{G})$ since
\bas
\mathrm{D}\pi(fX) &= \mathrm{D}\pi(\pi^*f~ X) = \pi^*f~ \mathrm{D}\pi(X) = 0.
\eas
Furthermore, $fX|_{\mathcal{G}_x}$ ($x \in M$) is left-invariant over $\mathcal{G}_x$ since $X|_{\mathcal{G}_x}$ is left-invariant and $f|_{\mathcal{G}_x} \equiv f(x) \in \mathbb{R}$. Thence, $fX \in L(\mathcal{G})$.
\end{proof}

\begin{corollaries}{$L(\mathcal{G})$ as sections of $e^*\mathrm{V}\mathcal{G}$, \newline\cite[\S 3.5, comment before Lemma 3.5.5, page 122; parts of Cor.\ 3.5.4, page 121]{mackenzieGeneralTheory}}{LeftInvVfToLAB}
Let $\mathcal{G} \stackrel{\pi}{\to} M$ be an LGB over a smooth manifold $M$, and denote with $e$ the identity section of $\mathcal{G}$. Then we have an isomorphism of $C^\infty(M)$-modules
\bas
L(\mathcal{G}) &\to \Gamma\mleft(e^*\mathrm{V}\mathcal{G}\mright),\\
X &\mapsto e^*X.
\eas
The inverse of this map is given by
\bas
\Gamma\mleft(e^*\mathrm{V}\mathcal{G}\mright) &\to L(\mathcal{G}),\\
\nu &\mapsto X_\nu,
\eas
where $X_\nu$ is given by
\bas
\mleft.X_\nu\mright|_g
&\coloneqq
\mathrm{D}_{e_x}L_g \bigl( \mathrm{pr}_2(\nu_x) \bigr)
\eas
for all $g \in \mathcal{G}$ and $x \coloneqq \pi(g)$, where $\mathrm{pr}_2$ is the projection onto the second component in $e^*\mathrm{V}\mathcal{G}$.
\end{corollaries}

\begin{remarks}{Abstract notation 2}{AbstractNotationTwoForLeftInvarVfs}
Since $e^*\mathrm{V}\mathcal{G} \cong \mathrm{V}\mathcal{G}|_{e_M}$ is a very natural isomorphism, we will often just write
\bas
\mleft.X_\nu\mright|_g
&=
\mathrm{D}_{e_x}L_g ( \nu_x ),
\eas
omitting $\mathrm{pr}_2$ and using that natural isomorphism without further mention.

Also recall that we can actually define a left translations by (local) sections of $\mathcal{G}$, \textit{i.e.}\ for $\sigma \in \Gamma(\mathcal{G})$ we define the left translation $L_\sigma$ as a map by
\bas
\mathcal{G} &\to \mathcal{G},\\
q &\mapsto \sigma_{\pi(q)} \cdot q.
\eas
This map is a diffeomorphism, and restricts to the fibres $\mathcal{G}_x$ as embedded submanifolds to the map $L_{\sigma_x}$; we discussed this in more generality in Rem.\ \ref{SmoothnessOfACtionTranslations}. Observe that for vertical vector fields $Y \in \Gamma(\mathrm{V}\mathcal{G})$ we have
%\bas
%\mathrm{D}\pi\bigl(\mathrm{D}L_\sigma(Y)\bigr)
%&=
%\mathrm{D}\underbrace{(\pi \circ L_\sigma)}_{= \pi}(Y)
%=
%\mathrm{D}\pi(Y)
%=
%0,
%\eas
%that is, we get $\mathrm{D}L_\sigma(Y)$ is vertical at each point $q \in \mathcal{G}$, especially also
\bas
\mathrm{D}_qL_\sigma(Y_q)
&=
\mleft.\frac{\mathrm{d}}{\mathrm{d}t}\mright|_{t=0}(L_\sigma \circ \gamma)
\equiv
\mleft.\frac{\mathrm{d}}{\mathrm{d}t}\mright|_{t=0}\mleft(L_{\sigma_{\pi(q)}} \circ \gamma\mright)
=
\mathrm{D}_qL_{g}(Y_q)
\eas
where $g \coloneqq \sigma_{\pi(q)}$ and $\gamma: I \to \mathcal{G}_{\pi(q)}$ ($I$ an open interval containing 0) is a curve with $\gamma(0)=q$ and $\mathrm{d}/\mathrm{d}t|_{t=0} \gamma = Y_q$. Therefore $L_\sigma$ restricts onto vertical vector fields and is then just the left translation via an element in the fibre over a fixed base point. In total one can then introduce the brief notation
\bas
X_\nu \circ \sigma
&=
\mathrm{D}L_\sigma (\nu)
=
\mathrm{D}L_\sigma|_{e_M} (\nu)
=
\mathrm{D}L_\sigma \circ \nu.
\eas
However, be careful, in general one cannot simply replace $L_g$ with $L_\sigma$, even if $\sigma_{\pi(g)} = g$. This only works with respect to vertical tangent vectors; once horizontal parts play a role things change, $L_g$ is a priori not even defined then. Once we turn to the definition of horizontal distributions we will come back to this.
\end{remarks}

\begin{proof}[Proof of Cor.\ \ref{cor:LeftInvVfToLAB}]
\leavevmode\newline
This map is clearly $C^\infty(M)$-linear, especially due to
\bas
e^*(fX)
&=
e^*( \pi^*f ~ X)
=
(f\circ \underbrace{\pi \circ e}_{= \mathds{1}_M}) ~ e^*X
=
f~ e^*X
\eas
for all $f \in C^\infty(M)$ and $X \in L(\mathcal{G})$; for this recall Cor.\ \ref{cor:LeftInvVefCMModule}.

We essentially only need to show that the suggested inverse $\nu \mapsto X_\nu$ is well-defined. First of all, that $X_\nu$ is vertical and left-invariant is clear by construction; $\nu_x$ ($x \in M$) is an element of the Lie algebra of $\mathcal{G}_x$, and thus $X_\nu|_{\mathcal{G}_x}$ is a left-invariant vector field on $\mathcal{G}_x$. $X$ is therefore an element of $L(\mathcal{G})$ once we know that $X$ is smooth. We show smoothness similar as in \cite[\S 1.5, proof of Lemma 1.5.13, page 42]{Hamilton}: Denote the multiplication in $\mathcal{G}$ by $\mu: \mathcal{G} * \mathcal{G} \to \mathcal{G}$. Then observe that $(0_g, \nu_x)$ ($0_g \in \mathrm{T}_g\mathcal{G}$, $g \in \mathcal{G}_x$, the zero vector field $0$ on $\mathrm{T}\mathcal{G}$) is an element of
\bas
&\mathrm{T}\mleft(\mathcal{G} * \mathcal{G}\mright)\\
&=
\left\{
	(Y, Z)
	~\middle|~
	Y \in \mathrm{T}_q \mathcal{G}, Z \in \mathrm{T}_h \mathcal{G} \text{ with } \mathrm{D}_q\pi(Y) = \mathrm{D}_h\pi(Z), \text{ where } q, h \in \mathcal{G} \text{ with } \pi(q)=\pi(h)
\right\}
\eas
because $\nu_x \in \mathrm{V}_{e_x}\mathcal{G}$.\footnote{If it is not clear how to derive the tangent bundle of $\mathcal{G}*\mathcal{G}$, then see later when we will discuss it in a more general manner. However, essentially recall that $\mathcal{G}*\mathcal{G} = \pi^*\mathcal{G}$.} Therefore we can calculate
\bas
\mathrm{D}_{(g, e_x)}\mu(0_g, \nu_x)
&=
\mleft.\frac{\mathrm{d}}{\mathrm{d}t}\mright|_{t=0}(g \cdot \gamma)
=
\mleft.\frac{\mathrm{d}}{\mathrm{d}t}\mright|_{t=0}(L_g \circ \gamma)
=
\mathrm{D}_{e_x}L_g(\nu_x)
=
X_\nu|_g,
\eas
where $\gamma: I \to \mathcal{G}_{x}$ ($I$ an open interval containing 0) is a curve with $\gamma(0)= e_x$ and $\mathrm{d}/\mathrm{d}t|_{t=0} \gamma = \nu_x$. Since $\mu$ is smooth, $\mathrm{D}\mu$ is smooth, and thus 
\bas
\mathcal{G} &\to \mathrm{T}(\mathcal{G} * \mathcal{G}),\\
g &\mapsto \mleft.\bigl(\mathrm{D}\mu \circ (0,\nu_\pi)\bigr)\mright|_g = \mathrm{D}_{(g, e_x)}\mu\mleft(0_g, \nu_{\pi(g)}\mright) = X_\nu|_g
\eas
is smooth, also using the smoothness of $\nu$, $\pi$ and $g \mapsto 0_g$.
%Fix an element $g$ of $\mathcal{G}$ and a trivialization of $\mathcal{G}$ around $g$, so, w.r.t.\ this trivialization $\mathcal{G} \cong U \times G$ ($U$ open subset of $M$). Then define a section $\sigma$ of this trivial bundle to be the constant section with values $g$. By Rem.\ \ref{rem:AbstractNotationTwoForLeftInvarVfs} we have
%\bas
%X_\nu \circ \sigma
%&=
%\mathrm{D}L_\sigma(\nu),
%\eas
%where $L_\sigma$ is a diffeomorphism. Thus, smoothness of $X_\nu$ follows.
%we have ($g \in \mathcal{G}_x$, $x \in M$)
%\bas
%\mathrm{D}\pi|_{\mathcal{G}_x} \circ \mathrm{D}L_g
%&=
%\mathrm{D}\underbrace{(\pi|_{\mathcal{G}_x} \circ L_g)}_{= \pi}
%=
%\mathrm{D}\pi,
%\eas

Finally, that $\phi: L(\mathcal{G}) \to \Gamma\mleft(e^*\mathrm{V}\mathcal{G}\mright)$, $X \mapsto e^*X$, is bijective is also clear, similar to typical gauge theory; we know that $X|_{\mathcal{G}_x}$ is a left-invariant vector field on $\mathcal{G}_x$ by Rem.\ \ref{rem:AbstractNotationForLeftInvarianceVf}. Hence, for $g \in \mathcal{G}_x$,
\bas
\mleft.\mleft(X|_{\mathcal{G}_x}\mright)\mright|_g
&=
\mathrm{D}_{e_x}L_g(X_{e_x})
=
\mathrm{D}_{e_x}L_g\mleft(e^*X|_x\mright).
\eas
This is precisely the structure of the suggested inverse, that is, 
\bas
X
&=
X_{e^*X}
=
(\psi \circ \phi)(X),
\eas
where $\psi: \Gamma(e^*\mathrm{V}\mathcal{G}) \to L(\mathcal{G}), \nu \mapsto X_\nu$. Hence, injectivity follows; surjectivity simply follows similarly by
\bas
(\phi \circ \psi)(\nu)|_x
&=
\mleft.e^*X_\nu\mright|_x
=
\underbrace{\mathrm{D}_{e_x}L_{e_x}}_{= \mathds{1}_{\mathrm{V}_{e_x}\mathcal{G}}}(\nu_x)
=
\nu_x
\eas
for all $\nu \in \Gamma(e^*\mathrm{V}\mathcal{G})$ and $x \in M$. This finishes the proof.
\end{proof}

This result shows the typical statement about that elements of $L(\mathcal{G})$ are uniquely determined by their values at $e_M$. It immediately follows, too, that:

\begin{corollaries}{LGBs induce an LAB structure, \newline\cite[\S 3.5, simplified version of the discussion after Cor.\ 3.5.4, page 121ff.]{mackenzieGeneralTheory}}{LGBToLAB}
Let $G \to \mathcal{G} \to M$ be an LGB over a smooth manifold $M$, and denote with $e$ the identity section of $\mathcal{G}$. Then $\mathcal{g} \coloneqq e^*\mathrm{V}\mathcal{G} \to M$ admits the structure as an LAB with structural Lie algebra $\mathfrak{g}$, the Lie algebra of $G$, and the fibres $\mathcal{g}_x$ ($x \in N$) are the Lie algebras of $\mathcal{G}_x$. The field of Lie algebra brackets $\mleft[ \cdot, \cdot \mright]_{\mathcal{g}}$ is given by
\bas
\mleft[ \nu, \mu \mright]_{\mathcal{g}}
&\coloneqq
e^*\bigl( [X_\nu, X_\mu] \bigr)
\eas
for all $\nu, \mu \in \Gamma(e^*\mathrm{V}\mathcal{G})$, where $X_\nu, X_\mu$ are elements of $L(\mathcal{G})$ as given in Cor.\ \ref{cor:LeftInvVfToLAB}. Point-wise
\bas
\mleft[ \nu_x, \mu_x \mright]_{\mathcal{g}}
&=
\bigl.[X_\nu, X_\mu]\bigr|_{e_x}
\eas
for all $x \in M$.
\end{corollaries}

\begin{proof}
\leavevmode\newline
As already discussed $e^*\mathrm{V}\mathcal{G}$ is a vector bundle. The fibres are given by
\bas
\mathcal{g}_{x}
&=
\mathrm{T}_{e_x}\mathcal{G}_x
\cong
\mathrm{T}_{e_G} G
=
\mathfrak{g}
\eas
for all $x \in M$, where we used that $\mathcal{G}_x$ is isomorphic to $G$ as a Lie group whose neutral element we denoted formally by $e_G$. All fibres are Lie algebras of the fibre Lie group, isomorphic to $\mathfrak{g}$. By construction, the Lie bracket is precisely the Lie bracket isomorphic to the one of $\mathfrak{g}$, and $\mleft[ \cdot, \cdot \mright]_{\mathcal{g}}$ is smooth. Therefore we conclude that $\mathcal{g}$ is an LAB with structural Lie algebra $\mathfrak{g}$. Alternatively see \cite[\S 3.5, Ex.\ 3.5.12, page 126]{mackenzieGeneralTheory} for an explicit construction of an LAB atlas.
\end{proof}

\begin{definitions}{The LAB of an LGB, \newline\cite[\S 3.5, special situation of Def.\ 3.5.1, page 120]{mackenzieGeneralTheory}}{LABOfAnLGB}
Let $\mathcal{G} \to M$ be an LGB over a smooth manifold $M$, and denote with $e$ the identity section of $\mathcal{G}$. Then we define the \textbf{LAB $\mathcal{g}$ of $\mathcal{G}$} as the vector bundle $e^*\mathrm{V}\mathcal{G}$.
\end{definitions}

In the following $\mathcal{g}$ will usually denote the LAB of $\mathcal{G}$.

\begin{examples}{Endomorphisms of a vector bundle as LAB of fibre-wise automorphisms}{EndosAreLABOfAutos}
Recall Ex.\ \ref{ex:AutOfVectorBundleAnLGB} and \ref{ex:EndVAnLAB}. For a vector bundle $V \to M$ one can show that the LAB of $\mathrm{Aut}(V)$ is given by $\mathrm{End}(V)$; the proof is precisely as for the automorphisms and endomorphisms of a vector space as in \cite[\S 1.5.4, page 45ff.]{Hamilton}, just canonically extended to a bundle language. 
\end{examples}

\begin{examples}{Associated LAB}{AssociatedLABsFromAssocLGBs}
We discussed associated LGBs in Subsection \ref{AssocLGBsSubSection}. So, let $G, H$ be Lie groups, $P \to M$ a principal $G$-bundle over a smooth manifold $M$, and $\psi$ a $G$-representation on $H$. Then we have the associated LGB $\mathcal{H} \coloneqq P \times_\psi H$.

Also recall Remark \ref{rem:WhyRepresentation}, $\psi_* \coloneqq \mleft[ g \mapsto \mathrm{D}_{e_G}\psi_g \mright]$ is a $G$-representation on $\mathfrak{h}$, the Lie algebra of $H$, where $e_G$ is the neutral element of $G$. Hence, we also have the associated bundle $\mathcal{h} \coloneqq P \times_{\psi_*} \mathfrak{h}$.

It is now natural to guess that $\mathcal{h}$ is the LAB of $\mathcal{H}$, and this is indeed the case, we will give a sketch; the proof's construction is similar to \cite[\S 3.1, Prop.\ 3.1.4, page 90]{mackenzieGeneralTheory}; also recall the short introduction of associated bundles at the beginning of Subsection \ref{AssocLGBsSubSection}. For $x \in M$ we have the fibre $\mathcal{H}_x = P_x \times_\psi H$ which is isomorphic to $H$ as a Lie group by $H \ni h \mapsto [p, h]$ for a fixed $p \in P_x$. Thus the complete flow of an LAB element of $\mathcal{H}$ (in the fibre over $x$) is equivalent to $[p, \gamma]$ where $\gamma: \mathbb{R} \to H$, $t \mapsto \e^{tX}$, for an $X\in \mathfrak{h}$. Thus, we can already conclude that the fibre of the LAB of $\mathcal{H}$ over $x$ is isomorphic to $\mathfrak{h} = \mathcal{h}_x$ as Lie algebra. By the definition of $\mathcal{H}$ we also have
\bas
[p, \gamma]
&=
\mleft[ p \cdot g, \psi_g \circ \gamma \mright]
\eas
for all $g\in G$. Denoting $[ \cdot, \cdot ] \eqqcolon \pi$ we get infinitesimally
\bas
\mathrm{D}_{\mleft(p, e_H\mright)}\pi\mleft( p, X \mright)
&=
\mathrm{D}_{\mleft(p \cdot g, e_H\mright)}\pi\bigl( p \cdot g, \mathrm{D}_e\psi_g(X) \bigr),
\eas
where $e_H$ is the neutral element of $H$. One can trivially check that this gives an equivalence relation on $P_x \times \mathfrak{h}$, precisely the one we need for $\mathcal{h}$. It finally follows that the LAB of $\mathcal{H}$ is given by $\mathcal{h}$.

As a special example, the LAB of the inner group bundle $c_G(P)$, Ex.\ \ref{ex:InnerLGBs}, is the \textbf{adjoint bundle $\mathrm{Ad}(P)$ of $P$}, $\mathrm{Ad}(P) \coloneqq P \times_{\mathrm{Ad}} \mathfrak{g}$, where $\mathrm{Ad}$ is the adjoint representation of $G$. This is also a special example of the generalized Atiyah sequence (of both, a short exact sequence of Lie groupoids and algebroids) in \cite[\S 3.5, Def.\ 3.5.19, page 130]{mackenzieGeneralTheory}.
\end{examples}

\subsection{Vertical Maurer-Cartan form of LGBs}

As one may expect, the last result gives hints about the vertical bundle structure of $\mathcal{G}$; this can be shown with the Maurer-Cartan form on LGBs, which we will call vertical Maurer-Cartan form. It will be clear later why we choose to add this adjective; however, as a first argument recall Rem.\ \ref{rem:AbstractNotationTwoForLeftInvarVfs}, especially the last paragraph.

\begin{corollaries}{Well-definedness of the vertical Maurer-Cartan form}{VertMCVormIsWellDefined}
Let $\mathcal{G} \stackrel{\pi}{\to} M$ be an LGB over a smooth manifold $M$. Define the following map
\bas
(\mu_\mathcal{G})_g(v)
&\coloneqq
\mleft( \mathrm{D}_g L_{g^{-1}} \mright)(v)
\eas
for all $g \in \mathcal{G}$ and $v \in V_g\mathcal{G}$. Then this map is an element of $\Gamma(\mathrm{V}^*\mathcal{G} \otimes \pi^*\mathcal{g})$, where $\mathrm{V}^*\mathcal{G}$ is the dual bundle of $\mathrm{V}\mathcal{G}$.
\end{corollaries}

\begin{proof}
\leavevmode\newline
Observe that
\bas
\mathrm{D}_g L_{g^{-1}}: \mathrm{V}_g\mathcal{G} &\to \mathrm{V}_{e_x}\mathcal{G} \cong \mleft( \pi^*\mathcal{g} \mright)_g
\eas
where $x \coloneqq \pi(g)$ and $e_x$ is the neutral element of $\mathcal{G}_x$. Smoothness follows similarly to the smoothness of left-invariant vector fields, that is, denote with $\Phi$ the multiplication $\mathcal{G} * \mathcal{G} \to \mathcal{G}$ and recall the arguments and the notation in the proof of Cor.\ \ref{cor:LeftInvVfToLAB}. We have
\bas
&\mathrm{T}\mleft(\mathcal{G} * \mathcal{G}\mright)\\
&=
\left\{
	(Y, Z)
	~\middle|~
	Y \in \mathrm{T}_q \mathcal{G}, Z \in \mathrm{T}_h \mathcal{G} \text{ with } \mathrm{D}_q\pi(Y) = \mathrm{D}_h\pi(Z), \text{ where } q, h \in \mathcal{G} \text{ with } \pi(q)=\pi(h)
\right\}
\eas
and thus
\bas
\mleft( 0_{g^{-1}}, v \mright) &\in \mathrm{T}(\mathcal{G} * \mathcal{G})
\eas
where $0$ is the zero vector field, $v \in \mathrm{V}_g\mathcal{G}$ and $g \in \mathcal{G}$. Therefore we can calculate
\bas
\mathrm{D}_{\mleft(g^{-1}, g \mright)}\Phi\mleft(0_{g^{-1}}, v\mright)
&=
\mleft.\frac{\mathrm{d}}{\mathrm{d}t}\mright|_{t=0}\mleft(g^{-1} \cdot \gamma \mright)
=
\mleft.\frac{\mathrm{d}}{\mathrm{d}t}\mright|_{t=0}\mleft(L_{g^{-1}} \circ \gamma \mright)
=
\mathrm{D}_{g}L_{g^{-1}}(v)
=
\mu_{\mathcal{G}}(v),
\eas
where $\gamma: I \to \mathcal{G}_{x}$ ($I$ an open interval containing 0, and $x \coloneqq \pi(g)$) is a curve with $\gamma(0)= g$ and $\mathrm{d}/\mathrm{d}t|_{t=0} \gamma = v$. Denote with $0^{-1}$ the vector field on $\mathcal{G}$ given by $g \mapsto 0_{g^{-1}}$ and with $\iota_{0^{-1}}$ the contraction with $0^{-1}$, that is,
\bas
\mleft.\mleft(\iota_{0^{-1}}\mathrm{D}\Phi\mright)\mright|_g
&\coloneqq
\mathrm{D}_{\mleft( g^{-1}, g \mright)}\Phi\mleft( 0_{g^{-1}}, \cdot \mright)
=
\Bigl[
	\mathrm{V}_g\mathcal{G} \ni v \mapsto 
	\mathrm{D}_{\mleft( g^{-1}, g \mright)}\Phi\mleft( 0_{g^{-1}}, v \mright)
\Bigr]
=
\mu_\mathcal{G}
\eas
for all $g \in \mathcal{G}$, using the structure of $\mathrm{T}(\mathcal{G} *\mathcal{G})$. Thus, we get in total that
\bas
\mu_{\mathcal{G}}
&=
\iota_{0^{-1}}\mathrm{D}\Phi 
\in \Gamma(V^*\mathcal{G} \otimes \pi^*\mathcal{g}),
\eas
using the smoothness of all involved parts, especially that $\Phi$ is smooth, hence also $\mathrm{D}\Phi$ is smooth as an element of $\Omega^1(\mathcal{G}*\mathcal{G}; \Phi^*\mathcal{G})$.
\end{proof}

\begin{definitions}{Vertical Maurer-Cartan form of LGBs, \newline \cite[generalization of Def.\ 3.5.2, page 148]{Hamilton}}{MCFormOnLGBs}
Let $\mathcal{G} \stackrel{\pi}{\to} M$ be an LGB over a smooth manifold $M$. The map defined in Cor.\ \ref{cor:VertMCVormIsWellDefined} is the \textbf{vertical Maurer-Cartan form $\mu_\mathcal{G}$}, \textit{i.e.}\ defined to be an element of $\Gamma\mleft( \mathrm{V}^*\mathcal{G} \otimes \pi^*\mathcal{g} \mright)$ given by
\bas
\mleft(\mu_\mathcal{G}\mright)_g(v)
&\coloneqq
\mleft( \mathrm{D}_g L_{g^{-1}} \mright)(v)
\eas
for all $g \in \mathcal{G}$ and $v \in V_g\mathcal{G}$,
where $\mathrm{V}^*\mathcal{G}$ is the dual bundle of $\mathrm{V}\mathcal{G}$.
\end{definitions}

\begin{remarks}{Recovering of the classical definition}{RecoveringofClassicalMCForm}
Observe that $\mu_{\mathcal{G}}|_{\mathcal{G}_x}$ ($x \in M$) is the typical Maurer-Cartan form of $\mathcal{G}_x$, hence, $\mu_{\mathcal{G}}$ restricts to the Maurer-Cartan form of Lie groups on each fibre.

Also recall Subsection \ref{BasicNotations}, we have a 1:1 correspondence of $\mu_\mathcal{G}$ to the following commuting diagram
\begin{center}
	\begin{tikzcd}
		\mathrm{V}\mathcal{G} \arrow{d} \arrow{r}{\mu_\mathcal{G}} & \mathcal{g} \arrow{d}\\
		\mathcal{G} \arrow{r}{\pi} & M
	\end{tikzcd}
\end{center}
which is the same diagram as in \cite[\S 3.5, special situation of Prop.\ 3.5.3, page 121]{mackenzieGeneralTheory}.
\end{remarks}

We can finally finish the discussion about the vertical bundle of an LGB.

\begin{corollaries}{Vertical tangent space of $\mathcal{G}$, \newline \cite[\S 3.5, a reformulation of Prop.\ 3.5.3, page 121]{mackenzieGeneralTheory}}{TLGBAsLGB}
Let $\mathcal{G} \stackrel{\pi}{\to} M$ be an LGB over a smooth manifold $M$. Then we have an isomorphism of vector bundles
\bas
\mathrm{V}\mathcal{G} &\cong \pi^*\mathcal{g}.
\eas
\end{corollaries}

\begin{remark}
\leavevmode\newline
Observe that by Cor.\ \ref{cor:PullBackLABIsLAB} we then know that $\mathrm{V}\mathcal{G}$ admits a unique LAB structure such that $\mathrm{V}\mathcal{G} \cong \pi^*\mathcal{g}$ is an isomorphism of LABs. This statement is also not in contradiction with $\mathcal{g} = e^*\mathrm{V}\mathcal{G}$ ($e$ the identity section of $\mathcal{G}$), because
\bas
e^*\mathrm{V}\mathcal{G}
&\cong
e^*\pi^*\mathcal{g}
\cong
(\pi \circ e)^* \mathcal{g}
\cong
\mathcal{g}.
\eas
By this we also know that $\mathrm{V}\mathcal{G}$ is trivial if and only if $\mathcal{g}$ is trivial; as also argued in \cite[\S 3.5, discussion after Cor.\ 3.5.4, page 121]{mackenzieGeneralTheory}. Compare this result with $\mathrm{T}G \cong G \times \mathfrak{g}$, where $G$ is a Lie group with Lie algebra $\mathfrak{g}$. We recover this result by restricting to the Lie group fibres $\mathcal{G}_x$ ($x \in M$), that is,
\bas
\mathrm{V}\mathcal{G}|_{\mathcal{G}_x}
&=
\mathrm{T}\mathcal{G}_x
\cong
\mathcal{G}_x \times \mathcal{g}_x
=
\pi^*\mathcal{g}|_{\mathcal{G}_x}.
\eas
%For sections we therefore get
%\bas
%\Gamma(\mathrm{V}\mathcal{G})
%&\cong
%C^\infty(\mathcal{G}) \otimes_{C^\infty(M)} \Gamma(\mathcal{g}).
%\eas

Last but not least, sections of $\mathrm{V}\mathcal{G}$ are therefore generated by sections of $\mathcal{g}$, the left-invariant vector fields.
\end{remark}

\begin{proof}[Proof of Cor.\ \ref{cor:TLGBAsLGB}]
\leavevmode\newline
This can be quickly shown by recalling Rem.\ \ref{rem:RecoveringofClassicalMCForm}, that is, we have the following commuting diagram
\begin{center}
	\begin{tikzcd}
	\mathrm{V}\mathcal{G} \arrow{d} \arrow{r}{\mu_\mathcal{G}} & \mathcal{g} \arrow{d}\\
	\mathcal{G} \arrow{r}{\pi} & M
	\end{tikzcd}
\end{center}
where $\mu_\mathcal{G}$ is defined as in Def.\ \ref{def:MCFormOnLGBs}, and $\mu_{\mathcal{G}}$ restricts to the Maurer-Cartan form of $\mathcal{G}_x$ ($x \in M$) on each fibre of $\mathcal{G}$; especially, $\mu_\mathcal{G}: \mathrm{V}\mathcal{G} \to \mathcal{g}$ is a fibre-wise isomorphism (since $\mathrm{D}_gL_{g^{-1}}$ is an isomorphism). Hence, as described in Subsection \ref{BasicNotations}, $\mu_\mathcal{G}$ as an element of $\Gamma(\mathrm{V}^*\mathcal{G} \otimes \pi^*\mathcal{G})$, \textit{i.e.}\ a vector bundle morphism $\mathrm{V}\mathcal{G} \to \pi^*\mathcal{g}$ (linearity of $\mu_G$ is clear), is a vector bundle isomorphism. This finishes the proof.
\end{proof}

\subsection{Exponential map of LGBs}\label{ExponentialMapSubsection}

By Remark \ref{rem:AbstractNotationForLeftInvarianceVf} it is clear that we have a natural exponential map, just given by the fibre-wise exponential. If one is interested into a more general exponential map, then see \cite[\S 3.6, page 132ff.]{mackenzieGeneralTheory}. However, since our situation is much simpler, we quickly finish this discussion just making use of the already existing exponential map in each fibre; a straightforward generalization on results as provided in \cite[\S 1.7, page 55ff.]{Hamilton}.

\begin{definitions}{Exponential map, \newline \cite[\S 3.6, second part of Ex.\ 3.6.2, page 133f.]{mackenzieGeneralTheory}}{ExpOfLGB}
Let $\mathcal{G} \to M$ be an LGB over a smooth manifold $M$, and $\mathcal{g} \to M$ its LAB. Then we define the \textbf{exponential map $\exp: \mathcal{g} \to \mathcal{G}$} by
\bas
\exp(X) &\coloneqq \e^X \coloneqq \exp_{\mathcal{G}_x}(X)
\eas
for all $x \in M$ and $X \in \mathcal{g}_x$, where $\exp_{\mathcal{G}_x}$ is the exponential map of the Lie group $\mathcal{G}_x$ as \textit{e.g.}\ provided in \cite[\S 1.7, Def.\ 1.7.6, page 57]{Hamilton}. Its extension to sections, also denoted by $\exp: \Gamma(\mathcal{g}) \to \Gamma(\mathcal{G})$, is canonically given by
\bas
\mleft.\exp(\nu)\mright|_x
&\coloneqq
\exp(\nu_x)
\eas
for all $x \in M$ and $\nu \in \Gamma(\mathcal{g})$.

Usually, the LGB is given by context, otherwise we will denote the exponential map by $\exp_{\mathcal{G}}$ instead.
\end{definitions}

The exponential map is well-defined, especially it is smooth because it describes the flow of left-invariant vector fields as it also happens for Lie groups; see for example \cite[Prop.\ 1.7.12, page 58]{Hamilton} for the Lie group statement. Also recall Cor.\ \ref{cor:LeftInvVfToLAB}, we denote left-invariant vector fields by $X_\nu$ where $\nu \in \Gamma(\mathcal{g})$ due to the 1:1 correspondence of $L(\mathcal{G})$ and $\Gamma(\mathcal{g})$.

\begin{corollaries}{The exponential map as flow of $L(\mathcal{G})$, \newline \cite[discussion at the beginning of \S 3.6, Prop.\ 3.6.1 and its discussion afterwards; page 132f.]{mackenzieGeneralTheory}}{ExpAsFlow}
Let $\mathcal{G} \stackrel{\pi}{\to} M$ be an LGB over a smooth manifold $M$, and $\mathcal{g} \to M$ its LAB. Then the flow of a left-invariant vector field $X_\nu \in L(\mathcal{G})$ ($\nu \in \Gamma(\mathcal{g})$) is a complete flow $\phi: \mathbb{R} \times \mathcal{G} \to \mathcal{G}$ given by
\bas
\phi(t, g)
&=
g \cdot \e^{t \nu_{\pi(g)}}
\eas
for all $t \in \mathbb{R}$ and $g \in \mathcal{G}$. Especially, the map
\bas
\mathbb{R} \times M &\to \mathcal{G},\\
(t, x) &\mapsto \e^{t\nu_x}
\eas
is smooth.
%, where we understand smoothness of $\exp: \Gamma(\mathcal{g}) \to \Gamma(\mathcal{G})$ as the smoothness of 
%\bas
%M &\to \mathcal{G},\\
%g &\mapsto \e^{}
%\eas
\end{corollaries}

\begin{proof}
\leavevmode\newline
As mentioned in Remark \ref{rem:AbstractNotationForLeftInvarianceVf}, $X_\nu$ is a vertical vector field so that $\mleft.X_\nu\mright|_{\mathcal{G}_x}$ is a left-invariant vector field of the Lie group $\mathcal{G}_x$ for all $x \in M$. $\mleft.X_\nu\mright|_{\mathcal{G}_x}$ is the left-invariant vector field in $\mathcal{G}_x$ related to $\nu_x \in \mathcal{g}_x$ by Cor.\ \ref{cor:LeftInvVfToLAB}, and the flow $\phi_x$ of $\mleft.X_\nu\mright|_{\mathcal{G}_x}$ is well-known, as \textit{e.g.}\ in \cite[\S 1.7, Prop.\ 1.7.12, page 58]{Hamilton}, that is,
\bas
\mathbb{R} \times \mathcal{G}_x &\to \mathcal{G}_x,\\
(t, g) &\mapsto \phi_x(t, g) = g \cdot \exp_{\mathcal{G}_x}\mleft(t \nu_x\mright),
\eas
where $\exp_{\mathcal{G}_x}$ is the exponential map of $\mathcal{G}_x$. Since this works for all fibres $\mathcal{G}_x$ we get by Def.\ \ref{def:ExpOfLGB} that the flow $\phi$ of $X_\nu$ is complete and given by
\bas
\phi(t, g)
&=
g \cdot \e^{t\nu_{\pi(g)}}
\eas
for all $t \in \mathbb{R}$ and $g \in \mathcal{G}$. We also have
\bas
\e^{t \nu_x}
&=
e_x \cdot \e^{t \nu_{\pi\mleft(e_x\mright)}}
=
\phi\mleft(t, e_x\mright)
\eas
for all $t \in \mathbb{R}$ and $x \in M$, such that smoothness of $(t, x) \mapsto \e^{t\nu_x}$ follows as composition of smooth maps.
\end{proof}

\begin{remarks}{Simplifying notation related to the exponential map}{SimplyExpNotation}
Due to Def.\ \ref{def:ExpOfLGB} we recover a lot of the typical properties of the exponential map as in \cite[\S 1.7, Prop.\ 1.7.9, page 57]{Hamilton}, if we understand these properties point-wise. Recall Remark \ref{rem:SectionMultiplication}, then for example
\bas
\e^{(t + s) \nu}
&=
\e^{t\nu} \cdot \e^{s\nu}
\eas
for all $t, s \in \mathbb{R}$ and $\nu \in \Gamma(\mathcal{g})$. As in \cite[\S 3.6, discussion after Prop.\ 3.6.1, page 133]{mackenzieGeneralTheory}, we say that $\mathbb{R} \ni t \mapsto \e^{t\nu}$ is smooth in the sense of the second statement of Cor.\ \ref{cor:ExpAsFlow}, and by construction we have
\bas
\nu_x
&=
\mleft.\frac{\mathrm{d}}{\mathrm{d}t}\mright|_{t=0} \e^{t\nu_x}
\coloneqq
\mleft.\frac{\mathrm{d}}{\mathrm{d}t}\mright|_{t=0} \mleft[ t \mapsto \e^{t\nu_x} \mright]
\eas
for all $x\in M$, so that we also write
\bas
\nu
&=
\mleft.\frac{\mathrm{d}}{\mathrm{d}t}\mright|_{t=0} \e^{t\nu}.
\eas

Since all of this is rather natural, we will make use of that without further mention.
\end{remarks}

This discussion also highlights that one could understand the infinite-dimensional Lie algebra $\Gamma(\mathcal{g})$ as the Lie algebra of the infinite-dimensional Lie group $\Gamma(\mathcal{G})$; thus, one could construct the LABs of LGBs by starting in that fashion, and then $\mathcal{g}$ is constructed by making use of the 1:1 correspondence of vector bundles and locally free sheaf of modules of constant rank.

\subsection{LABs of pullback LGBs}

We are going to define LGB representations and corresponding LAB representations. Since group representation are a special form of actions, we will have something similar in the case of LGB representations. Since actions are defined as maps on a pullback of an LGB $\mathcal{G}$, which is also an LGB by Cor.\ \ref{cor:PullbackLGB}, we expect that the corresponding LAB representation is related to the LAB of the pullback of $\mathcal{G}$. It is natural to think of this LAB as the pullback of $\mathcal{g}$, which is also an LAB by Cor.\ \ref{cor:PullBackLABIsLAB}:

\begin{corollaries}{LAB of pullback LGB is pullback LAB, \newline \cite[\S 3, Thm.\ 3.5, page 21]{PullbackLGBLAB}}{LABOfPullbackLGBIsPullbackLAB}
Let $\mathcal{G} \to M$ be an LGB over a smooth manifold $M$, and let $f: N \to M$ be a smooth map defined on another smooth manifold $N$. Then the LAB of $f^*\mathcal{G}$ is isomorphic to the pullback LAB $f^*\mathcal{g}$.
\end{corollaries}

\begin{proof}
\leavevmode\newline
By Cor.\ \ref{cor:PullbackLGB} we know that $\pi_2: f^*\mathcal{G} \to \mathcal{G}$, the projection onto the second factor, is an LGB morphism over $f$,
\begin{center}
	\begin{tikzcd}
		f^*\mathcal{G} \arrow{d} \arrow{r}{\pi_2} & \mathcal{G} \arrow{d}\\
		N \arrow{r}{f} & M
	\end{tikzcd}
\end{center}
and it is fibre-wise a Lie group isomorphism such that
\bas
\mathrm{D}_{(p,e_x)}\pi_2: \mathcal{h}_p &\to \mathcal{g}_x
\eas
is a Lie algebra isomorphism for all $p \in N$, where $e_x$ is the neutral element of $\mathcal{G}_x$ for $x \coloneqq f(p)$, and $\mathcal{h}$ and $\mathcal{g}$ are the LABs of $f^*\mathcal{G}$ and $\mathcal{G}$, respectively; for all of that recall that $\mleft(f^*\mathcal{G}\mright)_p$ and $\mathcal{G}_x$ are Lie groups. Hence, we have a vector bundle morphism over $f$ given by the following commuting diagram
\begin{center}
	\begin{tikzcd}[column sep=huge]
		\mathcal{h} \arrow{d} \arrow{r}{\mathrm{D}_{\mleft( \mathds{1}_N, e_f \mright)} \pi_2} & \mathcal{g} \arrow{d} \\
		N \arrow{r}{f} & M
	\end{tikzcd}
\end{center}
which describes fibre-wise a Lie algebra isomorphism, where 
\bas
\mathrm{D}_{\mleft( \mathds{1}_N, e_f \mright)} \pi_2
&\coloneqq
\mathrm{D}\pi_2 \circ \mleft( \mathds{1}_N, e_f \mright)
=
\mleft[
	N \ni p \mapsto \mathrm{D}_{\mleft( p, e_{f(p)} \mright)} \pi_2
\mright].
\eas
By our notes in Subsection \ref{BasicNotations}, we therefore achieve an LAB isomorphism $\mathcal{h} \to f^*\mathcal{g}$.
\end{proof}

\begin{remarks}{LAB of $f^*\mathcal{G}$}{LABofPullBackNotation}
Since this isomorphism is very natural, we always use that identification and will refer to $f^*\mathcal{g}$ as \textit{the} LAB of $f^*\mathcal{G}$.
\end{remarks}

With this we can quickly show the following familiar result.

\begin{corollaries}{Differentials of LGB morphisms are LAB morphisms, \newline \cite[\S 3.5, section about morphisms, page 124f.]{mackenzieGeneralTheory}}{LGBToLABHomomorphis}
Let $\mathcal{H} \to N$ and $\mathcal{G} \to M$ be two LGBs over two smooth manifolds $N$ and $M$, and we denote with $\mathcal{h}$ and $\mathcal{g}$ the LABs of $\mathcal{H}$ and $\mathcal{G}$, respectively. Furthermore, assume that we have an LGB morphism $F: \mathcal{H} \to \mathcal{G}$ over a smooth map $f: N \to M$. Then
\bas
\mleft.\mathrm{D} F\mright|_{\mathcal{h}}: \mathcal{h} &\to \mathcal{g}
\eas
is an LAB morphism over $f$.
%, where $e$ is the identity section of $\mathcal{H}$, that is,
%\bas
%\mathrm{D}_eF|_p
%&\coloneqq
%\mleft.\bigl(\mathrm{D}F|_{e_N}\bigr)\mright|_p
%=
%\mathrm{D}_{e_p}F: \mathcal{h}_p \to \mathcal{g}_{f(p)}
%\eas
%for all $p \in N$.
\end{corollaries}

\begin{proof}
\leavevmode\newline
Again by our notes in Subsection \ref{BasicNotations}, we can view $F$ as a base-preserving LGB morphism $F: \mathcal{H} \to f^*\mathcal{G}$, since $f^*\mathcal{G}$ is an LGB whose structure is naturally inherited by $\mathcal{G}$ as given in Cor.\ \ref{cor:PullbackLGB}; similarly for its LAB by Cor.\ \ref{cor:PullBackLABIsLAB}. Thus, it is fibre-wise a Lie group morphism, and its tangent map restricted to $\mathcal{h} = e^*\mathrm{V}\mathcal{H}$ ($e$ the identity section of $\mathcal{H}$) gives therefore fibre-wise a Lie algebra morphism. Thus, $\mleft.\mathrm{D} F\mright|_{\mathcal{h}}: \mathcal{h} \to f^*\mathcal{g}$ is an LAB morphism (using Cor.\ \ref{cor:LABOfPullbackLGBIsPullbackLAB}), and can be seen as an LAB morphism $\mathcal{h} \to \mathcal{g}$ over $f$. Alternatively, it is straightforward and trivial to show it directly.
\end{proof}

\section{LGB actions, part II}\label{LGBActionIISection}

Finally, we come to the last part of the \textit{basics} for LGBs and their notions needed.

\subsection{LGB and LAB representations}

As usual, representations are a special type of group action, with an infinitesimal analogue. 
%The approach in \cite[\S 1.7, page 43ff.]{mackenzieGeneralTheory} defines an LGB representation by a map which restricts to a typical representation on each fibre of the LGB. However, in alignment with the previous approach we want to keep the fibre-wise behaviour as a remark while the definition itself emphasizes the global structure. For this it will now be important that we have shown that the pullback of an LGB is again an LGB; recall Cor.\ \ref{cor:PullbackLGB}.

\begin{definitions}{LGB representations, \newline\cite[\S 1.7, special situation of the remark before Def.\ 1.7.1, page 43]{mackenzieGeneralTheory}}{LGBRep}
Let $\mathcal{G} \stackrel{\pi}{\to} M$ be an LGB over a smooth manifold $M$, $V \stackrel{p}{\to} M$ be a vector bundle, and assume that we have a left $\mathcal{G}$-action on $V$, $\Psi: \mathcal{G}*V \coloneqq p^*\mathcal{G}\to V$. Then we say that $\Psi$ is a \textbf{$\mathcal{G}$-representation on $V$} if it is linear, that is,
\bas
\Psi(g, \alpha v )
&=
\alpha \Psi(g, v),\\
\Psi(g, v + w)
&=
\Psi(g, v) + \Psi(g, w )
\eas
for all $\alpha \in \mathbb{R}$, and $(g, v), (g, w) \in \mathcal{G}*V$. In alignment with previous notations we may also write $\Psi(g, v) = \Psi_g(v)$, or $\Psi(g, v) = g \cdot v$.
\end{definitions}

\begin{remark}
\leavevmode\newline
Observe that $(g, v) \in \mathcal{G}*V$ means that $p(v) = \pi(g)$, same for $(g, w)$. Hence, given a base point in $x\in M$, the pairs $(g, v)$ in $\mathcal{G}*V|_{p^{-1}(\{x\})}$ are given by elements $g \in \mathcal{G}_x$ and $v \in V_x$. By Def.\ \ref{def:LiegroupACtion} we also have 
\bas
p\bigl(\Psi(g, v)\bigr)
&=
\pi(g) = p(v) = x.
\eas
Thence, linearity of $\Psi$ is well-defined. In fact, observe that $\mathcal{G} * V = p^*\mathcal{G} \cong \pi^*V$ as fibre bundle, therefore $\mathcal{G} * V$ carries not only the structure of an LGB but also of a vector bundle. That is, we have the following commuting diagram
\begin{center}
	\begin{tikzcd}
	\mathcal{G} * V \arrow{r}{\mathrm{pr}_1} \arrow{d}{\mathrm{pr}_2} & \mathcal{G} \arrow{d}{\pi}\\
	V \arrow{r}{p} & M
	\end{tikzcd}
\end{center}
the horizontal arrows describe the vector bundle structure (viewing $\mathcal{G} * V$ as the vector bundle $\pi^*V$), and the vertical ones the LGB structure (viewing $\mathcal{G} * V$ as the LGB $p^*\mathcal{G}$), where $\mathrm{pr}_i$ ($i \in \{1,2\}$) are the projections onto the $i$-th component.
\end{remark}

By fixing a base point $x \in M$ we clearly have the typical notion of a $\mathcal{G}_x$-representation on $V_x$. Equivalently, we could therefore obviously define LGB representations as a base-preserving LGB morphism $\Psi: \mathcal{G} \to \mathrm{Aut}(V)$ as also in \cite[\S 1.7, Def.\ 1.7.1, page 43]{mackenzieGeneralTheory}.

\begin{corollaries}{LGB representations as LGB morphisms, \newline \cite[\S 1.7, Prop.\ 1.7.2, page 43]{mackenzieGeneralTheory}}{LGBRepAsLGBMorph}
Let $\mathcal{G} \to M$ be an LGB over a smooth manifold $M$, $V \to M$ be a vector bundle. Then every $\mathcal{G}$-representation $\Psi: \mathcal{G} * V \to V$ is equivalent to a base-preserving LGB morphism $\widetilde{\Psi}: \mathcal{G} \to \mathrm{Aut}(V)$, related by
\bas
\Psi(g, v)
&=
\widetilde{\Psi}(g)(v)
\eas
for all $g \in \mathcal{G}_x$ ($x \in M$) and $v \in V_x$.
\end{corollaries}

\begin{remark}
\leavevmode\newline
We will usually denote both interpretations with the same notation.
\end{remark}

\begin{proof}[Proof of Cor.\ \ref{cor:LGBRepAsLGBMorph}]
\leavevmode\newline
This is an immediate consequence of Def.\ \ref{def:LGBRep} and \ref{def:LiegroupACtion}. $\Psi$ is smooth if and only if $\widetilde{\Psi}$ is smooth; this is due to the fact that the LGB atlas of $\mathrm{Aut}(V)$ is inherited by an atlas of $V$ as constructed in Ex.\ \ref{ex:AutOfVectorBundleAnLGB}, and due to that $\mathcal{G} * V$ is an embedded submanifold of $\mathcal{G} \times V$.
\end{proof}

\begin{examples}{Recovering Lie group representations on vector spaces}{LGRepsOnVecs}
Similarly as to Ex.\ \ref{ex:TrivialLGBAction}, if $\mathcal{G} = M \times G$ is a trivial LGB over $M$, $G$ its structural Lie group, and $V = M \times W$ a trivial vector bundle with structural vector space $W$, then $\Psi$ is equivalent to a $G$-representation on $W$. Backwards, every Lie group representation gives a representation of trivial LGBs on trivial vector bundles.
\end{examples}

Also recall Ex.\ \ref{ex:AutOfVectorBundleAnLGB}: We can similarly construct another LGB needed for the adjoint representation.

\begin{examples}{Another LGB example: Automorphisms of LABs, \newline \cite[\S 1.7, special situation of Ex.\ 1.7.12, page 46]{mackenzieGeneralTheory}}{AutosOfLABsAsLGB}
Let $\mathcal{g}$ be an LAB over the smooth manifold $M$. We denote with $\mathrm{Aut}(\mathcal{g}) \to M$ the bundle of fibre-wise Lie algebra automorphisms of $\mathcal{g}$, where the sections of $\mathrm{Aut}(\mathcal{g})$ are the base-preserving LAB automorphisms of $\mathcal{g}$. As in Ex.\ \ref{ex:AutOfVectorBundleAnLGB} one can show $\mathrm{Aut}(\mathcal{g})$ is an LGB by using LAB trivializations of $\mathcal{g}$ instead of vector bundle trivializations.

The notation of $\mathrm{Aut}(\mathcal{g})$ may be confusing with the notation as for vector bundles. We usually refer to this LAB automorphism when speaking of LABs and state it explicitly if we just mean the vector bundle version.
\end{examples}

For the next major example of an LGB representation recall Def.\ \ref{def:LeftRightTranslationConjugation}.
 %and Cor.\ \ref{cor:LGBToLABHomomorphis}. 

\begin{examples}{Adjoint LGB representation, \newline \cite[\S 3.5, special situation of Prop.\ 3.5.20, page 131]{mackenzieGeneralTheory}}{LGBAdjointRep}
For $g \in \mathcal{G}_x$ ($x \in M$) we have the conjugation $c_g$ which is a $\mathcal{G}_x$-automorphism. In the usual way we define the \textbf{adjoint representation of $\mathcal{G}$} as a base-preserving LGB morphism $\mathcal{G} \to \mathrm{Aut}(\mathcal{g})$ by
\bas
\mathrm{Ad}_g &\coloneqq \mathrm{D}_{e_x} c_g
\eas
for all $g \in \mathcal{G}_x$ ($x \in M$). 

That this is a $\mathcal{G}$-representation on $\mathcal{g}$ follows quickly: By construction $\mathrm{Ad}_g \in \mathrm{Aut}(\mathcal{g}_x) = \mleft.\mathrm{Aut}(\mathcal{g})\mright|_x$ so that the adjoint representation is clearly well-defined. As in the Lie group case, due to the fact that $c_{gq} = c_{g} \circ c_q$ for all $g, q\in \mathcal{G}_x$ and $c_g(e_x) = e_x$ we have
\bas
\mathrm{Ad}_{gq}
&=
\mathrm{D}_{e_x}\mleft( c_g \circ c_q \mright)
=
\mathrm{D}_{e_x} c_g \circ \mathrm{D}_{e_x} c_q
=
\mathrm{Ad}_{g} \circ \mathrm{Ad}_{q}.
\eas
Smoothness could either be shown in a bit more generalized way than the proof in \cite[\S 2.1, Thm.\ 2.1.45, page 101f.]{Hamilton}, but that would be a bit tedious; instead let us use the already-known smoothness of the adjoint representation in each fibre. Fix an open subset $U$ of $M$ such that $\mathcal{G}$ and $\mathcal{g}$ are trivial, that is, $\mathcal{G}|_U \cong U \times G$ and $\mathcal{g}|_U \cong U \times \mathfrak{g}$ as LGB and LAB, respectively, where $G$ is the structural Lie group with its Lie algebra $\mathfrak{g}$. Then also clearly $\mathrm{Aut}(\mathcal{g})|_U \cong U \times \mathrm{Aut}(\mathfrak{g})$ as LGBs. By construction we then have w.r.t.\ these trivializations
\bas
\mathrm{Ad}_{(x, g)} 
&= 
\Bigl( x, \mathrm{D}_{e_G} c^G_g \Bigr)
=
\Bigl( x, \mathrm{Ad}^G_g \Bigr)
\eas
for all $(x, g) \in U \times G$, where $\mathrm{Ad}^G_g: G \to \mathrm{Aut}(\mathfrak{g})$ is the adjoint representation of $G$ on $\mathfrak{g}$. Smoothness now follows trivially by the canonical manifold structure of product manifolds and the smoothness of $\mathrm{Ad}^G$.
%\eas
%However, smoothness has to be shown in a bit more generalized way than the proof in \cite[\S 2.1, Thm.\ 2.1.45, page 101f.]{Hamilton}, that is, we equivalently view $\mathrm{Ad}$ as a linear left action $\mathcal{G} * \mathcal{g} \to \mathcal{g}$ as in Def.\ \ref{def:LGBRep} given by
%\bas
%\mathrm{Ad}(g,\nu) &= \mathrm{Ad}_g(\nu)= \mathrm{D}_{e_x}c_g(\nu)
%\eas
%for all $g \in \mathcal{G}_x$ and $\nu \in \mathcal{g}_x$. Similarly we can view the conjugation as a smooth map\footnote{In fact it is obviously a smooth left $\mathcal{G}$-action on itself, and one could generalize Def.\ \ref{def:LieGroupActingOnLieGroup} and derive a similar statement as Cor.\ \ref{cor:LGBRepAsLGBMorph} related to the LGB of fibre-wise $\mathcal{G}$-automorphisms.}
%\bas
%\mathcal{G}*\mathcal{G} &\to \mathcal{G},\\
%(g,h) &\mapsto c(g, h) \coloneqq c_g(h) = g h g^{-1},
%\eas
%making use of Ex.\ \ref{ex:LGBActingOnItself} to derive smoothness as composition of smooth maps.
%Observe that $(0_g, \nu) \in \mathrm{T}_{(g, e)}(\mathcal{G} * \mathcal{G})$ because $\nu \in \mathcal{g}_x \cong \mathrm{V}_{e_x}\mathcal{G}$ is vertical, where $0_g$ is the zero vector of $\mathrm{T}_g\mathcal{G}$; see also Lemma \ref{lem:PullbackFibreBundleItsTangentSp} later for a general statement about bundles like $\mathrm{T}(\mathcal{G} * \mathcal{G})$, or alternatively recall the proof of Cor.\ \ref{cor:VertMCVormIsWellDefined}. Especially, take a curve $\gamma: I \to \mathcal{G}_x$ ($I$ an open interval of $\mathbb{R}$ containing 0) with $\gamma(0) = e_x$ and $\mathrm{d}/\mathrm{d}t|_{t=0}\gamma = \nu$, then
%\bas
%\mathrm{D}_{(g,e_x)}c(0_g, \nu)
%&=
%\mleft.\frac{\mathrm{d}}{\mathrm{d}t}\mright|_{t=0} c\mleft( g, \gamma \mright)
%=
%\mleft.\frac{\mathrm{d}}{\mathrm{d}t}\mright|_{t=0} c_g\mleft( \gamma \mright)
%=
%\mathrm{D}_{e_x}c_g(\nu)
%=
%\mathrm{Ad}(g, \nu).
%\eas
%Thus, $\mathrm{Ad}$ is the following composition of maps
%\ba\label{AdjointAsCompOfSmoothMaps}
%\mathcal{G} * \mathcal{g} &\to \mathrm{T}(\mathcal{G} * \mathcal{G}) \to \mathcal{g},\nonumber\\
%\Bigl(g, \mleft(e_{\pi(\nu)}, \nu\mright) \Bigr) &\mapsto \Bigl(\mleft(g, e_{\pi(\nu)}\mright), (0_g, \nu)\Bigr) \mapsto \mathrm{D}_{\mleft(g,e_{\pi(\nu)}\mright)}c\mleft(0_g, \nu\mright),
%\ea
%where $\pi$ denotes the projection of $\mathcal{g} \to M$, and for the first arrow we were carefully denoting the base points in the first component of tangent vectors. The second arrow is smooth because $c$ is smooth. The first arrow is also smooth: The map
%\ba\label{FirstArrowOfAdAsCompOfSmoothmaps}
%\mathcal{G} \times \mathcal{g} &\to \mathrm{T}(\mathcal{G} \times \mathcal{G}),\nonumber\\
%\Bigl(g, \mleft(e_{\pi(\nu)}, \nu\mright) \Bigr) &\mapsto \Bigl(\mleft(g, e_{\pi(\nu)}\mright), (0_g, \nu)\Bigr),
%\ea
%is clearly smooth, and so also its restriction onto $\mathcal{G} * \mathcal{g} = \pi^*\mathcal{G}$ as embedded submanifold of $\mathcal{G} \times \mathcal{g}$. Similarly, $\mathcal{G} * \mathcal{G}$ is an embedded submanifold of $\mathcal{G} \times \mathcal{G}$, and it is therefore well-known by using submanifold charts that then $\mathrm{T}(\mathcal{G} * \mathcal{G})$ is a submanifold of $\mathrm{T}(\mathcal{G} \times \mathcal{G})|_{\mathcal{G} * \mathcal{G}}$, which is itself a submanifold of $\mathrm{T}(\mathcal{G} \times \mathcal{G})$ (see \textit{e.g.}\ \cite[\S 4.1, Lemma 4.1.16, page 204]{Hamilton}). Thence, the restriction of the domain of definition and image of \eqref{FirstArrowOfAdAsCompOfSmoothmaps} is smooth, and this is precisely the first arrow of \eqref{AdjointAsCompOfSmoothMaps}. Finally it follows that $\mathrm{Ad}$ is smooth and therefore a $\mathcal{G}$-representation on $\mathcal{g}$.
\end{examples}

Infinitesimally, we have LAB actions and representations; recall Cor.\ \ref{cor:PullBackLABIsLAB}.

\begin{definitions}{LAB actions, \newline \cite[\S 4.1, reformulated version for LABs of Def.\ 4.1.1, page 149]{mackenzieGeneralTheory}}{LABACtions}
Let $M, N$ be smooth manifolds, $\mathcal{g} \stackrel{\pi}{\to} M$ an LAB, and $f: N \to M$ a smooth map. Then a \textbf{$\mathcal{g}$-action on $N$} is a base-preserving vector bundle morphism 
\bas
f^*\mathcal{g} &\to \mathrm{T}N,\\
\nu &\mapsto \rho(\nu),
\eas
satisfying
\ba\label{LABActionAlongFibres}
\mathrm{D}f \circ \rho
&=
0
\ea
and such that the induced map
\bas
\Gamma(\mathcal{g}) &\to \Gamma(f^*\mathcal{g}) \to \mathfrak{X}(N),\\
\mu &\mapsto f^*\mu \mapsto \rho\mleft( f^*\mu \mright)
\eas
is a homomorphism of Lie algebras, that is,
\ba\label{ActionLieAlgebroidButNonTrivial}
\rho\biggl( f^*\mleft(\mleft[ \mu, \nu \mright]_{\mathcal{g}}\mright) \biggr)
&=
\bigl[ \rho\mleft(f^*\mu\mright), \rho\mleft(f^*\nu\mright) \bigr]
\ea
for all $\mu, \eta \in \Gamma(\mathcal{g})$.
\end{definitions}

\begin{remark}
\leavevmode\newline
For the readers familiar with Lie algebroids, especially action Lie algebroid structures on trivial LABs, Eq.\ \eqref{ActionLieAlgebroidButNonTrivial} should look familiar; in fact, as shown in \cite[\S 4.1, Prop.\ 4.1.2, page 149f.]{mackenzieGeneralTheory}, given an LAB action one can construct a Lie algebroid structure on $f^*\mathcal{g}$. This leads to an action Lie algebroid structure on a possibly non-trivial bundle.
\end{remark}

As for Lie group and algebra actions, an LGB action induces an LAB action.

\begin{lemmata}{LGB actions induce LAB actions, \newline \cite[\S 4.1, special situation of Thm.\ 4.1.6, page 152]{mackenzieGeneralTheory}}{LGBInduceLABAction}
Let $M, N$ be smooth manifolds, $\mathcal{G} \stackrel{\pi}{\to} M$ an LGB over $M$ and $f: N \to M$ a smooth map. Then any right $\mathcal{G}$-action $\Phi: N * \mathcal{G} = f^*\mathcal{G} \to N$ on $N$ induces a $\mathcal{g}$-action $\rho$ on $N$ by 
\bas
\rho
&\coloneqq
\mleft.\mathrm{D}\Phi\mright|_{f^*\mathcal{g}},
\eas
\textit{i.e.}\
\bas
\rho(\eta)
&=
\mathrm{D}_{\mleft(p, e_{f(p)}\mright)}\Phi(\eta)
\eas
for all $\eta \in (f^*\mathcal{g})_p$ ($p\in N$).
\end{lemmata}

\begin{remark}\label{LeftActionsAndTheirSignProblem}
\leavevmode\newline
Similarly to the discussion as in \cite[\S 3.4, page 141ff.]{Hamilton}, one can show the same for left actions, but one has to use the multiplication with the inverse, that is, 
\bas
\rho
&\coloneqq
\mleft.\mathrm{D}\Phi^\prime\mright|_{f^*\mathcal{g}},
\eas
where
\bas
\Phi^\prime: \mathcal{G} * N &\to N,\\
(g, p) &\mapsto g^{-1} \cdot p.
\eas
\end{remark}

\begin{proof}[Proof of Lemma \ref{lem:LGBInduceLABAction}]
\leavevmode\newline
We generalize the proof as provided in \cite[\S 3.4, proof of Prop.\ 3.4.4, page 144f.]{Hamilton}. Observe that we have
\bas
\rho(p, \nu)
&=
\mathrm{D}_{\mleft( p, e_{f(p)} \mright)}\Phi(p,\nu)
\in
\mathrm{T}_p N
\eas
for all $(p, \nu) \in f^*\mathcal{g}$, and thus describes a base-preserving vector bundle morphism $f^*\mathcal{g} \to \mathrm{T}N$. We can rewrite Eq.\ \eqref{InvarianceOffUnderGAction} to
\bas
f \circ \Phi
&=
\pi \circ \mathrm{pr}_2^{f^*\mathcal{G}},
\eas
where $\mathrm{pr}_2^{f^*\mathcal{G}}$ is the projection onto the second factor in $f^*\mathcal{G}\subset N \times \mathcal{G}$. Thence, we have
\bas
\mathrm{D}f \circ \rho
&=
\mathrm{D}f \circ \mleft.\mathrm{D}\Phi\mright|_{f^*\mathcal{g}}
=
\mleft.\mathrm{D}(f \circ \Phi)\mright|_{f^*\mathcal{g}}
=
\mleft.\mleft(\mathrm{D}\pi \circ \mathrm{Dpr}_2^{f^*\mathcal{G}}\mright)\mright|_{f^*\mathcal{g}}
=
\mleft.\mathrm{D}\pi\mright|_{\mathcal{g}} \circ \mathrm{pr}_2^{f^*\mathcal{g}}
= 0,
\eas
making use of that $\mathcal{g}$ consists of vertical vectors, and where $\mathrm{pr}_2^{f^*\mathcal{g}}$ is the projection onto the second factor in $f^*\mathcal{g} \subset N \times \mathcal{g}$.

The proof of Eq.\ \eqref{ActionLieAlgebroidButNonTrivial} is as straightforward as usual, as a quick argument for the experienced reader observe that Eq.\ \eqref{ActionNeutralElement} and \eqref{ActionAssociative} imply that $\Phi$ induces a Lie group homomorphism $\Gamma(\mathcal{G}) \to \mathrm{Diff}(M)$ ($\mathrm{Diff}$ the group of diffeomorphisms) so that we have the desired Lie algebra homomorphism on an infinitesimal scale.\footnote{We use the notion of sections to avoid the possible lack of smooth structure on the preimages of $f$.} Let $\mu, \nu \in \Gamma(\mathcal{g})$, then we know that 
\bas
\bigl[ \mathrm{D}\Phi(f^*\mu), \mathrm{D}\Phi(f^*\nu) \bigr]
&=
\mathrm{D}\Phi\mleft(
	\mleft[ f^*\mu, f^*\nu \mright]_{f^*\mathcal{g}}
\mright),
\eas
if we can show that $\mathrm{D}\Phi(f^*\mu)$ and $\mathrm{D}\Phi(f^*\nu)$ are $\Phi$-related to $f^*\mu$ and $f^*\nu$, respectively; this is a common procedure, see for example \cite[\S A.1, Prop.\ A.1.49, page 615]{Hamilton}. That is, we need to show now that
\ba\label{PhiRelatedSections}
\mleft.\mathrm{D}\Phi(f^*\mu)\mright|_{\Phi(p, g) = p \cdot g}
&\stackrel{!}{=}
\mathrm{D}_{(p, g)}\Phi\mleft( \mleft.\mleft(f^*\mu\mright)\mright|_{(p,g)} \mright)
\ea
for all $(p, g) \in N*\mathcal{G}$, similarly for $\nu$; here we understand the LABs of LGBs as left-invariant vector fields as in Cor.\ \ref{cor:LeftInvVfToLAB}. Observe that we have in general for $\eta \in \Gamma(f^*\mathcal{g}) \cong L(f^*\mathcal{G})$, using the definition of $\rho$,
%($N$ canonically embedded into $f^*\mathcal{G}$ via the neutral section)
\bas
\mleft.\mathrm{D}\phi(\eta)\mright|_{p\cdot g}
&=
\mathrm{D}_{(p\cdot g, e_x)}\Phi\mleft( \eta_{(p \cdot g, e_x)} \mright)
=
\mleft.\frac{\mathrm{d}}{\mathrm{d}t}\mright|_{t=0}\Phi \mleft( p \cdot g, \e^{t \xi_{p \cdot g}} \mright)
=
\mleft.\frac{\mathrm{d}}{\mathrm{d}t}\mright|_{t=0}\mleft( p \cdot g \e^{t \xi_{p \cdot g}} \mright)
\eas
for all $(p, g) \in N* \mathcal{G}$,
where $t \in \mathbb{R}$ and we write in general $\eta_{(p, e_x)} = \mleft(p, \xi_{p}\mright)$ with $\xi_{p} \in \mathcal{g}_x$ ($x \coloneqq f(p)$), and the exponential $\e$ is the one of $\mathcal{G}$. But we also have similarly
\bas
\mathrm{D}_{(p,g)}\Phi\bigl(\underbrace{\eta_{(p, g)}}_{\mathclap{ = \mathrm{D}_{(p, e_x)}L_{(p, g)}\mleft( \eta_{(p, e_x)} \mright) }} \bigr)
&=
\mathrm{D}_{(p, e_x)}\mleft( \Phi \circ L_{(p, g)} \mright)\mleft( p, \xi_p \mright)
=
\mleft.\frac{\mathrm{d}}{\mathrm{d}t}\mright|_{t=0} \Phi\mleft(
	p, g \e^{t \xi_p}
\mright)
=
\mleft.\frac{\mathrm{d}}{\mathrm{d}t}\mright|_{t=0} \mleft(
	p \cdot g \e^{t \xi_p}
\mright),
\eas
where $L$ is the left-multiplication in $f^*\mathcal{G}$. So, in order to achieve Eq.\ \eqref{PhiRelatedSections} we now set $\eta = f^*\mu$. That is,
$
\eta_{(p, e_x)}
=
\mleft( p, \mu_{f(p)} \mright),
$
therefore $\xi_p = \mu_{f(p)}$,
and due to Eq.\ \eqref{InvarianceOffUnderGAction} and Rem.\ \ref{rem:ActionAndPullbackLGBs} we derive $\xi_p = \xi_{p\cdot g}$. In total, we see that Eq.\ \eqref{PhiRelatedSections} is satisfied.

Finally, as argued earlier, we can prove
\bas
\bigl[ \mathrm{D}\Phi(f^*\mu), \mathrm{D}\Phi(f^*\nu) \bigr]
&=
\mathrm{D}\Phi\mleft(
	\mleft[ f^*\mu, f^*\nu \mright]_{f^*\mathcal{g}}
\mright)
=
\mathrm{D}\Phi\mleft( f^*\mleft(
	\mleft[ \mu, \nu \mright]_{\mathcal{g}}
\mright) \mright),
\eas
making use of that the field of Lie brackets of $f^*\mathcal{g}$ is the $f$-pullback of $\mleft[ \cdot, \cdot \mright]_{\mathcal{g}}$ as a section. This finishes the proof.
\end{proof}

\begin{remarks}{Variants of the LAB action as Lie algebra homomorphism}{LABHomomorpInActionVariants}
It is clear that one has a local version of Eq.\ \eqref{ActionLieAlgebroidButNonTrivial}. In fact, as one sees in the proof of Lemma \ref{lem:LGBInduceLABAction}, the argument for Eq.\ \eqref{ActionLieAlgebroidButNonTrivial} works pointwise in $M$, so, one may also think of a homomorphism of Lie algebras
\bas
\mathcal{g}_{x} &\to \mathfrak{X}\mleft( f^{-1}(\{x\}) \mright),\\
\nu &\mapsto \mleft[ f^{-1}(\{x\}) \ni p \mapsto \rho(p, \nu) \mright]
\eas
for all $x \in M$, where we also used Eq.\ \eqref{LABActionAlongFibres}; that is, we expect that an LAB action is fibre-wise a Lie algebra action. Due to a possible lack of manifold structure on $f^{-1}(\{x\})$ we did not restrict to $x\in M$ to avoid technical difficulties. However, $f$ will be the projection of a bundle later, and in that case the fibre $f^{-1}(\{x\})$ is an embedded submanifold of $N$. As argued in Remark \ref{rem:LocalLGBAction}, the $\mathcal{G}$-action is a $\mathcal{G}_x$-action restricted on $f^{-1}(\{x\})$. By what we know about Lie group actions we immediately know that we have the aforementioned Lie algebra homomorphism, and so also Eq.\ \eqref{ActionLieAlgebroidButNonTrivial}, in total a vector bundle morphism $f^*\mathcal{g} \to \mathrm{T}N$ which gives fibre-wise over $x$ rise to a Lie algebra action.
\end{remarks}

An important example will be fundamental vector fields which we will introduce later. Another example are LAB representations induced by LGB representations; hence let us introduce LAB representations. We will introduce them in a reverted order than the LGB representations, that is, first defining them as certain LAB morphisms, and then trivially concluding a relation to LAB actions; for the following recall Ex.\ \ref{ex:EndVAnLAB}.

\begin{definitions}{LAB representations, \cite[\S 3.3, Def.\ 3.3.13, page 107]{mackenzieGeneralTheory}}{LABReps}
Let $\mathcal{g} \to M$ be an LAB over a smooth manifold $M$, and $V \to M$ be a vector bundle. A \textbf{$\mathcal{g}$-representation on $V$} is an LAB morphism $\psi: \mathcal{g} \to \mathrm{End}(V)$.
\end{definitions}

\begin{corollaries}{LAB representations are specific LAB actions, \newline \cite[\S 4.1, special consequence of Prop.\ 4.1.7 but we do not assume integrability of the LAB, page 153]{mackenzieGeneralTheory}}{LABRepsAreLABActions}
Let $\mathcal{g} \to M$ be an LAB over a smooth manifold $M$, $V \stackrel{p}{\to} M$ be a vector bundle, and $\psi: \mathcal{g} \to \mathrm{End}(V)$ a $\mathcal{g}$-representation on $V$. Then $\psi$ defines an LAB action $\widetilde{\psi}: p^*\mathcal{g} \to \mathrm{T}V$ by
\bas
\widetilde{\psi}(v, \nu)
&\coloneqq
-\psi(\nu)(v)
\eas
for all $(v, \nu) \in p^*\mathcal{g}$, where one makes use of the identification $\mathrm{T}_vV_x = \mathrm{V}_vV \cong V_x$ ($x \coloneqq p(v)$), so that $V_x \subset \mathrm{T}_vV$.
\end{corollaries}

\begin{remark}
\leavevmode\newline
In fact, similar to LGBs we could have defined LAB representations as a certain type of LAB action with values in certain \textit{linear vector fields}. However, it would exceed this work to introduce these vector fields; start with \cite[\S 3.4, page 110]{mackenzieGeneralTheory} for more elaborated details on how to do this.

As for LGB representations and linear actions, we denote both, LAB representation and its associated sense of action, in the same fashion.
\end{remark}

This is a known relationship in the case of Lie algebra representations on vector spaces and their associated actions; usually this is proven by assuming a Lie group representation, however, one can show it without assuming integrability of the Lie algebra action. See for example \cite[\S 2.1, proof of Prop.\ 2.1.16, page 22]{MyThesis}. Hence, we will just show that the proof of Cor.\ \ref{cor:LABRepsAreLABActions} breaks down to proving it for Lie algebras acting on vector spaces and refer to this reference for the remaining part of the proof.

\begin{proof}[Proof of Cor.\ \ref{cor:LABRepsAreLABActions}]
\leavevmode\newline
By construction we have that $\widetilde{\psi}$ has values in the vertical bundle $\mathrm{V}V$ of $V$, hence Eq.\ \eqref{LABActionAlongFibres} follows. 
Making use of $\mathrm{V}_vV \cong V_x$ for all $v\in V_x$ ($x \in M$), it is clear that $p^*V \cong \mathrm{V}V$ as vector bundles, and so it is trivial to see that $\widetilde{\psi}$ as a map $p^*\mathcal{g} \to p^*V$ is a smooth and base-preserving vector bundle morphism due to that $\psi$ is a smooth map with values in $\mathrm{End}(V)$ (making again use of that $p^*\mathcal{g}$ is an embedded submanifold of $V \times \mathcal{g}$, similarly for $p^*V$ as embedded submanifold of $V \times V$).

As argued in Rem.\ \ref{rem:LABHomomorpInActionVariants}, since $\mathcal{g}$ acts via a representation on a bundle $V$, $\widetilde{\psi}: p^*\mathcal{g} \to \mathrm{V}V$ will be an LAB action, if it induces a Lie algebra action over each base point $x \in M$ via
\bas
\mathcal{g}_x &\to \mathfrak{X}(V_x),\\
\nu &\mapsto \mleft[ V_x \ni v \mapsto \widetilde{\psi}\mleft(v, \nu\mright) \mright].
\eas
Since $\psi$ is just a $\mathcal{g}_x$-representation on $V_x$ over $x$ we know by \cite[\S 2.1, proof of Prop.\ 2.1.16, page 22]{MyThesis} that $\psi$ indeed gives rise to a $\mathcal{g}_x$-action on $V_x$ by
\bas
\widetilde{\psi}\mleft(v, \nu\mright)
&=
- \psi(\nu)(v)
\eas
for all $v \in V_x$ and $\nu \in \mathcal{g}_x$, which is precisely the form of $\widetilde{\psi}$. Thus, $\widetilde{\psi}$ is an LAB action.
\end{proof}

Usually statements like Cor.\ \ref{cor:LABRepsAreLABActions} are proven by assuming integrability: LGB representations as linear actions $V*\mathcal{G} \to V$ and as LGB morphisms $\mathcal{G} \to \mathrm{Aut}(V)$ are the same by Cor.\ \ref{cor:LGBRepAsLGBMorph}; the former induces LAB actions $p^*\mathcal{g} \to \mathrm{T}V$ by Lemma \ref{lem:LGBInduceLABAction}, and in the same manner the latter implies LAB morphisms $\mathcal{g} \to \mathrm{End}(V)$ by Cor.\ \ref{cor:LGBToLABHomomorphis}. Either way, LAB representations can be viewed as certain LAB actions. 

The adjoint representation is an important example inherited by an LGB representation; this example will conclude this subsection.

\begin{examples}{Adjoint LAB representations, \newline \cite[\S 3.3, special situation of Ex.\ 3.3.15, page 108]{mackenzieGeneralTheory}}{LABAdjointRep}
Let us assume the same situation as in Ex.\ \ref{ex:LGBAdjointRep}, especially we have the adjoint representation of the LGB $\mathcal{G}\to M$, $\mathrm{Ad}: \mathcal{G} \to \mathrm{Aut}(\mathcal{g})$. As discussed earlier, its infinitesimal version is a $\mathcal{g}$-representation on itself, the \textbf{adjoint representation $\mathrm{ad}$ of $\mathcal{g}$}. By construction,
\bas
\mathrm{ad}_\nu(\mu)
&\coloneqq
\mathrm{ad}(\nu)(\mu)
=
\mleft[ \nu, \mu \mright]_{\mathcal{g}_x}
\eas
for all $\nu, \mu \in \mathcal{g}_x$ ($x\in M$).
\end{examples}

\subsection{Fundamental vector fields}

The LAB actions inherited by LGB actions are also called fundamental vector fields; we are following \cite[\S 3.4, generalization of Def.\ 3.4.1, page 143]{Hamilton} for the labelling:

\begin{definitions}{Fundamental vector fields}{FundVecs}
Let $M$ and $N$ be two smooth manifolds, $\mathcal{G} \to M$ an LGB over $M$, $f: N \to M$ a smooth map, and assume we have a right $\mathcal{G}$-action on $N$, denoted as $\Phi: N * \mathcal{G} \to N$. For $\nu \in \Gamma(\mathcal{g})$ we define its induced \textbf{fundamental vector field $\widetilde{\nu}$} as an element of $\mathfrak{X}(N)$ by
\bas
\widetilde{\nu}_p
&\coloneqq
\mathrm{D}_{e_{f(p)}}\Phi_p\mleft(\nu_{f(p)}\mright)
\eas 
for all $p \in N$, where $\Phi_p$ is the orbit map defined in Def.\ \ref{def:LRTranslations} and $e_{f(p)}$ the neutral element of $\mathcal{G}_{f(p)}$. 

For left actions we define fundamental vector fields similarly by
\bas
\widetilde{\nu}_p
&\coloneqq
\mathrm{D}_{e_{f(p)}}\Phi^\prime_p\mleft(\nu_{f(p)}\mright)
\eas
for all $p \in N$,
where $\Phi^\prime_p$ is a slightly adjusted orbit map given by
\bas
\mathcal{G}_{f(p)} &\to N,\\
g &\mapsto g^{-1} \cdot p.
\eas
\end{definitions}

\begin{remarks}{Notation}{FundVecsNotations}
Again, point-wise over $x \coloneqq f(p)$ this is just the typical definition of a fundamental vector field with respect to $\nu_x \in \mathcal{g}_x$ (except that $f^{-1}(\{x\})$ may not be a manifold). Hence, one has also a point-wise definition which we will also denote similarly by $\widetilde{\nu_x}$. If $f^{-1}(\{x\})$ is not a manifold, then $\mleft.\widetilde{\nu_x}\mright|_p$ is just a formal notation, and it only defines an element of $\mathrm{T}_pN$ which may not be related to a vector field on $f^{-1}(\{x\})$, not even to a vector field on $N$ if $\nu_x$ does not formally come from a fixed section of $\mathcal{g}$.

However, if $f$ is \textit{e.g.}\ the projection of a bundle, then we have a $\mathcal{G}$-action on the fibre $f^{-1}(\{x\})$ as manifold, and therefore $\widetilde{\nu}|_{f^{-1}(\{x\})}$ and $\widetilde{\nu_x}$ give rise to a vector field on $f^{-1}(\{x\})$. In other words, fundamental vector fields are vertical vector fields as expected and are fibre-wise fundamental vector fields coming from a Lie group action.

For long expressions we use the following different font 
\bas
\oversortoftilde{\nu}
\eas
instead of $\widetilde{\nu}$.
\end{remarks}

\begin{remarks}{Map to fundamental vector fields an LAB action}{FundVecsAreLABActions}
As anticipated, this is related to the LAB action coming from the right $\mathcal{G}$-action $\Phi: N*\mathcal{G} \to N$. The following can also be shown for left actions in a similar manner by recalling Rem.\ \ref{LeftActionsAndTheirSignProblem}.

By Lemma \ref{lem:LGBInduceLABAction} we have an LAB action $\rho: f^*\mathcal{g} \to \mathrm{T}N$ given by $\mleft.\rho \coloneqq \mathrm{D}\Phi\mright|_{f^*\mathcal{g}}$. We have
\bas
\rho(p, \nu)
&=
\mleft. \frac{\mathrm{d}}{\mathrm{d}t} \mright|_{t=0} \Phi\mleft( p, \e^{t\nu} \mright)
=
\mleft. \frac{\mathrm{d}}{\mathrm{d}t} \mright|_{t=0} \mleft( p \cdot \e^{t\nu} \mright)
=
\mathrm{D}_{e_{f(p)}}\Phi_p(\nu)
=
\widetilde{\nu}_p
\eas
for all $(p, \nu) \in f^*\mathcal{g}$, where $t \in \mathbb{R}$, $\e$ is the exponential map of $\mathcal{G}$, and $e_{f(p)}$ the neutral element of $\mathcal{G}_{f(p)}$. Thus, the map to fundamental vector fields
\bas
\Gamma(\mathcal{g}) &\mapsto \mathfrak{X}(N),\\
\nu &\mapsto \widetilde{\nu},
\eas
is equivalent to the map
\bas
\Gamma(\mathcal{g}) &\to \mathfrak{X}(N),\\
\nu &\mapsto \rho\mleft( f^*\nu \mright)
\eas
induced by $\rho$ as in Def.\ \ref{def:LABACtions}, which also implies that the map to the fundamental vector fields is a homomorphism of Lie algebras. Last but not least, by Def.\ \ref{def:LABACtions} it follows that fundamental vector fields are in the kernel of $\mathrm{D}f$.
\end{remarks}

\subsection{Differential of smooth LGB actions}

\begin{lemmata}{Tangent bundle of pullback fibre bundles}{PullbackFibreBundleItsTangentSp}
Let $M$ and $N$ be two smooth manifolds, $F \stackrel{\pi}{\to} M$ a fibre bundle over $M$, and $f: N \to M$ a smooth map. Then we have for its tangent spaces
\bas
\mathrm{T}_{(p, v)}\mleft( f^*F \mright)
&=
\bigl\{
	(X, Y)
	~\big|~
	X \in \mathrm{T}_pN, Y\in \mathrm{T}_vF \text{ with } \mathrm{D}_pf(X) = \mathrm{D}_v\pi(Y)
\bigr\}
\eas
for all $(p, v) \in f^*F$.
\end{lemmata}

\begin{proof}
\leavevmode\newline
Recall that $(p, v) \in f^*F$ implies that
\bas
f(p) &= \pi(v)
\eas
such that we can immediately derive its infinitesimal version as
\bas
\mathrm{D}_pf(X) = \mathrm{D}_v\pi(Y)
\eas
for all $X \in \mathrm{T}_pN, Y\in \mathrm{T}_vF$. Hence, we have derived that $\mathrm{T}_{(p, v)}\mleft( f^*F \mright)$ is a subset of the set of such pairs $(X, Y)$. That this is an equivalent description quickly follows by the fact that $f$ and $\pi$ are transversal to each other (trivially, because $\pi$ is a surjective submersion). This means, the following linear map
\bas
\mathrm{T}_pN \times \mathrm{T}_vF&\to \mathrm{T}_{f(p)}M,\\
(X, Y) &\mapsto \mathrm{D}_pf(X) - \mathrm{D}_v\pi(Y)
\eas
is surjective because $\pi$ is a submersion; it is also well-defined because of $f(p) = \pi(v)$. Hence, the dimension of the kernel of this map has the dimension
\bas
\mathrm{dim}(N) + \mathrm{dim}(F) - \mathrm{dim}(M)
&=
\mathrm{dim}(N) + \mathrm{rk}(F)
=
\mathrm{dim}(f^*N),
\eas
where $\mathrm{dim}$ denotes the dimension as a manifold and $\mathrm{rk}$ the rank of a bundle (the dimension of its structural fibre). Its dimension is precisely the dimension of $f^*F$, and since it is about finite dimensions we can therefore identify $\mathrm{T}_{(p, v)}\mleft( f^*F \mright)$ with this kernel. This concludes the proof due to the fact that the kernel consists of $(X, Y)$ with $\mathrm{D}_pf(X) = \mathrm{D}_v\pi(Y)$.
\end{proof}

With this we can finally show the following theorem; also recall the notations introduced in Def.\ \ref{def:LRTranslations}, \ref{def:LeftRightTranslationConjugation}, \ref{def:MCFormOnLGBs} and \ref{def:FundVecs}.

\begin{theorems}{Differential of smooth LGB actions}{DiffOfLGBAction}
Let $M$ and $N$ be two smooth manifolds, $\mathcal{G} \stackrel{\pi}{\to} M$ an LGB over $M$, $f: N \to M$ a smooth map, and assume we have a right $\mathcal{G}$-action on $N$, $\Phi: N * \mathcal{G} \to N$. Then we have
\ba\label{NewDIffACtionwithMCForm}
\mathrm{D}_{(p, g)}\Phi(X, Y)
&=
\mathrm{D}_pr_\sigma(X)
	+ \mleft.{\oversortoftilde{\mleft( \mu_{\mathcal{G}}\mright)_g \mleft(Y - \mathrm{D}_{e_x}R_\sigma \bigl( \mathrm{D}_x e (\omega) \bigr)\mright)}}\mright|_{p \cdot g}
\ea
for all $(p, g) \in N * \mathcal{G}$ and $(X, Y) \in \mathrm{T}_{(p, g)}(N*\mathcal{G})$, where $x \coloneqq f(p) = \pi(g)$, $\sigma$ is any (local) section of $\mathcal{G}$ with $\sigma_{x} = g$, $e$ is the identity section of $\mathcal{G}$, and $\omega$ is an element of $\mathrm{T}_xM$ given by
\bas
\omega
&\coloneqq
\mathrm{D}_p f(X)
= 
\mathrm{D}_g\pi(Y).
\eas
We can also write instead
\ba
\mathrm{D}_{(p, g)}\Phi(X, Y)
&=
\mathrm{D}_pr_\sigma(X)
	+ \mleft.{\oversortoftilde{\mleft( \mu_{\mathcal{G}}\mright)_g \bigl(Y - \mathrm{D}_{x}\sigma (\omega)\bigr)}}\mright|_{p \cdot g}.
\ea

If $f$ is a surjective submersion, then we can also write
\ba\label{DIFfActionSimilarToTangenGroupoid}
\mathrm{D}_{(p, g)}\Phi(X, Y)
&=
\mathrm{D}_pr_\sigma(X)
	+ \mathrm{D}_g\Phi_\tau (Y)
	- \mathrm{D}_{e_x}(\Phi_\tau \circ R_\sigma)\bigl( \mathrm{D}_xe(\omega) \bigr)
\ea
and
\ba\label{DiffActionAsClassicalButWithExtraContribution}
\mathrm{D}_{(p, g)}\Phi(X, Y)
&=
\mathrm{D}_pr_g\Bigl( X - \mathrm{D}_{e_x}\Phi_\tau\bigl( \mathrm{D}_xe(\omega) \bigr) \Bigr)
	+ \mathrm{D}_g\Phi_p \Bigl( Y - \mathrm{D}_{e_x}R_\sigma \bigl( \mathrm{D}_x e (\omega) \bigr) \Bigr)
\nonumber\\
&\hspace{1cm}
	+ \mathrm{D}_{e_x} \mleft( \Phi_\tau \circ R_\sigma \mright)\bigl( \mathrm{D}_xe(\omega) \bigr),
\ea
where $\tau$ is any (local) section\footnote{That is, $f \circ \tau = \mathds{1}_M$ (locally).} of $f$ with $\tau_x = p$, and $\Phi_\tau$ is the orbit map through $\tau$.
\end{theorems}

\begin{remark}
\leavevmode\newline
The assumption about $f$ being a surjective submersion is being stated in order to assure the existence of $\tau$ and the manifold structure on $f^{-1}(\{x\})$ as an embedded submanifold of $N$; see the proof for more details. If the existence of $\tau$ and the embedded submanifold structure is known otherwise, then those equations can still be derived. Following the proof, one may also just need the structure of an immersed submanifold.
\end{remark}

\begin{proof}[Proof of Thm.\ \ref{thm:DiffOfLGBAction}]
\leavevmode\newline
We want to calculate the derivative of $\Phi$, and due to $N*\mathcal{G} = f^*\mathcal{G}$ we are going to use Lemma \ref{lem:PullbackFibreBundleItsTangentSp}. That is, fix $(p, g) \in N*\mathcal{G}$ and $X \in \mathrm{T}_pN$, $Y \in \mathrm{T}_g \mathcal{G}$ with
\bas
\mathrm{D}_p f(X) &= \mathrm{D}_g\pi(Y) \eqqcolon \omega \in \mathrm{T}_{f(p)}M.
\eas
Recall that we can localize LGB actions in sense of Rem.\ \ref{rem:LocalLGBAction}; so, let $x \coloneqq f(p) = \pi(g)$, and fix a trivialization of $\mathcal{G}$ around $x$. Then it is clear that there is a (local)\footnote{For simplicity of notation we omit the notation of restricting on some open subset of $M$.} section $\sigma \in \Gamma(\mathcal{G})$ with $\sigma_x = g$. Observe that we can write
\bas
Y
&=
Y 
	- \mathrm{D}_x \sigma (\omega)
	+ \mathrm{D}_x \sigma (\omega).
\eas
The two first summands result into a vertical tangent vector due to
\bas
\mathrm{D}_g\pi\bigl( Y - \mathrm{D}_x\sigma(\omega) \bigr)
&=
\mathrm{D}_g\pi( Y ) - \mathrm{D}_x\underbrace{(\pi \circ \sigma)}_{= \mathds{1}_M}(\omega)
=
\mathrm{D}_g \pi (Y) - \omega
=
0.
\eas
For the following recall the notations introduced in Def.\ \ref{def:LRTranslations} and \ref{def:LeftRightTranslationConjugation}; we can derive that
\bas
\mathrm{D}_{e_x}R_\sigma \circ \mathrm{D}_x e
&=
\mathrm{D}_x \underbrace{\mleft( R_\sigma \circ e \mright)}
	_{\mathclap{x \mapsto e_x \sigma_x = \sigma_x}}
=
\mathrm{D}_x \sigma.
\eas
So, we can also write 
\bas
Y^v
&\coloneqq
Y - \mathrm{D}_x \sigma(\omega)
=
Y - \mathrm{D}_{e_x}R_\sigma \bigl( \mathrm{D}_x e (\omega) \bigr).
\eas
We have proven that $Y^v \in \mathrm{V}_g\mathcal{G}$, and thus $(0_p, Y^v) \in \mathrm{T}_{(p, g)}(f^*\mathcal{G})$ by Lemma \ref{lem:PullbackFibreBundleItsTangentSp}, where $0_p$ is the zero tangent vector of $\mathrm{T}_pN$. In the same fashion we also have
\bas
\mleft(X, \mathrm{D}_{e_x}R_\sigma \bigl( \mathrm{D}_x e (\omega) \bigr) \mright)
&\in \mathrm{T}_{(p, g)}(f^*\mathcal{G})
\eas
because of
\bas
\mathrm{D}_g\pi \Bigl( \mathrm{D}_{e_x}R_\sigma \bigl( \mathrm{D}_x e (\omega) \bigr) \Bigr)
&=
\mathrm{D}_g \pi (Y - Y^v)
=
\mathrm{D}_g \pi (Y)
=
\mathrm{D}_p f (X),
\eas
thence we can write in $\mathrm{T}_{(p, g)}(f^*\mathcal{G})$
\bas
(X, Y)
&=
\mleft(X, \mathrm{D}_{e_x}R_\sigma \bigl( \mathrm{D}_x e (\omega) \bigr) + Y^v \mright)
=
\mleft(X, \mathrm{D}_{e_x}R_\sigma \bigl( \mathrm{D}_x e (\omega) \bigr) \mright)
	+ (0_p, Y^v).
\eas
Hence also
\ba\label{DifferentialOfActionSplitFirst}
\mathrm{D}_{(p, g)}\Phi(X, Y)
&=
\mathrm{D}_{(p, g)}\Phi\mleft(X, \mathrm{D}_{e_x}R_\sigma \bigl( \mathrm{D}_x e (\omega) \bigr) \mright)
	+ \mathrm{D}_{(p, g)}\Phi(0_p, Y^v)
\ea
The second summand is quickly calculated as
\bas
\mathrm{D}_{(p, g)}\Phi(0_p, Y^v)
&=
\mleft.\frac{\mathrm{d}}{\mathrm{d}t}\mright|_{t=0} \Phi(p, \gamma)
=
\mleft.\frac{\mathrm{d}}{\mathrm{d}t}\mright|_{t=0} \underbrace{(p \cdot \gamma)}
	_{\Phi_p(\gamma)}
=
\mathrm{D}_{g}\Phi_p (Y^v)
\eas
where $\gamma: I \to \mathcal{G}_x$ ($I$ open interval of $\mathbb{R}$ containing 0) is a curve with $\gamma(0) = g$ and $\mathrm{d}/\mathrm{d}t|_{t=0} \gamma = Y^v$; this is due to the verticality of $Y^v$, and so $\mathrm{D}_g\Phi_p(Y^v)$ is also well-defined since $\Phi_p$ is a map $\mathcal{G}_x \to N$. Now recall Def.\ \ref{def:MCFormOnLGBs} and \ref{def:FundVecs}, and then observe that we can write
\ba\label{ClassicalWayToWriteLeibnizRuleWithLeftPushForwardInsteadOfMCForm}
\mathrm{D}_{g}\Phi_p (Y^v)
&=
\mleft(\mathrm{D}_{g}\Phi_p \circ \mathrm{D}_{e_x}L_g \circ \mathrm{D}_g L_{g^{-1}} \mright)(Y^v)
=
\mathrm{D}_{e_x}\underbrace{\mleft( \Phi_p \circ L_g \mright)}
	_{\mathclap{ \mathcal{G} \ni q \mapsto p \cdot gq = \Phi_{p \cdot g}(q) }}
	\mleft(\mleft( \mu_{\mathcal{G}}\mright)_g (Y^v) \mright)
=
\mleft.{\oversortoftilde{\mleft( \mu_{\mathcal{G}}\mright)_g (Y^v)}}\mright|_{p \cdot g}
%\mleft.{\widetilde{\mleft( \mu_{\mathcal{G}}\mright)_g (Y^v)}}\mright|_{p \cdot g},
\ea
making use of the verticality of $Y^v$ such that operators like $\mathrm{D}L_g$ can act on $Y^v$.

For the first summand in Eq.\ \eqref{DifferentialOfActionSplitFirst} we use
\bas
\mathrm{D}_{(p, g)}\Phi\Bigl(X, \mathrm{D}_{e_x}R_\sigma \bigl( \mathrm{D}_x e (\omega) \bigr) \Bigr)
&=
\mathrm{D}_{(p, e_x)} \underbrace{\bigl( \Phi \circ \mleft( \mathds{1}_N, R_\sigma \mright) \bigr)}_{\mathclap{ N * \mathcal{G} \ni (p, g) \mapsto p \cdot g \sigma_x = r_\sigma(p \cdot g) }}
	\bigl( X, \mathrm{D}_x e(\omega) \bigr)
\\
&=
\mathrm{D}_{(p, e_x)}( r_\sigma \circ \Phi ) \bigl( X, \mathrm{D}_x e(\omega) \bigr)
\\
&=
\mathrm{D}_p r_\sigma \Bigl(
	\mathrm{D}_{(p, e_x)} \Phi \bigl( X, \mathrm{D}_x e(\omega) \bigr)
\Bigr)
\\
&=
\mathrm{D}_p r_\sigma \Bigl(
	\mathrm{D}_{(p, e_x)} \Phi \bigl( X, \mathrm{D}_p (e \circ f) (X) \bigr)
\Bigr)
\\
&=
\mathrm{D}_p r_\sigma \Bigl(
	\mathrm{D}_{(p, p)} \bigl(\Phi \circ (\mathds{1}_N, e\circ f)\bigr)( X, X )
\Bigr),
\eas
but on the diagonal $\mathrm{diag}(N\times N)$ of $N\times N$ we have
\bas
\Bigl[
\mathrm{diag}(N \times N) \ni (p,p)
\mapsto
\bigl(\Phi \circ (\mathds{1}_N, e\circ f)\bigr)(p,p)
=
\Phi\mleft(p, e_{f(p)}\mright)
=
p \cdot e_{f(p)}
=
p
\Bigr]
&=
\mleft.\mathrm{pr}_i\mright|_{\mathrm{diag}(N \times N)},
\eas
where $\mathrm{pr}_i$ is just the projection onto any of both $i$-th components ($i \in \{1,2\}$) in $N \times N$. Due to $(X, X) \in \mathrm{T}_{(p,p)}\bigl( \mathrm{diag}(N\times N) \bigr)$ we derive
\bas
\mathrm{D}_{(p, p)} \bigl(\Phi \circ (\mathds{1}_N, e\circ f)\bigr)( X, X )
&=
\mathrm{D}_{(p,p)} \mathrm{pr}_i( X, X )
=
X.
\eas
So, we get in total
\bas
\mathrm{D}_{(p, g)}\Phi\Bigl(X, \mathrm{D}_{e_x}R_\sigma \bigl( \mathrm{D}_x e (\omega) \bigr) \Bigr)
&=
\mathrm{D}_pr_\sigma(X),
\eas
therefore
\bas
\mathrm{D}_{(p, g)}\Phi(X, Y)
&=
\mathrm{D}_pr_\sigma(X)
	+ \mleft.{\oversortoftilde{\mleft( \mu_{\mathcal{G}}\mright)_g (Y^v)}}\mright|_{p \cdot g}.
\eas

If $f$ is a surjective submersion, then $x$ is a regular value and thus $f^{-1}(\{x\})$ is an embedded submanifold, and we can assure the existence of a smooth local section $\tau: U \to N$ of $f$ ($U$ an open neighbourhood of $x \in M$), \textit{i.e.}\ $f \circ \tau = \mathds{1}_U$ with $\tau_x = p$; in case of doubt, this can be shown as in \cite[\S 3.7, Lemma 3.7.4, page 152f.]{Hamilton}\ via the Regular Point Theorem. Recalling the arguments of Rem. \ref{rem:AbstractNotationTwoForLeftInvarVfs}, we can rewrite the second summand of Eq.\ \eqref{DifferentialOfActionSplitFirst} instead to
\bas
\mathrm{D}_g\Phi_p (Y^v)
&=
\mathrm{D}_g\Phi_\tau (Y^v)
\\
&=
\mathrm{D}_g\Phi_\tau (Y)
	- \mathrm{D}_g\Phi_\tau \mleft(\mathrm{D}_{e_x}R_\sigma \bigl( \mathrm{D}_x e (\omega) \bigr)\mright)
\\
&=
\mathrm{D}_g\Phi_\tau (Y)
	- \mathrm{D}_{e_x}(\Phi_\tau \circ R_\sigma)\bigl( \mathrm{D}_xe(\omega) \bigr)
\eas
making use of that $\mathrm{D}_g\Phi_\tau$ is linear map $\mathrm{T}_g\mathcal{G} \to \mathrm{T}_{p \cdot g}N$. In that case we would get in total
\bas
\mathrm{D}_{(p, g)}\Phi(X, Y)
&=
\mathrm{D}_pr_\sigma(X)
	+ \mathrm{D}_g\Phi_\tau (Y)
	- \mathrm{D}_{e_x}(\Phi_\tau \circ R_\sigma)\bigl( \mathrm{D}_xe(\omega) \bigr).
\eas
Alternatively, we do not rewrite Eq.\ \eqref{DifferentialOfActionSplitFirst} with the Maurer-Cartan form and instead apply the same trick to $X$ as for $Y$, that is,
\bas
X &= X^v + \mathrm{D}_{e_x}\Phi_\tau\bigl( \mathrm{D}_xe(\omega) \bigr),
\eas
where
\bas
X^v
&\coloneqq
X
	- \mathrm{D}_x \tau(\omega)
=
X
	- \mathrm{D}_{e_x}\Phi_\tau\bigl( \mathrm{D}_xe(\omega) \bigr),
\eas
and $X^v$ is vertical, too, that is,
\bas
\mathrm{D}_p f (X^v)
&=
\mathrm{D}_p f (X)
	- \mathrm{D}_x (f \circ \tau) (\omega)
=
\omega - \omega
=
0,
\eas
especially $X^v \in \mathrm{T}_p\mleft( f^{-1}(\{x\}) \mright)$ and so we can apply a similar argument as in Rem. \ref{rem:AbstractNotationTwoForLeftInvarVfs} to derive
\bas
\mathrm{D}_p r_\sigma(X)
&=
\mathrm{D}_p r_\sigma(X^v)
	+ \mathrm{D}_p r_\sigma\Bigl( \mathrm{D}_{e_x}\Phi_\tau\bigl( \mathrm{D}_xe(\omega) \bigr) \Bigr)
\\
&=
\mathrm{D}_p r_g(X^v)
	+ \mathrm{D}_{e_x} \mleft( r_\sigma \circ \Phi_\tau \mright)\bigl( \mathrm{D}_xe(\omega) \bigr).
\eas
Finally, using these expressions, the total formula would look like
\bas
\mathrm{D}_{(p, g)}\Phi(X, Y)
&=
\mathrm{D}_pr_g(X^v)
	+ \mathrm{D}_g\Phi_p (Y^v)
	+ \mathrm{D}_{e_x} \underbrace{\mleft( r_\sigma \circ \Phi_\tau \mright)}
		_{\mathclap{ \mathcal{G} \ni g \mapsto \tau_{\pi(g)} \cdot g \sigma_{\pi(g)} = (\Phi_\tau \circ R_\sigma)(g) }}
		\bigl( \mathrm{D}_xe(\omega) \bigr)
\\
&=
\mathrm{D}_pr_g(X^v)
	+ \mathrm{D}_g\Phi_p (Y^v)
	+ \mathrm{D}_{e_x} \mleft( \Phi_\tau \circ R_\sigma \mright)\bigl( \mathrm{D}_xe(\omega) \bigr).
\eas
\end{proof}

\begin{remark}
\leavevmode\newline
Eq.\ \eqref{NewDIffACtionwithMCForm} is very similar to the "classical" formula used in gauge theory, see \textit{e.g.}\ \cite[\S 3.5, Prop.\ 3.5.4, page 146]{Hamilton}: In the case of a Lie group action on $N$ we have
\bas
\mathrm{D}_{(p, g)}\Phi(X, Y)
&=
\mathrm{D}_pr_g(X)
	+ \mleft.\oversortoftilde{(\mu_G)_g(Y)}\mright|_{p \cdot g}
\eas
for all $p \in N$, $g \in G$, $X \in \mathrm{T}_pN$ and $Y \in \mathrm{T}_g G$.
However, in our general case the vector $Y$ is deformed by $\omega$, due to the fact that the action $\Phi$ has no "constant" Lie group factor anymore. This will be important later when we are going to derive the gauge transformations. Furthermore, already the first summand is now different than the classical formula, because we need to use LGB sections in order to define the push-forward of tangent vectors which are not vertical, that is, $X$ may not be a tangent vector of $f^{-1}(\{x\})$ (which is in the general case not even an embedded submanifold) such that $\mathrm{D}_p r_g(X)$ is in general not well-defined anymore.

The other two equations in the case of $f$ being a surjective submersion are mainly for reference; the last equation, Eq.\ \eqref{DiffActionAsClassicalButWithExtraContribution}, emphasises the contribution of non-vertical vectors measured by $\omega$. While the first two summands are the classical product rule on the vertical parts, the third summand shows the deformation of the product rule because of the new structure of an action without a "constant" Lie group factor.

Eq.\ \eqref{DIFfActionSimilarToTangenGroupoid} may be the most elegant formulation, making use of local sections $\sigma$ and $\tau$ and their advantage that these can act on all tangent vectors, not just the vertical ones; the reader who knows Lie groupoids may recognize this equation's structure with the one as given in \cite[\S 1.4, Thm.\ 1.4.14, page 28]{mackenzieGeneralTheory}, where it is about the induced multiplication structure on the tangent bundle of a Lie groupoid making use of bisections playing a similar role like $\sigma$ and $\tau$.

Those equations additionally show that we have a more general Leibniz rule with a third summand. However, by Ex.\ \ref{ex:TrivialLGBAction} we expect still a typical Leibniz rule once a trivialization of $\mathcal{G}$ around $g$ is fixed. Indeed, as we will understand and see also later, this is the case; fix such a trivialization, equip it with a canonical flat connection, and take $\sigma$ to be a parallel section, that is, $g$ as a constant section. Then one can calculate that the typical Leibniz rule is recovered. Hence, one can view this Leibniz rule as the "covariantized" version of the "classical" Leibniz rule, independent of a choice of "coordinate"/section on $\mathcal{G}$. 
\end{remark}

\section{Connections and curvature on LGB principal bundles}\label{ConnCurvOnPrincLGBBundle}

\subsection{Principal bundles with structural LGB}\label{PrincBundlLGBBased}

\subsubsection{Definition}

The principal bundle $\mathcal{P}$ we are interested into is still a fibre bundle related to the same Lie group as the one behind the LGB $\mathcal{G}$, especially also $\mathrm{dim}(\mathcal{P}) = \mathrm{dim}(\mathcal{G})$, but it is equipped with an LGB action.

\begin{definitions}{Principal bundles with structural LGB, \newline \cite[simplification of the beginning of \S 5.7, page 144f.]{GroupoidBasedPrincipalBundles}}{PrinciBdleWithStruLGB}
Let $G$ be a Lie group, $M$ a smooth manifold, and an LGB $\mathcal{G} \to M$ which acts on the right on another $G$-fibre bundle $\mathcal{P} \to M$
\begin{center}
	\begin{tikzcd}
	G \arrow{r} & \mathcal{P} \arrow{d}{\pi_{\mathcal{P}}} & \arrow[bend right]{l} \mathcal{G} \arrow{d}{\pi_{\mathcal{G}}} & \arrow{l} G \\
	& M & M
	\end{tikzcd}
\end{center}
where the right-action is defined on $\mathcal{P} * \mathcal{G}$ given by $\pi_{\mathcal{P}}^*\mathcal{G}$.
Then we call $\mathcal{P}$ a \textbf{principal $\mathcal{G}$-bundle} if 
\begin{enumerate}
	\item The right $\mathcal{G}$-action on $\mathcal{P}$ \textbf{is simply transitive on the fibres}, that is, the restriction of $\mathcal{P}*\mathcal{G} \to \mathcal{P}$ on $x\in M$
	\bas
	\mathcal{P}_x \times \mathcal{G}_x &\to \mathcal{P}_x
	\eas
	induces bijective orbit maps
	\bas
	\mathcal{G}_x &\to \mathcal{P}_x,\\
	g &\mapsto p \cdot g
	\eas
	for $p \in \mathcal{P}_x$.
	\item There exist base-preserving \textbf{$\mathcal{G}$-equivariant diffeomorphisms $\varphi_i: \mathcal{P}|_{U_i} \to \mathcal{G}|_{U_i}$}	subordinate to an open covering $\mleft( U_i \mright)_i$ of $M$, that is,
	\bas
	\pi_{\mathcal{G}} \circ \varphi_i &= \pi_{\mathcal{P}},\\
	\varphi_i(p \cdot g)
	&=
	\varphi_i(p) \cdot g,
	\eas
	where the multiplication on the right hand side is inherited by $\mathcal{G} * \mathcal{G} \to \mathcal{G}$, the multiplication on $\mathcal{G}$.
	%\textbf{bundle atlas of $\mathcal{G}$-equivariant bundle charts $\varphi_i: \mathcal{P}|_{U_i} \to U_i \times G$}, where $\mleft(U_i\mright)_i$ is an open covering of $M$ also subordinate to LGB charts $\vartheta_i: \mathcal{G}|_{U_i} \to U_i \times G$ of $\mathcal{G}$, that is, we have
	%\bas
	%\varphi_i(p \cdot g)
	%&=
	%\varphi_i(p) \cdot \vartheta_i(g)
	%\eas
	%for all $(p,g) \in \mleft.\mleft(\mathcal{P} * \mathcal{G}\mright)\mright|_{\pi_{\mathcal{P}}^{-1}(U_i)} = \mathcal{P} \times_{U_i} \mathcal{G}$, where on the right hand side the multiplication is the canonical multiplication of $U_i \times G$ as a trivial LGB; recall Ex.\ \ref{ex:TrivialLGBundle}.
	%
	%We will call such an atlas and its charts as \textbf{principal bundle atlas} and \textbf{principal bundle charts} for $\mathcal{P}$, respectively.
	%$(x, q) \in U_i \times G$ acts on $(x, v) \in U_i \times G$ via
	%\bas
	%(x, v) \cdot (x, q)
	%&=
	%(x, vq).
	%\eas
\end{enumerate}

The LGB $\mathcal{G}$ is the \textbf{structural LGB} of the principal bundle $\mathcal{P}$.
\end{definitions}

\begin{remarks}{Discussion about the definition of $\mathcal{G}$-principal bundles}{LGBPrincDefDiscussion}
There are several things in need to be discussed:
\begin{enumerate}
	\item The mentioned reference, \cite[simplification of the beginning of \S 5.7, page 144f.]{GroupoidBasedPrincipalBundles}, introduces these principal bundles in a different manner; we will come back later to this in Remark \ref{AlternativePrincBdlDef}.
	\item By Remark \ref{rem:ActionAndPullbackLGBs} the $\mathcal{G}$-action on $\mathcal{P}$ is fibre-preserving, thus, the restriction of $\mathcal{P}*\mathcal{G} \to \mathcal{P}$ is indeed a map $\mathcal{P}_x \times \mathcal{G}_x \to \mathcal{P}_x$.
	\item The $\mathcal{G}$-equivariant diffeomorphisms $\varphi_i$ give rise to a \textbf{bundle atlas of $\mathcal{G}$-equivariant bundle charts $\xi_i: \mathcal{P}|_{U_i} \to U_i \times G$}. Assume w.l.o.g.\ that $U_i$ is small enough so that there is an LGB chart $\phi_i: \mathcal{G}|_{U_i} \to U_i \times G$ of $\mathcal{G}$; recall Def.\ \ref{def:LieGroupBundle}. Then $\xi_i \coloneqq \phi_i \circ \varphi_i$, which clearly satisfies
	\bas
	\xi_i(p \cdot g)
	&=
	\xi_i(p) \cdot \phi_i(g),
	\eas
	where the multiplication on the right hand side is the canonical one for $U_i \times G$ as a trivial LGB (recall Ex.\ \ref{ex:TrivialLGBundle}),
	and this can be viewed as $\mathcal{G}$-equivariance (under the trivialization induced by $\phi_i$). We usually then speak of \textbf{$\mathcal{G}$-equivariance w.r.t.\ the LGB morphism $\phi_i$}.
	
	We will call such an atlas and its charts a \textbf{principal bundle atlas} and \textbf{principal bundle charts} for $\mathcal{P}$, respectively. For simplicity we may also refer to $\varphi_i$ as principal bundle chart giving rise to the principal bundle atlas.
	\item Due to the existence of $\varphi_i$ one does not need to claim upfront that $\mathcal{P}$ is a $G$-fibre bundle. However, for readability, we decided to structure the definition like this. We kept a similar style as for "typical" principal bundles as provided in \cite[\S 4.2, Def.\ 4.2.1, page 207f.]{Hamilton}.
	\item Since $\pi_{\mathcal{P}}$ is a surjective submersion we know by Remark \ref{SmoothnessOfACtionTranslations} that right-translations $r_g$ ($g \in \mathcal{G}_x$) are diffeomorphisms on $\mathcal{P}_{x}$. Furthermore, following \cite[\S 4.2, discussion after Def.\ 4.2.1, page 208f.]{Hamilton}, by definition we have a simply transitive $\mathcal{G}_x$-action (as a Lie group) on $\mathcal{P}_x$, and the isotropy group for each $p \in \mathcal{P}_x$ is trivial; the isotropy group consists in general of $g \in \mathcal{G}_x$ with $p \cdot g = p$, see \textit{e.g.}\ \cite[\S 3.2, third part of Def.\ 3.2.4, page 132]{Hamilton}. Therefore, and by \cite[\S 3.8, Thm.\ 3.8.8, page 165]{Hamilton}, the orbit map $\Phi_p$ gives rise to a $\mathcal{G}_x$-equivariant diffeomorphism
	\bas
	\mathcal{G}_x &\to \mathcal{P}_x,
	\eas
	where the action on $\mathcal{G}_x$ is the right-action on itself regarding the $\mathcal{G}_x$-equivariance. 
	\item Similarly by definition, for each $x \in U_i$ we know that $\mleft.\mleft(\varphi_i\mright)\mright|_x: \mathcal{P}_x \to \mathcal{G}_x$ is a $\mathcal{G}_x$-equivariant diffeomorphism. In fact, together with $\pi_{\mathcal{G}} \circ \varphi_i = \pi_{\mathcal{P}}$ this clearly gives an equivalent definition of $\varphi_i$ which we may make use in the following without further mention.
\end{enumerate}
\end{remarks}

\subsubsection{Examples}

Let us provide examples of such principal bundles.

\begin{examples}{The "classical" principal bundle}{TheCLassicalPrincAsEx}
We recover principal $G$-bundles as principal $\mathcal{G} \coloneqq M \times G$-bundles. Recall Ex.\ \ref{ex:TrivialLGBAction}, the LGB action of a trivial LGB $\mathcal{G} \cong M \times G$ is equivalent to a $G$-action, the right-translation with an element $g \in G$ is then the right-translation of the corresponding constant section in $\mathcal{G}$; this action is clearly simply transitive. The principal $G$-bundle atlas is then naturally inherited by the existing principal $\mathcal{G}$-bundle atlas. In fact, reverting this argument proves that principal $G$-bundles are equivalent to principal $\mathcal{G} \coloneqq M \times G$-bundles.

We often refer to principal bundles related to trivial LGBs as \textbf{typical} or \textbf{classical principal bundles}, or, as usual, principal $G$-bundle. 
%We will later clarify why "pre-classical" as adjective instead of "classical"; however, at this point both adjectives are permutable.
\end{examples}

\begin{examples}{The "trivial" principal $\mathcal{G}$-bundle}{TrivialPrincAsLGB}
$\mathcal{G}$ itself is a principal $\mathcal{G}$-bundle, equipped with its canonical right-action inherited by its multiplication $\mathcal{G}*\mathcal{G} \to \mathcal{G}$. The principal bundle atlas then just consists of the identity map $\mathds{1}_{\mathcal{G}}$.

By Ex.\ \ref{ex:TheCLassicalPrincAsEx}, it may be natural to call this the \textbf{trivial $\mathcal{G}$-principal bundle} even if $\mathcal{G}$ itself might not be trivial. It will be clearer later why one may choose to do so.

The remarkable property is that this example shows that we are going to define a gauge theory for which LGBs themselves are allowed as principal bundles. One might have wondered why classical gauge theory uses classical principal bundles instead of LGBs (especially including non-trivial LGBs), because LGBs could be viewed as a more natural choice due to the fact that they are an analogue to how vector bundles are the "bundle-construction" of vector spaces. The problems described in Subsection \ref{TheBigMotivationBehindEverything} show why it was easier to choose classical principal bundles, since these avoid the difficulties regarding the definition of a connection; however, we are going to solve these problems in such a way that one can either use LGBs or classical bundles or even something more general.
\end{examples}

Our main example will be the inner bundle of a classical principal bundle, recall Ex.\ \ref{ex:InnerLGBs}. To explain the "triviality" of Ex.\ \ref{ex:TrivialPrincAsLGB} we need to introduce morphisms of principal bundles.

\subsubsection{Morphism of principal LGB-bundles}

Let us define morphisms of LGB-principal bundles

\begin{definitions}{Morphism of principal bundles with structural LGB}{MorphOfPrincBundles}
Let $M$ and $N$ be smooth manifolds, $\mathcal{H} \stackrel{\pi_{\mathcal{H}}}{\to} N$ and $\mathcal{G} \stackrel{\pi_{\mathcal{G}}}{\to} M$ LGBs, and $\mathcal{H} \to \mathcal{P}^\prime \stackrel{\pi^\prime}{\to} N$ and $\mathcal{G} \to \mathcal{P} \stackrel{\pi}{\to} M$ principal $\mathcal{H}$- and $\mathcal{G}$-bundles, respectively. A \textbf{principal bundle morphism} between $\mathcal{P}^\prime$ and $\mathcal{P}$ is a triple of smooth maps $F: \mathcal{H} \to \mathcal{G}$, $f: N \to M$ and $H: \mathcal{P}^\prime \to \mathcal{P}$ such that the pair $(F, f)$ is an LGB morphism as in Def.\ \ref{def:LGB morphism} and
\ba
\pi \circ H &= f \circ \pi^\prime,\label{PrincMorphoverBaseMap}\\
H(p \cdot h) &= H(p) \cdot F(h)\label{PrincMorphLGBEquiv}
\ea
for all $(p, h) \in \mathcal{P}^\prime * \mathcal{H} = \mleft(\pi^\prime\mright)^*\mathcal{H}$. We also speak of a \textbf{principal bundle morphism over $f$ w.r.t.\ the LGB morphism $F$}.

We speak of a \textbf{principal bundle isomorphism (over $f$, w.r.t.\ $F$)} if $H$ is a diffeomorphism. 

If $H$ is a base-preserving isomorphism $\mathcal{P} \to \mathcal{P}$ w.r.t.\ $F = \mathds{1}_{\mathcal{G}}$, then we say that $H$ is a \textbf{(global) gauge transformation} or \textbf{principal bundle automorphism}, and the set of all such automorphisms is denoted by $\sAut(\mathcal{P})$. If $H$ is defined on an open subset of $M$, then we may also speak of a \textbf{local gauge transformation}.
\end{definitions}
%Following remark is wrong, an equivalent statement may be related to the quotient space
%\begin{remarks}{Equivalent definition}{EquivalentDefinitionForPrincMorphisms}
%Recall Cor.\ \ref{cor:PullbackLGB}, that is, $\mathcal{P}^\prime * \mathcal{H} \to \mathcal{P}^\prime$ and $\mathcal{P} * \mathcal{G}\to \mathcal{P}$ are LGBs as the pullbacks of $\mathcal{H}$ and $\mathcal{G}$, respectively. Furthermore, $(H, F)$ is a well-defined map $\mathcal{P}^\prime * \mathcal{H} \to \mathcal{P} * \mathcal{G}$ over $H$. Then Eq.\ \eqref{PrincMorphLGBEquiv} is equivalent to say that $(H, F)$ is an LGB morphism (over $H$).
%\end{remarks}

\begin{remark}
\leavevmode\newline
\indent $\bullet$ Observe that the right hand side of Eq.\ \eqref{PrincMorphLGBEquiv} is well-defined because of Eq.\ \eqref{PrincMorphoverBaseMap}, $(p, h) \in \mathcal{P}^\prime * \mathcal{H}$ and Def.\ \ref{def:LGB morphism}, that is,
\bas
\mleft( \pi_{\mathcal{G}} \circ F \mright)(h)
&\stackrel{\text{\ref{def:LGB morphism}}}{=}
\mleft( f \circ \pi_{\mathcal{H}} \mright)(h)
\stackrel{\pi^\prime(p) = \pi_{\mathcal{H}}(h)}{=}
\mleft( f \circ \pi^\prime \mright)(p)
\stackrel{\eqref{PrincMorphoverBaseMap}}{=}
\mleft( \pi \circ H \mright)(p),
\eas
thus,
\bas
\bigl( H(p), F(h) \bigr)
&\in
\mathcal{P} * \mathcal{G}
=
\pi^* \mathcal{G}.
\eas
Furthermore, by additionally using Remark \ref{rem:LGBPrincDefDiscussion} we know that LGB actions preserve the fibres of the principal bundles, therefore both, $H(p \cdot h)$ and $H(p) \cdot F(h)$, are over the same base point $\mleft(f \circ \pi^\prime\mright)(p)$, so that Eq.\ \eqref{PrincMorphLGBEquiv} as a whole is well-defined.

$\bullet$ Also observe that one can conclude that $(F,f)$ has to be an LGB morphism in order to have a satisfied and well-defined Eq.\ \eqref{PrincMorphLGBEquiv}, assuming that $H$ is a map over $f$. To well-define the right hand side, $F$ has to be a map over $f$, since $H$ is defined over $f$. Assuming Eq.\ \eqref{PrincMorphLGBEquiv}, for $h^\prime \in \mathcal{H}_{\pi^\prime(p)}$ we have
\bas
H\mleft(p \cdot h^\prime h \mright)
&=
H(p) \cdot F\mleft(h^\prime h\mright),
\eas
but we also get by associativity
\bas
H\mleft(p \cdot h^\prime h \mright)
&=
H\mleft(p \cdot h^\prime \mright) \cdot F(h)
=
H(p) \cdot F\mleft(h^\prime\mright) ~ F(h).
\eas
Using that $\mathcal{G}$ acts simply transitive on $\mathcal{P}$, we derive
\bas
F\mleft(h^\prime h\mright)
&=
F\mleft(h^\prime\mright) ~ F(h).
\eas

$\bullet$ Assume $H$ is a diffeomorphism, then $F$ is an LGB isomorphism. For this we only have to show that $F$ is a diffeomorphism by Def.\ \ref{def:LGB morphism}; by Remark \ref{LGBMOrphismRemark}, also $f$ is then a diffeomorphism. Thence, let us show that $F$ is a diffeomorphism. This follows by the fact that the LGB actions are simply transitive on the fibres of $\mathcal{P}^\prime$ and $\mathcal{P}$, \textit{i.e.}\ orbit maps are diffeomorphisms, also recall Remark \ref{rem:LGBPrincDefDiscussion}. Denoting the orbit maps inherited by the action on $\mathcal{P}^\prime$ and $\mathcal{P}$ by $\Phi^{\mathcal{P}^\prime}$ and $\Phi^{\mathcal{P}}$, respectively, we can write
\bas
F(h)
&=
\mleft(
	\mleft(\Phi^{\mathcal{P}}_{H(p)}\mright)^{-1} \circ H \circ \Phi^{\mathcal{P}^\prime}_p 
\mright)(h)
\eas
for arbitrary $p$ by rewriting Eq.\ \eqref{PrincMorphLGBEquiv}. Hence, $F$ is a diffeomorphism as the composition of diffeomorphisms.

Last but not least, it follows that $H^{-1}$ is then $\mathcal{G}$-equivariant in the sense of 
\bas
H^{-1}(q \cdot g) &= H^{-1}(q) \cdot F^{-1}(g)
\eas
for all $(q, g) \in \mathcal{P}*\mathcal{G}$, which is well-defined by similar arguments as before, especially because the inverses are maps over $f^{-1}$. To prove this, observe that there is a unique $(p, h) \in \mathcal{P}^\prime * \mathcal{H}$ such that $H(p) = q$ and $F(h) = g$ due to the fact that both are bijective maps over $f$.\footnote{In fact, it is easy to prove that $(H, F):\mathcal{P}^\prime * \mathcal{H} \to \mathcal{P} * \mathcal{G}$ is an LGB isomorphism (over $H$); also recall Cor.\ \ref{cor:PullbackLGB}.} Then
\bas
H^{-1}(q \cdot g)
&=
H^{-1}\bigl( \underbrace{H(p) \cdot F(h)}_{= H(p \cdot h)} \bigr)
=
p \cdot h
=
H^{-1}(q) \cdot F^{-1}(g).
\eas
Thence, Def.\ \ref{def:MorphOfPrincBundles} is a valid definition for a principal bundle morphism, because it is easy to check that the principal bundle atlas on $\mathcal{P}$ (consisting of $\varphi_i$ related to an open covering $\mleft(U_i\mright)_i$ of $M$) is related to the principal bundle atlas on $\mathcal{P}^\prime$ by
\bas
\mleft(F^{-1} \circ \varphi_i \circ H\mright)|_{f^{-1}(U_i)}.
\eas

$\bullet$ Finally, observe that we have locally an isomorphism of every principal bundle $\mathcal{P}$ to $\mathcal{G}$: Fix a principal bundle chart $\varphi: \mathcal{P}|_U \to \mathcal{G}|_U$ ($U$ some open subset of $M$). Then take $H \coloneqq \varphi$, $F \coloneqq \mathds{1}_{\mathcal{G}|_U}$ and $f \coloneqq \mathds{1}_U$; by the definition of a principal bundle chart this gives a principal bundle isomorphism. Therefore one could say that every principal bundle is locally "trivial" in the sense of being an LGB; recall Ex.\ \ref{ex:TrivialPrincAsLGB}. 

Additionally using Remark \ref{rem:LGBPrincDefDiscussion} we have locally an isomorphism of $\mathcal{P}$ to a trivial LGB; keeping the same notation as in Remark \ref{rem:LGBPrincDefDiscussion}, choose $H \coloneqq \xi_i$, $F \coloneqq \phi_i$ and $f \coloneqq \mathds{1}_{U_i}$. Hence, locally every principal bundle is also classical in the sense of Ex.\ \ref{ex:TheCLassicalPrincAsEx}.
\end{remark}

As expected, there is a natural isomorphism induced by local sections of $\mathcal{P}$, which are, however, no trivializations in general. In other words, by Remark \ref{rem:LGBPrincDefDiscussion} we know that orbit maps through the fibres of $\mathcal{P}$ are equivariant diffeomorphisms, we want to show the same for the orbit map through a section of $\mathcal{P}$.

\begin{lemmata}{Local sections of principal bundles induce isomorphisms to the structural LGB}{SectionsNowInduceIsomToLGBsNotNecTriv}
Let $\mathcal{G} \stackrel{\pi_{\mathcal{G}}}{\to} M$ be an LGB over a smooth manifold $M$, and $\mathcal{P} \to M$ a principal $\mathcal{G}$-bundle. Let $s: U \to \mathcal{P}$ be a smooth local section of $\mathcal{P}$ over an open subset $U$ of $M$. Then the orbit map $\Phi_s$ through $s$, given as in Def.\ \ref{def:LRTranslations} by
\bas
\mathcal{G}|_U &\to \mathcal{P}|_U,\\
g &\mapsto s_{\pi_{\mathcal{G}}(g)} \cdot g,
\eas
is a base-preserving principal bundle isomorphism w.r.t.\ $\mathds{1}_{\mathcal{G}|_U}$.
\end{lemmata}

\begin{remark}
\leavevmode\newline
As in the typical formulation of gauge theory, we have an isomorphism induced by sections, but it is not necessarily a trivialization as fibre bundle, so that we do not necessarily also have a "classical" bundle; it is a trivialization in the sense of Ex.\ \ref{ex:TrivialPrincAsLGB}. Due to the similarity with the "classical" statement, we therefore were speaking of the "trivial" principal bundle in Ex.\ \ref{ex:TrivialPrincAsLGB}. Of course, since every LGB is locally trivial, we can find a typical trivialization by taking a "local-enough" section.

The nomenclature about calling LGBs "trivial" principal bundles is also introduced and discussed in \cite[\S 5.7, third and fourth part of Remark 5.34, page 145]{GroupoidBasedPrincipalBundles} for groupoid-based principal bundles.
\end{remark}

\begin{proof}[Proof of Lemma \ref{lem:SectionsNowInduceIsomToLGBsNotNecTriv}]
\leavevmode\newline
The proof is similar to the "classical" statement as \textit{e.g.}\ given in \cite[\S 4.2, proof of Lemma 4.2.7, page 210ff.]{Hamilton}, but generalized since we have to treat a possible non-triviality of $\mathcal{G}|_U$. As already discussed, the orbit map $\Phi_s$ is well-defined because of $\pi_{\mathcal{P}} \circ s = \mathds{1}_{U}$ such that $\mleft(s_{\pi_{\mathcal{G}}(g)}, g\mright) \in \mathcal{P}*\mathcal{G}$. Via Rem.\ \ref{SmoothnessOfACtionTranslations} we also know that $\Phi_s$ is also smooth,
 %Let us denote with $\Phi: \mathcal{P}* \mathcal{G} \to \mathcal{G}$ the right $\mathcal{G}$-action $(p, g) \mapsto p \cdot g$ on $\mathcal{P}$, then
%\ba\label{RewritingLocalTrivOfPrinc}
%\Phi_s(g)
%&=
%\Phi\mleft( s_{\pi_{\mathcal{G}}(g)}, g\mright)
%=
%\bigl(\Phi \circ \mleft( \pi_{\mathcal{G}}^*s , \mathds{1}_{\mathcal{G}} \mright) \bigr) (g,g).
%\ea
%Thence, $\Phi_s$ is clearly a smooth map as the composition of smooth maps, that is, $\Phi$ is composed with the restriction of the smooth map $\mleft( \pi_{\mathcal{G}}^*s, \mathds{1}_{\mathcal{G}} \mright): \mathcal{G} \times \mathcal{G} \to \mathcal{G}$, which is given by $(g, q) \mapsto \mleft( s_{\pi_{\mathcal{G}(g)}} , q \mright)$, and it is clearly smooth if restricted to the diagonal as embedded submanifold.
%By Def.\ \ref{def:LiegroupACtion} we have
%\bas
%\mleft( \pi_{\mathcal{P}} \circ \Phi_s \mright)(g)
%&=
%\pi_{\mathcal{P}}\mleft( s_{\pi_{\mathcal{G}}(g)} \cdot g \mright)
%=
%\pi_{\mathcal{G}}(g),
%\eas
%thus, $\Phi_s$ is base-preserving, similarly it follows that $\Phi_s$ is $\mathcal{G}$-equivariant w.r.t.\ $\mathds{1}_{\mathcal{G}}$ by making use of $\pi_{\mathcal{G}}(g q) = \pi_{\mathcal{G}}(g)$ for all $g, q \in \mathcal{G}*\mathcal{G}$. Furthermore, $\Phi_s$ is bijective since the right $\mathcal{G}$-action on $\mathcal{P}$ is simply transitive, that is, when we restrict $\Phi_s$ to an $x \in U$, then it is a map
%\bas
%\mathcal{G}_x &\to \mathcal{P}_x,\\
%g &\mapsto s_x \cdot g,
%\eas
%thence, as expected 
and it is point-wise the orbit map $\Phi_{s_x}$ through $s_x$ which is a $\mathcal{G}_x$-equivariant diffeomorphism by Remark \ref{rem:LGBPrincDefDiscussion}. So, $\Phi_s$ is fibre-wise bijective and therefore bijective as a whole since it is base-preserving, and it is $\mathcal{G}$-equivariant w.r.t.\ $\mathds{1}_{\mathcal{G}}$.

Now we want to use the inverse function theorem to show that its inverse is also smooth. Once we know that the tangent map/total derivative $\mathrm{D}_g\Phi_s: \mathrm{T}_g \mathcal{G} \to \mathrm{T}_{s_{x} \cdot g}\mathcal{P}$ is an isomorphism of vector spaces for all $g \in \mathcal{G}|_U$, then we know by the inverse function theorem that $\Phi_s^{-1}$ is smooth. Hence, we will now show that $\mathrm{D}_g\Phi_s$ is injective, then it has to bijective by dimensional reasons ($\mathrm{dim}(\mathcal{G}) = \mathrm{dim}(\mathcal{P})$) so that we are done. Let us denote with $\Phi: \mathcal{P}* \mathcal{G} \to \mathcal{G}$ the right $\mathcal{G}$-action $(p, g) \mapsto p \cdot g$ on $\mathcal{P}$, then
\bas
\Phi_s(g)
&=
\Phi\mleft( s_{\pi_{\mathcal{G}}(g)}, g\mright)
=
\bigl(\Phi \circ \mleft( \pi_{\mathcal{G}}^*s , \mathds{1}_{\mathcal{G}} \mright) \bigr) (g,g).
\eas
Then $\mathrm{D}_g\Phi_s$ is given by Thm.\ \ref{thm:DiffOfLGBAction},
\ba
\mathrm{D}_g \Phi_s (Y)
&=
\mathrm{D}_{s_x}r_\sigma\bigl( \mathrm{D}_x s(\omega) \bigr)
	+ \mleft.\oversortoftilde{\mleft( \mu_{\mathcal{G}} \mright)_g\bigl( Y - \mathrm{D}_x \sigma(\omega) \bigr) }\mright|_{s_x \cdot g}
\nonumber\\
&=
\mathrm{D}_x \mleft( r_\sigma \circ s \mright)(\omega)
	+ \mleft.\oversortoftilde{\mleft( \mu_{\mathcal{G}} \mright)_g\bigl( Y - \mathrm{D}_x \sigma(\omega) \bigr) }\mright|_{s_x \cdot g}
	\label{DiffOfOrbitMap}
\ea
for all $Y \in \mathrm{T}_g \mathcal{G}$, where $x \coloneqq \pi_{\mathcal{G}}(g)$, $\sigma \in \Gamma(\mathcal{G}|_U)$\footnote{W.l.o.g.\ we assume that $\sigma$ is defined on $U$, otherwise "make $U$ smaller around $x$".} with $\sigma_{x} = g$ and $\omega \coloneqq \mathrm{D}_{g}\pi_{\mathcal{G}}(Y) \in \mathrm{T}_xM$. We want to decompose $\Phi_s$ now. Observe that $Y - \mathrm{D}_x \sigma(\omega) \in \mathrm{V}_g \mathcal{G}$, \textit{i.e.}\ it is vertical in $\mathcal{G}$ due to the fact that
\ba\label{DecomposingTheLGBVectorfieldsWithsections}
\mathrm{D}_g\pi_{\mathcal{G}}
\mleft(Y - \mathrm{D}_x \sigma(\omega)\mright)
&=
\underbrace{\mathrm{D}_g\pi_{\mathcal{G}}(Y)}_{= \omega}
	- \underbrace{\mathrm{D}_g\pi_{\mathcal{G}} \mleft(\mathrm{D}_x \sigma(\omega)\mright)}_{= \omega}
=
0
\ea
because of
\bas
\pi_{\mathcal{G}} \circ \sigma
&=
\mathds{1}_U
\eas
such that
\ba\label{DsigmaIsASPlitting}
\mathrm{D}\pi_{\mathcal{G}} \circ \mathrm{D}\sigma
&=
\mathds{1}_{\mathrm{T}M|_U}.
\ea
Usually, we just make use of that without further mention; we repeat this trivial fact in order to emphasize that $\mathrm{D}\sigma$ is injective because $\mathds{1}_{\mathrm{T}M|_U}$ is bijective, and to emphasize that we have
\bas
%\mathrm{T}_g\mathcal{G}
%&=
\mathrm{Im}\mleft( \mathrm{D}_x\sigma \mright)
	\cap \mathrm{Ker}\mleft( \mathrm{D}_{g}\pi_{\mathcal{G}} \mright)
&=
\{0\}
\eas
(the image of $\mathrm{D}_x\sigma$ intersects trivially with the kernel of $\mathrm{D}_{g}\pi_{\mathcal{G}}$). The injectivity of $\mathrm{D}\sigma$ implies that the dimension of its image satisfies
\bas
\mathrm{dim}\bigl(\mathrm{Im}\mleft( \mathrm{D}_x\sigma \mright)\bigr)
&=
\mathrm{dim}(M),
\eas
so that, in total, we know by dimensional reasons
\bas
\mathrm{T}_g\mathcal{G}
&\cong
\mathrm{Im}\mleft( \mathrm{D}_x\sigma \mright)
	\oplus \mathrm{Ker}\mleft( \mathrm{D}_{g}\pi_{\mathcal{G}} \mright)
=
\mathrm{Im}\mleft( \mathrm{D}_x\sigma \mright)
	\oplus \mathrm{V}_g \mathcal{G}.
\eas
We can decompose $Y$ accordingly by Eq.\ \eqref{DecomposingTheLGBVectorfieldsWithsections},
\bas
Y
&=
\underbrace{\mathrm{D}_x\sigma(\omega)}_{\eqqcolon Y^h}
	+ \underbrace{Y - \mathrm{D}_x\sigma(\omega)}_{\eqqcolon Y^v}
=
Y^h + Y^v,
\eas
$v$ stands for the vertical part and $h$ for its complementary part ("horizontal"). In fact, as it is well-known, a section $\sigma$ of a bundle induces a splitting of the short exact sequence of vector bundles
\bes
	\begin{tikzcd}
		\sigma^*\mathrm{V}\mathcal{G}|_U \arrow[hook]{r} & \sigma^*\mathrm{T}\mathcal{G}|_U \arrow[two heads]{r}{\mathrm{D}\pi_{\mathcal{G}}} & \mathrm{T}M|_U,
	\end{tikzcd}
\ees
and $\mathrm{D}\sigma$ is a splitting/section of this sequence
so that we have $\sigma^*\mathrm{T}\mathcal{G} \cong \mathrm{Im}(\mathrm{D}\sigma) \oplus \sigma^*\mathrm{V}\mathcal{G}$.\footnote{$\sigma$ is actually an embedding, and thus $\mathrm{Im}(\mathrm{D}\sigma)$ is a well-defined subbundle isomorphic to $\mathrm{T}M$.} Furthermore, again due to Eq.\ \eqref{DsigmaIsASPlitting}, we have $\mathrm{Im}(\mathrm{D}_x\sigma) \cong \mathrm{T}_xM$ as vector spaces by $Y^h \mapsto \mathrm{D}_g\pi_{\mathcal{G}}\mleft(Y^h\mright) = \mathrm{D}_g\pi_{\mathcal{G}}(Y) = \omega$ since $Y^v$ is vertical. Hence, we can rewrite Eq.\ \eqref{DiffOfOrbitMap} as a map
\bas
\mathrm{T}_xM \oplus \mathrm{V}_g \mathcal{G} &\to \mathrm{T}_{s_{x} \cdot g}\mathcal{P},\\
\mleft(\omega,~ Y^v\mright) &\mapsto
\mathrm{D}_x \mleft( r_\sigma \circ s \mright)(\omega)
	+ \mleft.\oversortoftilde{\mleft( \mu_{\mathcal{G}} \mright)_g\bigl( Y^v \bigr) }\mright|_{s_x \cdot g}.
\eas

These arguments apply to any section of a bundle, hence, we repeat this now. Observe that $r_\sigma \circ s$ is also a section of $\mathcal{P}|_U$ due to Def.\ \ref{def:LiegroupACtion}, that is,
\bas
\pi_{\mathcal{P}}\mleft(s_y \cdot \sigma_y\mright)
&=
\pi_{\mathcal{P}}\mleft(s_y\mright)
=
y
\eas
for all $y \in M$. As before, we split
\bas
\mathrm{T}_{s_x\cdot g} \mathcal{P}|_U
&\cong
\mathrm{Im}\bigl( \mathrm{D}_x(r_\sigma\circ s) \bigr)
	\oplus \mathrm{V}_{s_x \cdot g} \mathcal{P},
\eas
and we identify $\mathrm{T}_xM \cong \mathrm{Im}\bigl( \mathrm{D}_x(r_\sigma\circ s) \bigr)$ as vector spaces, now via $\omega \mapsto \mathrm{D}_x(r_\sigma\circ s) (\omega)$ due to Eq.\ \eqref{DsigmaIsASPlitting} but with $r_\sigma \circ s$ playing the role as section. Additionally by Def.\ \ref{def:FundVecs} and Remark \ref{rem:FundVecsNotations} we know that fundamental vector fields are vertical, and so we can further rewrite Eq.\ \eqref{DiffOfOrbitMap} as a map
\bas
\mathrm{T}_xM \oplus \mathrm{V}_g \mathcal{G} 
&\to 
\mathrm{T}_x M \oplus \mathrm{V}_{s_x \cdot g} \mathcal{P},
\\
\mleft(\omega, ~Y^v\mright) &\mapsto
\mleft(
	\omega,~
	\mleft.\oversortoftilde{\mleft( \mu_{\mathcal{G}} \mright)_g\bigl( Y^v \bigr) }\mright|_{s_x \cdot g}
\mright).
\eas
Thus, $\mathrm{D}_g\Phi_s$ is the pair of two linear independent maps $\mathrm{T}_x M \to \mathrm{T}_xM$ and $\mathrm{V}_g\mathcal{G} \to \mathrm{V}_{s_x \cdot g}\mathcal{P}$. The former is clearly an isomorphism, and the latter is by definition of fundamental vector fields and the Maurer-Cartan form of the shape as in Eq.\ \eqref{ClassicalWayToWriteLeibnizRuleWithLeftPushForwardInsteadOfMCForm}, \textit{i.e.}\
\bas
Y^v &\mapsto \mathrm{D}_g \Phi_{s_x}(Y^v).
\eas
By Remark \ref{rem:LGBPrincDefDiscussion}, $\Phi_{s_x}$ is a $\mathcal{G}_x$-equivariant diffeomorphism $\mathcal{G}_x \to \mathcal{P}_x$, thence, $\mathrm{D}_g \Phi_{s_x}: \mathrm{T}_g \mathcal{G}_x = \mathrm{V}_g\mathcal{G} \to \mathrm{T}_{s_x \cdot g}\mathcal{P}_x = \mathrm{V}_{s_x \cdot g}\mathcal{P}_x$ is an isomorphism of vector spaces. Finally we can conclude that $\mathrm{D}_g \Phi_s$ is a linear isomorphism as the sum of two linear independent isomorphisms. As argued earlier, the inverse function theorem finishes now the proof.
\end{proof}

Usually, one likes to think about the choice of sections as a choice of a coordinate transformation as in \cite[\S 4.2, Remark 4.2.21, page 220]{Hamilton}. This is due to that the gauge theory usually corresponds to a formulation via a trivial LGB, which we will understand later, but already have an idea of by \textit{e.g.}\ Ex.\ \ref{ex:TrivialLGBAction} and \ref{ex:TheCLassicalPrincAsEx}. Then we have a local trivialization of $\mathcal{P}$ such that one usually thinks of the choice of a section as a choice of coordinate system.

However, we now learned that on a more general scale this is not completely what is happening. The idea of LGBs and principal bundles are very similar; both are fibre bundles related to a Lie group and they carry an action which also restricts on each fibre. But the fibres of an LGB are Lie groups themselves, while the fibres of a principal bundle are "just" diffeomorphic to a Lie group in an equivariant way as outlined in Remark \ref{rem:LGBPrincDefDiscussion} and as given in their definition Def.\ \ref{def:PrinciBdleWithStruLGB}. One could view the fibres of $\mathcal{P}$ as having an "almost" Lie group structure, a Lie group structure without a designated neutral element. 

Lemma \ref{lem:SectionsNowInduceIsomToLGBsNotNecTriv} shows that the choice of a section of $\mathcal{P}$ is actually the choice of a designated neutral element, naturally inducing a Lie group structure in each fibre and thus an LGB structure, which may not be trivial. Aligning the more general definition of principal bundles with the definition of LGBs. For example set $\mathcal{P} = \mathcal{G} = c_G(P)$ as in Ex.\ \ref{ex:InnerLGBs}, where $P$ is a non-trivial classical principal bundle and the underlying Lie group $G$ is non-abelian; such an LGB is likely non-trivial but always carries a global section, for examples the neutral section $e$. We will see such an example later.

We want to use this now in order to study a certain pullback of principal bundles. For this we need to introduce general pullbacks of principal bundles; also recall again Cor.\ \ref{cor:PullbackLGB}.

\begin{corollaries}{Pullbacks of principal LGB-bundles are principal LGB-bundles}{PullBacksArePrincToo}
Let $M, N$ be smooth manifolds, $f: N \to M$ a smooth map, $\mathcal{G} \to M$ an LGB, and $\mathcal{P} \to M$ a principal $\mathcal{G}$-bundle. Then there is a unique (up to isomorphisms) principal $f^*\mathcal{G}$-bundle structure on $f^*\mathcal{P}$, such that the projection $\pi_2: f^*\mathcal{P} \to \mathcal{P}$ onto the second factor is a principal bundle morphism (over $f$) w.r.t.\ the projection $\pi_2^{\mathcal{G}}: f^*\mathcal{G} \to \mathcal{G}$ onto the second factor as LGB morphism, and such that $\pi_2|_x: (f^*\mathcal{P})_x \to \mathcal{P}_{f(x)}$ is a $\mathcal{G}_x$-equivariant diffeomorphism w.r.t.\ the LGB isomorphism $\mleft.\pi_2^{\mathcal{G}}\mright|_x$.
%
%$f^*\mathcal{P}$ is the \textbf{pullback principal bundle of $\mathcal{P}$ (under $f$)}.
\end{corollaries}

\begin{remark}
\leavevmode\newline
This was also stated in \cite[\S 5.7, second argument in Remark 5.34, page 145]{GroupoidBasedPrincipalBundles} for an even more general type of principal bundle, but without proof.
\end{remark}

\begin{proof}
\leavevmode\newline
As in the previous statements about pullback structures, this is a rather trivial and canonical construction. We have a right $f^*\mathcal{G}$-action on $f^*\mathcal{P}$ defined by
\bas
(x, p) \cdot (x, g)
&\coloneqq
(x, p \cdot g)
\eas
for all $(x, p) \in f^*\mathcal{P}$ and $(x, g) \in f^*\mathcal{G}$, that is, $x \in N$, $p\in \mathcal{P}_{f(x)}$ and $g \in \mathcal{G}_{f(x)}$. This is clearly an action $f^*\mathcal{P} * f^*\mathcal{G} \coloneqq \pi_1^*f^*\mathcal{G} \to f^*\mathcal{P}$, where $\pi_1$ is the canonical projection of $f^*\mathcal{P} \to N$ as fibre bundle. This action's restriction onto the fibres is simply transitive: A fibre of $f^*\mathcal{P}$ at $x$ is $\{x\} \times \mathcal{P}_{f(x)} \cong \mathcal{P}_{f(x)}$, similarly $(f^*\mathcal{G})_x \cong \mathcal{G}_{f(x)}$. Hence, the restriction of that action at $x \in N$ is the right $\mathcal{G}_{f(x)}$-action on $\mathcal{P}_{f(x)}$ such that the action is simply transitive.

A principal bundle atlas can be constructed by a pullback of principal bundle charts of $\mathcal{P}$, that is, let $\mleft(U_i\mright)_i$ be an open covering of $M$ over which we have $\mathcal{G}$-equivariant diffeomorphisms $\varphi_i: \mathcal{P}|_{U_i} \to \mathcal{G}|_{U_i}$. Then define
\bas
\mleft.f^*\mathcal{P}\mright|_{f^{-1}(U_i)} &\to \mleft.f^*\mathcal{G}\mright|_{f^{-1}(U_i)},\\
(x, p) &\mapsto
\mleft(f^*\varphi_i\mright)(x, p)
\coloneqq
\mleft( x, \mleft.\mleft(\varphi_i\mright)\mright|_{f(x)}(p) \mright),
\eas
which is well-defined and by construction a base-preserving $f^*\mathcal{G}$-equivariant smooth map, and this map is equivalent to $\mleft(\mathds{1}_{f^{-1}(U_i)}, \varphi_i\mright): N \times \mathcal{P} \to N \times \mathcal{G}$ restricted onto $f^*\mathcal{P}$ as an embedded submanifold of $N \times \mathcal{P}$. Therefore it is clearly a diffeomorphism, so that we can conclude that $f^*\mathcal{P}$ admits the structure as a principal $f^*\mathcal{G}$-bundle.

The last part of the proof about the uniqueness of the structure is precisely as in the proof of Cor.\ \ref{cor:PullbackLGB}, just replace the property of being an LGB morphism with being a principal bundle morphism w.r.t.\ the LGB morphism $\pi_2^{\mathcal{G}}$ (essentially, replace homomorphism with equivariance). Keeping the same notation as in the proof of Cor.\ \ref{cor:PullbackLGB}, we get analogously
\bas
\varphi_i \circ \pi_2 \circ \psi_i^{-1}
&=
\pi_2^{\mathcal{G}} \circ f^*\varphi_i \circ \psi_i^{-1}.
\eas
Then start by making use of the point-wise behaviour of $\pi_2$ and $\pi_2^{\mathcal{G}}$ and proceed similarly as in the proof of Cor.\ \ref{cor:PullbackLGB} to conclude the proof.
\end{proof}

\begin{definitions}{Pullback principal bundle}{PullbackPrincBundleDef}
Let $M, N$ be smooth manifolds, $f: N \to M$ a smooth map, $\mathcal{G} \to M$ an LGB, and $\mathcal{P} \to M$ a principal $\mathcal{G}$-bundle. Then we call the principal $f^*\mathcal{G}$-bundle structure on $f^*\mathcal{P}$ as given in Cor.\ \ref{cor:PullBacksArePrincToo} the \textbf{pullback principal bundle of $\mathcal{P}$ (under $f$)}.

We will refer to this structure often without further mention.
\end{definitions}

We can actually show that $\mathcal{P}*\mathcal{G}$ is not only $\pi^*\mathcal{G}$ by definition but it is also isomorphic to $\pi^*\mathcal{P}$. For this recall Ex.\ \ref{ex:TrivialPrincAsLGB}, \textit{i.e.}\ LGBs are "trivially" also principal bundles, hence we know that $\mathcal{P} * \mathcal{G}$ is a principal $\pi^*\mathcal{G}$-bundle.

\begin{corollaries}{$\mathcal{P}*\mathcal{G}$ is the pullback of $\mathcal{P}$ along its projection}{ProductSpaceIsPItself}
Let $\mathcal{G} \to M$ be an LGB over a smooth manifold $M$ and $\mathcal{P} \stackrel{\pi}{\to} M$ a principal $\mathcal{G}$-bundle. Then we have a base-preserving principal bundle isomorphism 
\bas
\mathcal{P}*\mathcal{G} &\cong \pi^*\mathcal{P}
\eas
w.r.t.\ the LGB isomorphism given as the identity map on $\mathcal{P} * \mathcal{G} = \pi^*\mathcal{G}$.
\end{corollaries}

\begin{proof}
\leavevmode\newline
We have a global section of the pullback principal bundle $\pi^*\mathcal{P} \to \mathcal{P}$ given by
\bas
\mathcal{P} &\to \pi^*\mathcal{P},\\
p &\mapsto (p, p).
\eas
This is clearly a well-defined global section of $\pi^*\mathcal{P}$, so that by Lemma \ref{lem:SectionsNowInduceIsomToLGBsNotNecTriv} we achieve the desired isomorphism $\pi^*\mathcal{P} \cong \pi^*\mathcal{G} = \mathcal{P} * \mathcal{G}$.
\end{proof}

\begin{remark}\label{AlternativePrincBdlDef}
\leavevmode\newline
By Lemma \ref{lem:SectionsNowInduceIsomToLGBsNotNecTriv} this isomorphism is explicitly given by
\bas
\mathcal{P}*\mathcal{G} &\to \pi^*\mathcal{P},\\
(p, g) &\mapsto (p,p) \cdot (p, g) = (p, p \cdot g).
\eas
In fact, the cited reference of Def.\ \ref{def:PrinciBdleWithStruLGB}, \cite[simplification of the beginning of \S 5.7, page 144f.]{GroupoidBasedPrincipalBundles}, takes the existence of such a diffeomorphism as the essential sole part of defining principal bundles. Such an approach essentially avoids the second part of Def.\ \ref{def:PrinciBdleWithStruLGB}.
\end{remark}

\begin{remarks}{Why "principal"?}{ItIsAPrincipalAction}
The last result and remark also outline why we are speaking of a \textit{principal} bundle; this is similar to \cite[\S 4.2.2, page 212ff.]{Hamilton}. As in \cite[\S 3.7, Def.\ 3.7.24, page 159]{Hamilton} one could say that a \textbf{principal} $\mathcal{G}$-action on a manifold $N$ (recall Def.\ \ref{def:LiegroupACtion}) is a \textbf{free} action, \textit{i.e.}\ orbit maps are injective, such that 
\bas
N * \mathcal{G} &\to N \times N,\\
(p, g) &\mapsto (p, p \cdot g)
\eas
is a closed map. In our case, $N = \mathcal{P}$, we clearly have a free action, and we just have shown that that map is closed, because we have the composition of maps
\bas
\mathcal{P} * \mathcal{G} &\to \pi^*\mathcal{P} \to \mathcal{P} \times \mathcal{P},\\
(p, g) &\mapsto (p, p \cdot g) \mapsto (p, p \cdot g).
\eas
By Remark \ref{AlternativePrincBdlDef} the first arrow is a diffeomorphism and thence a closed map. The second arrow is the inclusion, an embedding because $\pi^*\mathcal{P}$ is a closed embedded submanifold of $\mathcal{P} \times \mathcal{P}$ as the restriction of the fibre bundle $\mathcal{P} \times \mathcal{P} \stackrel{\mathds{1}_{\mathcal{P}} \times \pi}{\to} \mathcal{P} \times M$ along the graph of $\pi$; see \textit{e.g.}\ \cite[\S 4.1, proof of Thm.\ 4.1.17, page 204ff.]{Hamilton}. Additionally, by the continuity of $\pi$, the graph $\Gamma$ of $\pi$ is closed and thus $\pi^*\mathcal{P} = \mleft(\mathds{1}_{\mathcal{P}} \times \pi\mright)^{-1}(\Gamma)$, too. Using this, it is a quick exercise to show that a set closed in $\pi^*\mathcal{P}$ is also closed in $\mathcal{P} \times \mathcal{P}$, thus the second arrow as inclusion is also closed. Hence, the whole composition is closed.

This knowledge should allow us to carry over a lot of similar results related to principal actions, as in \cite[\S 3.7.5, page 159ff.]{Hamilton}\ and \cite[\S 4.2.2, page 212ff.]{Hamilton}. Essentially Remark \ref{AlternativePrincBdlDef} allows us to look at the quotient $\mathcal{P} \Big/ \mathcal{G}$ by using Godement's Theorem as given in \cite[\S 3.7, Thm.\ 3.7.10, page 155]{Hamilton}, leading to a manifold structure on $\mathcal{P} \Big/ \mathcal{G}$ and maybe similarly leading to a principal bundle structure on $\mathcal{P} \to \mathcal{P} \Big/ \mathcal{G}$.
\end{remarks}

Let us conclude our discussion about principal bundles by introducing a typical label.

\begin{definitions}{Gauges of a principal bundle}{GaugesOfPrincipalBundles}
Let $\mathcal{G} \to M$ be an LGB over a smooth manifold $M$, and $\mathcal{P} \to M$ a principal $\mathcal{G}$-bundle. A \textbf{gauge} of $\mathcal{P}$ is a section of $\mathcal{P}$. If the section is globally defined on $M$, then we speak of a \textbf{global gauge}, otherwise we may just say \textbf{local gauge} or just gauge.
\end{definitions}

By Lemma \ref{lem:SectionsNowInduceIsomToLGBsNotNecTriv}, a gauge corresponds to a "$\mathcal{G}$-ization" of $\mathcal{P}$, not necessarily a trivialization.

\subsection{Generalized distributions and connections}\label{ConnectionSubsection}

\begin{remarks}{References for connections on principal LGB-bundles}{AnotherPreprintWork}
It is important to mention beforehand that there is another great paper, \cite{OtherPreprintAboutConnection}, which also discusses the notion of connections on principal LGB-bundles. When I wrote this subsection about the connection, I was not aware of this preprint so that there will be some shared results which I found independently of the other authors. This will be obvious due to the fact that our approach is different. Both, this and the other paper, lead to the same sense of connection (based on the same idea of the behaviour of the parallel transport), but are explicitly not directly the same formulation. More like equivalent definitions of the same object, two sides of the same coin, so that it should be worth it to read both research works. \cite{OtherPreprintAboutConnection} sole purpose was in defining connections, while we want to discuss this sense of connection in the context of Yang-Mills gauge theory, also pinpointing what happens for Yang-Mills-Higgs gauge theories. Hence, our approach will be different, we will use sections of the LGB to formulate connections, simplifying certain explicit formulas; while \cite{OtherPreprintAboutConnection} avoids using sections and rather uses formulations just looking on the vertical parts of a vector or by rephrasing the notions via the total derivative of the LGB action. Another example of difference is that we are also introducing a generalisation of the Darboux derivative and phrasing important formulas with this derivative. Furthermore, we are not going to fix a specific connection on $\mathcal{G}$ for the principal bundle's connection (other than a horizontal distribution), while \cite{OtherPreprintAboutConnection} did fix a more specific type of connection on $\mathcal{G}$, and our curvature will be different and more general as already pointed out in research like \cite{CurvedYMH} and \cite{MyThesis}.
\end{remarks}

Finally, let us now define the gauge theory, starting with horizontal distributions. 
%For readability, we start with a very easy toy model which should help in understanding the general definitions. 
We expect a basic understanding of horizontal distributions and their relationship to what we call connections (on principal and vector bundles). The bare-bones start with horizontal distributions.

\begin{definitions}{Horizontal distribution, \newline \cite[\S 5.1.2, Def.\ 5.1.6, page 260; without the symmetry along right-translations here]{Hamilton}}{EhresmannConnectionBasics}
Let $F \to M$ be a fibre bundle over a smooth manifold $M$. Then a \textbf{horizontal distribution/bundle of $F$} is a smooth subbundle $\mathrm{H}F$ of $\mathrm{T}F$ with
\bas
\mathrm{T}F
&=
\mathrm{H}F \oplus \mathrm{V}F.
\eas
For $p \in M$ the fibre is denoted by $\mathrm{H}_p F$, which we may call a \textbf{horizontal tangent space}.
\end{definitions}

As usual for gauge theory we will understand connections as horizontal distributions with a certain symmetry along the fibres in order to assure a certain behaviour of the gauge transformation of what physicists call minimal coupling; in mathematical words, in order to assure to be able to define a connection on associated vector bundles, and to assure a certain transformation of the associated curvature.
%to assure that connection forms have a relationship to a form defined on the spacetime/basis itself.\footnote{Recall that connection 1-forms on principal bundles (in the classical gauge theory) can be viewed as a 1-form on the spacetime with values in the Atiyah bundle. We usually also want that the connection acting on a section $X$, denoted by $\nabla X$ in the case of vector bundles, still has values in that vector bundle as a bundle over the spacetime/basis and not over its tangent bundle, but this may be the case without further symmetry of the horizontal distribution.}
To do so it is useful if this symmetry is similar to the symmetry carried in the vertical structure; recall Def.\ \ref{def:FundVecs} and its remark \ref{rem:FundVecsNotations}. The following is a straightforward generalization of what one knows in the typical formulation of gauge theory, see \textit{e.g.}\ \cite[\S 5.1, part 2 to 4 of Prop.\ 5.1.3, page 258f]{Hamilton}.

\begin{corollaries}{The natural invariance of the vertical bundle of $\mathcal{P}$}{VerticalBundleOfPrincIsNearlyAsUsual}
Let $\mathcal{G} \to M$ be an LGB over a smooth manifold $M$, $\mathcal{g}$ its LAB, and $\mathcal{P}\stackrel{\pi}{\to} M$ a principal $\mathcal{G}$-bundle. Then
\ba\label{SymmetryOfTheVerticalBundle}
\mathrm{D}_p r_g\mleft( \mathrm{V}_p \mathcal{P} \mright)
&=
\mathrm{V}_{p \cdot g} \mathcal{P}
\ea
for all $(p, g) \in \mathcal{P} * \mathcal{G}$, and we have an isomorphism of vector bundles
\ba
\mathrm{V}\mathcal{P}
&\cong
\pi^*\mathcal{g}
\ea
given by
\ba
\pi^*\mathcal{g} &\to \mathrm{V}\mathcal{P},\nonumber\\
(p, \nu) &\mapsto \widetilde{\nu}_p.\label{FundVecAsMapIsIsom}
\ea
\end{corollaries}

\begin{remarks}{Extending the notation of fundamental vector fields}{FundVecNotationOnPullbackBundle}
For $\mu \coloneqq (p, \nu)$ we may also write
\bas
\widetilde{\mu}_p
&\coloneqq
\widetilde{\nu}_p,
\eas
which simplifies the notation in certain circumstances.
\end{remarks}

\begin{proof}[Proof of Cor.\ \ref{cor:VerticalBundleOfPrincIsNearlyAsUsual}]
\leavevmode\newline
Recall Rem.\ \ref{rem:LGBPrincDefDiscussion} for this proof. By definition it is clear that each fibre $\mathcal{P}_x$ ($x \in M$) is a principal $\mathcal{G}_x$-bundle over $\{x\}$ whose $\mathcal{G}_x$-action is the $\mathcal{G}$-action restricted to $x$, and thus we know
\bas
\mathrm{D}_pr_g\mleft(\mathrm{V}_p\mathcal{P}_x\mright)
=
\mathrm{V}_{p\cdot g} \mathcal{P}_x
\eas
for all $p \in \mathcal{P}_x$ and $g \in \mathcal{G}_x$; see \textit{e.g.}\ \cite[\S 5.1, fourth part of Prop.\ 5.1.3, page 258f.]{Hamilton}\ for such statements about principal Lie group bundles. Due to that $\mathcal{P}_x$ is a bundle over a point, we have $\mathrm{V}\mathcal{P}_x = \mathrm{T}\mathcal{P}_x = \mleft.\mathrm{V}\mathcal{P}\mright|_{\mathcal{P}_x}$. Thus,
\bas
\mathrm{D}_pr_g\mleft(\mathrm{V}_p\mathcal{P}\mright)
=
\mathrm{V}_{p\cdot g} \mathcal{P}.
\eas
Due to the fact that the $\mathcal{G}$-action is simply transitive we can derive that its induced $\mathcal{g}$-action $\rho$ is a vector bundle isomorphism: By Rem.\ \ref{rem:FundVecsAreLABActions} this LAB action $\rho$ is precisely \eqref{FundVecAsMapIsIsom} and $\rho$ has values in $\mathrm{V}\mathcal{P}$ by Eq.\ \eqref{LABActionAlongFibres}. We know that the orbit maps $\Phi_p: \mathcal{G}_x \to \mathcal{P}_x$ through $p \in \mathcal{P}_x$ are $\mathcal{G}_x$-equivariant diffeomorphisms, so that $\mathrm{D}_{e_x}\Phi_p: \mathcal{g}_x \to \mathrm{V}_p\mathcal{P}$ is a vector space isomorphism. But we also have
\bas
\rho(p, \nu)
&=
\mathrm{D}_{e_x}\Phi_p(\nu)
\eas
for all $\nu\in \mathcal{g}_x$, and therefore $\rho$ is fibre-wise an isomorphism of vector spaces such that it is an isomorphism of vector bundles.
\end{proof}

By this result, we would like to have Eq.\ \eqref{SymmetryOfTheVerticalBundle} also for a chosen horizontal distribution. However, its formulation leads to certain problems which we now want to discuss.

\subsubsection{Idea and motivation}\label{TheBigMotivationBehindEverything}

\begin{center}
%\begin{figure}[H]
%\caption{Pushforwards via right-multiplication of tangent vectors}  
%\label{figure:fibre bundle}
%\centering
\begin{tikzpicture}[scale=0.80]
\draw (0,-3.5) to[out=10,in=170] (4.25,-3.5) to[out=350,in=190] (8.5,-3.5);
\draw (1.25,2.5) to[out=280,in=90](1.5,0) to[out=270,in=80] (1.25,-2.5);
\filldraw [fill=gray!20!white, draw=white] (3.5,2.5) to[out=280,in=90](3.75,0) to[out=270,in=80] (3.5,-2.5) -- (5,-2.5)  to[out=80,in=270](5.25,0) to[out=90,in=280] (5,2.5) -- cycle;
\draw (4.25,2.5) to[out=280,in=95](4.41,1) to[out=275,in=85] (4.41,-1) to[out=265,in=80] (4.25,-2.5); %draw with middle section for precise point placement

\draw (4,-1) node {\textcolor[rgb]{1,0,0}{$p$}};
\draw [<-, draw=blue] (5,1) to[out=275,in=85] (5,-1);
\draw (5.5,0) node {\textcolor[rgb]{0,0,1}{$\cdot g$}};
\filldraw [fill=red, draw=red] (4.41,1) circle (1pt);
\filldraw [fill=red, draw=red] (4.41,-1) circle (1pt);
\draw (5.75,2.5) to[out=280,in=90](6,0) to[out=270,in=80] (5.75,-2.5);
\draw (7.25,2.5) to[out=280,in=90](7.5,0) to[out=270,in=80] (7.25,-2.5);
\draw (2.75,2.475) to[out=280,in=90](3,0) to[out=270,in=80] (2.75,-2.475);

%\draw (4.75,2) node {$\mathcal{P}_x$};
%\draw (4.95,1.5) node {$\cong G$};
\draw (9.5,-3.5) node {$M$};
\draw (5,2) node {$\mathcal{P}_U$};
\draw (3.5,-3.4) node {$($};
\filldraw [fill=red, draw=red] (4.25,-3.5) circle (1pt);
\draw (4.25,-3.9) node {\textcolor[rgb]{1,0,0}{$x$}};
\draw (5,-3.6) node {$)$};
\draw (4.25, -3) node {$U$};
\draw (0,0) node {$\mathcal{P}$};
\draw [->] (0,-0.4) -- (0,-3.1);
\draw (0.4,-1.75) node {$\pi$};
\draw (3.7,1) node {\textcolor[rgb]{1,0,0}{$p \cdot g$}};
%%%%%%%%%%%%%%%%%%%%%%%%%%%%%%%%%%%%%%%%%%%%%%%%%%%%%%%%%%%%%%%%%%%%%%%%%%%%%%%%%%%%
\draw (10.5,-3.5) to[out=10,in=170] (14.75,-3.5) to[out=350,in=190] (19,-3.5); %hori M
\filldraw [fill=gray!20!white, draw=white] (14,2.5) to[out=280,in=90](14.25,0) to[out=270,in=80] (14,-2.5) -- (15.5,-2.5)  to[out=80,in=270](15.75,0) to[out=90,in=280] (15.5,2.5) -- cycle; %gray area
\draw (11.75,2.5) to[out=280,in=90](12,0) to[out=270,in=80] (11.75,-2.5); %vert line 1
\draw (14.75,2.5) to[out=280,in=90](15,0) to[out=270,in=80] (14.75,-2.5); %3
\draw (16.25,2.5) to[out=280,in=90](16.5,0) to[out=270,in=80] (16.25,-2.5); %4
\draw (17.75,2.5) to[out=280,in=90](18,0) to[out=270,in=80] (17.75,-2.5); %5

\draw (13.25,2.475) to[out=280,in=90](13.5,0) to[out=270,in=80] (13.25,-2.475); %2
\filldraw [fill=blue, draw=blue] (15,0) circle (1pt);
\draw (15.3,0) node {\textcolor[rgb]{0,0,1}{$g$}};

\draw (19,0) node {$\mathcal{G}$}; %These three lines are for LGB projection
\draw [->] (19,-0.4) -- (19,-3.1);
%\draw (18.5,-1.75) node {$\pi_{\mathcal{G}}$};

\draw (15.5,2) node {$\mathcal{G}_U$};
\draw (14,-3.4) node {$($};
\filldraw [fill=red, draw=red] (14.75,-3.5) circle (1pt);
\draw (14.75,-3.9) node {\textcolor[rgb]{1,0,0}{$x$}};
\draw (15.5,-3.6) node {$)$};
\draw (14.75, -3) node {$U$};

\path[<-] (8.5,0) edge [bend left] (11,0);
\end{tikzpicture}
%\end{figure}
\end{center}

For $\mathcal{G} \to M$ an LGB over a smooth manifold $M$ and $\mathcal{P} \stackrel{\pi}{\to} M$ a principal $\mathcal{G}$-bundle we fix a point $p \in \mathcal{P}_x$ ($x \in M$) and can multiply that with an element $g \in \mathcal{G}_x$. Infinitesimally, we are interested into how this multiplication by $g$ affects tangent vectors, especially non-vertical ones.

However, as we have seen in Def.\ \ref{def:LRTranslations}, Rem.\ \ref{rem:AbstractNotationTwoForLeftInvarVfs} and Thm.\ \ref{thm:DiffOfLGBAction} (and its proof) the push-forward of horizontal vectors is not well-defined anymore on non-vertical vectors if one uses a fixed element of an LGB; $r_g$ is just a map $\mathcal{P}_x \to \mathcal{P}_x$. In order to study push-forwards of non-vertical vectors, we need information of the $\mathcal{G}$-action in an open neighbourhood $U$ around the fibres over $x$. Hence we want to use a section $\sigma \in \Gamma(\mathcal{G}|_U)$ with $\sigma_x= g$ instead.

\begin{center}
\begin{tikzpicture}[scale=0.80]
\draw (0,-3.5) to[out=10,in=170] (4.25,-3.5) to[out=350,in=190] (8.5,-3.5);
\draw (1.25,2.5) to[out=280,in=90](1.5,0) to[out=270,in=80] (1.25,-2.5);
\filldraw [fill=gray!20!white, draw=white] (3.5,2.5) to[out=280,in=90](3.75,0) to[out=270,in=80] (3.5,-2.5) -- (5,-2.5)  to[out=80,in=270](5.25,0) to[out=90,in=280] (5,2.5) -- cycle;
\draw (4.25,2.5) to[out=280,in=95](4.41,1) to[out=275,in=85] (4.41,-1) to[out=265,in=80] (4.25,-2.5); %draw with middle section for precise point placement

\draw (4,-1) node {\textcolor[rgb]{1,0,0}{$p$}};
\draw [<-, draw=blue] (5,1) to[out=275,in=85] (5,-1);
\draw [<-, draw=blue] (4.8,1) to[out=275,in=85] (4.8,-1);
\draw [<-, draw=blue] (4.6,1) to[out=275,in=85] (4.6,-1);
\draw (5.5,0) node {\textcolor[rgb]{0,0,1}{$\cdot \sigma$}};
\filldraw [fill=red, draw=red] (4.41,1) circle (1pt);
\filldraw [fill=red, draw=red] (4.41,-1) circle (1pt);
\draw (5.75,2.5) to[out=280,in=90](6,0) to[out=270,in=80] (5.75,-2.5);
\draw (7.25,2.5) to[out=280,in=90](7.5,0) to[out=270,in=80] (7.25,-2.5);
\draw (2.75,2.475) to[out=280,in=90](3,0) to[out=270,in=80] (2.75,-2.475);

%\draw (4.75,2) node {$\mathcal{P}_x$};
%\draw (4.95,1.5) node {$\cong G$};
\draw (9.5,-3.5) node {$M$};
\draw (5,2) node {$\mathcal{P}_U$};
\draw (3.5,-3.4) node {$($};
\filldraw [fill=red, draw=red] (4.25,-3.5) circle (1pt);
\draw (4.25,-3.9) node {\textcolor[rgb]{1,0,0}{$x$}};
\draw (5,-3.6) node {$)$};
\draw (4.25, -3) node {$U$};
\draw (0,0) node {$\mathcal{P}$};
\draw [->] (0,-0.4) -- (0,-3.1);
\draw (0.4,-1.75) node {$\pi$};
\draw (3.7,1) node {\textcolor[rgb]{1,0,0}{$p \cdot \sigma_x$}};
%%%%%%%%%%%%%%%%%%%%%%%%%%%%%%%%%%%%%%%%%%%%%%%%%%%%%%%%%%%%%%%%%%%%%%%%%%%%%%%%
\draw (10.5,-3.5) to[out=10,in=170] (14.75,-3.5) to[out=350,in=190] (19,-3.5); %hori M
\filldraw [fill=gray!20!white, draw=white] (14,2.5) to[out=280,in=90](14.25,0) to[out=270,in=80] (14,-2.5) -- (15.5,-2.5)  to[out=80,in=270](15.75,0) to[out=90,in=280] (15.5,2.5) -- cycle; %gray area
\draw (11.75,2.5) to[out=280,in=90](12,0) to[out=270,in=80] (11.75,-2.5); %vert line 1
\draw (14.75,2.5) to[out=280,in=90](15,0) to[out=270,in=80] (14.75,-2.5); %3
\draw (16.25,2.5) to[out=280,in=90](16.5,0) to[out=270,in=80] (16.25,-2.5); %4
\draw (17.75,2.5) to[out=280,in=90](18,0) to[out=270,in=80] (17.75,-2.5); %5

\draw (13.25,2.475) to[out=280,in=90](13.5,0) to[out=270,in=80] (13.25,-2.475); %2
\draw [draw=blue] (14.25,0) to[out=10,in=170] (15,0) to[out=350,in=190] (15.75,0);
%\filldraw [fill=blue, draw=blue] (15,0) circle (1pt);
\draw (13.9,0) node {\textcolor[rgb]{0,0,1}{$\sigma$}};

\draw (19,0) node {$\mathcal{G}$}; %These three lines are for LGB projection
\draw [->] (19,-0.4) -- (19,-3.1);
%\draw (18.5,-1.75) node {$\pi_{\mathcal{G}}$};

\draw (15.5,2) node {$\mathcal{G}_U$};
\draw (14,-3.4) node {$($};
\filldraw [fill=red, draw=red] (14.75,-3.5) circle (1pt);
\draw (14.75,-3.9) node {\textcolor[rgb]{1,0,0}{$x$}};
\draw (15.5,-3.6) node {$)$};
\draw (14.75, -3) node {$U$};

\path[<-] (8.5,0) edge [bend left] (11,0);
\end{tikzpicture}
\end{center}

But there are in general a plethora of sections with $\sigma_x= g$, thenceforth one expects that a definition of connections based on that may depend on the choice of sections and thus leading to conflicts once one looks at push-forwards with all possible $g \in \mathcal{G}$. Especially the tangential behaviour of the section's image (as an embedding of the base) may contribute to the push-forward; given a horizontal distribution, a horizontal vector may be still horizontal after a push-forward with one LGB section but not with respect to another LGB section. A similar problem may arise if we would work with local trivializations instead in order to use Ex.\ \ref{ex:TrivialLGBAction}. Thus, we need to adjust the typical definition of connections on principal bundles.

In order to understand what has to be changed, let us revisit the "typical" situation, that is, let $\mathcal{P}$ be a classical principal bundle as in Ex.\ \ref{ex:TheCLassicalPrincAsEx}, \textit{i.e.}\ $\mathcal{G} = M \times G$ is trivial with Lie group $G$. The $\mathcal{G}$-action is equivalent to a $G$-action on $\mathcal{P}$ by Ex.\ \ref{ex:TrivialLGBAction}, a push-forward with $g \in G$ w.r.t.\ the latter action is equivalent to the push-forward with a constant section in $\mathcal{G}$ for which we may still simply write $g$.

Equip $\mathcal{P}$ with a "typical" (Ehresmann) connection as in \cite[\S 5.1, Def.\ 5.1.6, page 260]{Hamilton}, that is, a horizontal distribution $\mathrm{H}\mathcal{P}$ of $\mathcal{P}$ with
\ba\label{ClassicalSymmetryOfHorizontalDistr}
\mathrm{D}_pr_g\mleft(\mathrm{H}\mathcal{P}_p\mright) &= \mathrm{H}\mathcal{P}_{p \cdot g}
\ea
for all $p \in \mathcal{P}$ and $g \in G$ as constant section. Recall that a connection has a 1:1 correspondence to a parallel transport, as presented in \cite[\S 5.8, page 286ff.]{Hamilton}. This means corresponding to a piece-wise smooth base curve $\alpha: [0, t] \to M$ ($t > 0$) with $\alpha(0) = x$ we have a \textbf{parallel transport along $\alpha$} as a map $\mathrm{PT}_\alpha^{\mathcal{P}}: \mathcal{P}_x \to \Gamma(\alpha^*\mathcal{P})$ satisfying
\ba\label{PTProp1}
\mleft.\mathrm{PT}^{\mathcal{P}}_{\alpha*\alpha^\prime}\mright|_t
&=
\mleft.\mathrm{PT}^{\mathcal{P}}_{\alpha^\prime}\mright|_t
	\circ \mleft.\mathrm{PT}_\alpha^{\mathcal{P}}\mright|_t,\\
\mleft.\mathrm{PT}_{\alpha^-}^{\mathcal{P}}\mright|_t
&=
\mleft.\mleft( \mathrm{PT}_\alpha^{\mathcal{P}} \mright)^{-1}\mright|_t,\label{PTProp2}
\ea
especially parallel transport is a diffeomorphism between the fibres, where $\alpha^\prime$ is just another similarly defined base curve with $\alpha^\prime(0)= \alpha(t)$, $\alpha*\alpha^\prime$ their concatenation ($\alpha$ coming first and with a suitable parametrization such that $\alpha*\alpha^\prime$ is a map $[0, t] \to M$), and $\alpha^-$ denotes $\alpha$ traversed backwards. Moreover, Eq.\ \eqref{ClassicalSymmetryOfHorizontalDistr} integrates and is equivalent to
\ba\label{ClassicalSymmetryofPTs}
\mleft.\mathrm{PT}_\alpha^{\mathcal{P}}(p \cdot g)\mright|_t
&=
\mleft.\mathrm{PT}_\alpha^{\mathcal{P}}(p)\mright|_t \cdot g.
\ea
Thinking of the associated $\mathcal{G}$-action, $g$ is an element of $\mathcal{G}_x$ such that the right hand side is in general not well-defined\footnote{Also here one could work with trivializations, especially since $\alpha^*\mathcal{P}$ is trivial as a fibre bundle due to the fact that the image of $\alpha$ is contractible. As before, this would just lead to other problems on a global scale, and we aim to provide a definition of connections on $\mathcal{P}$ without making use of trivializations.} anymore due to the fact that $\mleft.\mathrm{PT}_\alpha^{\mathcal{P}}(p)\mright|_t \in \mathcal{P}_{\alpha(t)}$; it is well-defined if interpreting $g$ as a constant section of $\mathcal{G} = M \times G$, denoted now by $\widetilde{g} \in \Gamma(\mathcal{G})$ for bookkeeping reasons. The left hand side uses $\widetilde{g}_{\alpha(0)} = \widetilde{g}_{x} = (x, g)$, the right hand side $\widetilde{g}_{\alpha(t)} = (\alpha(t), g)$, and by Ex.\ \ref{ex:TrivialLGBAction} we rewrite the $G$-action to a $\mathcal{G}$-action:
\bas
p \cdot g
&=
p \cdot (x, g)
=
p \cdot \widetilde{g}_x,&
\mathrm{PT}_\alpha^{\mathcal{P}}(p) \cdot g
&=
\mathrm{PT}_\alpha^{\mathcal{P}}(p) \cdot (\alpha, g)
=
\mathrm{PT}_\alpha^{\mathcal{P}}(p) \cdot \alpha^*\widetilde{g},
\eas
where we recall that $\alpha^*\mathcal{P}$ is a principal $\alpha^*\mathcal{G}$-bundle in sense of Def.\ \ref{def:PullbackPrincBundleDef}. 

Now we equip $\mathcal{G} \stackrel{\pi_{\mathcal{G}}}{\to} M$ with its \textbf{canonical flat connection} $\mathrm{H}\mathcal{G} \coloneqq \pi^*_{\mathcal{G}}\mathrm{T}M$ which induces a parallel transport $\mathrm{PT}_\alpha^{\mathcal{G}}: \mathcal{G}_{x} \to \Gamma\mleft(\alpha^*\mathcal{G}\mright)$ with $\mleft.\mathrm{PT}_\alpha^{\mathcal{G}}\mright|_t(x, g) = ( \alpha(t), g)$, especially $\mathrm{PT}_\alpha^{\mathcal{G}}\mleft( \widetilde{g}_x \mright) = \alpha^*\widetilde{g}$. In total, we can rewrite Eq.\ \eqref{ClassicalSymmetryofPTs} to
\bas
\mathrm{PT}_\alpha^{\mathcal{P}}\mleft(p \cdot \widetilde{g}_x\mright)
&=
\mathrm{PT}_\alpha^{\mathcal{P}}(p) \cdot \alpha^*\widetilde{g}
=
\mathrm{PT}_\alpha^{\mathcal{P}}(p) \cdot \mathrm{PT}_\alpha^{\mathcal{G}}\mleft( \widetilde{g}_x \mright).
\eas
This opens a gateway to define Eq.\ \eqref{ClassicalSymmetryofPTs} on general principal $\mathcal{G}$-bundles. That is, now let $\mathcal{P}$ be again a general principal $\mathcal{G}$-bundle. Fix any horizontal distribution $\mathrm{H}\mathcal{G}$ on $\mathcal{G}$ inducing a parallel transport $\mathrm{PT}_\alpha^{\mathcal{G}}$, then a connection on $\mathcal{P}$ should be equivalent to a parallel transport $\mathrm{PT}_\alpha^{\mathcal{P}}$ on $\mathcal{P}$ satisfying Eq.\ \eqref{PTProp1}, \eqref{PTProp2} and
\ba\label{PTHomomNEw}
\mathrm{PT}_\alpha^{\mathcal{P}}(p \cdot g)
&=
\mathrm{PT}_\alpha^{\mathcal{P}}(p) \cdot \mathrm{PT}_\alpha^{\mathcal{G}}(g)
\ea
for all $p \in \mathcal{P}_x$ and $g \in \mathcal{G}_x$. The right hand side is now well-defined since both, $\mathrm{PT}_\alpha^{\mathcal{P}}(p)$ and $\mathrm{PT}_\alpha^{\mathcal{G}}(g)$, are elements of $\Gamma(\alpha^*\mathcal{P})$ and $\Gamma\mleft(\alpha^*\mathcal{G}\mright)$, respectively. We now want to derive its infinitesimal analogue similar to Eq.\ \eqref{ClassicalSymmetryOfHorizontalDistr}. Recall that we can view the parallel transports like $\mathrm{PT}_\alpha^{\mathcal{P}}(p)$ also as a map $[0, t] \to \mathcal{P}$ with $\pi \circ \mathrm{PT}_\alpha^{\mathcal{P}}(p) = \alpha$ (recall Subsection \ref{BasicNotations}; alternatively use Lemma \ref{lem:PullbackFibreBundleItsTangentSp} and project onto the second component for the following derivatives). Then by construction
\bas
Y
\coloneqq
\mleft.\frac{\mathrm{d}}{\mathrm{d}t}\mright|_{t=0} \mathrm{PT}_\alpha^{\mathcal{G}}(g)
&\in
\mathrm{H}_g\mathcal{G}
\eas
for all $g \in \mathcal{G}$, that is, it is horizontal in $\mathcal{G}$; similarly for the parallel transport on $\mathcal{P}$,
\bas
X
\coloneqq
\mleft.\frac{\mathrm{d}}{\mathrm{d}t}\mright|_{t=0} \mathrm{PT}_\alpha^{\mathcal{P}}(p)
&\in
\mathrm{H}_p\mathcal{P}.
\eas
We want to use Thm.\ \ref{thm:DiffOfLGBAction} now in order to understand what Eq.\ \eqref{PTHomomNEw} implies about the corresponding horizontal distribution of $\mathcal{P}$. For this we differentiate both sides of Eq.\ \eqref{PTHomomNEw} with $\mathrm{d}/\mathrm{d}t|_{t=0}$; the left hand side implies that the right hand side is an element of $\mathrm{H}_{p\cdot g}\mathcal{P}$, while the right hand side gives
\bas
\mleft.\frac{\mathrm{d}}{\mathrm{d}t}\mright|_{t=0} \mleft(
	\mathrm{PT}_\alpha^{\mathcal{P}}(p) \cdot \mathrm{PT}_\alpha^{\mathcal{G}}(g)
\mright)
&=
\mathrm{D}_{(p, g)}\Phi\mleft(
	X, 
	Y
\mright)
\\
&=
\mathrm{D}_pr_\sigma\mleft( 
	X 
\mright)
	+ \mleft.{\oversortoftilde{\mleft( \mu_{\mathcal{G}}\mright)_g \Bigl(
		Y
		- \mathrm{D}_x \sigma \bigl( \dot{\alpha}(0) \bigr)
	\Bigr)}}\mright|_{p \cdot g}
\eas
where $\Phi: \mathcal{P}* \mathcal{G} \to \mathcal{P}$ denotes the $\mathcal{G}$-action on $\mathcal{P}$, $\sigma$ is any (local) section of $\mathcal{G}$ with $\sigma_x = g$ and
\bas
\dot{\alpha}(0)
&=
\mleft.\frac{\mathrm{d}}{\mathrm{d}t}\mright|_{t=0} \alpha
=
\mathrm{D}_p\pi (X)
\eas
because of $\pi \circ \mathrm{PT}_\alpha^{\mathcal{P}}(p) = \alpha$ (similarly w.r.t.\ $\mathrm{PT}_\alpha^{\mathcal{G}}$ so that it follows that $(X, Y)$ is a tangent vector of $\mathcal{P}*\mathcal{G}$). As we already have shown several times, for example recall the beginning of the proof of Thm.\ \ref{thm:DiffOfLGBAction}, we have
\bas
Y - \mathrm{D}_x \sigma \bigl( \dot{\alpha}(0) \bigr)
&\in
\mathrm{V}_g\mathcal{G},
\eas
so that the canonical projection $\pi^{\mathrm{vert}, \mathcal{G}}: \mathrm{T}\mathcal{G} \to \mathrm{V}\mathcal{G}$ onto the vertical structure of $\mathcal{G}$ acts as identity on it. By making use of the horizontality of $Y$, we can write
\bas
\mleft( \mu_{\mathcal{G}}\mright)_g \Bigl(
		Y - \mathrm{D}_x \sigma \bigl( \dot{\alpha}(0) \bigr)
	\Bigr)
&=
\mleft( \mu_{\mathcal{G}} \circ \pi^{\mathrm{vert}, \mathcal{G}}\mright)_g \bigl(
		Y
		- \mathrm{D}_p (\sigma \circ \pi)(X)
	\bigr)
\\
&=
-
\mleft( \mu_{\mathcal{G}} \circ \pi^{\mathrm{vert}, \mathcal{G}}\mright)_g \bigl(
		\mathrm{D}_p (\sigma \circ \pi)(X)
	\bigr).
\eas
It may not surprise that $\mu_{\mathcal{G}} \circ \pi^{\mathrm{vert}, \mathcal{G}}$ is by construction actually \textit{the} connection 1-form on $\mathcal{G}$ corresponding to $\mathrm{H}\mathcal{G}$, therefore we will denote $\mu_{\mathcal{G}} \circ \pi^{\mathrm{vert}, \mathcal{G}}$ as the \textbf{total Maurer-Cartan form $\mu_{\mathcal{G}}^{\mathrm{tot}}$ of $\mathcal{G}$}. Hence we get in total (see also Figure \ref{fig:Pushforwardwandwithoutparallelsection})
\ba\label{OiTHatIsHowWeFormulateHorizSymmetry}
\mathrm{D}_pr_\sigma\mleft( 
	X 
\mright)
	- \mleft.{\oversortoftilde{
		\mleft( \mu_{\mathcal{G}}^{\mathrm{tot}}\mright)_g \bigl(
		\mathrm{D}_p (\sigma \circ \pi)(X)
	\bigr)
	}}\mright|_{p \cdot g}
&\in
\mathrm{H}_{p \cdot g}\mathcal{P}
\ea
for all $X \in \mathrm{H}_p\mathcal{P}$, and we would like to take this as the starting point of defining a connection on $\mathcal{P}$. We are going to prove that such a definition is independent of the chosen section $\sigma$, and thence this gives the fundamental formula for the following parts of this paper. Last but not least, our starting point was Eq.\ \eqref{SymmetryOfTheVerticalBundle}; there is no contradiction between the approaches of Eq.\ \eqref{SymmetryOfTheVerticalBundle} and \eqref{OiTHatIsHowWeFormulateHorizSymmetry}. If $X$ is a vertical vector, then
\bas
\mathrm{D}_p\pi(X) &= 0,
\eas
and
\bas
\mathrm{D}_p r_\sigma (X)
&=
\mathrm{D}_p r_g (X)
\eas
by Remark \ref{rem:AbstractNotationTwoForLeftInvarVfs}. Thence, the symmetry behind Eq.\ \eqref{OiTHatIsHowWeFormulateHorizSymmetry} will be compatible with the one of Eq.\ \eqref{SymmetryOfTheVerticalBundle} on the vertical bundle. Let us first study the second summand in Eq.\ \eqref{OiTHatIsHowWeFormulateHorizSymmetry} via the \textit{Darboux derivative}.

\begin{figure}
\begin{minipage}[]{0.5\textwidth} 
%\begin{figure}[H]
\centering
%\begin{center}
\begin{tikzpicture}[scale=1.2]
		%Gray area
		\filldraw [fill=gray!20!white, draw=white] (3.5,2.5) to[out=280,in=95](3.66,1) to[out=275,in=85] (3.66,-1) to[out=265,in=80] (3.5,-2.5) -- (5,-2.5) to[out=80, in=265] (5.16, -1) to[out=85, in=275] (5.16, 1) to[out=95,in=280] (5,2.5) -- cycle;
		%lines
		\draw (4.25,2.5) to[out=280,in=95](4.41,1) to[out=275,in=85] (4.41,-1) to[out=265,in=80] (4.25,-2.5);
		\definecolor{darkgreen}{rgb}{0,0.58,0}
		\draw [draw=darkgreen] (3.66,-1) to[out=280,in=200] (4.41,-1) to[out=20,in=100] (5.16, -1);
		\draw [draw=darkgreen] (3.66,1) to[out=280,in=200] (4.41,1) to[out=20,in=100] (5.16,1);
		%Auxiliary stuff like arrows and dots
		\filldraw [fill=red, draw=red] (4.41,1) circle (1pt);
		\filldraw [fill=red, draw=red] (4.41,-1) circle (1pt);
		\draw [<-, draw=blue] (5.16,1) to[out=275,in=85] (5.16,-1);
		\draw [->, thick] (4.41,-1) -- (4.71,-0.88);
		%\draw [->] (4.41,1) -- (4.63,2);
		\draw [->, thick] (4.41,1) -- (4.71,1.12);
		%\draw [<-,dotted,thick] (4.63,2) -- (4.71,1.12);
		%Labels
		\draw (3.1,1) node {\textcolor[rgb]{0,0.58,0}{$\mathrm{H}_{p \cdot g}\mathcal{P}$}};
		\draw (3.2,-1) node {\textcolor[rgb]{0,0.58,0}{$\mathrm{H}_p\mathcal{P}$}};
		\draw (5.5,0) node {\textcolor[rgb]{0,0,1}{$\cdot g$}};
		\draw (3,2.2) node {$\mathcal{P}_U$};
		\draw (4.71,-0.6) node {$X$};
		\draw (5,1.4) node {$\mathrm{D}_pr_g(X)$};
		%\draw (5.5,1.56) node {$\thicksim \mathrm{D}_x\sigma(X)$};
		%Text
	\end{tikzpicture}
	%\end{center}
	%\caption{Push-forward with constant section.}
	%\label{PushyWithConst}
	%\end{figure}
	\end{minipage}\hfill
	%\begin{minipage}[]{0.5\textwidth}
	%\vfill
	%$
	%\end{minipage}
	%\hfill
	\begin{minipage}[]{0.5\textwidth}
	\begin{center}
	\begin{tikzpicture}[scale=1.2]
		%Gray area
		\filldraw [fill=gray!20!white, draw=white] (3.5,2.5) to[out=280,in=95](3.66,1) to[out=275,in=85] (3.66,-1) to[out=265,in=80] (3.5,-2.5) -- (5,-2.5) to[out=80, in=265] (5.16, -1) to[out=85, in=275] (5.16, 1) to[out=95,in=280] (5,2.5) -- cycle;
		%lines
		\draw (4.25,2.5) to[out=280,in=95](4.41,1) to[out=275,in=85] (4.41,-1) to[out=265,in=80] (4.25,-2.5);
		\definecolor{darkgreen}{rgb}{0,0.58,0}
		\draw [draw=darkgreen] (3.66,-1) to[out=280,in=200] (4.41,-1) to[out=20,in=100] (5.16, -1);
		\draw [draw=darkgreen] (3.66,1) to[out=280,in=200] (4.41,1) to[out=20,in=100] (5.16,1);
		%Auxiliary stuff like arrows and dots
		\filldraw [fill=red, draw=red] (4.41,1) circle (1pt);
		\filldraw [fill=red, draw=red] (4.41,-1) circle (1pt);
		\draw [<-, draw=blue] (5.16,1) to[out=275,in=85] (5.16,-1);
		\draw [->, thick] (4.41,-1) -- (4.71,-0.88);
		\draw [->, thick] (4.41,1) -- (4.63,2);
		\draw [->,dotted,thick] (4.41,1) -- (4.71,1.12);
		\draw [<-,dotted,thick] (4.63,2) -- (4.71,1.12);
		%Labels
		\draw (3,1) node {\textcolor[rgb]{0,0.58,0}{$\mathrm{H}_{p \cdot \sigma_x}\mathcal{P}$}};
		\draw (3.2,-1) node {\textcolor[rgb]{0,0.58,0}{$\mathrm{H}_p\mathcal{P}$}};
		\draw (5.5,0) node {\textcolor[rgb]{0,0,1}{$\cdot \sigma$}};
		\draw (3,2.2) node {$\mathcal{P}_U$};
		\draw (4.71,-0.6) node {$X$};
		\draw (5,2.25) node {$\mathrm{D}_pr_\sigma(X)$};
		\draw (5.9,1.56) node {$\thicksim \mathrm{D}_p(\sigma\circ\pi)(X)$};
		%Text
	\end{tikzpicture}
		%\caption{Push-forward with constant section.}
	\end{center}
	\end{minipage}
	\caption{Push-forward of a horizontal tangent vector $X$ with constant section (left) and general section (right), where $\mathcal{P}$ is a classical principal bundle as in Ex.\ \ref{ex:TheCLassicalPrincAsEx} equipped with a "typical" connection $\mathrm{H}\mathcal{P}$ of principal $G$-bundles ($G$ the structural Lie group).}
	\label{fig:Pushforwardwandwithoutparallelsection}
\end{figure}

 %a very trivial LGB and an action thereof; we will omit discussions about smoothness in the following. We will now discuss the trivial LGB $\mathcal{G}$ given by
%\begin{center}
	%\begin{tikzcd}
	%\mathbb{R}^* \arrow{r}& \mathbb{R} \times \mathbb{R}^* \arrow{d}{\mathrm{pr}_1}\\
	%& \mathbb{R}
	%\end{tikzcd}
%\end{center}
%where $\mathbb{R}^* \coloneqq \mathbb{R} \setminus \{0\}$ and $\mathrm{pr}_1$ is the projection onto the first component, and whose multiplication is fibre-wise given by
%\bas
%(x, g) \cdot (x, q) &\coloneqq (x, gq)
%\eas
%for all $x \in \mathbb{R}$ and $g, q \in \mathbb{R}^*$. We are interested into an action of $\mathcal{G}$ on another bundle $\mathcal{P}$, and we define $\mathcal{P}$ to be the same bundle as $\mathcal{G}$ for simplicity. The action is a right-action $\Phi: \mathcal{P} * \mathcal{G} \to \mathcal{P}$ whose explicit form is precisely the same as the multiplication on $\mathcal{G}$. For bookkeeping reasons we keep the notation $\mathcal{P}$ because this bundle will later be interpreted as the principal bundle; that $\mathcal{P}$ is the LGB $\mathcal{G}$ itself makes it easier to express certain things explicitly in the following.
%
%The tangent bundle $\mathrm{T}\mathcal{P}$ of $\mathcal{P}$ is canonically isomorphic to
%\begin{center}
	%\begin{tikzcd}
	%\mathbb{R}^2 \arrow{r} & \mathcal{P} \times \mathbb{R}^2 \arrow{d} \\
	%& \mathcal{P}
	%\end{tikzcd}
%\end{center}
%where the projection is given by the projection onto the $\mathcal{P}$-component. 
%The vertical bundle $\mathrm{V}\mathcal{P}$ of $\mathcal{P}$ is canonically isomorphic to
%\begin{center}
	%\begin{tikzcd}
	%\mathbb{R} \arrow{r} & \mathcal{P} \times \{0\} \times \mathbb{R} \arrow{d} \\
	%& \mathcal{P}
	%\end{tikzcd}
%\end{center}
%as a natural subbundle of $\mathrm{T}\mathcal{P}$ via $\{0\} \times \mathbb{R} \subset \mathbb{R}^2$. In this case we have $\mathcal{P} = \mathcal{G}$, and thus we have $\mathrm{V}\mathcal{P}\cong \mathrm{pr}_1^*\mathcal{g}$ by Cor.\ \ref{cor:TLGBAsLGB}; though one expects something also in general by using the notion of fundamental vector fields. Hence, we can think of the LAB $\mathcal{g}$ as the vertical bundle.
%
%We want to define a horizontal bundle as a "suitable" complement to the vertical bundle. One way would be the \textit{\textbf{canonical flat horizontal distribution/connection $\mathrm{H}^{\mathrm{flat}}\mathcal{P}$}} of $\mathcal{P}$ given by
%\begin{center}
	%\begin{tikzcd}
	%\mathbb{R} \arrow{r} & \mathcal{P} \times \mathbb{R} \times \{0\} \arrow{d} \\
	%& \mathcal{P}
	%\end{tikzcd}
%\end{center}
%We clearly have $\mathrm{T}\mathcal{P} = \mathrm{H}^{\mathrm{flat}}\mathcal{P} \oplus \mathrm{V}\mathcal{P}$ (Whitney sum). So, we will denote elements of $\mathrm{T}_{(x, g)}\mathcal{P}$ ($(x, g) \in \mathcal{P}$) by $(v_x, v_g) \in \mathbb{R}^2 \cong \mathrm{T}_x\mathbb{R} \oplus \mathrm{T}_g\mathbb{R}^*$, and we interpret the second component as a vertical component (but not necessarily the projection onto the vertical part of $(v_x, v_g)$). In the case of $\mathrm{H}^{\mathrm{flat}}\mathcal{P}$, we have $(v_x, v_g) \in \mathrm{H}^{\mathrm{flat}}_{(x, g)}\mathcal{P}$ if and only if $v_g = 0$.
%
%We want to describe horizontal distributions by a projection onto the vertical bundle, $\mathrm{T}\mathcal{P} \to \mathrm{V}\mathcal{P}$; for $\mathrm{H}^{\mathrm{flat}}\mathcal{P}$ this projection is simply given by $(v_x, v_g) \mapsto (0, v_g)$. However, one can think of course of other horizontal distributions $\mathrm{H}\mathcal{P}$ with $\mathrm{T}\mathcal{P} = \mathrm{H}\mathcal{P} \oplus \mathrm{V}\mathcal{P}$ given by a more general projection $\mathrm{T}\mathcal{P} \to \mathrm{V}\mathcal{P}$. Imagine we have
%\ba
%\pi_v: \mathrm{T}\mathcal{P} &\to \mathrm{V}\mathcal{P},\nonumber\\\label{ToyModelProjectionConn}
%(v_x, v_g) &\mapsto 
%\Bigl(0, v_g + \mathrm{D}_{(x, e)}R_{(x, g)}\bigl(\omega_{x}(v_x)\bigr) \Bigr)
%\ea
%for all $(x, g) \in \mathcal{P}$, where $(x, e)$ is the neutral element of $\mathcal{G}_x$ and
%\bas
%\omega \in \Omega^1(M; \mathcal{g}).
%\eas
%By definition $\mathrm{D}_{(x, e)}R_{(x, g)}: \mathcal{g}_x \to \mathrm{V}_{(x,g)}\mathcal{P}$; that its image is vertical quickly follows by $\mathcal{g}_x = \mathrm{V}_{(x,e)}\mathcal{G}$ (Def.\ \ref{def:LABOfAnLGB}) and
%\bas
%\mathrm{D}_{(x,g)}\mathrm{pr}_1 \circ \mathrm{D}_{(x,e)}R_{(x,g)}
%&=
%\mathrm{D}_{(x, e)}\underbrace{\mleft( \mathrm{pr}_1 \circ R_{(x,g)} \mright)}
	%_{= \mathrm{pr_1}}
%=
%\mathrm{D}_{(x, e)} \mathrm{pr}_1
%\eas
%such that vertical vectors as the kernel of $\mathrm{D}\mathrm{pr}_1$ stay vertical under $\mathrm{D}_{(x,e)}R_{(x,g)}$; or simply recall the proof of Cor.\ \ref{cor:TLGBAsLGB}. Hence, Def.\ \eqref{ToyModelProjectionConn} is well-defined. Since $\mathbb{R}^*$ is a (rather trivial) abelian matrix group one knows that $\mathrm{D}R_{(x, g)}$ is just the multiplication with $(x, g)$, that is, we can write
%\bas
%\pi_v(v_x, v_g)
%&=
%\mleft(
	%0, v_g + w_x(v_x) ~ g
%\mright).
%\eas
%
%Let us now define a \textit{\textbf{horizontal distribution $\mathrm{H}\mathcal{P}$}} as the kernel of $\pi_v$. By $\pi_v(0, v_g) = (0, v_g)$ we know that $\pi_v$ is surjective, and thus the dimension of the kernel is
%\bas
%\mathrm{dim}\mleft( \mathrm{Ker} \mleft( \mleft.\pi_v\mright|_{(x, g)} \mright) \mright)
%&=
%\mathrm{dim}\mleft( \mathrm{T}_{(x, g)}\mathcal{P} \mright)
	%- \mathrm{dim}\mleft( \mathrm{V}_{(x, g)} \mathcal{P} \mright)
%=
%\mathrm{dim}(M),
%\eas
%and its elements are of the form
%\bas
%\mleft( v_x, - \omega_x(v_x) ~ g \mright),
%\eas
%desribing a lift of $v_x$ as a tangent vector of the basis.
%Therefore we constructed a subbundle $\mathrm{H}\mathcal{P}$ of $\mathrm{T}\mathcal{P}$ with
%\bas
%\mathrm{T}\mathcal{P}
%&=
%\mathrm{H}\mathcal{P} \oplus \mathrm{V}\mathcal{P}.
%\eas
%Let us now investigate the behaviour of this horizontal distribution under push-forwards of right-multiplications; as we already discussed before, we are interested into such push-forwards using sections of $\mathcal{G}$. That is, let $\widetilde{\sigma}$ a (local) section of $\mathcal{G}$; for the section we also write $\widetilde{\sigma}_x = (x, \sigma_x)$, where $\sigma$ is a (locally) defined map with values in $\mathbb{R}^*$; for simplicity we also write $\sigma$ for $\widetilde{\sigma}$, it should be clear by context what we mean. Then
%\bas
%\mathrm{D}_{(x, g)}r_\sigma(v_x, v_g)
%&=
%\mleft.\frac{\mathrm{d}}{\mathrm{d}t}\mright|_{t=0}\bigl( 
	%(x + tv_x, g + tv_g) \cdot \mleft(x + tv_x, \sigma_{x + tv_x} \mright)
%\bigr)
%\\
%&=
%\mleft.\frac{\mathrm{d}}{\mathrm{d}t}\mright|_{t=0} 
	%\bigl(x + tv_x, (g + tv_g) ~ \sigma_{x + tv_x} \bigr)
%\\
%&=
%\bigl(
	%v_x, v_g \sigma_{x} + g ~ \mathrm{D}_x \sigma(v_x)
%\bigr).
%\eas
%Also let $X = \mleft(X_x, -\omega_x(X_x) ~ g\mright) \in \mathrm{H}_{(x, g)}\mathcal{P}$, then
%\ba\label{SectionContributionToHorizontalStuff}
%\mathrm{D}_{(x, g)}r_\sigma(X)
%&=
%\bigl(
	%X_x, -\omega_x(X_x) ~ g \sigma_{x} + g ~ \mathrm{D}_x \sigma(X_x)
%\bigr).
%\ea
%In order to have $\mathrm{D}_{(x, g)}r_\sigma(X) \in \mathrm{H}_{(x, g \sigma_x)}\mathcal{P}$ we would need
%\bas
%\mathrm{D}_{(x, g)}r_\sigma(X)
%&\stackrel{!}{=}
%\bigl(
	%X_x, -\omega_x(X_x) ~ g \sigma_{x}
%\bigr),
%\eas
%and we would like that for all $X_x$, hence, $\mathrm{D}_{(x, g)}r_\sigma(X) \in \mathrm{H}_{(x, g \sigma_x)}\mathcal{P}$ for all $X_x$ holds if and only if $\mathrm{D}_x\sigma = 0$. Varying with respect to $x$, $\sigma$ needs to be a constant section of $\mathcal{P} = \mathbb{R} \times \mathbb{R}^*$. A constant section is equivalent to an element of $\mathbb{R}^*$, also recall Ex.\ \ref{ex:TrivialLGBAction}, and thus we are back at the classical formulation of horizontal distributions. However, due to the lack of a suitable notion of a constant section in the general case, especially if $\mathcal{G}$ is non-trivial, we need to work differently with Eq.\ \eqref{SectionContributionToHorizontalStuff}.
%
%Rewrite Eq.\ \eqref{SectionContributionToHorizontalStuff} to
%\bas
%\mathrm{D}_{(x, g)}r_\sigma(X)
%&=
%\bigl(
	%X_x, -\omega_x(X_x) ~ g \sigma_{x}
%\bigr)
	%+  \bigl(0, g ~ \mathrm{D}_x \sigma(X_x) \bigr)
%\\
%&=
%\bigl(
	%X_x, -\omega_x(X_x) ~ g \sigma_{x}
%\bigr)
	%+  \mleft.\oversortoftilde{\mu_{\mathcal{G}}^{\mathrm{tot}}\bigl( \mathrm{D}_x\sigma(X_x) \bigr)}\mright|_{(x, g \sigma_x)},
%\eas
%where 
%\bas
%\mu_{\mathcal{G}}^{\mathrm{tot}}(Y)
%&\coloneqq
%\mu_{\mathcal{G}}\mleft( \pi^{\mathrm{flat}}(Y) \mright)
%\eas
%for all $Y \in \mathrm{T}\mathcal{G}$, were $\pi^{\mathrm{flat}}$ is the projection onto $\mathrm{V}\mathcal{G}$ along the canonically flat horizontal distribution. The notation $\mu_{\mathcal{G}}^{\mathrm{tot}}$ comes from that it is essentially the Maurer-Cartan form but defined on the \textbf{tot}al tangent bundle of $\mathcal{G}$; in fact, by construction, it is \textbf{the} connection-1-form corresponding to the canonical flat connection on $\mathcal{G}$. So, in total we get that
%\bas
%\mathrm{D}_{(x, g)}r_\sigma(X)
	%- \mleft.\oversortoftilde{\mu_{\mathcal{G}}^{\mathrm{tot}}\bigl( \mathrm{D}_x\sigma(X_x) \bigr)}\mright|_{(x, g \sigma_x)}
%\eas
%is horizontal. 
%
%Recall Ex.\ \ref{ex:TrivialLGBAction}, observe that $\mathcal{P}$ clearly admits the structure of an $\mathbb{R}^*$-principal bundle, and we have proven that $\mathrm{H}\mathcal{P}$ is a "typical" horizontal distribution on $\mathcal{P}$,\footnote{We have shown that the typical symmetry of horizontal distributions applies w.r.t.\ constant sections; then argue with Ex.\ \ref{ex:TrivialLGBAction}.} such that we have a 1:1 correspondence to a field of gauge bosons corresponding to $\mathrm{H}\mathcal{P}$. Thus, this example has obviously a strong correspondence to the classical formulation which allows us to compare the notions.
%
%That is, altogether, we have derived that the pushforward of a horizontal vector is horizontal after subtracting a form very similar to the Maurer-Cartan form making use of a horizontal distribution on $\mathcal{G}$ itself. This horizontal distribution is the canonical flat one on $\mathcal{G}$, and the "classical" notion of a connection on $\mathcal{P}$ uses this canonical flat horizontal distribution in order to remediate the contribution of $\sigma$ to the horizontality of $\mathrm{D}_{(x, g)}r_\sigma(X)$.
%
%Henceforth, one now asks what happens if one chooses a different connection on $\mathcal{G}$. This is the notion we need because such a notion also allows non-trivial LGBs $\mathcal{G}$. The answer is clearly provided by Thm.\ \ref{thm:DiffOfLGBAction}: Define $Y \coloneqq \mathrm{D}_x\sigma(X_x) = \mathrm{D}_{e_x}R_\sigma\bigl( \mathrm{D}_xe(X_x) \bigr)$, then
%\bas
%\mathrm{D}_{\sigma_x}\mathrm{pr}_1(Y)
%&=
%X_x
%=
%\mathrm{D}_{(x, g)}\mathrm{pr}_1(X),
%\eas
%and by Thm.\ \ref{thm:DiffOfLGBAction} we get
%\bas
%\mathrm{D}_{(x, g)}r_\sigma(X)
%&=
%\mathrm{D}_{((x, g), \sigma_x)}\Phi(X, Y).
%\eas

\subsubsection{Darboux derivative on LGBs}\label{DiscussingDarbouxDerivativeGeneral}

By Cor.\ \ref{cor:VerticalBundleOfPrincIsNearlyAsUsual} we know that $\mathrm{V}\mathcal{P}$ is the pullback of $\mathcal{g}$. Since connections are projections onto the vertical bundle we are interested into the pullback situation, and therefore we will put some remarks after some definitions in the following in order to discuss the situation of pullback LGBs. Hence let us start with a general remark about the pullback situation to which we will later refer in the other remarks regarding this situation.

\begin{remarks}{Pullback LGBs and their connections}{GeneralPullBackLGBConnSituation}
Let $\mathcal{G} \stackrel{\pi_\mathcal{G}}{\to} M$ be an LGB over a smooth manifold $M$, and $\mathrm{H}\mathcal{G}$ be a horizontal distribution of $\mathcal{G}$, where we denote with $\pi^{\mathrm{vert}}: \mathrm{T}\mathcal{G} \to \mathrm{V}\mathcal{G}$ the corresponding (base-preserving) projection onto its vertical bundle; we will view $\pi^{\mathrm{vert}}$ as an element of $\Omega^1(\mathcal{G}; \mathrm{V}\mathcal{G})$. Furthermore, let $f: N \to M$ be a smooth map defined on another smooth manifold $N$. By Lemma \ref{lem:PullbackFibreBundleItsTangentSp} we know that $\mathrm{T}(f^*\mathcal{G})$ consists of pairs of tangent vectors $(X, Y) \in \mathrm{T}_{(p, g)}(f^*\mathcal{G})$ ($(p, g) \in f^*\mathcal{G}$) with $X \in \mathrm{T}_pN$, $Y \in \mathrm{T}_g\mathcal{G}$ and
\bas
\mathrm{D}_pf(X) &= \mathrm{D}_g\pi_{\mathcal{G}}(Y).
\eas
In the following we denote with $\mathrm{pr}_i$ ($i \in \{1,2\}$) the projections onto the $i$-th component of $f^*\mathcal{G}$.

The projection of $f^*\mathcal{G} \to N$ is $\mathrm{pr}_1$, thus $\mathrm{Dpr}_1(X, Y) = X$. The vertical bundle $\mathrm{V}(f^*\mathcal{G})$ as the kernel of $\mathrm{Dpr}_1$ then consists of pairs $(X = 0, Y)$, and therefore $\mathrm{D}_g\pi_{\mathcal{G}}(Y) = 0$ which implies $Y \in \mathrm{V}_g\mathcal{G}$. Thence,
\bas
\mathrm{V}\mleft( f^*\mathcal{G} \mright)
&\cong
\mathrm{pr}_2^*\mleft( \mathrm{V}\mathcal{G} \mright).
\eas 
It is then trivial to check that
\bas
\mathrm{pr}_2^!\pi^{\mathrm{vert}} 
&\in 
\Omega^1\bigl(f^*\mathcal{G}; \mathrm{pr}_2^*\mleft( \mathrm{V}\mathcal{G} \mright)\bigr)
\cong
\Omega^1\bigl(f^*\mathcal{G}; \mathrm{V}\mleft( f^*\mathcal{G} \mright)\bigr)
\eas
is a projection onto $\mathrm{V}\mleft( f^*\mathcal{G} \mright)$, and gives therefore rise to a horizontal distribution on $f^*\mathcal{G}$, the \textbf{pullback connection on $f^*\mathcal{G}$}. Especially we have
\bas
\mleft(\mathrm{pr}_2^!\pi^{\mathrm{vert}}\mright)_{(p,g)}(X, Y)
&=
\mleft( 0, \pi^{\mathrm{vert}}_g(Y) \mright),
%=
%\mleft.\mleft( 0, \pi^{\mathrm{vert}} \circ \mathrm{Dpr}_2 \mright)\mright|_{(p,g)}(X, Y),
\eas
and thus
\bas
\mleft.\mleft(
\mleft(\mathrm{Dpr}_1, \mathrm{Dpr}_2 \mright) \circ \mathrm{pr}_2^!\pi^{\mathrm{vert}}
\mright)\mright|_{(p, g)}(X, Y)
&= 
\mleft( 0, \pi^{\mathrm{vert}}_g(Y) \mright)
=
\mleft(\mathrm{pr}_2^!\pi^{\mathrm{vert}}\mright)_{(p,g)}(X, Y).
\eas 
%We may just shortly write $\mathrm{pr}_2^!\pi^{\mathrm{vert}}$, meaning its canonical extension to $\mathrm{T}(f^*\mathcal{G}) \supset \mathrm{pr}_2^*\mathrm{T}\mathcal{G}$.
\end{remarks}

As we have seen, we need a slight adjustment of the vertical Maurer-Cartan as presented in Def.\ \ref{def:MCFormOnLGBs}.

\begin{definitions}{Total Maurer-Cartan form}{TotMCFormOnLGB}
Let $\mathcal{G} \stackrel{\pi_{\mathcal{G}}}{\to} M$ be an LGB over a smooth manifold $M$, and $\mathrm{H}\mathcal{G}$ be a horizontal distribution of $\mathcal{G}$, where we denote with $\pi^{\mathrm{vert}}: \mathrm{T}\mathcal{G} \to \mathrm{V}\mathcal{G}$ the corresponding projection onto its vertical bundle. Then we define the \textbf{total Maurer-Cartan form $\mu_{\mathcal{G}}^{\mathrm{tot}} \in \Omega^1\mleft(\mathcal{G}; \pi_{\mathcal{G}}^*\mathcal{g}\mright)$ of $\mathcal{G}$} as the connection 1-form corresponding to $\mathrm{H}\mathcal{G}$, \textit{i.e.}\
\bas
\mleft( \mu_{\mathcal{G}}^{\mathrm{tot}} \mright)_g(Y)
&\coloneqq
\mleft.\mleft(\mu_\mathcal{G} \circ \pi^{\mathrm{vert}}\mright)\mright|_g(Y)
=
\mleft( \mathrm{D}_g L_{g^{-1}} \mright)\mleft(\pi^{\mathrm{vert}}(Y)\mright)
\eas
for all $g \in \mathcal{G}$ and $Y \in \mathrm{T}_g\mathcal{G}$.
\end{definitions}
%
%By definition it is clear that $\mleft.\mu^{\mathrm{tot}}_{\mathcal{G}}\mright|_{\mathcal{g}}$ is the identity on $\mathcal{g}$.

This is clearly well-defined by construction; also recall our discussion of the vertical Maurer-Cartan form, especially Cor.\ \ref{cor:VertMCVormIsWellDefined}.

\begin{remarks}{Pullback situation: Part I}{PullbackTotMCForm}
Given the situation as in Remark \ref{rem:GeneralPullBackLGBConnSituation}, then we have
\bas
L_{(p, g)}\bigl( (p, q) \bigr)
&=
\bigl(p, L_g(q)\bigr)
\eas
for all $(p, g), (p, q) \in f^*\mathcal{G}$, where the left-translation on the left and right hand side are the ones of $f^*\mathcal{G}$ and $\mathcal{G}$, respectively. Thence,
\bas
L_{(p, g)}
&=
\mleft(
	\mathrm{pr}_1, L_{\mathrm{pr_2}(p, g)} \circ \mathrm{pr}_2
\mright),
\eas
and hence,
\bas
\mathrm{D}_{(p, g)} L_{(p, g)^{-1}}
&=
\mleft.
\mleft(
	\mathrm{D}_{(p,g)}\mathrm{pr}_1, \mathrm{D}_{\mathrm{pr}_2(p, g)}L_{\mathrm{pr}_2\mleft(p, g^{-1}\mright)} \circ \mathrm{D}_{(p,g)}\mathrm{pr}_2
\mright)
\mright|_{\mathrm{V}_{(p, g)}\mleft( f^*\mathcal{G} \mright)}
\\
&=
\mleft.
\mleft(
	0, 
	\mleft( \mu_{\mathcal{G}} \mright)_{\mathrm{pr}_2(p, g)} \circ \mathrm{D}_{(p,g)}\mathrm{pr}_2
\mright)\mright|_{\mathrm{V}_{(p, g)}\mleft( f^*\mathcal{G} \mright)},
%\\
%&=
%\mleft(
	%\mathrm{Dpr}_1, 
	%\mathrm{Dpr}_2^!\mu_G
%\mright),
\eas
making use of that 
%$f^*\mathcal{G}$ is an embedded submanifold of $N \times \mathcal{G}$ and that 
$L_{(p, g)}: \mleft(f^*\mathcal{G}\mright)_p \to \mleft(f^*\mathcal{G}\mright)_p$ so that $\mathrm{D}_{(p, g)}L_{(p, g)^{-1}}:\mathrm{V}_{(p, g)}\mleft( f^*\mathcal{G} \mright) \to \mathrm{V}_{(p, e_x)}\mleft( f^*\mathcal{G} \mright) = \mleft(f^*\mathcal{g}\mright)_{p}$, where $x \coloneqq f(p)$ and $e_x$ is the neutral element of $\mathcal{G}_x$.
Altogether we get
\bas
\mleft(\mu_{f^*\mathcal{G}}^{\mathrm{tot}}\mright)_{(p, g)}
&=
\mathrm{D}_{(p, g)} L_{(p, g)^{-1}} \circ \mleft.\mathrm{pr}_2^!\pi^{\mathrm{vert}}\mright|_{(p, g)}
\\
&=
\mleft(
	0, 
	\mleft( \mu_{\mathcal{G}} \mright)_{\mathrm{pr}_2(p, g)} \circ \mathrm{D}_{(p,g)}\mathrm{pr}_2
\mright)
	\circ \mleft.\mathrm{pr}_2^!\pi^{\mathrm{vert}}\mright|_{(p, g)}
\\
&=
\mleft(
	0, 
	\mleft.\mleft( \mu_{\mathcal{G}} \circ \pi^{\mathrm{vert}}\mright)\mright|_{\mathrm{pr}_2(p, g)} \circ \mathrm{D}_{(p,g)}\mathrm{pr}_2
\mright)
\\
&=
\mleft(
	0, 
	\mleft( \mu_{\mathcal{G}}^{\mathrm{tot}}\mright)_{\mathrm{pr}_2(p, g)} \circ \mathrm{D}_{(p,g)}\mathrm{pr}_2
\mright)
\\
&=
\mleft.\mleft(\mathrm{pr}_2^!\mu_{\mathcal{G}}^{\mathrm{tot}}\mright)\mright|_{(p, g)}.
\eas
\end{remarks}

\begin{remarks}{Total Maurer-Cartan form just typical form on trivial LGBs}{TrivialLGBsAndTheirMCForm}
Let $G$ be a Lie group with Lie algebra $\mathfrak{g}$, and $\mathcal{G} \coloneqq M \times G \stackrel{\mathrm{pr}_1}{\to} M$ be the trivial LGB equipped with its canonical flat connection $\mathrm{H}\mathcal{G} \coloneqq \mathrm{pr}_1^*\mathrm{T}M$, where $\mathrm{pr}_1$ is the projection onto the first component in $M \times G$. Its LAB $\mathcal{g}$ is also a trivial bundle, $M \times \mathfrak{g}$, and we have several identities (recall Cor.\ \ref{cor:TLGBAsLGB})
\bas
\mathrm{T}\mathcal{G}
&\cong
\mathrm{pr}_1^*\mathrm{T}M \oplus \mathrm{pr}_2^*\mathrm{T}G
\cong
\mathrm{pr}_1^*\mathrm{T}M \oplus \mathfrak{g}
\cong
\mathrm{pr}_1^*\mathrm{T}M \oplus \mathrm{pr}_1^*\mathcal{g}
=
\mathrm{H}\mathcal{G} \oplus \mathrm{V}\mathcal{G},
\eas
where $\mathrm{pr}_2: M \times G \to G$ is the projection onto the second component,
and the projection onto the vertical bundle is then equivalent to $\mathrm{Dpr}_2 \in \Omega^1\mleft(\mathcal{G}; \mathrm{pr}_2^*\mathrm{T}G\mright)$. 

Now let us view $G$ as an LGB over a point $\{*\}$, then $\mathcal{G} = f^*G$, where $f: M \to \{*\}$. Making use of the uniqueness of $\pi^{\mathrm{vert}} = \mathds{1}_{\mathrm{T}G}$ for $G \to \{*\}$, we have for the pullback connection $\mathrm{pr}_2^!\pi^{\mathrm{vert}} = \mathrm{Dpr}_2$ which is precisely the projection for $\mathcal{G}$, and therefore we can use Remark \ref{rem:PullbackTotMCForm} to derive
\bas
\mu^{\mathrm{tot}}_{\mathcal{G}}
&=
\mathrm{pr}_2^!\mu_G,
\eas
where $\mu_G$ is the Maurer-Cartan form of $G$.
%
%We also have the typical projection properties
%\bas
%\mathrm{Dpr}_1 \circ \mathrm{Dpr}_2 &= 0,&
%\mathrm{Dpr}_2 \circ \mathrm{Dpr}_2 &= \mathrm{Dpr}_2,
%\eas
%and for $(x, g) \in M \times G$
%\bas
%L_{(x, g)}\bigl( (x, q) \bigr)
%&=
%\bigl(x, L_g(q)\bigr)
%\eas
%for all $(x, q) \in M \times G$, where the left-tanslation on the right hand side is the one in $G$. Thence,
%\bas
%L_{(x, g)}
%&=
%\mleft(
	%\mathrm{pr}_1, L_{\mathrm{pr_2}(x. g)} \circ \mathrm{pr}_2
%\mright),
%\eas
%and hence
%\bas
%\mathrm{D}_{(x, g)} L_{(x, g)^{-1}}
%&=
%\mleft(
	%\mathrm{D}_{(x,g)}\mathrm{pr}_1, \mathrm{D}_{\mathrm{pr}_2(x, g)}L_{\mathrm{pr}_2\mleft(x, g^{-1}\mright)} \circ \mathrm{D}_{(x,g)}\mathrm{pr}_2
%\mright)
%\\
%&=
%\mleft(
	%\mathrm{D}_{(x,g)}\mathrm{pr}_1, 
	%\mleft( \mu_G \mright)_{\mathrm{pr}_2(x, g)} \circ \mathrm{D}_{(x,g)}\mathrm{pr}_2
%\mright),
%%\\
%%&=
%%\mleft(
	%%\mathrm{Dpr}_1, 
	%%\mathrm{Dpr}_2^!\mu_G
%%\mright),
%\eas
%where $\mu_G$ is the Maurer-Cartan form of $G$. Altogether we get
%\bas
%\mleft(\mu_{\mathcal{G}}^{\mathrm{tot}}\mright)_{(x, g)}
%&=
%\mathrm{D}_{(x, g)} L_{(x, g)^{-1}} \circ \mathrm{D}_{(x,g)}\mathrm{pr}_2
%\\
%&=
%\mleft(
	%\mathrm{D}_{(x,g)}\mathrm{pr}_1, 
	%\mleft( \mu_G \mright)_{\mathrm{pr}_2(x, g)} \circ \mathrm{D}_{(x,g)}\mathrm{pr}_2
%\mright)
	%\circ \mathrm{D}_{(x,g)}\mathrm{pr}_2
%\\
%&=
%\mleft(
	%0, 
	%\mleft( \mu_G \mright)_{\mathrm{pr}_2(x, g)} \circ \mathrm{D}_{(x,g)}\mathrm{pr}_2
%\mright)
%\\
%&=
%\mleft.\mathrm{pr}_2^!\mu_G\mright|_{(x, g)},
%\eas
%where $\mu_G$ is the Maurer-Cartan form of $G$. In general we played around a bit with the isomorphisms provided at the beginning of this remark; it can be \textit{e.g.}\ useful to think of the value of an element of $\mathrm{T}G$ as an element of $\mathrm{pr}_2^*\mathrm{T}G$: In the first line we view $\mathrm{Dpr}_2$ as a \textbf{base-preserving} morphism $\mathrm{T}\mathcal{G} \to \mathrm{pr}_2^*\mathrm{T}G \subset \mathrm{T}\mathcal{G}$, while later on as a morphism $\mathrm{T}\mathcal{G} \to \mathrm{T}G$ \textbf{over the base map $\mathrm{pr}_2$}. Since all of this is straightforward, such kind of stuff should have been clear by context.
\end{remarks}

The Maurer-Cartan form is important for gauge transformations because it induces a derivative, which we also need now.

\begin{remarks}{Maurer-Cartan form inducing a natural derivative: Part I}{MCFormAGeneralizationOfDerivative}
If $G$ is a Lie group and $\mathfrak{g}$ its Lie algebra, then there is actually some stance that the typical Maurer-Cartan form $\mu_G \in \Omega^1(G; \mathfrak{g})$ describes \textit{the} generalization of the total derivative of smooth maps $M \to \mathbb{R}^n$ $(n \in \mathbb{N})$, given by the \textbf{Darboux derivative $\Delta$} as given in \cite[\S 5.1, page 182ff.]{mackenzieGeneralTheory}. For a smooth section $\sigma: M \to G$ of the trivial LGB over $M$, this is $\Delta \sigma \in \Omega^1(M; \mathfrak{g})$ given by
\bas
\Delta \sigma
&\coloneqq
\sigma^! \mu_G,
\eas
that is,
\bas
(\Delta \sigma)_p(X)
&=
\mathrm{D}_{\sigma_x}L_{\sigma_x^{-1}}\bigl( \mathrm{D}_x \sigma(X) \bigr)
\eas
for all $p \in M$ and $X \in \mathrm{T}_p M$. For $G = \mathbb{R}^n$ one usually shows that $\mathrm{D}\sigma$ is the actual total derivative (Jacobian) by making use of $\mathrm{T}\mathbb{R}^n \cong \mathbb{R}^n \times \mathbb{R}^n$; however, if one views the triviality of $\mathrm{T}\mathbb{R}^n$ as the trivialization of general $\mathrm{T}G$ as given by $G \times \mathfrak{g}$, then it is more natural to think of the total derivative as $\mathrm{D}\sigma$ followed by $\mathrm{D}_{\sigma_x}L_{\sigma_x^{-1}}$ in order to receive information about the essential Lie algebra element. Then the classical total derivative of $\mathbb{R}^n$-valued maps is actually more naturally given by the Darboux derivative.
\end{remarks}

We now have a similar behaviour in our case but related to arbitrary connections on $\mathbb{R}^n$ (as a trivial bundle over $M$).

\begin{definitions}{Generalised Darboux derivative}{DarbouxDerivativeOnLGBs}
Let $\mathcal{G} \stackrel{\pi_{\mathcal{G}}}{\to} M$ be an LGB over a smooth manifold $M$, and $\mathrm{H}\mathcal{G}$ be a horizontal distribution of $\mathcal{G}$.
For $\sigma \in \Gamma(\mathcal{G})$ we define the \textbf{Darboux derivative $\Delta \sigma \in \Omega^1(M; \mathcal{g})$}
\bas
\Delta \sigma
&=
\sigma^! \mu_{\mathcal{G}}^{\mathrm{tot}}
=
\mleft(\sigma^* \mu_{\mathcal{G}}^{\mathrm{tot}} \mright) \circ \mathrm{D}\sigma.
\eas

We may also write $\Delta^{\mathcal{G}}$ instead of $\Delta$ in order to accentuate the LGB.
\end{definitions}

\begin{remark}\label{RemarkABoutDarbouxNotationWRTPullback}
\leavevmode\newline
The notation with the pullback $\sigma^*$ is only needed if one wants to view $\Delta \sigma$ as a $C^\infty(M)$-linear map $\mathfrak{X}(M) \to \Gamma(\mathcal{g})$; in this case it is a composition of base-preserving vector bundle morphisms $\mathrm{D}\sigma: \mathrm{T}M \to \sigma^*\mathrm{T}\mathcal{G}$ and $\sigma^*\mu_{\mathcal{G}}^{\mathrm{tot}}: \sigma^*\mathrm{T}\mathcal{G} \to \sigma^*\pi_{\mathcal{G}}^*\mathcal{g} \cong \mathcal{g}$ which can be extended to sections, where $\pi$ is the projection of $\mathcal{G}$.

Of course one can view $\Delta \sigma$ as a map $\mathrm{T}M \to \mathcal{g}$ which one can also write as the composition of $\mathrm{D}\sigma: \mathrm{T}M \to \mathrm{T}\mathcal{G}$ and $\mu_{\mathcal{G}}^{\mathrm{tot}}: \mathrm{T}\mathcal{G} \to \mathcal{g}$, that is,
\bas
\Delta \sigma
&=
\mu_{\mathcal{G}}^{\mathrm{tot}} \circ \mathrm{D}\sigma.
\eas
If one wants to emphasize base points of the involved pullback bundles, that is, one views $\mu_{\mathcal{G}}^{\mathrm{tot}}$ as a map $\mathrm{T}\mathcal{G} \to \pi_{\mathcal{G}}^*\mathcal{g}$, then one can write point-wise
\bas
\mleft(\mu_{\mathcal{G}}^{\mathrm{tot}}\mright)_{\sigma_x} \circ \mathrm{D}_x\sigma
&=
\bigl(
	\sigma_x, \mleft.\mleft(\Delta \sigma\mright)\mright|_x
\bigr)
\eas
for all $x \in M$.
\end{remark}

As a derivative we expect a Leibniz rule as for the classical Darboux derivative (see \cite[\S 5.1, Eq.\ 2, page 182]{mackenzieGeneralTheory}). However, we cannot make a general statement about that yet, as long as we do not fix a more specific type of connection on $\mathcal{G}$. We will come back to this later; at this point just be sure of that there is of course a certain Leibniz rule, simply due to that it generalizes the "classical" Darboux derivative, as we are going to see.

Let us now discuss the pullback situation for the Darboux derivative.

\begin{remarks}{Maurer-Cartan form inducing a natural derivative: Part II}{MCAsDerivativePartII}
We have now something similar to Remark \ref{rem:MCFormAGeneralizationOfDerivative}. Let $\mathcal{G} = M \times \mathbb{R}^n$ ($n \in \mathbb{N}$) be the trivial abelian LGB, and $\nabla$ a vector bundle connection on $M \times \mathbb{R}^n$, for which we have the associated projection onto the vertical bundle $\pi^{\mathrm{vert}}:\mathrm{T}\mathcal{G} \to \mathrm{V}\mathcal{G}$. For $\sigma \in \Gamma(\mathcal{G})$ we then have
\ba\label{NablaByProjection}
\nabla_{X_x} \sigma
&=
\mleft(\sigma^*\pi^{\mathrm{vert}}\mright)\bigl( \mathrm{D}_x \sigma\mleft( X_x \mright)  \bigr)
\ea
for all $x \in M$ and $X_x \in \mathrm{T}_xM$, by making use of $\sigma^*\mathrm{V}\mathcal{G} \cong M \times \mathbb{R}^n$ as vector bundles such that the right hand side has again values in $\mathcal{G}$ due to "enough triviality" and equals the left hand side.

Alternatively, one could use the isomorphism as given in Cor.\ \ref{cor:TLGBAsLGB}. 
This is also more natural in the sense of that one wants that $\nabla_{X_x} \sigma$ is a section over $M$; $M$ can be viewed as the image of the neutral section $e$ (the zero vector here), so that $\nabla_{X_x} \sigma$ should have values in $e^*\mathrm{V}\mathcal{G} = \mathcal{g}$. Henceforth one could say it is more natural to "pull $\mleft(\sigma^*\pi^{\mathrm{vert}}\mright)\bigl( \mathrm{D}_x \sigma\mleft( X_x \mright)  \bigr)$ back" to $\mathcal{g}$ by left-translation in order to define $\nabla_{X_x} \sigma$, \textit{i.e.}\
\bas
\mathrm{D}_{\sigma_x} L_{\sigma_x^{-1}} \Bigl(
	\mleft(\sigma^*\pi^{\mathrm{vert}}\mright)\bigl( \mathrm{D}_x \sigma\mleft( X_x \mright)  \bigr)
\Bigr)
&=
\mleft.\mleft( \sigma^*\mu_{\mathcal{G}} \circ \sigma^*\pi^{\mathrm{vert}} \mright)\mright|_x
\bigl( \mathrm{D}_x \sigma\mleft( X_x \mright) \bigr)
\\
&=
\mleft.(\Delta \sigma)\mright|_x(X_x).
\eas
Thus, as in Remark \ref{rem:MCFormAGeneralizationOfDerivative} one may say that $\Delta \sigma$ is \textit{the} generalization of vector bundle connections to general LGBs $\mathcal{G}$. Furthermore, this argument is not based on a given (local) trivialization to handle the arising pullback in Eq.\ \eqref{NablaByProjection}.
\end{remarks}

\begin{remarks}{Pullback situation: Part II}{PullBackDarboux}
Following Remark \ref{rem:PullbackTotMCForm} and its notation and results (see Remark \ref{rem:GeneralPullBackLGBConnSituation} for the initial setup), we can calculate $\Delta^{f^*\mathcal{G}}\mleft( f^*\sigma \mright) \in \Omega^1(N; f^*\mathcal{g})$ for $\sigma \in \mathcal{G}$, first we write
\bas
\mleft.f^*\sigma\mright|_p
&=
\mleft( p, \sigma_{f(p)} \mright)
\eas
for all $p \in N$, and thus
%\bas
%\mathrm{D}_p\mleft( f^*\sigma \mright)(X)
%&=
%\mleft( X, \mathrm{D}_x\sigma\bigl( \mathrm{D}_pf(X) \bigr) \mright)
%\eas
%for all $X \in \mathrm{T}_pN$, where $x \coloneqq f(p)$. Then
\ba\label{PullBackDarbouxOnPullbackSections}
\Delta^{f^*\mathcal{G}}\mleft( f^*\sigma \mright)
&=
%\mleft( \mleft( f^*\sigma \mright)^* \mu_{f^*\mathcal{G}}^{\mathrm{tot}} \mright) \circ \mathrm{D}\mleft( f^*\sigma \mright)
\mleft( f^*\sigma \mright)^! \mu_{f^*\mathcal{G}}^{\mathrm{tot}}
\nonumber
\\
&=
\mleft( f^*\sigma \mright)^! \mathrm{pr}_2^! \mu_{\mathcal{G}}^{\mathrm{tot}}
\nonumber
\\
&=
\mleft( \mathrm{pr}_2\circ f^*\sigma \mright)^! \mu_{\mathcal{G}}^{\mathrm{tot}}
\nonumber
\\
&=
\mleft( \sigma \circ f \mright)^!\mu_{\mathcal{G}}^{\mathrm{tot}}
\nonumber
\\
&=
\mleft(\mleft( \sigma \circ f \mright)^*\mu_{\mathcal{G}}^{\mathrm{tot}} \mright)
	\circ \mathrm{D}(\sigma\circ f)
\nonumber
\\
&=
\underbrace{\mleft( f^* \sigma^*\mu_{\mathcal{G}}^{\mathrm{tot}} \mright) \circ f^*\mleft(\mathrm{D}\sigma\mright)}_{= f^*\mleft( \mleft(\sigma^*\mu_{\mathcal{G}}^{\mathrm{tot}} \mright) \circ \mathrm{D}\sigma \mright)} \circ ~\mathrm{D}f
\nonumber
\\
&=
f^!\mleft(\mleft( \sigma^*\mu_{\mathcal{G}}^{\mathrm{tot}} \mright) \circ \mathrm{D}\sigma\mright)
\nonumber
\\
&=
f^!\mleft( \Delta^{\mathcal{G}} \sigma \mright),
\ea
where we rewrote the chain rule
\bas
\mathrm{D}(\sigma \circ f)
&=
f^*\mleft(\mathrm{D}\sigma\mright) \circ \mathrm{D}f,
\eas
but as in Remark \ref{RemarkABoutDarbouxNotationWRTPullback} one can decide to omit this notation. Such Darboux derivatives related to the pullback connection on $f^*\mathcal{G}$ we call the \textbf{pullback Darboux derivative}, and similar to the notation of pullback vector bundle connections we write
\bas
f^*\Delta^{\mathcal{G}}
&\coloneqq
\Delta^{f^*\mathcal{G}}.
\eas
\end{remarks}

\begin{remarks}{Canonical flat Darboux derivative}{DarbouxOnCanonFlat}
Given the situation as in Remark \ref{rem:TrivialLGBsAndTheirMCForm} we can use Remark \ref{rem:PullBackDarboux} to derive
\bas
\Delta^{\mathcal{G}}
&=
f^*\Delta^0,
\eas
where $\Delta^0$ on the right hand side is the Darboux derivative of the Lie group $G$ as a bundle over point. Trivially, $\Delta^0 g \equiv 0$ for all $g \in G$, and thus
\bas
\Delta^{\mathcal{G}}(f^*g)
&=
f^!\mleft( \Delta^0 g\mright)
=
0
\eas
for all $g \in G$ viewed as a section of $G \to \{0\}$, so $f^*g$ are the constant sections of $\mathcal{G}$. In sense of Remark \ref{rem:MCAsDerivativePartII} it makes sense to say that \textbf{$\sigma$ is parallel w.r.t.\ $\Delta$}. Since constant sections generate all sections of $\mathcal{G}$, we then speak of the \textbf{canonical flat Darboux derivative}, and constant sections are its parallel sections.

However, just because $\Delta^0$ is a zero map, does not mean that also $\Delta^{\mathcal{G}}$ is zero. On one hand because of a Leibniz rule which we will discuss later in more detail, and on the other hand, as mentioned in Subsection \ref{BasicNotations}, we can view general sections $\sigma \in \Gamma(\mathcal{G})$ (not necessarily constant) equivalently as a smooth map $\mathrm{pr}_2 \circ \sigma: M \to G$, denoted by $\widetilde{\sigma}$ for bookkeeping reasons. By Remark \ref{rem:TrivialLGBsAndTheirMCForm} we then get
\bas
\Delta^{\mathcal{G}} \sigma
&=
\sigma^! \mathrm{pr}_2^!\mu_G
=
\mleft( \mathrm{pr}_2 \circ \sigma \mright)^!\mu_G
=
\Delta^G \widetilde{\sigma}
\eas
for all $\sigma \in \Gamma(\mathcal{G})$, where $\Delta^G$ is the "classical" Darboux derivative as in Remark \ref{rem:MCFormAGeneralizationOfDerivative} related to $\mu_G$. This emphasizes why we can speak of a canonical flat derivative due to the fact that $\mu_G$ is flat (Maurer-Cartan equation), and we may simply write $\Delta^{\mathcal{G}} = \Delta^G$.
\end{remarks}

We can actually rewrite some important equations now; recall the notation introduced in Remark \ref{rem:FundVecNotationOnPullbackBundle}.

\begin{remarks}{Darboux derivative in the infinitesimal LGB action}{DarbouxInActionInfinit}
Recall Thm.\ \ref{thm:DiffOfLGBAction}; keeping the same notation as in this theorem but denoting the projection of $\mathcal{P}$ by $\pi$, we checked several times that $Y - \mathrm{D}_{x}\sigma (\omega)$ is vertical, and thus we can now write
\bas
\mathrm{D}_{(p, g)}\Phi(X, Y)
&=
\mathrm{D}_pr_\sigma(X)
	+ \mleft.{\oversortoftilde{\mleft( \mu_{\mathcal{G}}\mright)_{g} \bigl(Y - \mathrm{D}_{x}\sigma (\omega)\bigr)}}\mright|_{p \cdot g}
\\
&=
\mathrm{D}_pr_\sigma(X)
	+ \mleft.{\oversortoftilde{\mleft( \mu_{\mathcal{G}} \circ \pi^{\mathrm{vert}}\mright)_{g} \bigl(Y - \mathrm{D}_{x}\sigma (\omega)\bigr)}}\mright|_{p \cdot g}
\\
&=
\mathrm{D}_pr_\sigma(X)
	+ \mleft.{\oversortoftilde{\mleft( \mu_{\mathcal{G}} \circ \pi^{\mathrm{vert}}\mright)_{g} (Y)}}\mright|_{p \cdot g}
	- \mleft.{\oversortoftilde{\mleft( \mu_{\mathcal{G}} \circ \pi^{\mathrm{vert}}\mright)_{\sigma_x} \bigl( \mathrm{D}_{x}\sigma (\omega)\bigr)}}\mright|_{p \cdot g}
\\
&=
\mathrm{D}_pr_\sigma(X)
	+ \mleft.{\oversortoftilde{\mleft( \mu_{\mathcal{G}}^{\mathrm{tot}}\mright)_{g} (Y)}}\mright|_{p \cdot g}
	- \mleft.{\oversortoftilde{ \mleft.(\Delta \sigma)\mright|_x (\omega)}}\mright|_{p \cdot g}
\\
&=
\mathrm{D}_pr_\sigma(X)
	- \mleft.{\oversortoftilde{ \mleft.\mleft(\pi^!\Delta \sigma\mright)\mright|_p (X)}}\mright|_{p \cdot g}
	+ \mleft.{\oversortoftilde{\mleft( \mu_{\mathcal{G}}^{\mathrm{tot}}\mright)_{g} (Y)}}\mright|_{p \cdot g}
\eas
If $\mathcal{G}$ is a trivial LGB, then this emphasizes again that we recover the typical Leibniz rule by choosing a constant section $\sigma$, using Remarks \ref{rem:TrivialLGBsAndTheirMCForm} and \ref{rem:DarbouxOnCanonFlat}.
\end{remarks}

\begin{remarks}{Idea behind the notion of connection on principal bundles using the Darboux derivative}{DefOfConnectionIdeaWithDarboux}
Recall the notation and discussion around the terms in Eq.\ \eqref{OiTHatIsHowWeFormulateHorizSymmetry} which will be important for our definition of a connection on principal LGB-bundles. The terms in Eq.\ \eqref{OiTHatIsHowWeFormulateHorizSymmetry} can be similarly rewritten as in
\bas
\mathrm{D}_pr_\sigma\mleft( 
	X 
\mright)
	- \mleft.{\oversortoftilde{
		\mleft( \mu_{\mathcal{G}}^{\mathrm{tot}}\mright)_g \bigl(
		\mathrm{D}_p (\sigma \circ \pi)(X)
	\bigr)
	}}\mright|_{p \cdot g}
&=
\mathrm{D}_pr_\sigma\mleft( 
	X 
\mright)
	- \mleft.{\oversortoftilde{
		\bigl(\mleft(\Delta\sigma\mright)_x \circ \mathrm{D}_p\pi\bigr) (X)
	}}\mright|_{p \cdot g}
\\
&=
\mathrm{D}_pr_\sigma\mleft( 
	X 
\mright)
	- \mleft.{\oversortoftilde{
		\mleft. \mleft( \pi^!\Delta\sigma \mright) \mright|_p(X)
	}}\mright|_{p \cdot g}.
\eas
We will use this form for the definition of the principal bundle connection. As in Remark \ref{rem:DarbouxInActionInfinit}, if $\mathcal{G}$ is trivial and $\sigma$ a constant section corresponding to a Lie group element $g$, then these terms are just
\bas
\mathrm{D}_p r_g (X),
\eas
making use of Ex.\ \ref{ex:TrivialLGBAction}.
\end{remarks}


In fact, the Darboux derivative naturally induces a connection on $\mathcal{g}$ as LAB of $\mathcal{G}$. One may have expected that since $\mathrm{H}\mathcal{G}$ should infinitesimally induce a horizontal distribution on $\mathcal{g}$; recall the exponential map introduced in Subsection \ref{ExponentialMapSubsection}. Also recall that by definition of vertical bundles we know for the vertical bundle of the LAB $\mathcal{g} \stackrel{\pi_{\mathcal{g}}}{\to} M$ that $\mathrm{V}\mathcal{g} \cong \pi_{\mathcal{g}}^*\mathcal{g}$, making use of that LABs are vector bundles; in the following we will use the natural projection onto the second component of $\pi_{\mathcal{g}}^*\mathcal{g}$ but now defined on $\mathrm{V}\mathcal{g}$, denoted by $\mathrm{pr}_2: \mathrm{V}\mathcal{g} \to \mathcal{g}$.

\begin{propositions}{LGB connection induces LAB connection}{FinallyTheNablaInduction}
Let $\mathcal{G} \stackrel{\pi_{\mathcal{G}}}{\to} M$ be an LGB over a smooth manifold $M$, and $\mathrm{H}\mathcal{G}$ be a horizontal distribution of $\mathcal{G}$. Then the map $\nabla^{\mathcal{G}}: \Gamma(\mathcal{g}) \to \Omega^1(M; \mathcal{g})$, $\nu \mapsto \nabla^{\mathcal{G}}\nu$ denoted as an element of $\Omega^1(M; \mathcal{g})$ by $X \mapsto \nabla^{\mathcal{G}}_X \nu$, defined by
\bas
\mleft.\nabla^{\mathcal{G}}_X \nu\mright|_x
&\coloneqq
\mathrm{pr}_2 \mleft(
\mleft.\frac{\mathrm{d}}{\mathrm{d}t}\mright|_{t=0} \Bigl( \mleft(\Delta \e^{t \nu}\mright)_x (X) \Bigr)
\mright)
\eas
for all $x \in M$, $X \in \mathrm{T}_xM$ and $\nu \in \Gamma(\mathcal{g})$, is a well-defined vector bundle connection on $\mathcal{g}$, where $t \in \mathbb{R}$, and $\mathrm{pr}_2: \mathrm{V}\mathcal{g} \to \mathcal{g}$ is the projection onto the second component of $\mathrm{V}\mathcal{g}$ naturally viewed as pullback bundle.
\end{propositions}

\begin{remark}\label{RemarkAboutPrTwoInDarbouxDerivative}
\leavevmode\newline
The notation of $\mathrm{pr}_2$ is usually omitted in such constructions due to the fact that 
\bas
t \mapsto \mleft(\Delta \e^{t \nu}\mright)_x (X)
\eas
is a curve with values in $\mathcal{g}_x$, a vector space, so that one canonically uses the identification of tangent spaces of vector spaces with itself to show that $\mleft.\nabla^{\mathcal{G}}_X \nu\mright|_x \in \mathcal{g}_x$. We will keep $\mathrm{pr}_2$ for the proof for the sake of rigorousness, but we will drop it after this proposition for simplicity without further mention.
\end{remark}

In order to prove this we need to apply Schwarz's Theorem in order to switch $\Delta$ with $\mathrm{d}/\mathrm{d}t$. To do this rigorously we need to introduce the canonical involution/flip on double tangent bundles; since this does not completely fit into this paper's subject and may be already known by the reader, you can learn about the double tangent bundle and its flip map in Appendix \ref{DoubleTangentFlip} if needed.

\begin{proof}[Proof of Prop.\ \ref{prop:FinallyTheNablaInduction}]
\leavevmode\newline
%First of all, it is well-defined because of $\Delta \e^{t\nu} \in \Omega^1(M; \mathcal{g})$ such that $\mleft.\nabla^{\mathcal{G}}_X \nu\mright|_x \in \mathcal{g}_x$ as the velocity of a curve in the vector space $\mathcal{g}_x$. In fact
%\bas
%\nabla^{\mathcal{G}} \nu
%&=
%\mleft.\frac{\mathrm{d}}{\mathrm{d}t}\mright|_{t=0} \Delta\e^{t\nu},
%\eas
%where $[t \mapsto \Delta \exp(t \nu)]$ is a smooth curve in the vector space $\Omega^1(M; \mathcal{g})$, so that $\nabla^{\mathcal{G}} \nu\in \Omega^1(M; \mathcal{g})$. Thus, smoothness is clear and the tensorial behaviour w.r.t.\ $X$. 
%
Let us begin with well-definedness. Similar to Diagram \eqref{DoubleTangentAsDiagram} we have the double vector bundle
\be\label{DiagramForNablaConn}
	\begin{tikzcd}
		 \mathrm{TT}\mathcal{G} \arrow{r}{\mathrm{D}\pi_{\mathrm{T}\mathcal{G}}} \arrow{d}{\pi_{\mathrm{TT}\mathcal{G}}} & \mathrm{T}\mathcal{G} \arrow{d}{\pi_{\mathrm{T}\mathcal{G}}} \\
		\mathrm{T}\mathcal{G} \arrow[r, "\pi_{\mathrm{T}\mathcal{G}}"]& \mathcal{G}
	\end{tikzcd}
\ee
We also have
\ba\label{EqForNablaConnectionWithFlip}
\mleft.\frac{\mathrm{d}}{\mathrm{d}t}\mright|_{t=0} \Bigl( \mleft(\Delta \e^{t \nu} \mright)_x(X) \Bigr)
&=
\mleft.\frac{\mathrm{d}}{\mathrm{d}t}\mright|_{t=0} \Bigl( \mleft( \mu_{\mathcal{G}}^{\mathrm{tot}} \circ \mathrm{D}_x\e^{t\nu} \mright) (X) \Bigr)
\nonumber
\\
&=
\mathrm{D}_{\mathrm{D}_x e(X)}\mu_{\mathcal{G}}^{\mathrm{tot}}\mleft( \mleft.\frac{\mathrm{d}}{\mathrm{d}t}\mright|_{t=0} \Bigl( \mathrm{D}_x\e^{t\nu}  (X) \Bigr)\mright)
\nonumber
\\
&=
\mathrm{D}_{\mathrm{D}_x e(X)}\mu_{\mathcal{G}}^{\mathrm{tot}} \Bigl( S_{\mathcal{G}} \bigl( \mathrm{D}_x\nu (X) \bigr) \Bigr) 
\nonumber
\\
&=
\mathrm{D}_{\mathrm{D}_x e(X)}\mu_{\mathcal{G}}^{\mathrm{tot}} \bigl( \nu_T (X) \bigr) 
\\
&\in
\mathrm{T}_{\mu_{\mathcal{G}}^{\mathrm{tot}}\mleft( \mathrm{D}_xe(X) \mright) }\mathcal{g}
\nonumber
\ea
for all $x \in M$ and $X \in \mathrm{T}_xM$, where $e$ is the neutral section of $\mathcal{G}$, and we viewed $\mu_{\mathcal{G}}^{\mathrm{tot}}$ as a map $\mathrm{T}\mathcal{G} \to \mathcal{g}$ when applying the chain rule; recall Remark \ref{RemarkABoutDarbouxNotationWRTPullback}. $S_{\mathcal{G}}$ is the linear canonical flip map on $\mathrm{TT}\mathcal{G}$, especially see the last part of Remark \ref{rem:SchwarzThmInDiffGeo}. We also introduced the notation $\nu_T$ similar to Remark \ref{rem:TangentLifts} for simplicity; in fact $\nu_T$ is a vector field on $\mathrm{T}\mathcal{G}$ only defined over $\mathrm{T}M$ which is canonically embedded into $\mathrm{T}\mathcal{G}$ by embedding $M$ into $\mathcal{G}$ via $e$. We also naturally embed $\mathrm{T}\mathcal{g}$ into $\mathrm{TT}\mathcal{G}$. In total, we will work with these embeddings now so that everything is embedded into $\mathrm{TT}\mathcal{G}$ as the "total space"; hence, you will  also see $e$ and its total derivative $\mathrm{D}e$ acting as an embedding several times.

We get by Eq.\ \eqref{DiagramForNablaConn}
%omitting the base points in the differentials for readability,
\bas
\mathrm{D}\pi_{\mathrm{T}\mathcal{G}}\mleft( \mleft.\nabla^{\mathcal{G}}_X \nu\mright|_x \mright)
&=
\mathrm{D}_{\mathrm{D}_x e(X)}\underbrace{\mleft( \pi_{\mathrm{T}\mathcal{G}} \circ \mu_{\mathcal{G}}^{\mathrm{tot}} \mright)}
	_{= e \circ \pi_{\mathcal{G}} \circ \pi_{\mathrm{T}\mathcal{G}} } 
 \bigl( \nu_T (X) \bigr)
%\\
%&=
%\mleft(\mathrm{D}_xe \circ \mathrm{D}_{e_x}\pi_{\mathcal{G}} \circ \pi_{\mathrm{TT}\mathcal{G}}\mright) \bigl( \mathrm{D}_x\nu (X) \bigr)
\\
&=
\mleft(\mathrm{D}_xe \circ \mathrm{D}_{e_x}\pi_{\mathcal{G}}\mright) \mleft( \nu_x \mright)
\\
&=
0 \in \mathrm{T}_{e_x}\mathcal{G}
\eas
viewing $S_{\mathcal{G}}$ as a base-preserving isomorphism from $\pi_{\mathrm{TT}G}: \mathrm{TT}\mathcal{G} \to \mathrm{T}\mathcal{G}$ to $\mathrm{D}\pi_{\mathrm{T}\mathcal{G}}: \mathrm{TT}\mathcal{G} \to \mathrm{T}\mathcal{G}$, and using that $\mathcal{g} = e^*\mathrm{V}\mathcal{G}$.
The projection of $\mathcal{g}\to M$ is canonically the restriction of $\pi_{\mathrm{T}\mathcal{G}}$, hence we can conclude that
\bas
\mleft.\frac{\mathrm{d}}{\mathrm{d}t}\mright|_{t=0} \Bigl( \mleft(\Delta \e^{t \nu} \mright)_x(X) \Bigr)
&\in
\mathrm{V}_{\mu_{\mathcal{G}}^{\mathrm{tot}}\mleft( \mathrm{D}_xe(X) \mright) }\mathcal{g},
\eas
and so we can derive that $\mathrm{pr}_2: \mathrm{V}\mathcal{g} \to \mathcal{g}$ is defined on this; in total $\nabla^{\mathcal{G}}\nu$ is well-defined. By Eq.\ \eqref{EqForNablaConnectionWithFlip} we also trivially know that 
\bas
\pi_{\mathrm{TT}\mathcal{G}}\mleft( \mleft.\frac{\mathrm{d}}{\mathrm{d}t}\mright|_{t=0} \Bigl( \mleft(\Delta \e^{t \nu} \mright)_x(X) \Bigr) \mright)
&=
\mu_{\mathcal{G}}^{\mathrm{tot}}\bigl( \mathrm{D}_x e(X) \bigr)
\in
\mathcal{g}_x.
\eas
Viewing $\mathrm{V}\mathcal{g}$ naturally as the pullback of $\mathcal{g}$ along its projection we can therefore write
\bas
\mathrm{D}_{\mathrm{D}_x e(X)}\mu_{\mathcal{G}}^{\mathrm{tot}} \bigl( \nu_T (X) \bigr)  
&\stackrel{ \text{Eq.\ \eqref{EqForNablaConnectionWithFlip}} }{=}
\mleft.\frac{\mathrm{d}}{\mathrm{d}t}\mright|_{t=0} \Bigl( \mleft(\Delta \e^{t \nu} \mright)_x(X) \Bigr)
=
\mleft(
	\mu_{\mathcal{G}}^{\mathrm{tot}}\bigl( \mathrm{D}_x e(X) \bigr),
	\mleft.\nabla^{\mathcal{G}}_X \nu \mright|_x
\mright),
\eas
so that smoothness of $\nabla^{\mathcal{G}} \nu$ follows.
In order to understand whether $\nabla^{\mathcal{G}}$ is a vector bundle connection we are hence interested into the restriction of Diagram \eqref{DiagramForNablaConn} onto
\begin{center}
	\begin{tikzcd}
		 \mathrm{V}\mathcal{g} \arrow{r}{\mathrm{D}\pi_{\mathrm{T}\mathcal{G}}} \arrow{d}{\pi_{\mathrm{TT}\mathcal{G}}} & \widetilde{M} \arrow{d}{\cong} \\
		\mathcal{g} \arrow[r, "\pi_{\mathrm{T}\mathcal{G}}"]& M
	\end{tikzcd}
\end{center}
where $M$ is canonically embedded into $\mathcal{G}$ by $e$, and $\widetilde{M}$ is the further canonical embedding of this (the image of $e$) into $\mathrm{T}\mathcal{G}$ by the zero section. As in Appendix \ref{DoubleTangentFlip}, the addition of vectors and the scalar multiplication of the left vertical arrow is denoted as usual, while the one of the upper horizontal arrow will be denoted by $\RPlus$ and $\boldsymbol{\cdot}$, respectively. For $Y \in \mathrm{T}_xM$ we now have
\bas
\mleft.\nabla^{\mathcal{G}}_{\lambda X + \kappa Y} \nu \mright|_x
&=
\mathrm{pr}_2\Bigl(\mathrm{D}_{\mathrm{D}_x e(\lambda X + \kappa Y)}\mu_{\mathcal{G}}^{\mathrm{tot}} \bigl( \nu_T (\lambda X + \kappa Y) \bigr) \Bigr)
\\
&=
\mathrm{pr}_2\Bigl(\mathrm{D}_{\mathrm{D}_x e(\lambda X + \kappa Y)}\mu_{\mathcal{G}}^{\mathrm{tot}} \bigl( 
	\lambda \boldsymbol{\cdot} \nu_T ( X )
	\RPlus \kappa \boldsymbol{\cdot} \nu_T (Y) 
\bigr) \Bigr)
\\
&=
\mathrm{pr}_2\Bigl(
	\lambda \boldsymbol{\cdot} \mathrm{D}_{\mathrm{D}_x e(X)}\mu_{\mathcal{G}}^{\mathrm{tot}} \bigl( \nu_T(X) \bigr)
	\RPlus \kappa \boldsymbol{\cdot} \mathrm{D}_{\mathrm{D}_x e(Y)}\mu_{\mathcal{G}}^{\mathrm{tot}} \bigl( \nu_T(Y) \bigr)
\Bigr)
\\
&=
\lambda ~ \mathrm{pr}_2\Bigl( \mathrm{D}_{\mathrm{D}_x e(X)}\mu_{\mathcal{G}}^{\mathrm{tot}} \bigl( \nu_T(X) \bigr) \Bigr)
	+ \kappa ~ \mathrm{pr}_2 \Bigl(\mathrm{D}_{\mathrm{D}_x e(Y)}\mu_{\mathcal{G}}^{\mathrm{tot}} \bigl( \nu_T(Y) \bigr) \Bigr)
\\
&=
\lambda ~ \mleft.\nabla^{\mathcal{G}}_{X} \nu \mright|_x
	+ \kappa ~ \mleft.\nabla^{\mathcal{G}}_{Y} \nu \mright|_x
\eas
for all $\lambda, \kappa \in \mathbb{R}$, using Remark \ref{rem:TotalDerivativesAreLinearWithRTOtherLinearStructure}, \ref{rem:BothLinearStructuresTheSameOnTheVerticalBundle} and \ref{rem:TangentLifts}. By these remarks we also derive for another section $\mu \in \Gamma(\mathcal{g})$
\bas
\mleft.\nabla^{\mathcal{G}}_{X} (\lambda \nu + \kappa \mu) \mright|_x
&=
\mathrm{pr}_2\Bigl(\mathrm{D}_{\mathrm{D}_x e(X)}\mu_{\mathcal{G}}^{\mathrm{tot}} \bigl( \underbrace{(\lambda \nu + \kappa \mu)_T}_{\mathclap{ = \lambda \nu_T + \kappa \mu_T }} (X) \bigr) \Bigr)
\\
&=
\lambda~ \mathrm{pr}_2\Bigl(
	\mathrm{D}_{\mathrm{D}_x e(X)}\mu_{\mathcal{G}}^{\mathrm{tot}} \bigl( 
		\nu_T (X) 
	\bigr) 
\Bigr)
	+ \kappa~ \mathrm{pr}_2\Bigl(
	\mathrm{D}_{\mathrm{D}_x e(X)}\mu_{\mathcal{G}}^{\mathrm{tot}} \bigl( 
		\mu_T (X) 
	\bigr) 
\Bigr)
\\
&=
\lambda \mleft.\nabla^{\mathcal{G}}_{X} \nu \mright|_x
	+ \kappa \mleft.\nabla^{\mathcal{G}}_{X} \mu \mright|_x,
\eas
and
\bas
\mleft.\nabla^{\mathcal{G}}_{X} (f \nu) \mright|_x
&=
\mathrm{pr}_2\Bigl(
	\mathrm{D}_{\mathrm{D}_x e(X)}\mu_{\mathcal{G}}^{\mathrm{tot}} \bigl( (f\nu)_T (X) \bigr)
\Bigr)
\\
&=
\mathrm{pr}_2\mleft(
	\mathrm{D}_{\mathrm{D}_x e(X)}\mu_{\mathcal{G}}^{\mathrm{tot}} \mleft( f(x) ~ \nu_T(X) 
	+ X(f) ~ \nu^a \mleft.\frac{\partial}{\partial \xi^a}\mright|_{\mathrm{D}_x e(X)} \mright)
\mright)
\\
&=
f(x) ~ \mleft.\nabla^{\mathcal{G}}_{X} \nu \mright|_x
	+ X(f) ~ \mathrm{pr}_2\mleft(
	\mathrm{D}_{\mathrm{D}_x e(X)}\mu_{\mathcal{G}}^{\mathrm{tot}} \mleft( \nu^a \mleft.\frac{\partial}{\partial \xi^a}\mright|_{\mathrm{D}_x e(X)} \mright)
\mright)
\eas
for all $f \in C^\infty(M)$, where $\mleft( \xi^a \mright)_a$ are fibre coordinates of $\mathcal{g}$.
%so that we can write $\bigl($$\mleft( x^i \mright)_i$ coordinates on $M$$\bigr)$
%\bas
%\mathrm{D}_x \nu (X)
%&=
%X^i \mleft.\frac{\partial}{\partial \mleft(\pi_{\mathrm{T}\mathcal{G}}^*x^i\mright)}\mright|_{\nu_x}
	%+ X^i \mleft.\frac{\partial \nu^a}{\partial x^i}\mright|_x \mleft.\frac{\partial}{\partial \xi^a}\mright|_{\nu_x}
%\eas
It was well-defined to use the linearity of $\mathrm{pr}_2$, since we know by what we have shown earlier that both, $\mathrm{D}_{\mathrm{D}_x e(X)}\mu_{\mathcal{G}}^{\mathrm{tot}} \bigl( (f\nu)_T (X) \bigr)$ and $\mathrm{D}_{\mathrm{D}_x e(X)}\mu_{\mathcal{G}}^{\mathrm{tot}} \bigl( f \nu_T (X) \bigr) = f ~ \mathrm{D}_{\mathrm{D}_x e(X)}\mu_{\mathcal{G}}^{\mathrm{tot}} \bigl( \nu_T (X) \bigr)$, are elements of $\mathrm{V}\mathcal{g}$, so that 
\bas
\mathrm{D}_{\mathrm{D}_x e(X)}\mu_{\mathcal{G}}^{\mathrm{tot}} \mleft( \nu^a \mleft.\frac{\partial}{\partial \xi^a}\mright|_{\mathrm{D}_x e(X)} \mright)
\eas
is vertical, too.

%As fibre coordinates of $\mathcal{g}$, there are curves $\gamma^a: I \to \mathcal{g}$ ($I$ an open interval containing 0) with 
%\bas
%\gamma^a(0) &= e_a,&
%\mleft. \frac{\mathrm{d}}{\mathrm{d}t} \mright|_{t=0} \gamma^a &= \mleft.\frac{\partial}{\partial \xi^a}\mright|_{\mathrm{D}_x e(X)},
%\eas
%where $\mleft( e_a \mright)_a$ is a local frame of $\mathcal{g}$ dual to $\xi^a$. 
Making use of the aforementioned isomorphism $\mathrm{V}\mathcal{g} \cong \pi^*_{\mathrm{T}\mathcal{G}}\mathcal{g}$, we can write similar as in Remark \ref{rem:BothLinearStructuresTheSameOnTheVerticalBundle}
\bas
\mleft.\frac{\partial}{\partial \xi^a}\mright|_{\mathrm{D}_x e(X)}
&\cong
\mleft(
	\mathrm{D}_x e(X), e_a
\mright),
\eas
where $\mleft(e_a\mright)_a$ is a local frame of $\mathcal{g}$ dual to $\xi^a$. Thus, 
\bas
\mathrm{pr}_2\mleft( \mleft.\frac{\partial}{\partial \xi^a}\mright|_{\mathrm{D}_x e(X)}\mright)
&=
e_a.
\eas
By definition, $\mu_{\mathcal{G}}^{\mathrm{tot}}$ acts as identity on $\mathcal{g} = e^*\mathrm{V}\mathcal{G}$, and so $\mathrm{D}\mu_{\mathcal{G}}^{\mathrm{tot}}$ is the identity on $\mathrm{T}\mathcal{g}$. Thus, we derive
\bas
\mathrm{pr}_2\mleft(
	\mathrm{D}_{\mathrm{D}_x e(X)}\mu_{\mathcal{G}}^{\mathrm{tot}} \mleft( \nu^a \mleft.\frac{\partial}{\partial \xi^a}\mright|_{\mathrm{D}_x e(X)} \mright)
\mright)
&=
\mathrm{pr}_2\mleft(
	\nu^a \mleft.\frac{\partial}{\partial \xi^a}\mright|_{\mathrm{D}_x e(X)}
\mright)
=
\nu^a e_a
=
\nu,
\eas
thus, finally,
\bas
\mleft.\nabla^{\mathcal{G}}_{X} (f \nu) \mright|_x
&=
f(x) ~ \mleft.\nabla^{\mathcal{G}}_{X} (f \nu) \mright|_x
	+ X(f) ~ \nu.
\eas
This finishes the proof.
%\bas
%\mathrm{D}_{\mathrm{D}_x e(X)}\mu_{\mathcal{G}}^{\mathrm{tot}} \mleft( \nu^a \mleft.\frac{\partial}{\partial \xi^a}\mright|_{\mathrm{D}_x e(X)} \mright)
%&=
%\mleft. \frac{\mathrm{d}}{\mathrm{d}t} \mright|_{t=0}
%\eas
\end{proof}

\begin{definitions}{LGB connection on its LAB}{ConnectionOnLAB}
Let $\mathcal{G} \stackrel{\pi_{\mathcal{G}}}{\to} M$ be an LGB over a smooth manifold $M$, and $\mathrm{H}\mathcal{G}$ be a horizontal distribution of $\mathcal{G}$. Then we call the vector bundle connection $\nabla^{\mathcal{G}}$ on $\mathcal{g}$ of Prop.\ \ref{prop:FinallyTheNablaInduction}, shortly denoted by
\bas
\nabla^{\mathcal{G}} \nu
&=
\mleft.\frac{\mathrm{d}}{\mathrm{d}t}\mright|_{t=0} \Delta \e^{t \nu}
\eas
for all $\nu \in \Gamma(\mathcal{g})$, the \textbf{$\mathcal{G}$-connection (on the LAB $\mathcal{g}$)}.
\end{definitions}

\begin{remark}\label{PTLABComingFromPTLGB}
\leavevmode\newline
Such a construction of vector bundle connections on LABs also arises in \cite[\S 4.5, Prop.\ 4.22]{LAURENTGENGOUXStienonXuMultiplicativeForms}, but w.r.t.\ a more specific $\mathrm{H}\mathcal{G}$, see Subsubsection \ref{MultiplicativeForms} later; in this context that reference also shows that the parallel transport associated to $\nabla^{\mathcal{G}}$ is the infinitesimal version of the parallel transport associated with $\mathrm{H}\mathcal{G}$.
\end{remark}

\begin{examples}{Canonical flat $\mathcal{G}$-connection}{CanonicalFlatGConnection}
If we again focus on trivial LGBs with their canonical flat connection as in Remark \ref{rem:TrivialLGBsAndTheirMCForm} and \ref{rem:DarbouxOnCanonFlat}, we can quickly derive that $\nabla^{\mathcal{G}}$ is then the canonical flat connection on the trivial LAB $\mathcal{g} = M \times \mathfrak{g}$ of $\mathcal{G} = M \times G$, $\mathfrak{g}$ the Lie algebra of the Lie group $G$. Let $\nu$ be a constant section of $\mathcal{g}$, then $\e^{t\nu}$ is a constant section of $\mathcal{G}$ for all $t \in \mathbb{R}$, so that by Remark \ref{rem:DarbouxOnCanonFlat}
\bas
\Delta^{\mathcal{G}} \e^{t\nu}
&=
0,
\eas
and thus
\bas
\nabla^{\mathcal{G}} \nu
&=
0
\eas
for all constant sections $\nu \in \Gamma(\mathcal{g})$. By the uniqueness of the canonical flat connection (w.r.t.\ a trivialisation) we conclude that $\nabla^{\mathcal{G}}$ is the canonical flat connection on $\mathcal{g} = M \times \mathfrak{g}$.
\end{examples}

\begin{remarks}{Pullback situation: Part III}{PullBackConnectionOfNablaG}
Following Remark \ref{rem:PullBackDarboux} (see Remark \ref{rem:GeneralPullBackLGBConnSituation} for the initial setup) and all the involved notation we can again derive something related to the pullback LGB $f^*\mathcal{G}$. First of all, let us denote the exponential of $f^*\mathcal{G}$ and $\mathcal{G}$ by $\e_{f^*\mathcal{G}}$ and $\e_{\mathcal{G}}$, respectively, then we clearly have
\bas
\e_{f^*\mathcal{G}}^{(p, \nu_x)}
&=
\exp_{\mleft(f^*\mathcal{G}\mright)_p}(p, \nu_x)
=
\mleft( p, \exp_{\mathcal{G}_x}(\nu_x) \mright)
=
\mleft( p, \e_{\mathcal{G}}^{\nu_x} \mright)
\eas
for all $(p, \nu_x) \in \mleft(f^*\mathcal{g}\mright)_p$, where $x \coloneqq f(p)$ and so $\nu_x \in \mathcal{g}_x$, and where we made use of that $\mleft(f^*\mathcal{G}\mright)_p \cong \mathcal{G}_x$ as Lie groups via the projection $\mathrm{pr}_2$ onto the second component (recall Cor.\ \ref{cor:PullbackLGB}), that is, we have used the well-known relation
\bas
\mathrm{pr}_2\mleft( \exp_{\mleft(f^*\mathcal{G}\mright)_p}(p, \nu_x) \mright)
&=
\mleft(\exp_{\mathcal{G}_x} \circ~ \mathrm{D}_{(p, e_x)}\mathrm{pr}_2\mright)(p, \nu_x)
=
\exp_{\mathcal{G}_x}(\nu_x),
\eas
see \textit{e.g.}\ \cite[\S 1.7, Thm.\ 1.7.16, page 59]{Hamilton} for the general formula; then use that $\mathrm{pr}_2|_{\mleft( f^*\mathcal{G} \mright)_p}$ is bijective. Hence, we get
\bas
\e_{f^*\mathcal{G}}^{f^*\nu}
&=
f^*\mleft( \e_{\mathcal{G}}^\nu \mright)
\eas
for all $\nu \in \Gamma(\mathcal{g})$. Due to clatch of notation we relabel the projection $\mathrm{V}\mathcal{g} \to \mathcal{g}$ in Prop.\ \ref{prop:FinallyTheNablaInduction} to $\pi_2$, and then observe, by using Eq.\ \eqref{PullBackDarbouxOnPullbackSections},
\bas
\mleft.\nabla^{f^*\mathcal{G}}_Y (f^*\nu)\mright|_p
&=
\pi_2 \mleft(
\mleft.\frac{\mathrm{d}}{\mathrm{d}t}\mright|_{t=0} \Bigl( \mleft(\Delta^{f^*\mathcal{G}} \e_{f^*\mathcal{G}}^{t \cdot f^*\nu}\mright)_p (Y) \Bigr)
\mright)
\\
&=
\pi_2 \mleft(
\mleft.\frac{\mathrm{d}}{\mathrm{d}t}\mright|_{t=0} \Bigl( \mleft(\Delta^{f^*\mathcal{G}} \bigl( f^*\mleft( \e_{\mathcal{G}}^{t\nu} \mright) \bigr) \mright)_p (Y) \Bigr)
\mright)
\\
&=
\pi_2 \mleft(
\mleft.\frac{\mathrm{d}}{\mathrm{d}t}\mright|_{t=0} \Bigl( \mleft( f^!\mleft( \Delta^{\mathcal{G}} \e_{\mathcal{G}}^{t\nu} \mright) \mright)_p (Y) \Bigr)
\mright)
\\
&=
\pi_2 \mleft(
\mleft.\frac{\mathrm{d}}{\mathrm{d}t}\mright|_{t=0} \biggl( p, 
\mleft( \Delta^{\mathcal{G}} \e_{\mathcal{G}}^{t\nu} \mright)_{f(p)} \bigl( \mathrm{D}_pf(Y) \bigr) 
\biggr)
\mright)
\\
&=
\pi_2 \mleft( 0,
\mleft.\frac{\mathrm{d}}{\mathrm{d}t}\mright|_{t=0} \biggl( 
\mleft( \Delta^{\mathcal{G}} \e_{\mathcal{G}}^{t\nu} \mright)_{f(p)} \bigl( \mathrm{D}_pf(Y) \bigr) 
\biggr)
\mright)
\\
&=
\pi_2 \mleft(
\mleft.\frac{\mathrm{d}}{\mathrm{d}t}\mright|_{t=0} \biggl( 
\mleft( \Delta^{\mathcal{G}} \e_{\mathcal{G}}^{t\nu} \mright)_{f(p)} \bigl( \mathrm{D}_pf(Y) \bigr) 
\biggr)
\mright)
\\
&=
\mleft.\nabla^{\mathcal{G}}_{\mathrm{D}_pf(Y)} \nu\mright|_{f(p)}
\eas
for all $\nu \in \Gamma(\mathcal{g})$, $p \in N$ and $Y \in \mathrm{T}_p N$. By the uniqueness of pullback vector bundle connections we therefore get
\bas
\nabla^{f^*\mathcal{G}}
&=
f^*\nabla^{\mathcal{G}}.
\eas
\end{remarks}

\subsubsection{First step towards towards associated bundles}

In order to provide a concise definition of connection 1-forms on principal bundles, we will need to introduce some canonical form of action which will be obviously related to the action needed for an analogue to the notion of associated bundles related to typical principal bundles. However, we will neither discuss nor introduce a more general notion of associated bundles in this paper; so, there will be just the "first step". The following action can be seen as another canonical action on the pullback of a principal bundle, but coming from $\mathcal{G}$ itself instead of its pullback, if $\mathcal{G}$ acts on the manifold one makes a pullback to.

\begin{propositions}{Canonical LGB action on pullback bundles over principal LGB-bundles}{CanonicalLGBActionForAssociatedBundles}
Let $\mathcal{G} \stackrel{\pi_{\mathcal{G}}}{\to} M$ be an LGB over a smooth manifold $M$ and $\mathcal{P} \stackrel{\pi}{\to} M$ a principal $\mathcal{G}$-bundle. Furthermore let $N$ be another smooth manifold and $f: N \to M$ a smooth map on which $\mathcal{G}$ acts on the left as in Def.\ \ref{def:LiegroupACtion}. Then on the pullback manifold $\mathcal{P} \times_M N \coloneqq f^*\mathcal{P}$, whose pairs of points are now reordered as in
\bas
\mathcal{P} \times_M N
&\coloneqq
\left\{
	(p, x) \in \mathcal{P} \times N
	~\middle|~
	\pi(p) = f(x)
\right\},
\eas
we have a right $\mathcal{G}$-action given by
\bas
\mleft(\mathcal{P} \times_M N\mright) * \mathcal{G} &\to \mathcal{P} \times_M N,\\
(p, x, g) &\mapsto (p, x) \cdot g \coloneqq \mleft( p \cdot g, g^{-1} \cdot x \mright)
\eas
where $\mleft(\mathcal{P} \times_M N\mright) * \mathcal{G}$ is given as pullback of $\mathcal{G}$ w.r.t.\ the map $\widetilde{\pi}: \mathcal{P} \times_M N \to M$, $(p,x) \mapsto \pi(p)$.
\end{propositions}

\begin{proof}
\leavevmode\newline
By Def.\ \ref{def:LiegroupACtion} it is clear that this action is well-defined, due to 
\bas
\pi(p\cdot g) = \pi_{\mathcal{G}}(g) = f\mleft( g^{-1} \cdot x \mright)
\eas
for all $(p,x,g) \in \mleft(\mathcal{P} \times_M N\mright) * \mathcal{G}$,
and this also implies
\bas
\widetilde{\pi}\bigl( (p, x) \cdot g \bigr)
&=
\pi(p\cdot g)
=
\pi_{\mathcal{G}}(g),
\eas
so that Eq.\ \eqref{InvarianceOffUnderGAction} is satisfied. By construction we also have a smooth action, since it is the composition of maps
\bas
\mleft(\mathcal{P} \times_M N\mright) * \mathcal{G} &\to (\mathcal{P} * \mathcal{G}) \times (\mathcal{G} * N) \to \mathcal{P} \times_M N,\\
(p, x, g) &\mapsto \bigl( (p, g), (g, x) \bigr) \mapsto \mleft( p \cdot g, g^{-1} \cdot x \mright).
\eas
The first arrow is clearly an embedding, and the second arrow is just the restriction of the smooth diagonal action,
\bas
(\mathcal{P} * \mathcal{G}) \times (\mathcal{G} * N) &\to \mathcal{P} \times N,\\
\bigl( (p, g), (q, p) \bigr) &\mapsto \mleft( p \cdot g, q^{-1} \cdot p \mright),
\eas
onto an embedded submanifold, and for its image make use of that $\mathcal{P} \times_M N$ is an embedded submanifold of $\mathcal{P} \times N$, as we already did several times for pullback bundles.

Associativity follows simply by
\bas
\bigl((p, x) \cdot g \bigr) \cdot q
&=
\mleft( p \cdot g, g^{-1} \cdot x \mright) \cdot q
=
\mleft( p \cdot gq, (gq)^{-1} \cdot x \mright)
=
\mleft( p, x \mright) \cdot (gq)
\eas
for all $g, q \in \mathcal{G}_y$ ($y \coloneqq \pi(p) = f(x)$); it is trivial to check that $\mleft( p, x \mright) \cdot e_y = (p, x)$.
\end{proof}

As mentioned in Remark \ref{rem:ItIsAPrincipalAction}, constructions of quotients related to the LGB action may be possible here. Thus, using the last proposition, one should be able to construct associated bundles in this more general setting. We mainly need this proposition for the following examples.

\begin{examples}{Adjoint action on the vertical bundle of $\mathcal{P}$}{AdjointACtionOnVP}
Recall the adjoint representation of $\mathcal{G}$ on its LAB $\mathcal{g} \stackrel{\pi_{\mathcal{g}}}{\to} M$, Ex.\ \ref{ex:LGBAdjointRep}. Using the notation of Prop.\ \ref{prop:CanonicalLGBActionForAssociatedBundles} we have a right $\mathcal{G}$-action on $\mathcal{P} \times_M \mathcal{g} = \{(p, v) \in \mathcal{P} \times \mathcal{g} ~ | ~ \pi(p) = \pi_{\mathcal{g}}(v) \}$ given by
\bas
(p, v) \cdot g
&\coloneqq
\mleft( p \cdot g, \mathrm{Ad}_{g^{-1}}(v) \mright)
\eas
for all $p \in \mathcal{P}_x$ ($x \in M$), $v \in \mathcal{g}_x$ and $g \in \mathcal{G}_x$. Observe that $\mathcal{P} \times_M \mathcal{g} = \pi^*\mathcal{g}$ which is isomorphic to $\mathrm{V}\mathcal{P}$ by Cor.\ \ref{cor:VerticalBundleOfPrincIsNearlyAsUsual}. We will denote this action shortly by
\bas
\sAd_{g^{-1}}(p, v)
&\coloneqq
(p,v) \cdot g,
\eas
the \textbf{adjoint representation of $\mathcal{G}$ on $\mathrm{V}\mathcal{P}$}; not to be confused with the adjoint representation of $\pi^*\mathcal{G}$ on $\pi^*\mathcal{g}$. In fact, it is trivial to check that $\sAd: \mathcal{G} \to \mathrm{Aut}(\pi^*\mathcal{g})$, $g \mapsto \sAd_{g}$, is a $\mathcal{G}$-representation on $\mathcal{P} \times_M \mathcal{g} = \pi^*\mathcal{g} \cong \mathrm{V}\mathcal{P}$ in the sense of Cor.\ \ref{cor:LGBRepAsLGBMorph}.

The quotient of $\mathcal{P} \times_M \mathcal{g}$ w.r.t.\ this group action should lead to a structure which is the generalization of the adjoint bundle in typical formulations of gauge theory.
\end{examples}

Similarly, we can define the conjugation action on its integrated bundle; recall the conjugation defined in Def.\ \ref{def:LeftRightTranslationConjugation}. Especially observe that the conjugation defines a left $\mathcal{G}$-action on itself by
\bas
\mathcal{G} * \mathcal{G} &\to \mathcal{G},\\
(g,q) &\mapsto c_g(q) = g q g^{-1}.
\eas

\begin{examples}{Conjugation action over $\mathcal{P}$}{ConjugationActionForTheGeneralInnerGroupBundle}
Let us look at $\mathcal{P} \times_M \mathcal{G} \cong \pi^*\mathcal{G}$. Then we have a right $\mathcal{G}$-action on $\mathcal{P} \times_M \mathcal{G}$ given by
\bas
(p, q) \cdot g
&\coloneqq
\mleft( p \cdot g, c_{g^{-1}}(q) \mright)
=
\mleft( p \cdot g, g^{-1} q g \mright)
\eas
for all $p \in \mathcal{P}_x$ ($x \in M$), and $q,g \in \mathcal{G}_x$.
We will denote this also by
\bas
\mathcal{c}_{g^{-1}}(p, q)
\coloneqq
(p, q) \cdot g,
\eas
the \textbf{conjugation of $\mathcal{G}$ on the integral of $\mathrm{V}\mathcal{P}$}.

Taking a quotient of $\mathcal{P} \times_M \mathcal{G}$ over this action should lead to a generalization of inner group bundles as introduced in Ex.\ \ref{ex:InnerLGBs}.
\end{examples}

Let us finally define connections on principal bundles.

\subsubsection{Generalized connection 1-forms on principal bundles}

As also stated in \cite[\S 5.1, Prop.\ 5.1.5, page 260]{Hamilton}, for a given horizontal distribution $\mathrm{H}\mathcal{P}$ of a principal $\mathcal{G}$-bundle $\mathcal{P} \stackrel{\pi}{\to} M$ over a smooth manifold $M$ we have by construction that
\bas
\mleft.\mathrm{D}_p \pi\mright|_{\mathrm{H}_p\mathcal{P}}: \mathrm{H}_p\mathcal{P} \to \mathrm{T}_xM
\eas
is a vector space isomorphism for all $x \in M$ and $p \in \mathcal{P}_x$; similarly for $\mathcal{G}$ itself. Using that, one has some sort of identification between the horizontal tangent spaces; we want to provide another identification in the sense of Eq.\ \eqref{OiTHatIsHowWeFormulateHorizSymmetry}, also recall Rem.\ \ref{rem:DefOfConnectionIdeaWithDarboux}; especially recall the Darboux derivative introduced in Subsubsection \ref{DiscussingDarbouxDerivativeGeneral}. The following proposition and definition will be needed.

\begin{propositions}{The right-pushforward modified by the Darboux derivative is a well-defined isomorphism}{IsomorphismRightPushAndDarboux}
Let $\mathcal{G} \to M$ be an LGB over a smooth manifold $M$ and $\mathcal{P} \stackrel{\pi}{\to} M$ a principal $\mathcal{G}$-bundle, and we have horizontal a distribution $\mathrm{H}\mathcal{G}$ on $\mathcal{G}$. Furthermore, for $g \in \mathcal{G}_x$ ($x \in M$) define the map
\bas
\mleft.\mathrm{T}\mathcal{P}\mright|_{\mathcal{P}_x} &\to \mleft.\mathrm{T}\mathcal{P}\mright|_{\mathcal{P}_x},\\
X 
&\mapsto 
\mathcal{r}_{g*}(X) 
\coloneqq
\mathrm{D}_pr_\sigma\mleft( 
	X 
\mright)
	- \mleft.{\oversortoftilde{
		\mleft. \mleft( \pi^!\Delta\sigma \mright) \mright|_p(X)
	}}\mright|_{p \cdot g},
\eas
where $p \in \mathcal{P}_x$, $X \in \mathrm{T}_p \mathcal{P}$, and $\sigma$ is any (local) section of $\mathcal{G}$ with $\sigma_x = g$. Then $\mathcal{r}_{g*}$ is independent of the choice of the local section $\sigma$, and it is a vector bundle automorphism over the right-translation $r_g$. 

Furthermore, if we also have a horizontal distribution $\mathrm{H}\mathcal{P}$ on $\mathcal{P}$, then $\mathrm{H}_p\mathcal{P}$ is isomorphic via $\mathcal{r}_{g*}$ to a complement of $\mathrm{V}_{p \cdot g} \mathcal{P}$ in $\mathrm{T}_{p \cdot g}\mathcal{P}$ (this complement is not necessarily $\mathrm{H}_{p \cdot g}\mathcal{P}$).
\end{propositions}

\begin{proof}
\leavevmode\newline
In the following we have $(p, g) \in \mathcal{P} * \mathcal{G}$, $X \in \mathrm{T}_p \mathcal{P}$, and $\sigma$ is any (local) section of $\mathcal{G}$ with $\sigma_x = g$ ($x \coloneqq \pi(p)$).

$\bullet$ Let $\pi_{\mathcal{G}}$ be the projection of $\mathcal{G}$, then as mentioned before Prop.\ \ref{prop:IsomorphismRightPushAndDarboux} there is a unique $Y \in \mathrm{H}_g\mathcal{G}$ with
\bas
\mathrm{D}_g\pi_{\mathcal{G}}(Y)
&=
\mathrm{D}_p\pi(X)
\in
\mathrm{T}_x M.
\eas
By Remark \ref{rem:DarbouxInActionInfinit} we can derive
\bas
\mathrm{D}_{(p, g)}\Phi(X,Y)
&=
\mathcal{r}_{g*}(X),
\eas
where we made use of that $Y$ is horizontal in $\mathcal{G}$ so that $\mleft(\mu_{\mathcal{G}}^{\mathrm{tot}}\mright)_g(Y) = 0$, and where $\Phi$ denotes the right $\mathcal{G}$-action on $\mathcal{P}$ as a map $\mathcal{P}*\mathcal{G} \to \mathcal{P}$. Therefore the independence of $\mathcal{r}_{g*}$ w.r.t.\ the choice of $\sigma$ follows.

$\bullet$ $\mathcal{r}_{g*}$ is clearly linear by construction. Let us now show fibre-wise that $\mathcal{r}_{g*}$ is injective, then it is also bijective by dimensional reasons. By Remark \ref{SmoothnessOfACtionTranslations} we know that $r_\sigma$ is a diffeomorphism, and thus $\mathrm{D}_pr_\sigma: \mathrm{T}_p\mathcal{P} \to \mathrm{T}_{p \cdot g}\mathcal{P}$ is a vector space isomorphism. Restricted onto the vertical subspace $\mathrm{V}_p\mathcal{P}$ we have by Remark \ref{rem:AbstractNotationTwoForLeftInvarVfs}
\bas
\mleft.\mathrm{D}_pr_\sigma\mright|_{\mathrm{V}_p\mathcal{P}}
&=
\mathrm{D}_pr_g:\mathrm{V}_p\mathcal{P} \to \mathrm{V}_{p\cdot g}\mathcal{P},
\eas
which is a vector space isomorphism by Cor.\ \ref{cor:VerticalBundleOfPrincIsNearlyAsUsual} and dimensional reasons (which implies that $\mathrm{D}_pr_g$ is surjective and thus bijective). We rewrite $\mathrm{D}_pr_\sigma$ as
\bas
\mathrm{H}_p\mathcal{P} \oplus \mathrm{V}_p\mathcal{P} &\to \mathrm{T}_{p \cdot g}\mathcal{P},\\
\mleft(X^{\mathrm{H}},~ X^{\mathrm{V}}\mright) &\mapsto 
\mleft.\mathrm{D}_pr_\sigma\mright|_{\mathrm{H}_p\mathcal{P}}\mleft(X^{\mathrm{H}}\mright)
	+ \mathrm{D}_pr_g\mleft(X^{\mathrm{V}}\mright),
\eas
where we fix just any horizontal distribution $\mathrm{H}\mathcal{P}$.
By dimensional reasons and due to the bijectivity of $\mathrm{D}_pr_\sigma$ and $\mathrm{D}_pr_g$ the image $\mathrm{Im}\mleft( \mleft.\mathrm{D}_pr_\sigma\mright|_{\mathrm{H}_p\mathcal{P}} \mright)$ of $\mleft.\mathrm{D}_pr_\sigma\mright|_{\mathrm{H}_p\mathcal{P}}$ has to be a complement subspace of $\mathrm{V}_{p \cdot g}\mathcal{P}$ in $\mathrm{T}_{p \cdot g}\mathcal{P}$; $\mathrm{Im}\mleft( \mleft.\mathrm{D}_pr_\sigma\mright|_{\mathrm{H}_p\mathcal{P}} \mright)$ is not necessarily equal to $\mathrm{H}_{p \cdot g}\mathcal{P}$ in general but of the same dimension, especially of the same dimension as $\mathrm{H}_p \mathcal{P}$. Thus, $\mleft.\mathrm{D}_pr_\sigma\mright|_{\mathrm{H}_p\mathcal{P}}$ is a vector space isomorphism onto its image which is complementary to $\mathrm{V}_{p\cdot g}\mathcal{P}$. Furthermore, observe
\bas
\mleft. \mleft( \pi^!\Delta\sigma \mright) \mright|_p(X)
&=
\mleft. \mleft( \Delta\sigma \mright) \mright|_x \mleft( \mathrm{D}_p\pi \mleft( X^{\mathrm{H}} + X^{\mathrm{V}} \mright) \mright)
=
\mleft. \mleft( \Delta\sigma \mright) \mright|_x \mleft( \mathrm{D}_p\pi \mleft( X^{\mathrm{H}} \mright) \mright)
=
\mleft. \mleft( \pi^!\Delta\sigma \mright) \mright|_p\mleft( X^{\mathrm{H}} \mright),
\eas
using that the vertical subbundle is given by the kernel of $\mathrm{D}\pi$, and
where we split again $X = X^{\mathrm{H}} + X^{\mathrm{V}}$ for all $X\in\mathrm{T}_p\mathcal{P} = \mathrm{H}_p\mathcal{P} \oplus \mathrm{V}_p\mathcal{P}$. Hence, we get in total that $\mathcal{r}_{g*}$ at $p$ is equivalent to
\bas
\mathrm{H}_p\mathcal{P} \oplus \mathrm{V}_p\mathcal{P} 
&\to 
\mathrm{Im}\mleft( \mleft.\mathrm{D}_pr_\sigma\mright|_{\mathrm{H}_p\mathcal{P}} \mright) \oplus \mathrm{V}_{p \cdot g}\mathcal{P},\\
\mleft(X^{\mathrm{H}},~ X^{\mathrm{V}}\mright) 
&\mapsto 
\mleft(
	\mleft.\mathrm{D}_pr_\sigma\mright|_{\mathrm{H}_p\mathcal{P}}\mleft(X^{\mathrm{H}}\mright),~
		\mathrm{D}_pr_g\mleft(X^{\mathrm{V}}\mright)
		- \mleft.{\oversortoftilde{
		\mleft. \mleft( \pi^!\Delta\sigma \mright) \mright|_p\mleft( X^{\mathrm{H}} \mright)
	}}\mright|_{p \cdot g}
\mright).
\eas
Let $\mleft(X^{\mathrm{H}},~ X^{\mathrm{V}}\mright)$ be now in the kernel of $\mathcal{r}_{g*}$, then $X^{\mathrm{H}} = 0$ by the bijectivity of $\mleft.\mathrm{D}_pr_\sigma\mright|_{\mathrm{H}_p\mathcal{P}}$ onto its image. The second component of $\mathcal{r}_{g*}$ is then just $\mathrm{D}_pr_g\mleft(X^{\mathrm{V}}\mright)$; again by the bijectivity of $\mathrm{D}_pr_g: \mathrm{V}_p\mathcal{P} \to \mathrm{V}_{p\cdot g}\mathcal{P}$ we also get $X^{\mathrm{V}}=0$. Thus, $\mathcal{r}_{g*}$ is injective and therefore defines vector space isomorphisms $\mathrm{T}_p\mathcal{P} \to \mathrm{T}_{p \cdot g}\mathcal{P}$. It follows that $\mathcal{r}_{g*}$ is an automorphism of $\mleft.\mathrm{T}\mathcal{P}\mright|_{\mathcal{P}_x}$ over $r_g$, independent of the choice of $\mathrm{H}\mathcal{P}$.

$\bullet$ It is then clear that $\mleft.\mathcal{r}_{g*}\mright|_{\mathrm{H}_p\mathcal{P}}$, given by
\bas
\mathrm{H}_p\mathcal{P} 
&\to 
\mathrm{Im}\mleft( \mleft.\mathrm{D}_pr_\sigma\mright|_{\mathrm{H}_p\mathcal{P}} \mright) \oplus \mathrm{V}_{p \cdot g}\mathcal{P},\\
X^{\mathrm{H}}
&\mapsto 
\mleft(
	\mleft.\mathrm{D}_pr_\sigma\mright|_{\mathrm{H}_p\mathcal{P}}\mleft(X^{\mathrm{H}}\mright),~
		- \mleft.{\oversortoftilde{
		\mleft. \mleft( \pi^!\Delta\sigma \mright) \mright|_p\mleft( X^{\mathrm{H}} \mright)
	}}\mright|_{p \cdot g}
\mright),
\eas
is also an isomorphism onto its image; its image is complementary to $\mathrm{V}_{p\cdot g}\mathcal{P}$ because its intersection with $\mathrm{V}_{p \cdot g}\mathcal{P}$ would require that
\bas
\mleft.\mathrm{D}_pr_\sigma\mright|_{\mathrm{H}_p\mathcal{P}}\mleft(X^{\mathrm{H}}\mright)
&=
0,
\eas
which implies that $X^{\mathrm{H}} = 0$ due to the fact that $\mleft.\mathrm{D}_pr_\sigma\mright|_{\mathrm{H}_p\mathcal{P}}$ is a vector space isomorphism onto its image, as we discussed before. But then
\bas
\mleft.\mathcal{r}_{g*}\mright|_{\mathrm{H}_p\mathcal{P}}(0)
&=
0,
\eas
which implies that the image of $\mleft.\mathcal{r}_{g*}\mright|_{\mathrm{H}_p\mathcal{P}}$ is a complement of $\mathrm{V}_{p \cdot g} \mathcal{P}$ in $\mathrm{T}_{p \cdot g}\mathcal{P}$. This finishes the proof.
\end{proof}

Hence, we formally define:

\begin{definitions}{Modified pushforward via right-translation}{WeModTheRIghtPush}
Let $\mathcal{G} \to M$ be an LGB over a smooth manifold $M$ and $\mathcal{P} \stackrel{\pi}{\to} M$ a principal $\mathcal{G}$-bundle, and we have a horizontal distribution $\mathrm{H}\mathcal{G}$ on $\mathcal{G}$. Then we define the \textbf{modified right-pushforward\footnote{The font of $\mathcal{r}$ is a calligraphic r.} $\mathcal{r}_{g*}$ (with $g \in \mathcal{G}_x$, $x \in M$)} as the vector bundle isomorphism $\mleft.\mathrm{T}\mathcal{P}\mright|_{\mathcal{P}_x} \to \mleft.\mathrm{T}\mathcal{P}\mright|_{\mathcal{P}_x}$ over $r_g$ as given in Prop.\ \ref{prop:IsomorphismRightPushAndDarboux} by
\bas
\mathcal{r}_{g*}(X)
&\coloneqq
\mathrm{D}_pr_\sigma\mleft( X \mright)
	- \mleft.{\oversortoftilde{
		\mleft. \mleft( \pi^!\Delta\sigma \mright) \mright|_p(X)
	}}\mright|_{p \cdot g}
\eas
for all $p \in \mathcal{P}_x$ and $X \in \mathrm{T}_p \mathcal{P}$, where $\sigma$ is any (local) section of $\mathcal{G}$ with $\sigma_x = g$.

Similarly, for a (local) section $\sigma$ of $\mathcal{G}$ we define the \textbf{modified right-pushforward $\mathcal{r}_{\sigma*}$ with $\sigma$} as a (local) vector bundle isomorphism $\mathrm{T}\mathcal{P} \to \mathrm{T}\mathcal{P}$ by
\bas
\mathcal{r}_{\sigma*}(X)
&\coloneqq
\mathcal{r}_{\sigma_x*}(X)
=
\mathrm{D}_pr_\sigma\mleft( X \mright)
	- \mleft.{\oversortoftilde{
		\mleft. \mleft( \pi^!\Delta\sigma \mright) \mright|_p(X)
	}}\mright|_{p \cdot \sigma_{x}}
\eas
for all $X \in \mathrm{T}_p \mathcal{P}$ ($p \in \mathcal{P}_x$).
\end{definitions}

\begin{remarks}{Restriction onto vertical bundle gives typical right-pushforward}{ModRightPushOnVertic}
It is trivial to check that we have
\bas
\mathcal{r}_{g*}(X)
&=
\mathrm{D}_pr_g(X)
\eqqcolon
r_{g*}(X)
\eas
for all $X \in \mathrm{V}_p\mathcal{P}$; we actually have proven this directly after Eq.\ \eqref{OiTHatIsHowWeFormulateHorizSymmetry}. As also pinpointed in Remark \ref{rem:DefOfConnectionIdeaWithDarboux}, if $\mathcal{G}$ is a trivial LGB equipped with its canonical flat connection and $\sigma$ a constant section, then it is easy to see that $\mathcal{r}_{\sigma*} = \mathcal{r}_{g*} = r_{g*}$, where $r_g$ has to be understood as the right-translation of the canonical Lie group action inherited by $\mathcal{G}$ in the sense of Ex.\ \ref{ex:TrivialLGBAction}.
\end{remarks}

\begin{remark}\label{RSigmaAnAuto}
\leavevmode\newline
Prop.\ \ref{prop:IsomorphismRightPushAndDarboux} trivially extends to $\mathcal{r}_{\sigma*}$, that is, $\mathcal{r}_{\sigma*}$ is a (local) automorphism of $\mathrm{T}\mathcal{P}$ over $r_\sigma$. Thus, we can view $\mathcal{r}_{\sigma*}$ as an element of $\Omega^1\mleft(\mathcal{P}; r_\sigma^*\mathrm{T}\mathcal{P}\mright)$.
\end{remark}

Hence, the following definition makes sense, especially if thinking about what we discussed for Eq.\ \eqref{OiTHatIsHowWeFormulateHorizSymmetry}, also recall Rem.\ \ref{rem:DefOfConnectionIdeaWithDarboux}.

\begin{definitions}{Ehresmann connection on principal LGB-bundles}{FinallyTheConnection}
Let $\mathcal{G} \to M$ be an LGB over a smooth manifold $M$ and $\mathcal{P} \stackrel{\pi}{\to} M$ a principal $\mathcal{G}$-bundle, and we have horizontal distributions $\mathrm{H}\mathcal{G}$ and $\mathrm{H}\mathcal{P}$ on $\mathcal{G}$ and $\mathcal{P}$, respectively. We call $\mathrm{H}\mathcal{P}$ an \textbf{Ehresmann connection} or a \textbf{connection on $\mathcal{P}$} if it is \textbf{right-invariant (w.r.t.\ modified right-pushforward)}, \textit{i.e.}\
\bas
\mathcal{r}_{g*}\mleft( \mathrm{H}_p\mathcal{P} \mright)
&=
\mathrm{H}_{p\cdot g}\mathcal{P}
\eas
for all $p \in \mathcal{P}_x$ and $g \in \mathcal{G}_x$ ($x \coloneqq \pi(p)$).
\end{definitions}

\begin{remark}
\leavevmode\newline
There is also a definition of connections on such and more general principal bundles in \cite[\S 5.7, paragraph before Prop.\ 5.38, page 148]{GroupoidBasedPrincipalBundles}. However, this reference provides a different type of definition; translated to our situation, it is based on assuming that $\mathcal{G}$ is defined over another base manifold $N$. In order to define the LGB action on $\mathcal{P}$ this reference introduces a moment map $\mu: \mathcal{P} \to N$ so that the action of an element $\mathcal{G}_y$ ($y \in N$) is defined on $\mu^{-1}(\{y\})$, especially the infinitesimal action $r_g$ of a fixed LGB element $g$ acts on tangent vectors of $\mu^{-1}(\{y\})$, not necessarily on the vertical structure of $\mathcal{P}$. Hence, in order to circumvent the problem we discussed in Subsubsection \ref{TheBigMotivationBehindEverything}, the reference's definition of a connection is then based on assuming that the "fibres" $\mu^{-1}(\{y\})$ are complementary to the vertical structure of $\mathcal{P}$, so these fibres' tangent spaces define a horizontal distribution while the infinitesimal action $r_g$ now acts well-defined on the horizontal structure. Henceforth, the reference does not need to look at using sections $\sigma$ and their actions.

However, this is not a suitable definition for us, because our moment map is the projection of $\mathcal{P}$ itself such that $r_g$ acts on the vertical structure. The reference's definition is rather restrictive, while our definition works for all principal $\mathcal{G}$-bundles by fixing a connection on $\mathcal{G}$.
\end{remark}

Let us discuss several examples.

\begin{examples}{Recovering of the classical definition}{OurConnectionIsReallyMoreGeneral}
As we already discussed several times, especially recall Remark \ref{rem:ModRightPushOnVertic} and Ex.\ \ref{ex:TheCLassicalPrincAsEx}, but ultimately by the discussion for Eq.\ \eqref{OiTHatIsHowWeFormulateHorizSymmetry} which also applies to classical principal bundles and their notion of connection and parallel transport: We recover the typical definition of a connection on a principal bundle if $\mathcal{P}$ is a classical principal bundle and if the trivial LGB $\mathcal{G}$ is equipped with its canonical flat connection $\mathrm{H}\mathcal{G}$. We call such a connection a \textbf{classical connection}.
\end{examples}

\begin{examples}{Associated LGBs}{AssociatedLGBsAndTheirCanonicalConnection}
Recall Def.\ \ref{def:AssociatedLGB}, and recall that LGBs themselves are principal bundles as in Ex.\ \ref{ex:TrivialPrincAsLGB}. That is, let $G, H$ be Lie groups, $P \to M$ a principal $G$-bundle over a smooth manifold $M$, and $\psi$ a $G$-representation on $H$. Then we have the LGB associated to the principal bundle $P$ and the representation $\psi$ on $H$
\begin{center}
	\begin{tikzcd}
	H \arrow{r}& \mathcal{H} \coloneqq P \times_\psi H \arrow{d} \\
	& M
	\end{tikzcd}
\end{center}
Fix a typical classical connection on $P$; as also introduced in Subsubsection \ref{TheBigMotivationBehindEverything} we have an associated parallel transport $\mathrm{PT}_\alpha^P: P_x \to \Gamma(\alpha^*P)$ along a curve $\alpha: [0, t] \to M$ ($t > 0$), where $\alpha(0) \eqqcolon x$. As proven in \cite[\S 5.9, Thm.\ 5.9.1, page 289f.]{Hamilton}, we have a canonical well-defined parallel transport $\mathrm{PT}_\alpha^\mathcal{H}: \mathcal{H}_x \to \Gamma(\alpha^*\mathcal{H})$ given by
\bas
\mathrm{PT}_\alpha^\mathcal{H}\bigl( [p, h] \bigr)
&=
\mleft[ \mathrm{PT}_\alpha^P(p), h \mright]
\eas
for all $[p, h] \in \mathcal{H}_x$. This has a 1:1 correspondence to a horizontal distribution $\mathrm{H}\mathcal{H}$ on $\mathcal{H}$. Observe that we have
\bas
\mathrm{PT}_\alpha^\mathcal{H}\Bigl( [p, h] \cdot \mleft[ p, h^\prime \mright] \Bigr)
&=
\mathrm{PT}_\alpha^\mathcal{H}\Bigl( \mleft[ p, h h^\prime \mright] \Bigr)
\\
&=
\mleft[ \mathrm{PT}_\alpha^P(p), h h^\prime \mright]
\\
&=
\mleft[ \mathrm{PT}_\alpha^P(p), h \mright]
	\cdot \mleft[ \mathrm{PT}_\alpha^P(p), h^\prime \mright]
\\
&=
\mathrm{PT}_\alpha^\mathcal{H}\bigl( [p, h] \bigr)
	\cdot \mathrm{PT}_\alpha^\mathcal{H}\Bigl( \mleft[p, h^\prime\mright] \Bigr)
\eas
for all $[p, h], \mleft[ p, h^\prime \mright] \in \mathcal{H}_x$. Now recall that the whole motivation behind Def.\ \ref{def:FinallyTheConnection} comes from Eq.\ \eqref{OiTHatIsHowWeFormulateHorizSymmetry} (also recall Remark \ref{rem:DefOfConnectionIdeaWithDarboux}) which itself stems from Eq.\ \eqref{PTHomomNEw}. We see that that Eq.\ \eqref{PTHomomNEw} is satisfied here, and thus Eq.\ \eqref{OiTHatIsHowWeFormulateHorizSymmetry} follows, \textit{i.e.}\
\bas
\mathcal{r}_{\mleft[ p, h^\prime \mright]*}(X)
&\in
\mathrm{H}_{[p,h] \cdot \mleft[ p, h^\prime \mright]}\mathcal{H}
\eas
for all $X \in \mathrm{H}_{[p,h]}\mathcal{H}$.
By Prop.\ \ref{prop:IsomorphismRightPushAndDarboux} and dimensional reasons (horizontal subspaces are of the same dimension) it follows immediately that $\mathrm{H}\mathcal{H}$ is an (Ehresmann) connection on $\mathcal{H}$ in sense of Def.\ \ref{def:FinallyTheConnection}.

We call such connections on LGBs associated to a classical principal bundle an \textbf{associated connection}.
\end{examples}

As expected, we have a corresponding connection 1-form. For this we need to formally define the pullback of 1-forms with respect to the modified right-pushforward.

\begin{definitions}{The pullback of forms via modified right-pushforward}{PullbackOfFormsViaModRight}
Let $\mathcal{G} \to M$ be an LGB over a smooth manifold $M$ and $\mathcal{P} \stackrel{\pi}{\to} M$ a principal $\mathcal{G}$-bundle, and we have a fixed horizontal distribution $\mathrm{H}\mathcal{G}$ on $\mathcal{G}$. For $\omega \in \Omega^k(\mathcal{P}; \pi^*\mathcal{g})$ ($k \in \mathbb{N}_0$) we define the \textbf{pullback via the modified right-pushforward $\mathcal{r}_g^!\mleft(\mleft.\omega\mright|_{\mathcal{P}_x}\mright)$ (with $g\in\mathcal{G}_x$, over $\mathcal{P}_x$, $x \in M$)} as an element of $\Gamma \mleft( \Lambda^k\mleft.\mathrm{T}^*\mathcal{P}\mright|_{\mathcal{P}_x} \otimes \mathcal{g}_x \mright)$ by
\bas
\mleft.\mleft(\mathcal{r}_g^!\mleft(\mleft.\omega\mright|_{\mathcal{P}_x}\mright)\mright)\mright|_p \mleft( Y_1, \dotsc, Y_k\mright)
&\coloneqq
\omega_{p \cdot g} \bigl( \mathcal{r}_{g*}(Y_1), \dotsc, \mathcal{r}_{g*}(Y_k)\bigr)
\eas
for all $p \in \mathcal{P}_x$ and $Y_1, \dotsc, Y_k \in \mathrm{T}_p\mathcal{P}$. Similarly we define the \textbf{pullback via the modified right-pushforward $\mathcal{r}^!_{\sigma}\omega$ with a (local) section $\sigma \in \Gamma(\mathcal{G})$} as an element of $\Omega^k(\mathcal{P}; \pi^*\mathcal{g})$ by 
\bas
\mleft.\mleft(\mathcal{r}_\sigma^!\omega\mright)\mright|_p \mleft( Y_1, \dotsc, Y_k\mright)
&\coloneqq
\mleft.\mleft(\mathcal{r}_{\sigma_x}^!\mleft(\mleft.\omega\mright|_{\mathcal{P}_x}\mright)\mright)\mright|_p \mleft( Y_1, \dotsc, Y_k\mright)
=
\mleft.\mleft(r_\sigma^*\omega\mright)\mright|_{p} \bigl( \mathcal{r}_{\sigma*}(Y_1), \dotsc, \mathcal{r}_{\sigma*}(Y_k)\bigr)
\eas
for all $p \in \mathcal{P}_x$ and $Y_1, \dotsc, Y_k \in \mathrm{T}_p\mathcal{P}$.
\end{definitions}

\begin{remark}
\leavevmode\newline
By Prop.\ \ref{prop:IsomorphismRightPushAndDarboux} (and Remark \ref{RSigmaAnAuto}) these definitions are well-defined. A short note about the notation on the very right hand side of the second definition: This notation allows us to extend it to vector fields, that is,
\bas
\mleft(\mathcal{r}_\sigma^!\omega\mright) \mleft( Y_1, \dotsc, Y_k\mright)
=
\mleft(r_\sigma^*\omega\mright)\bigl( \mathcal{r}_{\sigma*}(Y_1), \dotsc, \mathcal{r}_{\sigma*}(Y_k)\bigr)
\eas
for all $p \in \mathcal{P}_x$ and $Y_1, \dotsc, Y_k \in \mathfrak{X}(\mathcal{P})$. Observe that we have $\mathcal{r}_{\sigma*}(Y_l) \in \Gamma\mleft(r_\sigma^*\mathrm{T}\mathcal{P}\mright)$ ($l \in \{1, \dotsc, k\}$) and $r_\sigma^*\omega \in \Gamma\mleft( \Lambda^k r_\sigma^*\mathrm{T}\mathcal{P} \otimes \pi^*\mathcal{g} \mright)$, using $r_\sigma^*\pi^*\mathcal{g} \cong (\pi \circ r_\sigma)^*\mathcal{g} = \pi^*\mathcal{g}$, such that the right hand side is well-defined.
\end{remark}

For the following definition recall the adjoint $\mathcal{G}$-representation on $\mathrm{V}\mathcal{P}$, Ex.\ \ref{ex:AdjointACtionOnVP}, and the isomorphism for $\mathrm{V}\mathcal{P}$ in Cor.\ \ref{cor:VerticalBundleOfPrincIsNearlyAsUsual}.

\begin{definitions}{Connection 1-forms on principal LGB-bundles}{GaugeBosonsOnLGBPrincies}
Let $\mathcal{G} \to M$ be an LGB over a smooth manifold $M$ and $\mathcal{P} \stackrel{\pi}{\to} M$ a principal $\mathcal{G}$-bundle, and we have a fixed horizontal distribution $\mathrm{H}\mathcal{G}$ on $\mathcal{G}$. A \textbf{connection 1-form} or \textbf{gauge field} on $\mathcal{P}$ is a 1-form $A \in \Omega^1(\mathcal{P}; \pi^*\mathcal{g})$ satisfying:
\begin{itemize}
	\item \textbf{($\mathcal{G}$-equivariance, but w.r.t.\ modified right-pushforward)}
		\bas 
			\mathcal{r}_\sigma^! A
			&=
			\sAd_{\sigma^{-1}} \circ A
		\eas
	for all (local) $\sigma \in \Gamma(\mathcal{G})$.
	\item \textbf{(Identity on $\mathrm{V}\mathcal{P}$)}
	\bas
	A\mleft(\widetilde{\nu}\mright)
	&=
	\pi^*\nu
	\eas
	for all (local) $\nu \in \Gamma(\mathcal{g})$.
\end{itemize}
\end{definitions}

\begin{remark}\label{PointwiseNotationOfConnectioNOneForms}
\leavevmode\newline
\indent $\bullet$ Due to the fact that we formulated the $\mathcal{G}$-equivariance using Ex.\ \ref{ex:AdjointACtionOnVP}, one may already be able to show an analogue of the 1:1 correspondence of typical (classical) connection 1-forms to connection reforms and splittings of the Atiyah sequence; that is, a generalization of \cite[\S 3.2, page 90 ff.]{mackenzieGeneralTheory}.

$\bullet$ The $\mathcal{G}$-equivariance reads point-wise for $g \coloneqq \sigma_x$ ($x\in M$)
\bas
\mleft.\mleft(\mathcal{r}_g^! \mleft(A|_{\mathcal{P}_x}\mright)\mright)\mright|_{p}(X)
&=
\mleft(
	p \cdot g, 
	\mathrm{Ad}_{g^{-1}}\mleft( \hat{A}_p(X) \mright)
\mright)
\eas
for all $p \in \mathcal{P}_x$ and $X \in \mathrm{T}_p\mathcal{P}$,
where we wrote $A_p = \mleft(p, \hat{A}_p\mright)$ with $\hat{A}_p \in \Gamma\mleft( \mathrm{T}^*_p\mathcal{P} \otimes \mathcal{g}_x \mright)$, and $\mathrm{Ad}$ is the adjoint representation of $\mathcal{G}$. In the typical formulation of gauge theory the base point component is usually omitted due to that $\mathcal{g}$ is then trivial and so also $\mathrm{V}\mathcal{P} (\cong \pi^*\mathcal{g})$.

The identity on $\mathrm{V}\mathcal{P}$ reads pointwise
\bas
A_p(\widetilde{v_p})
&=
(p, v_p)
\eas
for all $v_p \in \mathcal{g}_x$. For readability we may also omit the basepoint information and just write
\bas
A_p(\widetilde{v_p})
&\equiv
v_p.
\eas
Furthermore, recall the notation introduced in Remark \ref{rem:FundVecNotationOnPullbackBundle}, then for $\mu \coloneqq (p, v_p)$ we can therefore write
\bas
A_p\mleft( \widetilde{\mu}_p \mright)
&=
\mu.
\eas
\end{remark}

Finally, we identify (Ehresmann) connections and connection 1-forms on principal LGB-bundles in the typical way; for this recall Cor.\ \ref{cor:VerticalBundleOfPrincIsNearlyAsUsual}.

\begin{theorems}{1:1 correspondence of Ehresmann connections and connection 1-forms}{OurConnectionHasAUniqueoneForm}
Let $\mathcal{G} \to M$ be an LGB over a smooth manifold $M$ and $\mathcal{P} \stackrel{\pi}{\to} M$ a principal $\mathcal{G}$-bundle, and let $\mathrm{H}\mathcal{G}$ be a horizontal distribution on $\mathcal{G}$. Then there is a 1:1 correspondence between Ehresmann connections and connection 1-forms on $\mathcal{P}$:
\begin{itemize}
	\item Let $\mathrm{H}\mathcal{P}$ be an Ehresmann connection on $\mathcal{P}$. Then $\mathrm{H}\mathcal{P}$ defines a connection 1-form $A \in \Omega^1(\mathcal{P}; \pi^*\mathcal{g})$ by
	\bas
	A_p\bigl( \widetilde{v}_p + X_p \bigr)
	&=
	(p, v)
	\eas
	for all $p \in \mathcal{P}_x$ ($x \in M$), $v \in \mathcal{g}_x$ and $X \in \mathrm{H}_p\mathcal{P}$.
	\item Let $A \in \Omega^1(\mathcal{P}; \pi^*\mathcal{g})$ be a connection 1-form on $\mathcal{P}$. Then $A$ defines an Ehresmann connection $\mathrm{H}\mathcal{P}$ on $\mathcal{P}$ via its kernel $\mathrm{Ker}(A)$, that is,
	\bas
	\mathrm{H}_p\mathcal{P}
	&=
	\mathrm{Ker}(A_p)
	\eas
	for all $p \in \mathcal{P}$.
\end{itemize}
\end{theorems}

\begin{proof}
\leavevmode\newline
The proof is very similar to the proof of "typical connections", as \textit{e.g.}\ provided in \cite[\S 5.2, Thm.\ 5.2.2, page 262]{Hamilton}.

$\bullet$ For the first bullet point we need to show that Def.\ \ref{def:GaugeBosonsOnLGBPrincies} is satisfied, and the identity behaviour on $\mathrm{V}\mathcal{P}$ quickly follows by definition: We have
\bas
\mleft.A\mleft( \widetilde{\nu} \mright)\mright|_{p}
&=
A_p\mleft( \widetilde{\nu}_p \mright)
=
(p, \nu_x)
=
\mleft.\mleft(\pi^*\nu\mright)\mright|_p
\eas
for all $\nu \in \Gamma(\mathcal{g})$ and $p \in \mathcal{P}_x$ ($x \in M$). Therefore it is only left to show the $\mathcal{G}$-equivariance. As mentioned in Remark \ref{rem:FundVecsNotations}, for $v\in \mathcal{g}_x$ the vector field $\widetilde{v}$ on $\mathcal{P}_x$ is a fundamental vector field coming from the $\mathcal{G}_x$-action on $\mathcal{P}_x$. Hence, we know by \cite[\S 3.4, Prop.\ 3.4.6, page 145f.]{Hamilton}\ that
\bas
\mathcal{r}_{g*}\mleft( \widetilde{v} \mright)
&=
r_{g*}\mleft( \widetilde{v} \mright)
=
\oversortoftilde{\mathrm{Ad}_{g^{-1}}(v)}
\eas
for all $g \in \mathcal{G}_x$, also using Remark \ref{rem:ModRightPushOnVertic}.\footnote{This is very straightforward to prove.} Thus,
\bas
\mleft.\mleft(\mathcal{r}^!_\sigma A\mright)\mright|_p\bigl( \widetilde{v}_p + X_p \bigr)
&=
A_{p \cdot \sigma_x}\bigl( 
	\underbrace{\mathcal{r}_{\sigma_x*}\mleft(\widetilde{v}_p\mright)}
		_{= \mleft. \mathcal{r}_{\sigma_x*}\mleft( \widetilde{v} \mright) \mright|_{p \cdot  \sigma_x}}
	+ \underbrace{\mathcal{r}_{\sigma_x*}(X_p) }_{\in \mathrm{H}_{p \cdot \sigma_x}\mathcal{P}}
\bigr)
\\
&=
\mleft( p \cdot \sigma_x, \mathrm{Ad}_{\sigma_x^{-1}}(v) \mright)
\\
&=
\sAd_{\sigma_x^{-1}}(p, v)
\\
&=
\sAd_{\sigma_x^{-1}}\mleft(A_p\bigl( \widetilde{v}_p + X_p \bigr)\mright)
\\
&=
\mleft.\mleft(\sAd_{\sigma^{-1}} \circ A\mright)\mright|_p \bigl( \widetilde{v}_p + X_p \bigr)
\eas
for all $p \in \mathcal{P}_x$, $\sigma \in \Gamma(\mathcal{G})$, $v \in \mathcal{g}_x$ and $X_p \in \mathrm{H}_p\mathcal{P}$. This finishes the proof for the first bullet point.

$\bullet$ For the second bullet point we make use of Cor.\ \ref{cor:VerticalBundleOfPrincIsNearlyAsUsual} and
\bas
A\mleft( \widetilde{\nu} \mright)
&=
\pi^*\nu
\eas
for all (local) sections $\nu \in \Gamma(\mathcal{g})$. This implies that $A$ has not only values in $\mathrm{V}\mathcal{P}$, but acts also as identity on $\mathrm{V}\mathcal{P}$. Thus, $\mathrm{H}\mathcal{P} = \mathrm{Ker}(A_p)$ is a complementary subspace of $\mathrm{V}_p\mathcal{P}$ in $\mathrm{T}_p\mathcal{P}$ for all $p \in \mathcal{P}$. In order to show that $\mathrm{H}\mathcal{P}$ is a horizontal distribution of $\mathcal{P}$, we fix a local frame $\mleft( e_a \mright)_a$ of $\mathcal{g}$ over $U$ (an open subset of $M$), so that $\mleft( \pi^*e_a \mright)_a$ is a frame of $\pi^*\mathcal{g}|_{\pi^{-1}(U)}$. Therefore we write
\bas
A
&=
A^a \otimes \pi^*e_a,
\eas
where $A^a \in \Omega^1\mleft( \mathcal{P}|_U \mright)$. Due to $A\mleft( \widetilde{\nu} \mright) = \pi^*\nu$ we get
\bas
A^a\mleft( \widetilde{e_b} \mright)
&=
\delta^a_b,
\eas
where $\delta^a_b$ is the Kronecker delta. Hence, $\mleft( A^a \mright)_a$ is the dual frame of $\mleft( \widetilde{e_a} \mright)_a$, which is a frame of $\mathrm{V}\mathcal{P}$ by Cor.\ \ref{cor:VerticalBundleOfPrincIsNearlyAsUsual}, and so the $A^a$ are linear independent to each other. Fix an auxiliary fibre metric $\langle \cdot, \cdot \rangle$ on $\mathrm{T}\mathcal{P}$, and denote with $\mleft(V^a\mright)_a$ the $\langle \cdot, \cdot \rangle$-dual frame to $\mleft(A^a\mright)_a$, \textit{i.e.}\
\bas
A^a 
&=
\langle V^a, \cdot \rangle.
\eas
Due to the smoothness of $A$ and due to that $\langle \cdot, \cdot \rangle$ is a fibre metric, $\mleft(V^a\mright)_a$ are smooth (local) vector fields on $\mathcal{P}$, linear independent to each other. Observe that any orthogonal frame $\mleft(W^\alpha\mright)_\alpha$ to $\mleft(V^a\mright)_a$ satisfies
\bas
A^a(W^\alpha)
&=
\langle V^a, W^\alpha \rangle
=
0.
\eas
Thus, we derived that $\mathrm{Ker}(A)|_U$ is spanned by such frames $\mleft(W^\alpha\mright)_\alpha$, in total, $\mathrm{Ker}(A)$ is spanned by a locally free sheaf of modules of constant rank. By the 1:1 correspondence of vector bundle and locally free sheaf of modules of constant rank, we can conclude that $\mathrm{H}\mathcal{P} = \mathrm{Ker}(A)$ is a subbundle of $\mathrm{T}\mathcal{P}$; complementary to $\mathrm{V}\mathcal{P}$ due to what we have shown earlier.

It is only left to show the right-invariance of $\mathrm{H}\mathcal{P}$. So let $X \in \mathrm{H}_p\mathcal{P}$ for $p \in \mathcal{P}$. Then
\bas
A_{p \cdot \sigma_x}\bigl( \mathcal{r}_{\sigma*}(X) \bigr)
&=
\mleft(\mathcal{r}^!_\sigma(A)\mright)_p(X)
=
\sAd_{\sigma_x^{-1}}\bigl( A_p(X) \bigr)
=
0
\eas
for all (local) $\sigma \in \Gamma(\mathcal{G})$, where $x \coloneqq \pi(p)$. Thence, $\mathcal{r}_{\sigma*}(X) = \mathcal{r}_{\sigma_x*}(X) \in \mathrm{H}_{p \cdot \sigma_x}\mathcal{P}$, and therefore we derive by Prop.\ \ref{prop:IsomorphismRightPushAndDarboux} and dimensional reasons that $\mathcal{r}_{\sigma_x*}\mleft(\mathrm{H}_p \mathcal{P}\mright) = \mathrm{H}_{p \cdot \sigma_x}\mathcal{P}$. We eventually conclude that $\mathrm{H}\mathcal{P}$ is an Ehresmann connection.
\end{proof}

We finish this subsection with the following technical and useful corollary.

\begin{corollaries}{Commutation of modified push-forward and projections}{ModifiedRightPushyCommutesWithProj}
Let $\mathcal{G} \to M$ be an LGB over a smooth manifold $M$ and $\mathcal{P} \to M$ a principal $\mathcal{G}$-bundle, also let $\mathrm{H}\mathcal{G}$ be a horizontal distribution on $\mathcal{G}$ and $\mathrm{H}\mathcal{P}$ an Ehresmann connection on $\mathcal{P}$; denote with $\pi_h$ and $\pi_v$ the associated projections on the horizontal and vertical bundle of $\mathcal{P}$, respectively. Then we have
\bas
\mathcal{r}_{g*} \circ \pi_h
&=
\pi_h \circ \mathcal{r}_{g*},\\
\mathcal{r}_{g*} \circ \pi_v
&=
\pi_v \circ \mathcal{r}_{g*}
\eas
for all $g \in \mathcal{G}$.
\end{corollaries}

\begin{remark}\label{RemOohThesePullbacksConfusOrNotToConfus}
\leavevmode\newline
This extends of course to (local) sections $\sigma \in \Gamma(\mathcal{G})$, and as discussed in Remark \ref{RemarkABoutDarbouxNotationWRTPullback} we can extend these equations to vector fields on $\mathcal{P}$ via pullbacks, that is,
\bas
\mathcal{r}_{\sigma*} \circ \pi_h
&=
r_\sigma^*\pi_h \circ \mathcal{r}_{\sigma*},\\
\mathcal{r}_{\sigma*} \circ \pi_v
&=
r_\sigma^*\pi_v \circ \mathcal{r}_{\sigma*},
\eas
making use of that $\mathcal{r}_{\sigma*}$ is an isomorphism over $r_{\sigma*}$, recall Prop.\ \ref{prop:IsomorphismRightPushAndDarboux}.
\end{remark}

\begin{proof}
\leavevmode\newline
By Cor.\ \ref{cor:VerticalBundleOfPrincIsNearlyAsUsual} (also recall Remark \ref{rem:ModRightPushOnVertic}) and Def.\ \ref{def:FinallyTheConnection} we have
\bas
\mathcal{r}_{g*}\mleft(\mathrm{V}_p\mathcal{P}\mright)
&=
\mathrm{V}_{p\cdot g}\mathcal{P},\\
\mathcal{r}_{g*}\mleft( \mathrm{H}_p\mathcal{P} \mright)
&=
\mathrm{H}_{p\cdot g}\mathcal{P},
\eas
thus, $\mathcal{r}_{g*}$ preserves the splitting $\mathrm{T}\mathcal{P} = \mathrm{H}\mathcal{P} \oplus \mathrm{V}\mathcal{P}$. This concludes the proof.
\end{proof}

\subsection{Gauge transformations}\label{GaugeTrafoForA}

Let us now look at how gauge transformations of $A$ look like in this setting; for this recall the definition of gauge transformations in Def.\ \ref{def:MorphOfPrincBundles}. As in Remark \ref{rem:ClassGaugeTrafosAndcgPMulti} we expect a relationship between gauge transformations and certain LGB valued maps. For the following also recall Ex.\ \ref{ex:ConjugationActionForTheGeneralInnerGroupBundle}.

\begin{definitions}{LGB-valued conjugation maps}{GaugeTrafosLGBMaps}
Let $\mathcal{G} \to M$ be an LGB over a smooth manifold $M$, and $\mathcal{P} \stackrel{\pi}{\to} M$ a principal $\mathcal{G}$-bundle. Then we define the group \textbf{$C^\infty(\mathcal{P}; \mathcal{G})^{\mathcal{G}}$ of $\mathcal{G}$-valued conjugation maps} as a set by
\bas
C^\infty(\mathcal{P}; \mathcal{G})^{\mathcal{G}}
&=
\bigl\{
	\sigma \in \Gamma(\pi^*\mathcal{G})
	~\big|~
	\sigma_{p \cdot g}
	=
	\mathcal{c}_{g^{-1}}( \sigma_p )
	\text{ for all } (p, g) \in \mathcal{P}*\mathcal{G}
\bigr\}.
\eas
Its group structure is inherited by the point-wise group structure of $\Gamma(\pi^*\mathcal{G})$; recall Cor.\ \ref{cor:PullbackLGB}.
\end{definitions}

\begin{remarks}{Group structure on $C^\infty(\mathcal{P}; \mathcal{G})^{\mathcal{G}}$}{GroupStructureOnCInftPGG}
It is trivial to check that $C^\infty(\mathcal{P}; \mathcal{G})^{\mathcal{G}}$ is indeed a subgroup of $\Gamma(\pi^*\mathcal{G})$.
\end{remarks}

$C^\infty(\mathcal{P}; \mathcal{G})^{\mathcal{G}}$ canonically acts on $\mathcal{P}$ on the right via the given $\mathcal{G}$-action: Let $\sigma \in C^\infty(\mathcal{P}; \mathcal{G})^{\mathcal{G}}$, then we define
\ba\label{MultiWithPulli}
p \cdot \sigma_p
&\coloneqq
p \cdot \mathrm{pr}_2\mleft( \sigma_p \mright)
\ea
for all $p \in \mathcal{P}$, where $\mathrm{pr}_2: \pi^*\mathcal{G} \to \mathcal{G}$ is the projection onto the second component. In essence, we drop the notation of $\mathrm{pr}_2$, as if we view $\sigma$ as a map $\mathcal{P} \to \mathcal{G}$ of fibre bundles over $\pi$; recall Subsection \ref{BasicNotations}. If we denote the right $\mathcal{G}$-action on $\mathcal{P}$ by $\Phi$, then observe by $\sigma_p = (p, \mathrm{pr}_2(\sigma_p))$ that we could also write
\bas
p \cdot \sigma_p
&=
\Phi \bigl( p, \mathrm{pr}_2\mleft( \sigma_p \mright) \bigr)
=
\Phi\mleft( \sigma_p \mright).
\eas

As in the case of typical/classical principal bundles, we have the following statement.

\begin{propositions}{Gauge transformations as $\mathcal{G}$-valued conjugation maps}{GaugeTrafoAsBundleIsomIsASectionOfConjugationMaps}
Let $\mathcal{G} \to M$ be an LGB over a smooth manifold $M$, and $\mathcal{P} \stackrel{\pi}{\to} M$ a principal $\mathcal{G}$-bundle. Then there is a well-defined group isomorphism of gauge transformations and $\mathcal{G}$-valued conjugation maps given by
\bas
\sAut(\mathcal{P}) &\to C^\infty(\mathcal{P}; \mathcal{G})^{\mathcal{G}},\\
H &\mapsto \sigma^H,
\eas
where $\sigma^H \in C^\infty(\mathcal{P}; \mathcal{G})^{\mathcal{G}}$ is defined by
\bas
H(p) &= p \cdot \sigma^H_p
\eas
for all $p \in \mathcal{P}$.
\end{propositions}

\begin{proof}
\leavevmode\newline
This result can be proven similarly as for Lie group based principal bundles; however smoothness needs to be discussed a bit. In the following we make use of that pullback manifolds like $\mathcal{P}*\mathcal{G}$ are embedded submanifolds of product manifolds like $\mathcal{P}\times\mathcal{G}$, we will not further mention it.

$\bullet$ First of all, due to the fact that $H$ is base-preserving we know that $H(p)$ is in the same fibre as $p$ ($p \in \mathcal{P}$), and due to that the $\mathcal{G}$-action on $\mathcal{P}$ is simply transitive there is a unique element of $\pi^*\mathcal{G}|_p$, denoted by $\sigma^H_p$, such that
\bas
H(p)
&=
p \cdot \sigma_p^H
=
p \cdot \mathrm{pr}_2\mleft(\sigma_p^H\mright),
\eas
where $\mathrm{pr}_2: \mathcal{P}*\mathcal{G} = \pi^*\mathcal{G} \to \mathcal{G}$ is the smooth projection onto the second component.
Let us first show smoothness of $p \mapsto \sigma_p^H$: We can write
\bas
H(p) &= \Phi_p \Bigl( \mathrm{pr}_2\mleft(\sigma_p^H\mright) \Bigr),
\eas
where $\Phi_p: \mathcal{G}_x \to \mathcal{P}_x$ is the orbit map through $p \in \mathcal{P}_x$ ($x \in M$). By Remark \ref{rem:LGBPrincDefDiscussion}, $\Phi_p$ is a $\mathcal{G}_x$-equivariant diffeomorphism, especially invertible, so that
\bas
\mathrm{pr}_2\mleft(\sigma_p^H\mright) = \mleft( \Phi_p \mright)^{-1}\bigl( H(p) \bigr).
\eas
Let us define the map
\bas
\pi^*\mathcal{P} &\to \mathcal{G},\\
\mleft( p, p^\prime \mright) &\mapsto \Phi^{-1}\mleft( p, p^\prime \mright) \coloneqq \mleft( \Phi_p \mright)^{-1}\mleft( p^\prime \mright)
\eas
which is a map over $\pi$ by construction. If we can show smoothness of $\Phi^{-1}$, then smoothness of $\sigma^H$ follows additionally due to smoothness of $H$ and
\bas
\sigma_p^H
&=
\mleft( p, \mathrm{pr}_2\mleft(\sigma_p^H\mright) \mright)
=
\mleft( p, \mleft( \Phi_p \mright)^{-1}\bigl( H(p) \bigr) \mright)
=
\Bigl( p, \Phi^{-1}\bigl(p, H(p) \bigr) \Bigr).
\eas
Observe that we have
\bas
\mathrm{pr}_2(p, g)
=
g
&=
\mleft( \Phi_p \mright)^{-1}\bigl( \Phi_p(g) \bigr)
=
\Phi^{-1}\bigl( p, \Phi(p, g) \bigr)
\eas
for all $(p, g) \in \mathcal{P}*\mathcal{G}$, where $\Phi$ denotes the right $\mathcal{G}$-action on $\mathcal{P}$. The map
\bas
L: \mathcal{P}*\mathcal{G} &\to \pi^*\mathcal{P},\\
(p, g) &\mapsto \bigl( p, \Phi(p, g) \bigr),
\eas
is a base-preserving principal bundle isomorphism (w.r.t.\ the LGB isomorphism given as the identity map on $\mathcal{P} * \mathcal{G}$), recall Cor.\ \ref{cor:ProductSpaceIsPItself} and Remark \ref{AlternativePrincBdlDef}. Thus, we have in total
\bas
\Phi^{-1} \circ L
&=
\mathrm{pr}_2,
\eas
and thus
\bas
\Phi^{-1}
&=
\mathrm{pr}_2 \circ L^{-1},
\eas
so $\Phi^{-1}$ is smooth, and therefore $\sigma^H$ is smooth, too.

$\bullet$ That $\sigma^H$ is an element of $C^\infty(\mathcal{P};\mathcal{G})^{\mathcal{G}}$ follows as usual:
\bas
(p \cdot g) \cdot \mathrm{pr}_2\mleft(\sigma^H_{p \cdot g}\mright)
&=
H(p \cdot g)
=
H(p) \cdot g
=
p \cdot \mleft(\mathrm{pr}_2\mleft(\sigma^H_p\mright) ~ g \mright)
\eas
for all $(p, g) \in \mathcal{P}*\mathcal{G}$,
so that, by the simply transitivity of the action,
\bas
g ~ \mathrm{pr}_2\mleft(\sigma^H_{p \cdot g}\mright)
&=
\mathrm{pr}_2\mleft(\sigma^H_p\mright) ~ g,
\eas
and thus
\bas
\sigma^H_{p \cdot g}
&=
\Bigl( p \cdot g, \mathrm{pr}_2\mleft(\sigma^H_{p \cdot g}\mright) \Bigr)
=
\Bigl( p \cdot g, g^{-1} ~ \mathrm{pr}_2\mleft(\sigma^H_p\mright) ~ g \Bigr)
=
\mathcal{c}_{g^{-1}}\mleft( \sigma^H_p \mright).
\eas
Thus, $\sigma^H \in C^\infty(\mathcal{P}; \mathcal{G})^{\mathcal{G}}$.

$\bullet$ The inverse of $H \mapsto \sigma^H$ is clearly given by
\bas
C^\infty(\mathcal{P}; \mathcal{G})^{\mathcal{G}} &\to \sAut(\mathcal{P}),\\
\sigma &\mapsto H^\sigma,
\eas
where 
\bas
H^\sigma(p)
&\coloneqq
p \cdot \sigma_p
\eas
for all $p \in \mathcal{P}$. Smoothness here is now obvious, it is also clearly base-preserving, and we have 
\bas
H^\sigma(p \cdot g)
&=
p \cdot g ~ \underbrace{\sigma_{p \cdot g}}_{\mathclap{ = \mathcal{c}_{g^{-1}}\mleft(\sigma_p\mright) }}
=
p \cdot g ~ \underbrace{\mathrm{pr}_2\mleft( \mathcal{c}_{g^{-1}}\mleft(\sigma_p\mright) \mright)}_{\mathclap{ = c_{g^{-1}}\mleft( \mathrm{pr}_2(\sigma_p) \mright) }}
=
p \cdot \mathrm{pr}_2\mleft( \sigma_p \mright) ~ g
=
p \cdot \sigma_p ~ g
=
H^\sigma(p) \cdot g
\eas
for all $(p, g) \in \mathcal{P} * \mathcal{G}$. Thus, $H^\sigma \in \sAut(\mathcal{P})$, and so this finishes the proof.
%
%As we already argued several times, $\mathcal{P}* \mathcal{G} = \pi^*\mathcal{G}$ is an embedded submanifold of $\mathcal{P} \times \mathcal{G}$, and thus $\Phi_p$ is smooth in $p$, if it is point-wise in $\mathcal{G}$ smooth. That is, its smoothness comes from the smoothness of the right $\mathcal{G}$-action $\Phi$ on $\mathcal{P}$, and the derivative of $\Phi_p$ in $p$ along a tangent vector $X \in \mathrm{T}_p\mathcal{P}$ is determined by the following, using Thm.\ \ref{thm:DiffOfLGBAction},
%\bas
%\mathrm{D}_{(p, g)}\Phi\bigl(X, \mathrm{D}_x\sigma(\omega)\bigr)
%&=
%\mathrm{D}_pr_\sigma(X)
%\eas
%for all $(p, g) \in \mathcal{P} * \mathcal{G}$, where $x \coloneqq \pi(p)$, $\omega \coloneqq \mathrm{D}\pi(X)$, and $\sigma$ is any (local) section of $\mathcal{G}$ with $\sigma_x = g$; the tangent vector $Y$ as denoted in Thm.\ \ref{thm:DiffOfLGBAction} can be chosen in an arbitrary way, as long as $(X, Y) \in \mathrm{T}(\mathcal{P}* \mathcal{G})$. Thus, smoothness of $\Phi_p$ in $p$ can also 
\end{proof}

Using this, we can finally formulate the gauge transformations of connection 1-forms; for this recall Cor.\ \ref{cor:PullBacksArePrincToo} and the Darboux derivative, Def.\ \ref{def:DarbouxDerivativeOnLGBs}. Also recall the pullback Darboux derivative, Remark \ref{rem:PullBackDarboux}.

\begin{theorems}{Gauge transformations of connection 1-forms}{GaugeTrafoOfGaugeBoson}
Let $\mathcal{G} \to M$ be an LGB over a smooth manifold $M$ and $\mathcal{P} \stackrel{\pi}{\to} M$ a principal $\mathcal{G}$-bundle, also let $\mathrm{H}\mathcal{G}$ be a horizontal distribution on $\mathcal{G}$ and $A \in \Omega^1(\mathcal{P}; \pi^*\mathcal{g})$ be a connection 1-form on $\mathcal{P}$. Furthermore, let $H \in \sAut(\mathcal{P})$. We then have that $H^!A$ is a connection 1-form on $\mathcal{P}$ and
\bas
H^!A
&=
{\sAd_{\mathrm{pr}_2\circ\mleft(\sigma^H\mright)^{-1}}} \circ A 
	+ \mleft(\pi^*\Delta\mright)\sigma^H,
\eas
where $\sigma^H \in C^\infty(\mathcal{P}; \mathcal{G})^{\mathcal{G}}$ is defined as in Prop.\ \ref{prop:GaugeTrafoAsBundleIsomIsASectionOfConjugationMaps} and $\mathrm{pr}_2: \pi^*\mathcal{G} \to \mathcal{G}$ is the projection onto the second component.

Similar to Def.\ \eqref{MultiWithPulli} we may shortly just write 
\bas
H^!A
&=
{\sAd_{\mleft(\sigma^H\mright)^{-1}}} \circ A
	+ \mleft(\pi^*\Delta\mright)\sigma^H.
\eas
\end{theorems}

\begin{proof}
\leavevmode\newline
First of all observe that $H^!A \in \Omega^1(\mathcal{P}; H^*\pi^*\mathcal{g})$, and trivially $H^*\pi^*\mathcal{g} \cong (\pi\circ H)^*\mathcal{g} = \pi^*\mathcal{g}$. We will make use of this and similar isomorphisms in the following without further mentioning it. We will often not bookkeep the basepoint component in the following pairs and triples; so, everything has to be read in such a way that we have again values in the correct space. It would just be cumbersome to keep track of all these natural isomorphisms, and we decided for readability to avoid this bookkeeping most of the time.

$\bullet$ We then have
\bas
\mleft(\mathcal{r}_\sigma^!H^!A\mright)_p(X)
&=
\mleft( H^!A \mright)_{p \cdot \sigma_x}\bigl(
	\mathcal{r}_{\sigma*}(X)
\bigr)
\\
&=
\mleft( H^!A \mright)_{p \cdot \sigma_x}\mleft(
	\mathrm{D}_pr_\sigma\mleft( 
	X 
\mright)
	- \mleft.{\oversortoftilde{
		\mleft. \mleft( \pi^!\Delta\sigma \mright) \mright|_p(X)
	}}\mright|_{p \cdot \sigma_{x}}
\mright)
\\
&=
A_{H\mleft(p \cdot \sigma_x\mright)}\mleft( \mathrm{D}_{p \cdot \sigma_x}H\mleft(
	\mathrm{D}_pr_\sigma\mleft( 
	X 
\mright)
	- \mleft.{\oversortoftilde{
		\mleft. \mleft( \pi^!\Delta\sigma \mright) \mright|_p(X)
	}}\mright|_{p \cdot \sigma_{x}}
\mright)\mright)
\eas
for all (local) $\sigma \in \Gamma(\mathcal{G})$, $p \in \mathcal{P}_x$ ($x \in M$) and $X \in \mathrm{T}_p\mathcal{P}$. We also get by definition of $\sAut(\mathcal{P})$
\bas
H(p \cdot \sigma_x)
&=
H(p) \cdot \sigma_x,
\eas
that is,
\bas
H\circ r_\sigma
&=
r_\sigma \circ H
\eas
and thus
\bas
\mathrm{D}_{p\cdot \sigma_x}H \circ \mathrm{D}_p r_\sigma
&=
\mathrm{D}_p \mleft( H \circ r_\sigma \mright)
=
\mathrm{D}_p \mleft( r_\sigma \circ H \mright)
=
\mathrm{D}_{H(p)} r_\sigma \circ \mathrm{D}_pH.
\eas
Now also observe that
\ba\label{GaugeTrafoDiffOnFUndVect}
\mathrm{D}_p H \mleft( \widetilde{\nu}_p \mright)
&=
\mleft(\mathrm{D}_p H \circ \mathrm{D}_{e_x}\Phi_p\mright) (\nu)
=
\mathrm{D}_{e_x}\underbrace{\mleft(H \circ\Phi_p \mright)}_{\mathclap{ \mathcal{G}_x \ni g \mapsto H(p \cdot g) = H(p) \cdot g }} (\nu)
=
\mathrm{D}_{e_x} \Phi_{H(p)} (\nu)
=
\widetilde{\nu}_{H(p)}
\ea
for all $\nu \in \mathcal{g}_x$, where $\Phi_p$ is the orbit map through $p$. Due to that $H$ is base-preserving we derive
\bas
\mleft. \mleft( \pi^!\Delta\sigma \mright) \mright|_{H(p)}\bigl(\mathrm{D}_pH(X)\bigr)
&=
\mleft( H^! \pi^! \Delta \sigma \mright)_p (X)
=
\mleft( (\pi \circ H)^! \Delta \sigma \mright)_p (X)
=
\mleft( \pi^! \Delta \sigma \mright)_p (X).
\eas
In total we get
\bas
\mleft(\mathcal{r}_\sigma^!H^!A\mright)_p(X)
&=
A_{H(p) \cdot \sigma_x}\mleft( 
	\mathrm{D}_{H(p)}r_\sigma \bigl(\mathrm{D}_pH( X )\bigr)
	- \mleft.{\oversortoftilde{
		\mleft. \mleft( \pi^!\Delta\sigma \mright) \mright|_{H(p)}\bigl( \mathrm{D}_pH(X) \bigr)
	}}\mright|_{H(p) \cdot \sigma_{x}}
\mright)
\\
&=
A_{H(p) \cdot \sigma_x}\mleft(
	\mathcal{r}_{\sigma*}\bigl(\mathrm{D}_pH( X )\bigr)
\mright)
\\
&=
\mleft( \mathcal{r}_\sigma^!A \mright)_{H(p)}\bigl(\mathrm{D}_pH( X )\bigr)
\\
&=
\mleft( \sAd_{\sigma^{-1}} \circ A \mright)_{H(p)}\bigl(\mathrm{D}_pH( X )\bigr)
\\
&=
\mleft( H(p) \cdot \sigma_x, ~\mathrm{Ad}_{\sigma_x^{-1}}\mleft( \widehat{A}_{H(p)}\bigl(\mathrm{D}_pH( X )\bigr) \mright) \mright)
\\
&=
\sAd_{\sigma^{-1}}
\mleft( H(p), ~\widehat{A}_{H(p)}\bigl(\mathrm{D}_pH( X )\bigr) \mright)
\\
&=
\sAd_{\sigma^{-1}}
\mleft( \mleft(H^!A\mright)_p(X) \mright),
\eas
where we wrote $A = \mleft( p, \widehat{A}_p \mright)$ with $\widehat{A} \coloneqq \pi_2 \circ A$ ($\pi_2: \pi^*\mathcal{g} \to \mathcal{g}$ the projection onto the second component), and thus
\bas
\mathcal{r}_\sigma^!H^!A
&=
\sAd_{\sigma^{-1}}\circ~ H^!A.
\eas
By Eq.\ \eqref{GaugeTrafoDiffOnFUndVect} we also get $\mathrm{D}H(\widetilde{\nu}) = H^*\widetilde{\nu}$ for all (local) $\nu \in \Gamma(\mathcal{g})$ and thus
\bas
\mleft(H^!A\mright)\mleft( \widetilde{\nu} \mright)
&=
\mleft(H^*A\mright)\bigl( \mathrm{D}H\mleft( \widetilde{\nu} \mright) \bigr)
=
\mleft(H^*A\mright)\mleft( H^*\widetilde{\nu} \mright)
=
H^*\bigl( A\mleft( \widetilde{\nu}\mright) \bigr)
=
H^*\pi^*\nu
=
(\pi \circ H)^*\nu
=
\pi^*\nu.
\eas
Hence, $H^!A$ is a connection 1-form on $\mathcal{P}$.

$\bullet$ For the last part recall Prop.\ \ref{prop:GaugeTrafoAsBundleIsomIsASectionOfConjugationMaps}, especially we have a unique $\sigma^H \in C^\infty(\mathcal{P}; \mathcal{G})^{\mathcal{G}}$ such that
\bas
H(p)
&=
p \cdot \sigma^H_p
=
p \cdot \mathrm{pr}_2\mleft(\sigma^H_p\mright)
=
\Phi\mleft(p, \mathrm{pr}_2\mleft(\sigma^H_p \mright)\mright)
=
\Bigl( \Phi \circ \mleft( \mathds{1}_{\mathcal{P}}, \mathrm{pr}_2\circ\sigma^H \mright) \Bigr)(p)
\eas
for all $p \in \mathcal{P}_x$ ($x\in M$), where $\Phi$ is the right $\mathcal{G}$-action on $\mathcal{P}$, also recall Def.\ \ref{MultiWithPulli}, that is, $\mathrm{pr}_2: \pi^*\mathcal{G} \to \mathcal{G}$ is the projection onto the second component. Define $\widetilde{\sigma} \coloneqq \mathrm{pr}_2 \circ \sigma^H$, $\mathcal{P} \ni p \mapsto \widetilde{\sigma}_p \in \mathcal{G}$; by Remark \ref{rem:PullbackTotMCForm}, \ref{rem:PullBackDarboux} and Thm.\ \ref{thm:DiffOfLGBAction}, especially Remark \ref{rem:DarbouxInActionInfinit}, we can calculate
\ba\label{DiffOfPrincAutom}
\mathrm{D}_p H(X)
&=
\mathrm{D}_{\mleft(p, \widetilde{\sigma}_p\mright)}\Phi \bigl( X, \mathrm{D}_p\widetilde{\sigma}(X) \bigr)
\nonumber
\\
&=
\mathrm{D}_pr_\sigma(X)
	- \mleft.{\oversortoftilde{ \mleft.\mleft(\pi^!\Delta \sigma\mright)\mright|_p (X)}}\mright|_{p \cdot \widetilde{\sigma}_p}
	+ \mleft.{\oversortoftilde{\mleft( \mu_{\mathcal{G}}^{\mathrm{tot}}\mright)_{\widetilde{\sigma}_p} \bigl(\mathrm{D}_p\widetilde{\sigma}(X) \bigr)}}\mright|_{p \cdot \widetilde{\sigma}_p}
\nonumber
\\
&=
\mathcal{r}_{\widetilde{\sigma}_p*}(X)
	+ \mleft.{\oversortoftilde{\mleft( \mu_{\mathcal{G}}^{\mathrm{tot}}\mright)_{\mathrm{pr}_2\mleft(\sigma^H_p\mright)} \mleft(\mleft(\mathrm{D}_{\sigma^H_p}\mathrm{pr}_2 \circ \mathrm{D}_p\sigma^H\mright)(X) \mright)}}\mright|_{H(p)}
\nonumber
\\
&=
\mathcal{r}_{\widetilde{\sigma}_p*}(X)
	+ \mleft.{\oversortoftilde{\mleft( \mathrm{pr}_2^!\mu_{\mathcal{G}}^{\mathrm{tot}}\mright)_{\sigma^H_p} \mleft(\mathrm{D}_p\sigma^H(X) \mright)}}\mright|_{H(p)}
\nonumber
\\
&=
\mathcal{r}_{\widetilde{\sigma}_p*}(X)
	+ \mleft.{\oversortoftilde{\mleft( \mu_{\pi^*\mathcal{G}}^{\mathrm{tot}}\mright)_{\sigma^H_p} \mleft(\mathrm{D}_p\sigma^H(X) \mright)}}\mright|_{H(p)}
\nonumber
\\
&=
\mathcal{r}_{\widetilde{\sigma}_p*}(X)
	+ \mleft.{\oversortoftilde{ \mleft(\Delta^{\pi^*\mathcal{G}}\sigma^H\mright)_p(X) }}\mright|_{H(p)}
\nonumber
\\
&=
\mathcal{r}_{\widetilde{\sigma}_p*}(X)
	+ \mleft.{\oversortoftilde{ \mleft(\mleft(\pi^*\Delta\mright)\sigma^H\mright)_p(X) }}\mright|_{H(p)}
\ea
for all $X \in \mathrm{T}_p\mathcal{P}$, where $\sigma$ is a (local) section of $\mathcal{G}$ so that $\sigma_x = \widetilde{\sigma}_p = \mathrm{pr}_2\mleft( \sigma^H_p \mright)$.

Then
\bas
\mleft(H^!A\mright)_p(X)
&=
A_{H(p)}\bigl( \mathrm{D}_pH(X) \bigr)
\\
&=
A_{H(p)}\mleft( 
	\mathcal{r}_{\widetilde{\sigma}_p*}(X)
	+ \mleft.{\oversortoftilde{ \mleft(\mleft(\pi^*\Delta\mright)\sigma^H\mright)_p(X) }}\mright|_{H(p)}
\mright)
\\
&=
A_{p \cdot \widetilde{\sigma}_p}\mleft( 
	\mathcal{r}_{\widetilde{\sigma}_p*}(X)
\mright)
	+ \mleft(\mleft(\pi^*\Delta\mright)\sigma^H\mright)_p(X)
\\
&=
\mleft(\mathcal{r}_{\widetilde{\sigma}_p}^!A\mright)_p(X)
	+ \mleft(\mleft(\pi^*\Delta\mright)\sigma^H\mright)_p(X)
\\
&=
\sAd_{\widetilde{\sigma}_p^{-1}}\bigl(A_p(X)\bigr)
	+ \mleft(\mleft(\pi^*\Delta\mright)\sigma^H\mright)_p(X)
\\
&=
\sAd_{\mathrm{pr}_2\mleft( \mleft(\sigma_p^H\mright)^{-1} \mright)}\bigl(A_p(X)\bigr)
	+ \mleft(\mleft(\pi^*\Delta\mright)\sigma^H\mright)_p(X),
\eas
which finishes the proof, where we used the trivial relation (alternatively recall Cor.\ \ref{cor:PullbackLGB})
\bas
\mleft(\mathrm{pr}_2\mleft( \sigma_p^H \mright)\mright)^{-1}
&=
\mathrm{pr}_2\mleft( \mleft(\sigma_p^H\mright)^{-1} \mright).
\eas
%where we used the previously-mentioned isomorphism $H^*\pi^*\mathcal{g} \cong \pi^*\mathcal{g}$ to write
%\bas
%A_{H(p)}\mleft( 
	%\mleft.{\oversortoftilde{ \mleft(\mleft(\pi^*\Delta\mright)\sigma^H\mright)_p(X) }}\mright|_{H(p)}
%\mright)
%&=
%\mleft(  \mright)
%\eas
\end{proof}

Of course there is also the sense of gauge transformation with respect to gauges as in typical gauge theory (see \textit{e.g.}\ \cite[\S 5.4, page 270ff.]{Hamilton}), recall Def.\ \ref{def:GaugesOfPrincipalBundles}.

\begin{definitions}{Local gauge field}{LocalGaugeField}
Let $\mathcal{G} \to M$ be an LGB over a smooth manifold $M$ and $\mathcal{P} \stackrel{\pi}{\to} M$ a principal $\mathcal{G}$-bundle, also let $\mathrm{H}\mathcal{G}$ be a horizontal distribution on $\mathcal{G}$ and $A \in \Omega^1(\mathcal{P}; \pi^*\mathcal{g})$ be a connection 1-form on $\mathcal{P}$. Furthermore, let $s \in \Gamma(\mathcal{P}|_U)$ be a (local) gauge over an open subset $U \subset M$. Then we define the \textbf{local connection 1-form} of \textbf{local gauge field $A_s \in \Omega^1\mleft(U; \mleft.\mathcal{g}\mright|_U\mright)$ (w.r.t.\ $s$)} by
\bas
A_s
&\coloneqq
s^!A.
\eas
\end{definitions}

\begin{remark}\label{PullBackGaugeFieldRemark}
\leavevmode\newline
$A$ has values in $\pi^*\mathcal{g}$, and thus $A_s$ has values in $s^*\pi^*\mathcal{g} \cong (\pi \circ s)^*\mathcal{g} = \mathds{1}_U^*\mathcal{g} \cong \mleft.\mathcal{g}\mright|_U$.
\end{remark}

Gauge transformations now naturally arise in a change of the gauge $s$. So, let $U_i$ and $U_j$ be two open subsets of $M$ so that $U_i \cap U_j \neq \emptyset$. For two gauges $s_i \in \Gamma\mleft(\mathcal{P}|_{U_i}\mright)$ and $s_j \in \Gamma\mleft(\mathcal{P}|_{U_j}\mright)$ there is then a unique $\sigma_{ji} \in \Gamma\mleft( \mleft.\mathcal{G}\mright|_{U_i \cap U_j} \mright)$ such that
\bas
s_i
&=
s_j \cdot \sigma_{ji}
=
\Phi_{s_j} \circ \sigma_{ji}
\eas
on $U_i \cap U_j$,
where $\Phi_{s_j}$ is the orbit through $s_j$, especially also recall Lemma \ref{lem:SectionsNowInduceIsomToLGBsNotNecTriv}.
The unique existence is clear by the definition of principal bundles, while the smoothness follows by Lemma \ref{lem:SectionsNowInduceIsomToLGBsNotNecTriv} so that we can write $\sigma_{ji}$ as the composition of smooth maps
\bas
\sigma_{ji}
&=
\Phi^{-1}_{s_j} \circ s_i.
\eas
In the following we also introduce the notation
\bas
\mleft(\Phi^{-1}_{s_j}\mright)^{-1}(p)
&\coloneqq
\mleft(\Phi^{-1}_{s_j}(p)\mright)^{-1}
\eas
for all $p \in \mathcal{P}_{U_j}$, that is, $\Phi^{-1}_{s_j}$ is the inverse of the orbit map $\Phi_{s_j}$, but $(\cdot)^{-1}$ is the inverse of $\Phi^{-1}_{s_j}(p)$ as an element of $\mathcal{G}_{U_j}$.
%Similarly, there is a unique $\sigma_i \in \Gamma\mleft(\mathcal{G}_{U_i}\mright)$ such that
%\bas
%p
%&=
%\mleft.s_i\mright|_{\pi(p)} \cdot \mleft.\sigma_i\mright|_p
%\eas
%for all $p \in \mathcal{P}_{U_i}$, that is, $\sigma_i \coloneqq \Phi^{-1}_{s_i}$.

Instead of calculating directly how $A_{s_i}$ and $A_{s_j}$ are related we want to use Thm.\ \ref{thm:GaugeTrafoOfGaugeBoson}. For this it will be useful to understand that gauge transformations $\sAut(\mathcal{P})$ are locally isomorphic to the group of sections $\Gamma(\mathcal{G})$ via a gauge, analogously to the typical formulation of gauge theory as \textit{e.g.}\ illustrated in \cite[\S 5.3.2, page 268f.]{Hamilton}. While the isomorphism of $\sAut(\mathcal{P})$ to $C^\infty(\mathcal{P};\mathcal{G})^{\mathcal{G}} \subset \Gamma(\pi^*\mathcal{G})$ is global, the following isomorphism is in general only local.
%
%\begin{definitions}{Physical gauge transformation}{PhysicalGaugeTrafo}
%Let $\mathcal{G} \to M$ be an LGB over a smooth manifold $M$, and $\mathcal{P} \stackrel{\pi}{\to} M$ a principal $\mathcal{G}$-bundle. A \textbf{physical gauge transformation $\tau$} is an element of $\Gamma(\mathcal{G})$.
%\end{definitions}

\begin{propositions}{Gauge transformations and sections of the structural LGB}{GaugeTrafosAsLGBSectionsLocal}
Let $\mathcal{G} \stackrel{\pi_{\mathcal{G}}}{\to} M$ be an LGB over a smooth manifold $M$, and $\mathcal{P} \stackrel{\pi}{\to} M$ a principal $\mathcal{G}$-bundle. Also let $s \in \Gamma\mleft(\mleft.\mathcal{P}\mright|_U\mright)$ be a gauge defined over some open subset $U$ of $M$. Then we have a group isomorphism given by
\bas
C^\infty \mleft( \mleft.\mathcal{P}\mright|_U; \mleft.\mathcal{G}\mright|_U \mright)^{\mleft.\mathcal{G}\mright|_U} 
&\to 
\Gamma\mleft(\mleft.\mathcal{G}\mright|_U\mright),
\\
\sigma &\mapsto \mathrm{pr}_2 \circ \sigma \circ s,
\eas
where $\mathrm{pr}_2: \pi^*\mathcal{G} \to \mathcal{G}$ is the projection onto the second component,
with inverse
\bas
\Gamma\mleft(\mleft.\mathcal{G}\mright|_U\mright)
&\to
C^\infty \mleft( \mleft.\mathcal{P}\mright|_U; \mleft.\mathcal{G}\mright|_U \mright)^{\mleft.\mathcal{G}\mright|_U},
\\ 
\tau &\mapsto \sigma^\tau,
\eas
where 
\bas
\sigma^\tau_p
&\coloneqq
\mleft(p, c_{\mleft(\Phi^{-1}_{s}(p)\mright)^{-1}} \mleft( \tau_{\pi(p)} \mright) \mright)
=
\mleft(p, \mleft(\Phi^{-1}_{s}(p)\mright)^{-1} ~ \tau_{\pi(p)} ~ \Phi^{-1}_{s}(p) \mright)
\eas
for all $p \in \mathcal{P}_U$,
with $\Phi_s$ being the orbit map through $s$.
\end{propositions}

\begin{remarks}{Other notation}{OtherNotationForLGBSectionsAsGaugeTrafo}
Due to the fact that $s^*\pi^*\mathcal{G} \cong (\pi \circ s)^*\mathcal{G} = \mathds{1}_U^*\mathcal{G} \cong \mleft.\mathcal{G}\mright|_U$, and due to what we discussed in Subsection \ref{BasicNotations} about pullback sections, we could rewrite the first map to
\bas
C^\infty \mleft( \mleft.\mathcal{P}\mright|_U; \mleft.\mathcal{G}\mright|_U \mright)^{\mleft.\mathcal{G}\mright|_U} 
&\to 
\Gamma\mleft(\mleft.\mathcal{G}\mright|_U\mright),
\\
\sigma &\mapsto s^*\sigma.
\eas
\end{remarks}

\begin{proof}[Proof of Prop.\ \ref{prop:GaugeTrafosAsLGBSectionsLocal}]
\leavevmode\newline
For $\sigma \in C^\infty \mleft( \mleft.\mathcal{P}\mright|_U; \mleft.\mathcal{G}\mright|_U \mright)^{\mleft.\mathcal{G}\mright|_U} \subset \Gamma\mleft( \pi^*\mleft.\mathcal{G}\mright|_U \mright)$ observe that $\mathrm{pr}_2 \circ \sigma \circ s$ is by construction a section of $\mleft.\mathcal{G}\mright|_U$ due to 
\bas
\pi_{\mathcal{G}} \circ \mathrm{pr}_2 \circ \sigma \circ s
&=
\pi \circ \mathrm{pr}_1 \circ \sigma \circ s
=
\pi \circ s
=
\mathds{1}_U,
\eas
where $\mathrm{pr}_1: \pi^*\mathcal{G} \to \mathcal{P}$ is the projection onto the first component.
For the supposed inverse we have similarly that $\sigma^\tau$ is a section of $\Gamma(\pi^*\mleft.\mathcal{G}\mright|_U)$ by construction; then observe, using Lemma \ref{lem:SectionsNowInduceIsomToLGBsNotNecTriv},
\bas
\sigma_{p \cdot g}^\tau
&=
\mleft(p \cdot g, c_{\mleft(\Phi^{-1}_{s}(p \cdot g)\mright)^{-1}} \mleft( \tau_{\pi(p \cdot g)} \mright) \mright)
\\
&=
\mleft(p \cdot g, c_{g^{-1} ~ \mleft(\Phi^{-1}_{s}(p)\mright)^{-1}} \mleft( \tau_{\pi(p)} \mright) \mright)
\\
&=
\mleft(p \cdot g, \mleft(c_{g^{-1}} \circ c_{ \mleft(\Phi^{-1}_{s}(p)\mright)^{-1}} \mright) \mleft( \tau_{\pi(p)} \mright) \mright)
\\
&=
\mathcal{c}_{g^{-1}}
\mleft(p, c_{ \mleft(\Phi^{-1}_{s}(p)\mright)^{-1}} \mleft( \tau_{\pi(p)} \mright) \mright)
\\
&=
\mathcal{c}_{g^{-1}}
\mleft( \sigma^\tau_p \mright)
\eas
for all $(p, g) \in \mleft.\mathcal{P}\mright|_U * \mleft.\mathcal{G}\mright|_U$, thus, $\sigma^\tau \in C^\infty \mleft( \mleft.\mathcal{P}\mright|_U; \mleft.\mathcal{G}\mright|_U \mright)^{\mleft.\mathcal{G}\mright|_U}$. Recall
\bas
p &= s_{\pi(p)} \cdot \Phi^{-1}_s(p)
\eas 
for all $p \in \mleft.\mathcal{P}\mright|_U$,
so that
\bas
\Phi^{-1}_{s} \circ s
&=
e|_U,
\eas
where $e$ is the neutral section of $\mathcal{G}$. On one hand
\bas
\mleft(\mathrm{pr_2} \circ \sigma^\tau\mright)(s_x)
&=
\mathrm{pr}_2\mleft(\sigma^\tau_{s_x}\mright)
=
c_{\mleft(\Phi^{-1}_{s}(s_x)\mright)^{-1}} \mleft( \tau_{\pi(s_x)} \mright)
=
c_{e_x}\mleft( \tau_x \mright)
=
\tau_x
\eas
for all $x \in M$ and $\tau \in \Gamma(\mleft.\mathcal{G}\mright|_U)$, and by Def.\ \ref{def:GaugeTrafosLGBMaps} and Ex.\ \ref{ex:ConjugationActionForTheGeneralInnerGroupBundle} we have one the other hand
\bas
\sigma^{\mathrm{pr}_2 \circ \sigma \circ s}_p
&=
\Biggl(p, \mleft(c_{\mleft(\Phi^{-1}_{s}(p)\mright)^{-1}} \circ \mathrm{pr}_2 \mright) \mleft(\sigma_{s_{\pi(p)}}\mright) \Biggr)
\\
&=
\Biggl(p, \mleft(c_{\mleft(\Phi^{-1}_{s}(p)\mright)^{-1}} \circ \mathrm{pr}_2 \mright) \mleft(\sigma_{p \cdot \mleft( \Phi^{-1}_s(p) \mright)^{-1}}\mright) \Biggr)
\\
&=
\Biggl(p, \mleft(c_{\mleft(\Phi^{-1}_{s}(p)\mright)^{-1}} \circ \mathrm{pr}_2 \circ \mathcal{c}_{\Phi^{-1}_{s}(p)} \mright) \mleft(\sigma_{p}\mright) \Biggr)
\\
&=
\Biggl(p, \mleft(c_{\mleft(\Phi^{-1}_{s}(p)\mright)^{-1}} \circ c_{\Phi^{-1}_{s}(p)} \circ \mathrm{pr}_2 \mright) \mleft(\sigma_{p}\mright) \Biggr)
\\
&=
\bigl( p, \mathrm{pr}_2 (\sigma_p) \bigr)
\\
&=
\sigma_p
\eas
for all $p \in \mleft.\mathcal{P}\mright|_U$ and $\sigma \in C^\infty \mleft( \mleft.\mathcal{P}\mright|_U; \mleft.\mathcal{G}\mright|_U \mright)^{\mleft.\mathcal{G}\mright|_U}$. Thus, $\sigma \mapsto \mathrm{pr}_2 \circ \sigma \circ s$ is bijective with inverse $\tau \mapsto \sigma^\tau$, and so it is only left to show that $\sigma \mapsto \mathrm{pr}_2 \circ \sigma \circ s$ is a group homomorphism. For $\sigma^\prime \in C^\infty \mleft( \mleft.\mathcal{P}\mright|_U; \mleft.\mathcal{G}\mright|_U \mright)^{\mleft.\mathcal{G}\mright|_U}$ we have
\bas
\mleft.\sigma \sigma^\prime\mright|_p
&=
\mleft( p, \mathrm{pr}_2( \sigma_p) ~ \mathrm{pr}_2\mleft( \sigma^\prime_p \mright) \mright),
\eas
thus,
\bas
\mleft(\mathrm{pr}_2\circ \mleft( \sigma \sigma^\prime \mright) \circ s\mright)_x
&=
\mathrm{pr}_2\mleft( \sigma_{s_x}\mright) ~ \mathrm{pr}_2\mleft( \sigma^\prime_{s_x} \mright)
=
\mleft( \mleft( \mathrm{pr}_2 \circ \sigma \circ s \mright) \cdot \mleft( \mathrm{pr}_2 \circ \sigma^\prime \circ s \mright) \mright)_x
\eas
for all $x \in M$. This finishes the proof.
\end{proof}

\begin{propositions}{Change of gauge as a local bundle automorphism}{GaugeChangeAsBundleAutomorph}
Let $\mathcal{G} \to M$ be an LGB over a smooth manifold $M$, and $\mathcal{P} \stackrel{\pi}{\to} M$ a principal $\mathcal{G}$-bundle. Also let $U_i$ and $U_j$ be two open subsets of $M$ so that $U_i \cap U_j \neq \emptyset$, two gauges $s_i \in \Gamma\mleft(\mathcal{P}|_{U_i}\mright)$ and $s_j \in \Gamma\mleft(\mathcal{P}|_{U_j}\mright)$, and the unique $\sigma_{ji} \in \Gamma\mleft( \mleft.\mathcal{G}\mright|_{U_i \cap U_j} \mright)$ with $s_i = s_j \cdot \sigma_{ji}$ on $U_i \cap U_j$.

Then the map $H_{ji}$ defined by
\bas
\mleft.\mathcal{P}\mright|_{U_i \cap U_j} &\to \mleft.\mathcal{P}\mright|_{U_i \cap U_j},\\
p &\mapsto \mleft.s_i\mright|_{\pi(p)} \cdot \Phi_{s_j}^{-1}(p),
\eas
is an element of $\sAut\mleft( \mleft.\mathcal{P}\mright|_{U_i \cap U_j} \mright)$,
where $\Phi_{s_j}$ is the orbit map through $s_j$.

Using the notation of Prop.\ \ref{prop:GaugeTrafoAsBundleIsomIsASectionOfConjugationMaps}, we also get that $\sigma^{H_{ji}}$ is related to $\sigma_{ji}$ via $s_j$ in sense of Prop.\ \ref{prop:GaugeTrafosAsLGBSectionsLocal}, that is,
\bas
\sigma^{H_{ji}}_p
&=
\mleft(p, c_{\mleft(\Phi^{-1}_{s_j}(p)\mright)^{-1}} \mleft( \mleft.\sigma_{ji}\mright|_{\pi(p)} \mright) \mright)
\eas
for all $p \in \mleft.\mathcal{P}\mright|_{U_i \cap U_j}$, and
\bas
\sigma_{ji}
&=
\mathrm{pr}_2 \circ \sigma^{H_{ji}} \circ s_j,
\eas
where $\mathrm{pr}_2: \pi^*\mathcal{G} \to \mathcal{G}$ is the projection onto the second component.
\end{propositions}

\begin{proof}
\leavevmode\newline
Recall Lemma \ref{lem:SectionsNowInduceIsomToLGBsNotNecTriv} for the following proof.

We have $\mleft.s_i\mright|_{\pi(p)} \in \mathcal{P}_{\pi(p)}$ and $\Phi_{s_j}$ is base-preserving, and thus $\Phi_{s_j}^{-1}(p) \in \mathcal{G}_{\pi(p)}$, so that $H_{ji}$ is well-defined and also base-preserving because the right $\mathcal{G}$-action on $\mathcal{P}$ preserves the fibres.
$H_{ji}$ is clearly smooth by construction. Finally, 
\bas
H_{ji}(p \cdot g)
&=
\mleft.s_i\mright|_{\pi(p\cdot g)} \cdot \Phi_{s_j}^{-1}(p \cdot g)
=
\mleft.s_i\mright|_{\pi(p)} \cdot \Phi_{s_j}^{-1}(p) \cdot g
=
H_{ji}(p) \cdot g
\eas
for all $(p, g) \in \mleft.\mleft( \mathcal{P} * \mathcal{G} \mright)\mright|_{U_i \cap U_j}$, thus, $H_{ji}$ is a principal bundle morphism. The inverse $H^{-1}_{ji}$ is given by
\bas
H^{-1}_{ji}
&=
H_{ij}
\eas
due to
\bas
\mleft(H_{ij} \circ H_{ji}\mright)(p)
&=
\mleft.s_j\mright|_x \cdot \Phi^{-1}_{s_i}\mleft( \mleft.s_i\mright|_{x} \cdot \Phi_{s_j}^{-1}(p) \mright)
=
\mleft.s_j\mright|_x \cdot {\underbrace{\Phi^{-1}_{s_i}\mleft( \mleft.s_i\mright|_{x} \mright)}_{= e_x}} \cdot \Phi_{s_j}^{-1}(p)
=
\mleft.s_j\mright|_x \cdot \Phi_{s_j}^{-1}(p)
=
p
\eas
for all $p \in \mathcal{P}_{U_i \cap U_j}$, where $x \coloneqq \pi(p)$; by symmetry of course similarly when swapping $j$ and $i$. Thus, $H_{ji}$ is bijective and its inverse smooth, so that $H_{ji} \in \sAut\mleft( \mleft.\mathcal{P}\mright|_{U_i \cap U_j} \mright)$.

Now write
\bas
H_{ji}(p)
&=
\mleft.s_i\mright|_{x} \cdot \Phi_{s_j}^{-1}(p)
\\
&=
\mleft.s_j\mright|_{x} \cdot \mleft.\sigma_{ji}\mright|_x \cdot \Phi_{s_j}^{-1}(p)
\\
&=
\mleft.s_j\mright|_{x} \cdot \Phi_{s_j}^{-1}(p) \cdot \mleft( \Phi_{s_j}^{-1}(p) \mright)^{-1} \cdot \mleft.\sigma_{ji}\mright|_x \cdot \Phi_{s_j}^{-1}(p)
\\
&=
p \cdot c_{\mleft(\Phi^{-1}_{s_j}(p)\mright)^{-1}} \mleft( \mleft.\sigma_{ji}\mright|_{x} \mright),
\eas
and due to $H_{ji}(p) = p \cdot \mathrm{pr}_2\mleft(\sigma^{H_{ji}}_p\mright)$ we can immediately derive
\bas
\sigma^{H_{ji}}_p
&=
\mleft(p, c_{\mleft(\Phi^{-1}_{s_j}(p)\mright)^{-1}} \mleft( \mleft.\sigma_{ji}\mright|_{\pi(p)} \mright) \mright),
\eas
thus, $\sigma^{H_{ji}}$ is related to $\sigma_{ji}$ w.r.t.\ $s_j$ in sense of Prop.\ \ref{prop:GaugeTrafosAsLGBSectionsLocal} and so the remaining part of the proposition's statement follows by Prop.\ \ref{prop:GaugeTrafosAsLGBSectionsLocal}.
\end{proof}

Using this gauge transformation we can now relate $A_{s_i}$ and $A_{s_j}$.

\begin{theorems}{Gauge transformations as a change of gauge in the local gauge field}{LocalGaugeTrafoChangeGauge}
Let $\mathcal{G} \to M$ be an LGB over a smooth manifold $M$ and $\mathcal{P} \stackrel{\pi}{\to} M$ a principal $\mathcal{G}$-bundle, also let $\mathrm{H}\mathcal{G}$ be a horizontal distribution on $\mathcal{G}$ and $A \in \Omega^1(\mathcal{P}; \pi^*\mathcal{g})$ be a connection 1-form on $\mathcal{P}$. Also let $U_i$ and $U_j$ be two open subsets of $M$ so that $U_i \cap U_j \neq \emptyset$, two gauges $s_i \in \Gamma\mleft(\mathcal{P}|_{U_i}\mright)$ and $s_j \in \Gamma\mleft(\mathcal{P}|_{U_j}\mright)$, and the unique $\sigma_{ji} \in \Gamma\mleft( \mleft.\mathcal{G}\mright|_{U_i \cap U_j} \mright)$ with $s_i = s_j \cdot \sigma_{ji}$ on $U_i \cap U_j$.

Then we have over $U_i \cap U_j$ that
\bas
A_{s_i}
&=
\mathrm{Ad}_{\sigma_{ji}^{-1}}\circ A_{s_j}
	+ \Delta\sigma_{ji}.
\eas
\end{theorems}

\begin{proof}
\leavevmode\newline
Define $H_{ji} \in \sAut\mleft( \mleft.\mathcal{P}\mright|_{U_i \cap U_j} \mright)$ as in Prop.\ \ref{prop:GaugeChangeAsBundleAutomorph} (also following its notation), then observe
\bas
\mleft(H_{ji} \circ s_j\mright)_x
&=
\mleft.s_i\mright|_x \cdot {\underbrace{\Phi^{-1}_{s_j}\mleft( \mleft.s_j\mright|_x \mright)}_{= e_x}}
=
\mleft.s_i\mright|_x
\eas
for all $x \in U_i \cap U_j$, and thus
\bas
s_j^! H_{ji}^!A
&=
\mleft( H_{ji} \circ s_j \mright)^!A
=
A_{s_i}.
\eas
By Thm.\ \ref{thm:GaugeTrafoOfGaugeBoson} we also have
\bas
H_{ji}^!A
&=
{\sAd_{\mathrm{pr}_2\circ\mleft(\sigma^{H_{ji}}\mright)^{-1}}} \circ A
	+ \mleft(\pi^*\Delta\mright)\sigma^{H_{ji}}.
\eas
We now want to apply $s_j^!$ on both hand sides.
Recall Remark \ref{PullBackGaugeFieldRemark} and \ref{rem:OtherNotationForLGBSectionsAsGaugeTrafo}, we will now use several natural isomorphisms without further mention, especially the base-point component will be treated in a "sloppy" way; the advantage will be a cleaner notation, and this is the reason why $\sAd$ becomes $\mathrm{Ad}$ by applying $s_j^!$ on the right hand side.
Using Prop.\ \ref{prop:GaugeChangeAsBundleAutomorph} and Remark \ref{rem:PullBackDarboux} we can derive
\bas
\Delta \sigma_{ji}
&=
\bigl( \mleft(\pi \circ s_j\mright)^* \Delta \bigr)\mleft( s_j^*\sigma^{H_{ji}} \mright)
=
\bigl( s_j^*\pi^* \Delta \bigr)\mleft( s_j^*\sigma^{H_{ji}} \mright)
=
s_j^!\mleft(\bigl( \pi^* \Delta \bigr)\mleft( \sigma^{H_{ji}} \mright)\mright),
\eas
and
\bas
s_j^!\mleft( {\sAd_{\mathrm{pr}_2\circ\mleft(\sigma^{H_{ji}}\mright)^{-1}}} \circ A \mright)
&=
{\mathrm{Ad}_{\mathrm{pr}_2\circ\mleft(\sigma^{H_{ji}}\mright)^{-1}\circ s_j}} \circ s_j^!A
=
{\mathrm{Ad}_{\mleft(\mathrm{pr}_2\circ\sigma^{H_{ji}}\circ s_j\mright)^{-1}}} \circ s_j^!A
=
{\mathrm{Ad}_{\sigma_{ji}^{-1}}} \circ A_{s_j},
\eas
where we used again the trivial relation (alternatively recall Cor.\ \ref{cor:PullbackLGB})
\bas
\mleft(\mathrm{pr}_2 \circ \sigma^{H_{ji}} \mright)^{-1}
&=
\mathrm{pr}_2 \circ \mleft(\sigma^{H_{ji}}\mright)^{-1}.
\eas
Altogether, concluding the proof with
\bas
A_{s_i}
&=
s_j^! H_{ji}^!A
=
{\mathrm{Ad}_{\sigma_{ji}^{-1}}} \circ A_{s_j}
	+ \Delta \sigma_{ji}.
\eas
\end{proof}

This allows us to finally show the first relation to the infinitesimal gauge theory developed by Alexei Kotov and Thomas Strobl.

\begin{remarks}{Integrating curved Yang-Mills gauge theories, part I}{IntegratingKotovStrobl}
Let the situation be as in Thm.\ \ref{thm:LocalGaugeTrafoChangeGauge}, then set 
\bas
\sigma_{ji}
&\coloneqq
\e^{t \varepsilon}
\eas
for a $\varepsilon \in \Gamma\mleft( \mleft.\mathcal{g}\mright|_{U_i \cap U_j} \mright)$ and $t \in \mathbb{R}$. Then we define the \textbf{infinitesimal gauge transformation $\delta_\varepsilon A_{s_j}$ of $A$ along $\varepsilon$ (w.r.t.\ $s_j$)} as an element of $\Omega^1\mleft( U_i \cap U_j; \mleft.\mathcal{g}\mright|_{U_i \cap U_j} \mright)$ by
\bas
\delta_\varepsilon A_{s_j}
&\coloneqq
\mleft.\frac{\mathrm{d}}{\mathrm{d}t}\mright|_{t=0} A_{s_i}.
\eas
The corresponding derivative related to the adjoint representation can be calculated as usual, and for the corresponding term related to the Darboux derivative we use the vector bundle connection introduced in Def.\ \ref{def:ConnectionOnLAB}. Thus, we get\footnote{Rigorously, one has to define $\delta_\varepsilon A_{s_j}$ via the natural projection $\mathrm{V}\mathcal{g} \to \mathcal{g}$ as we did in Prop.\ \ref{prop:FinallyTheNablaInduction}.}
\bas
\delta_\varepsilon A_{s_j}
&=
\nabla^{\mathcal{G}} \varepsilon 
	- \mleft[ \varepsilon, A_{s_j} \mright]_{\mathcal{g}}.
\eas
This is \textit{precisely} the infinitesimal gauge transformation developed by Alexei Kotov and Thomas Strobl, see \cite{CurvedYMH} for a concise summary or \cite{MyThesis} for an extended introduction. These references are for the very general situation using Lie algebroids, hence see alternatively \cite{My1stpaper} for this type of gauge theory restricted to Lie algebra bundles.

Thus, we successfully integrated this definition of infinitesimal gauge transformations.
\end{remarks}

Alexei's and Thomas's theory also generalizes the curvature/field strength related to gauge theory. Hence, let us now turn to this notion.

\subsection{Generalized curvature/field strength}\label{CurvatureSubsection}

\subsubsection{Yang-Mills connections on LGBs}\label{MultiplicativeForms}

In the following we will introduce the curvature; in order to achieve gauge invariance later in the physical gauge theory, we will now need to assume what we will call \textbf{compatibility conditions} on the horizontal distribution of the structural LGB $\mathcal{G}$. Here in this work we will just present these compatibility conditions, and you will be able to see in the proofs why this leads to gauge invariance later. However, if you want to understand the following conditions coming from a point of view not knowing the solution beforehand, then either see \cite{CurvedYMH} or \cite{MyThesis}; the latter also provides a physical motivation using what I called the \textbf{field redefinitions} which naturally motivate the gauge theory presented here by transforming a classical gauge theory in such a way that the physics is preserved; now recall Def.\ \ref{def:ConnectionOnLAB} for the following.

We start with the infinitesimal version of the compatibility conditions, already known and pointed out in the previously-mentioned references.

\begin{definitions}{Infinitesimal Yang-Mills connection}{YangMillsConnection}
Let $\mathcal{G} \to M$ be an LGB over a smooth manifold $M$, and $\mathrm{H}\mathcal{G}$ be a horizontal distribution of $\mathcal{G}$. Then we say that $\mathrm{H}\mathcal{G}$ is an \textbf{infinitesimal Yang-Mills connection (on $\mathcal{G}$)} if there is a $\zeta \in \Omega^2(M; \mathcal{g})$ such that the $\mathcal{G}$-connection $\nabla^{\mathcal{G}}$ on the associated LAB $\mathcal{g}$ with bracket $\mleft[ \cdot, \cdot \mright]_{\mathcal{g}}$ satisfies the \textbf{infinitesimal compatibility conditions}
\ba\label{CondSGleichNullLAB}
\nabla^{\mathcal{G}}\mleft( \mleft[ \mu, \nu \mright]_{\mathcal{g}} \mright)
&=
\mleft[ \nabla^{\mathcal{G}} \mu, \nu \mright]_{\mathcal{g}}
	+ \mleft[ \mu, \nabla^{\mathcal{G}} \nu \mright]_{\mathcal{g}},
\\
R_{\nabla^{\mathcal{G}}}(X, Y)\mu
&=
\mleft[ \zeta(X, Y), \mu \mright]_{\mathcal{g}}\label{CondKruemmungmitBLAB}
\ea
for all $\mu, \nu \in \Gamma(\mathcal{g})$ and $X, Y \in \mathfrak{X}(M)$, where $R_{\nabla^{\mathcal{G}}}$ is the curvature of $\nabla^{\mathcal{G}}$. We then also call $\nabla^{\mathcal{G}}$ an \textbf{(infinitesimal) Yang-Mills connection} and denote it by $\nabla^{\mathrm{YM}}$.

In the following, $\zeta$ is always such a 2-form as given here, and even though it is not uniquely given we will treat it as if it is fixed for a given $\nabla^{\mathrm{YM}}$. 
\end{definitions}

\begin{remark}\label{MackenziesStuffRelation}
\leavevmode\newline
As illustrated in \cite[\S 5.1]{MyThesis} and \cite{My1stpaper} such connections also arise in \cite[\S 7.2, page 271ff.]{mackenzieGeneralTheory}: There these connections are called \textbf{Lie derivation laws covering a coupling} due to the fact that these are important in the construction of "coupling" an LAB to a Lie algebroid via a Whitney sum in such a way that the outcome is again a Lie algebroid. There is a strong relationship of this study with the gauge theory we are going to define, however, we will not need this so that we will not repeat what has been discussed in \cite[\S 5.1]{MyThesis} and \cite{My1stpaper}. 

One may want to write that $\mleft( \nabla^{\mathrm{YM}}, \zeta \mright)$ is a Yang-Mills connection, giving an emphasis on the non-uniqueness of $\zeta$. However, we are not going to do so, and the mentioned references and relations to Mackenzie's work actually show that there is just one object behind all of that describing this Yang-Mills connection and its relation to the following gauge theory. It would take too much time introducing all of this again; see the references.

The relation to Mackenzie's study can be useful for the interested reader to understand the importance of these compatibility conditions, too. In my opinion, the compatibility conditions together with a trivial what Mackenzie calls the obstruction class (and an ad-invariant fibre metric on the LAB) are needed to define infinitesimal Yang-Mills gauge theories from an axiomatic point of view. If the reader is interested into that, then I would appreciate an email so that I can send these details.
\end{remark}
%For this we will start 
%We will need several auxiliary result, also related to the Darboux derivative w.r.t.\ Yang-Mills connections. Especially also recall all the discussion around the modified push-forward $\mathcal{r}$ via right-translations, Def.\ \ref{def:WeModTheRIghtPush}.

We want to integrate these infinitesimal compatibility conditions to conditions on $\mathrm{H}\mathcal{G}$ directly.
Let us initially focus on the first infinitesmal compatibility condition \eqref{CondSGleichNullLAB}. In the following we will also view $\mathcal{G}$ as a principal $\mathcal{G}$-bundle which is equipped with the same horizontal distribution as $\mathcal{G}$ as LGB; related notions and notations carry over. One can show that Ehresmann connections satisfy \eqref{CondSGleichNullLAB}, and in fact Ehresmann connections on LGBs (and Lie groupoids) were already discussed in \cite{LAURENTGENGOUXStienonXuMultiplicativeForms}; and there is also a rather recent preprint related to a similar subject, see \cite{FernandesMarcutMultiplicativeForms}. \textbf{Special thanks to Camille Laurent-Gengoux for pointing out these references and discussing related subjects, this helped me tremendously.}

\begin{lemmata}{Ehresmann connections induce Lie bracket derivations, \newline \cite[\S 4.5, Prop.\ 4.21]{LAURENTGENGOUXStienonXuMultiplicativeForms}}{YangMillsConnAnEhresmann}
Let $\mathcal{G} \to M$ be an LGB over a smooth manifold $M$, and $\mathrm{H}\mathcal{G}$ a horizontal distribution on $\mathcal{G}$. If $\mathrm{H}\mathcal{G}$ is an (Ehresmann) connection on $\mathcal{G}$ as principal bundle, then the infinitesimal compatibility condition \eqref{CondSGleichNullLAB} is satisfied.
\end{lemmata}

\begin{proof}
\leavevmode\newline
This can be quickly answered by recalling how the parallel transport behaves; recall that Def.\ \ref{def:FinallyTheConnection} comes from Eq.\ \eqref{OiTHatIsHowWeFormulateHorizSymmetry} which itself stems from Eq.\ \eqref{PTHomomNEw}; that is, the associated parallel transport to an Ehresmann connection $\mathrm{H}\mathcal{G}$ satisfies Eq.\ \eqref{PTHomomNEw}. This means that the corresponding parallel transport $\mathrm{PT}^{\mathcal{G}}$ of $\mathcal{G}$ (omitting the notation of the involved curve) is a homomorphism, \textit{i.e.}\
\bas
\mathrm{PT}^{\mathcal{G}}(gq)
&=
\mathrm{PT}^{\mathcal{G}}(g) ~ \mathrm{PT}^{\mathcal{G}}(q)
\eas
for all $(g, q) \in \mathcal{G} * \mathcal{G}$. Thus, infinitesimally the associated parallel transport on $\mathcal{g}$ is a homomorphism w.r.t.\ the Lie bracket, and thus $\nabla^{\mathcal{G}}$ is a Lie bracket derivation.
\end{proof}

\begin{remarks}{Base manifold a horizontal leave}{baseManifoldHoriz}
Observe that the proof's argument implies that a parallel transported neutral element is again the neutral element of the corresponding final fibre, if $\mathrm{H}\mathcal{G}$ is an Ehresmann connection. Thus, $M$ naturally embedded via the neutral section $e$ can then be viewed as a horizontal leave of $\mathrm{H}\mathcal{G}$.
\end{remarks}

Due to this, given compatibility condition \eqref{CondSGleichNullLAB}, the following is clear by definition.

\begin{corollaries}{The total Maurer-Cartan form as connection 1-form}{TotMCFormIsConnectionForm}
Let $\mathcal{G} \to M$ be an LGB over a smooth manifold $M$, and $\mathrm{H}\mathcal{G}$ an Ehresmann connection on $\mathcal{G}$. Then the total Maurer-Cartan form $\mu_{\mathcal{G}}^{\mathrm{tot}}$ is the connection 1-form corresponding to $\mathrm{H}\mathcal{G}$.
\end{corollaries}

\begin{proof}
\leavevmode\newline
This simply follows by Def.\ \ref{def:TotMCFormOnLGB} and Thm.\ \ref{thm:OurConnectionHasAUniqueoneForm}.
\end{proof}

\begin{remarks}{Darboux derivative as local gauge field}{DavouxderivativeALocalGaugeField}
If $\mathrm{H}\mathcal{G}$ is an Ehresmann connection on $\mathcal{G}$, we know by Cor.\ \ref{cor:TotMCFormIsConnectionForm} that the Darboux derivative $\Delta \sigma$ of a (local) section $\sigma$ of $\mathcal{G}$ (Def.\ \ref{def:DarbouxDerivativeOnLGBs}) is a local gauge field (w.r.t.\ $\sigma$) corresponding to $\mu_{\mathcal{G}}^{\mathrm{tot}}$, recall Def.\ \ref{def:LocalGaugeField}. That is,
\bas
\Delta \sigma
&=
\mleft(\mu_{\mathcal{G}}^{\mathrm{tot}}\mright)_\sigma.
\eas
\end{remarks}

Furthermore, following \cite{LAURENTGENGOUXStienonXuMultiplicativeForms}, we can derive that $\mu_{\mathcal{G}}^{\mathrm{tot}}$ is a multiplicative form if and only if $\mathrm{H}\mathcal{G}$ is an Ehresmann connection; let us introduce multiplicative 1-forms. Recall Lemma \ref{lem:PullbackFibreBundleItsTangentSp}.

\begin{definitions}{Multiplicative forms, \cite[\S 2.1, special situation of Def.\ 2.1]{crainic2015multiplicative}}{MultiplicativeFormsDef}
Let $\mathcal{G} \stackrel{\pi_{\mathcal{G}}}{\to} M$ be an LGB over a smooth manifold $M$. Then we call an $\omega \in \Omega^1\mleft( \mathcal{G}; \pi_{\mathcal{G}}^*\mathcal{g} \mright)$ a \textbf{multiplicative form} if
\bas
\omega_{gq}\mleft( \mathrm{D}_{(g, q)}\Phi(X, Y)  \mright)
&=
\sAd_{q^{-1}}\bigl( \omega_{g}(X) \bigr)
	+ \bigl( gq, \widehat{\omega}_{q}(Y) \bigr)
\eas
for all $(g, q) \in \mathcal{G}*\mathcal{G}$ and $(X, Y) \in \mathrm{T}_{(g, q)}(\mathcal{G}*\mathcal{G})$,
where $\Phi: \mathcal{G} * \mathcal{G} \to \mathcal{G}$ is the multiplication in $\mathcal{G}$, and $\widehat{\omega} \coloneqq \mathrm{pr}_2 \circ \omega$, where $\mathrm{pr}_2: \pi_{\mathcal{G}}^*\mathcal{g} \to \mathcal{g}$ is the natural projection onto the second component.
\end{definitions}

\begin{remark}\label{sloppynotationformultiplicativity}
\leavevmode\newline
The shape of the second summand is again just because of technical reasons due to the definition of pullback bundles and their base points. If we do not care so much about the involved base points since these are clear by context, then we can also shortly write
\bas
\mleft(\Phi^!\omega\mright)_{(g, q)}(X, Y)
&=
\sAd_{q^{-1}}\bigl( \omega_{g}(X) \bigr)
	+ \omega_{q}(Y).
\eas
Also recall what we mentioned in Subsection \ref{BasicNotations} about pullback sections.
\end{remark}

\begin{theorems}{Ehresmann connection 1-forms on LGBs are multiplicative, \newline \cite[\S 4.4, implication of Lemma 4.14]{LAURENTGENGOUXStienonXuMultiplicativeForms}}{TotMaurerIsMultiplicativeIfSEqual0}
Let $\mathcal{G} \stackrel{\pi_{\mathcal{G}}}{\to} M$ be an LGB over a smooth manifold $M$, and $\mathrm{H}\mathcal{G}$ a horizontal distribution on $\mathcal{G}$ with corresponding total Maurer-Cartan form $\mu_{\mathcal{G}}^{\mathrm{tot}}$. Then $\mathrm{H}\mathcal{G}$ is an (Ehresmann) connection on $\mathcal{G}$ as principal bundle if and only if $\mu_{\mathcal{G}}^{\mathrm{tot}}$ is multiplicative.
\end{theorems}

\begin{proof}
\leavevmode\newline
The general approach in the mentioned reference is different to ours; hence, let us provide a proof suitable to our constructions. We will view $\mathcal{G}$ as a principal $\mathcal{G}$-bundle whose horizontal distribution is also $\mathrm{H}\mathcal{G}$. We will also use the same notation as in Def.\ \ref{def:MultiplicativeFormsDef}.

%Let us first assume that $\mathrm{H}\mathcal{G}$ is an Ehresmann connection, therefore we now initially show that the associated total Maurer-Cartan form is multiplicative. 
To answer whether or not $\mathrm{H}\mathcal{G}$ is an Ehresmann connection we just need to discuss whether $\mu_{\mathcal{G}}^{\mathrm{tot}}$ is a connection 1-form on $\mathcal{G}$ due to the fact that $\mathrm{H}\mathcal{G}$ is the kernel of $\mu_{\mathcal{G}}^{\mathrm{tot}}$ by definition (see also Cor.\ \ref{cor:TotMCFormIsConnectionForm}), also using Thm.\ \ref{thm:OurConnectionHasAUniqueoneForm}. Also by definition, $\mu_{\mathcal{G}}^{\mathrm{tot}}$ already satisfies
\bas
\mu_{\mathcal{G}}^{\mathrm{tot}}\mleft( \widetilde{\nu} \mright)
&=
\pi_{\mathcal{G}}^*\nu.
\eas
%By Cor.\ \ref{cor:TotMCFormIsConnectionForm} we know that $\mu_{\mathcal{G}}^{\mathrm{tot}}$ satisfies Def.\ \ref{def:GaugeBosonsOnLGBPrincies}; also recall the second part of Rem.\ \ref{PointwiseNotationOfConnectioNOneForms}. 
By additionally using Thm.\ \ref{thm:DiffOfLGBAction} (especially the associated Rem.\ \ref{rem:DarbouxInActionInfinit}) we show
\bas
\mleft(\mu_{\mathcal{G}}^{\mathrm{tot}}\mright)_{gq}\mleft( \mathrm{D}_{(g,q)}\Phi(X, Y) \mright)
&=
\mleft(\mu_{\mathcal{G}}^{\mathrm{tot}}\mright)_{gq}\mleft(
	\mathrm{D}_gR_\sigma(X)
		- \mleft.{\oversortoftilde{ \mleft.\mleft(\pi_{\mathcal{G}}^!\Delta \sigma\mright)\mright|_g (X)}}\mright|_{gq}
		+ \mleft.{\oversortoftilde{\mleft( \mu_{\mathcal{G}}^{\mathrm{tot}}\mright)_{q} (Y)}}\mright|_{gq}
\mright)
\\
&=
\mleft(\mu_{\mathcal{G}}^{\mathrm{tot}}\mright)_{gq}\bigl(
	\mathcal{r}_{q*}(X)
\bigr)
		+ \mleft( gq, \mleft( \widehat{\mu_{\mathcal{G}}^{\mathrm{tot}}}\mright)_{q} (Y) \mright)
\\
&=
\mleft(\mathcal{r}_q^!\mu_{\mathcal{G}}^{\mathrm{tot}}\mright)_{g}(X)
		+ \mleft( gq, \mleft( \widehat{\mu_{\mathcal{G}}^{\mathrm{tot}}}\mright)_{q} (Y) \mright)
%\\
%&=
%\sAd_{q^{-1}}\mleft(\mleft(\mu_{\mathcal{G}}^{\mathrm{tot}}\mright)_{g}(X)\mright)
		%+ \mleft( gq, \mleft( \widehat{\mu_{\mathcal{G}}^{\mathrm{tot}}}\mright)_{q} (Y) \mright)
\eas 
for all $(g, q) \in \mathcal{G}*\mathcal{G}$ and $(X, Y) \in \mathrm{T}_{(g, q)}(\mathcal{G}*\mathcal{G})$, where $\sigma$ is a (local) section of $\mathcal{G}$ such that $\sigma_{x} = q$. Thus, $\mu_{\mathcal{G}}^{\mathrm{tot}}$ is multiplicative if and only if
\bas
\sAd_{q^{-1}}\mleft(\mleft(\mu_{\mathcal{G}}^{\mathrm{tot}}\mright)_{g}(X)\mright)
		+ \mleft( gq, \mleft( \widehat{\mu_{\mathcal{G}}^{\mathrm{tot}}}\mright)_{q} (Y) \mright)
&=
\mleft(\mathcal{r}_q^!\mu_{\mathcal{G}}^{\mathrm{tot}}\mright)_{g}(X)
		+ \mleft( gq, \mleft( \widehat{\mu_{\mathcal{G}}^{\mathrm{tot}}}\mright)_{q} (Y) \mright)
\eas
if and only if
\bas
\sAd_{q^{-1}}\mleft(\mleft(\mu_{\mathcal{G}}^{\mathrm{tot}}\mright)_{g}(X)\mright)
&=
\mleft(\mathcal{r}_q^!\mu_{\mathcal{G}}^{\mathrm{tot}}\mright)_{g}(X)
\eas
if and only if $\mu_{\mathcal{G}}^{\mathrm{tot}}$ is a connection 1-form. This finishes the proof.
%
%$\bullet$ Now we assume that $\mu_{\mathcal{G}}^{\mathrm{tot}}$ is multiplicative. 
\end{proof}

This Theorem also implies the Leibniz rule of the Darboux derivative.


\begin{propositions}{Leibniz rule of the Darboux derivative}{DarbouxLeibnizRule}
Let $\mathcal{G} \stackrel{\pi_{\mathcal{G}}}{\to} M$ be an LGB over a smooth manifold $M$, and $\mathrm{H}\mathcal{G}$ an Ehresmann connection on $\mathcal{G}$. Then we have
\bas
\Delta(\sigma \tau)
&=
\mathrm{Ad}_{\tau^{-1}} \circ \Delta \sigma
	+ \Delta\tau
\eas
and
\bas
\Delta\mleft(\sigma^{-1}\mright)
&=
- \mathrm{Ad}_{\sigma} \circ \Delta \sigma
\eas
for all $\sigma, \tau \in \Gamma(\mathcal{G})$.
\end{propositions}

\begin{remark}
\leavevmode\newline
These are precisely the formulas one expects, see \textit{e.g.}\ \cite[\S 5.1, Eq.\ (2) and (3), page 182]{mackenzieGeneralTheory} for the "classical" case; beware that the reference uses right- instead of left-invariant vector fields to describe the Lie algebra, so that the referenced formulas look explicitly a bit different. Nevertheless, the similarity of the structure should be obvious.

The mentioned reference for Thm.\ \ref{thm:TotMaurerIsMultiplicativeIfSEqual0}, \cite[\S 4.4, Lemma 4.14]{LAURENTGENGOUXStienonXuMultiplicativeForms}, actually uses the Leibniz rule of the Darboux derivative as an equivalent description of Ehresmann connections.
\end{remark}

\begin{proof}[Proof of Prop.\ \ref{prop:DarbouxLeibnizRule}]
\leavevmode\newline
For this calculation we will take care about base points of pullback bundles w.r.t.\ the Darboux derivative, thus recall the last part of Remark \ref{RemarkABoutDarbouxNotationWRTPullback}. Using the same notation as in Def.\ \ref{def:MultiplicativeFormsDef}, this proposition follows simply by Thm.\ \ref{thm:TotMaurerIsMultiplicativeIfSEqual0}, that is, set $Y \coloneqq \mathrm{D}_x\sigma(X)$ and $Z \coloneqq \mathrm{D}_x\tau(X)$ for $X \in \mathrm{T}_xM$ ($x \in M$), then
\bas
\bigl( \sigma_x\tau_x, \Delta(\sigma\tau)_x(X) \bigr)
&=
\mleft(\mu_{\mathcal{G}}^{\mathrm{tot}}\mright)_{\sigma_x\tau_x}\bigl( \mathrm{D}_x(\sigma \tau)(X) \bigr)
\\
&=
\mleft(\mu_{\mathcal{G}}^{\mathrm{tot}}\mright)_{\sigma_x\tau_x}\bigl( \mathrm{D}_{(\sigma_x, \tau_x)}\Phi(Y, Z) \bigr)
\\
&=
\sAd_{\tau_{x}^{-1}}\mleft(\mleft(\mu_{\mathcal{G}}^{\mathrm{tot}}\mright)_{\sigma_x}(Y)\mright)
		+ \mleft( \sigma_x\tau_x, \mleft( \widehat{\mu_{\mathcal{G}}^{\mathrm{tot}}}\mright)_{\tau_x} (Z) \mright)
\\
&=
\mleft( 
	\sigma_x\tau_x, 
	\mathrm{Ad}_{\tau_x^{-1}}\mleft( \mleft( \widehat{\mu_{\mathcal{G}}^{\mathrm{tot}}}\mright)_{\sigma_x} (Y) \mright) 
	+ \mleft( \widehat{\mu_{\mathcal{G}}^{\mathrm{tot}}}\mright)_{\tau_x} (Z) 
\mright)
\\
&=
\mleft( 
	\sigma_x\tau_x, 
	\mathrm{Ad}_{\tau_x^{-1}}\bigl( \mleft.\mleft( \Delta \sigma \mright)\mright|_x(X) \bigr) 
	+ \mleft.\mleft( \Delta \tau \mright)\mright|_x(X)
\mright).
\eas
Hence,
\bas
\Delta(\sigma \tau)
&=
\mathrm{Ad}_{\tau^{-1}} \circ \Delta \sigma
	+ \Delta\tau.
\eas
%
%We have
%\bas
%\mathrm{D}_x\underbrace{(\sigma \tau)}_{\mathclap{ M \ni y \mapsto \sigma_x \tau_x = R_\tau(\sigma_x) }}(X)
%&=
%\mathrm{D}_x\mleft( R_\tau \circ \sigma \mright)(X)
%\\
%&=
%\mathrm{D}_{\sigma_x}R_\tau \bigl( \mathrm{D}_x\sigma(X) \bigr)
%\\
%&=
%\mathcal{r}_{\tau_x*}\bigl( \mathrm{D}_x\sigma(X) \bigr)
	%+ \mleft.{\oversortoftilde{
		%\mleft. \mleft( \pi_{\mathcal{G}}^!\Delta\tau \mright) \mright|_{\sigma_x}\bigl( \mathrm{D}_x\sigma(X) \bigr)
	%}}\mright|_{\sigma_x \cdot \tau_x}
%\\
%&=
%\mathcal{r}_{\tau_x*}\bigl( \mathrm{D}_x\sigma(X) \bigr)
	%+ \mleft.{\oversortoftilde{
		%\mleft. \mleft( \Delta\tau \mright) \mright|_{x}(X)
	%}}\mright|_{\sigma_x \cdot \tau_x}
%\eas
%for all $x \in M$, $X \in \mathrm{T}_xM$ and $\sigma, \tau \in \Gamma(\mathcal{G})$, where we used that $\pi_{\mathcal{G}} \circ \sigma = \mathds{1}_M$. Viewing $\mathcal{G}$ as a principal $\mathcal{G}$-bundle we can apply Cor.\ \ref{cor:ModifiedRightPushyCommutesWithProj}. Together with recalling Def.\ \ref{def:FundVecs} we derive
%\bas
%\Delta(\sigma\tau)_x(X)
%&=
%\mleft(\mu_{\mathcal{G}}^{\mathrm{tot}}\mright)_{\sigma_x\tau_x}\bigl( \mathrm{D}_x(\sigma \tau)(X) \bigr)
%\\
%&=
%\mleft(\mathrm{D}_{\sigma_x\tau_x} L_{(\sigma_x\tau_x)^{-1}} \circ \pi_v \circ \mathrm{D}_x(\sigma\tau) \mright)(X)
%\\
%&=
%\mleft(\mathrm{D}_{\sigma_x\tau_x} L_{\tau_x^{-1}\sigma_x^{-1}} \circ \mathcal{r}_{\tau_x*} \circ \pi_v \circ \mathrm{D}_x\sigma\mright)(X)
	%+ \mleft. \mleft( \Delta\tau \mright) \mright|_{x}(X)
%\\
%&=
%\mleft(\mathrm{D}_{\sigma_x\tau_x} L_{\tau_x^{-1}\sigma_x^{-1}} \circ \mathrm{D}_{\sigma_x}R_{\tau_x}
%\circ \pi_v \circ \mathrm{D}_x\sigma\mright)(X)
	%+ \mleft. \mleft( \Delta\tau \mright) \mright|_{x}(X)
%\\
%&=
%\Bigl(\mathrm{D}_{\sigma_x} 
%{\underbrace{\mleft( L_{\tau_x^{-1}\sigma_x^{-1}} \circ R_{\tau_x} \mright)}
	%_{\mathclap{ \mathcal{G} \ni g \mapsto c_{\tau_x^{-1}} \circ L_{\sigma_x^{-1}} }}}
%\circ \pi_v \circ \mathrm{D}_x\sigma\Bigr)(X)
	%+ \mleft. \mleft( \Delta\tau \mright) \mright|_{x}(X)
%\\
%&=
%\mleft(\mathrm{D}_{e_x} c_{\tau_x^{-1}} \circ \mathrm{D}_{\sigma_x}L_{\sigma_x^{-1}}
%\circ \pi_v \circ \mathrm{D}_x\sigma\mright)(X)
	%+ \mleft. \mleft( \Delta\tau \mright) \mright|_{x}(X)
%\\
%&=
%\mathrm{Ad}_{\tau_x^{-1}}\bigl( \mleft(\Delta\sigma\mright)_x(X) \bigr)
	%+ \mleft. \mleft( \Delta\tau \mright) \mright|_{x}(X)
%\eas
%where $\pi_v$ is the projection onto the vertical structure in $\mathcal{G}$, $e_x$ the neutral element of $\mathcal{G}_x$, and using Remark \ref{rem:ModRightPushOnVertic}, \textit{i.e.} ${\mathcal{r}_{\tau_x*}} \circ \pi_v = r_{\tau_x*} \circ \pi_v$.

Now recall Remark \ref{rem:baseManifoldHoriz}; this implies for the neutral element section $e$ that
\bas
\Delta e
&\equiv
0,
\eas
and therefore, using the previous result just derived,
\bas
0
&=
\Delta e
=
\Delta \mleft( \sigma \sigma^{-1} \mright)
=
\mathrm{Ad}_{\sigma} \circ \Delta \sigma
	+ \Delta\mleft(\sigma^{-1}\mright),
\eas
thus, concluding the proof with
\bas
\Delta\mleft(\sigma^{-1}\mright)
&=
- \mathrm{Ad}_{\sigma} \circ \Delta \sigma.
\eas
\end{proof}

The Leibniz rule of the Darboux derivative extends to $\nabla^{\mathcal{G}}$.

\begin{lemmata}{The conjugation of the canonical LAB connection is a Yang-Mills connection}{FieldRedefViaLGBSection}
Let $\mathcal{G} \stackrel{\pi_{\mathcal{G}}}{\to} M$ be an LGB over a smooth manifold $M$, and $\mathrm{H}\mathcal{G}$ an Ehresmann connection on $\mathcal{G}$. Then we have
\ba\label{SpecialFieldRedef}
\mathrm{Ad}_{\sigma^{-1}} \circ \nabla^{\mathcal{G}} \circ \mathrm{Ad}_\sigma
&=
\nabla^{\mathcal{G}}
	+ \mathrm{ad}_{\Delta \sigma}
\ea
for all $\sigma \in \Gamma(\mathcal{G})$, that is,
\bas
\mleft(\mathrm{Ad}_{\sigma^{-1}} \circ \nabla^{\mathcal{G}}_X \circ \mathrm{Ad}_\sigma\mright)\nu
&=
\nabla^{\mathcal{G}}_X\nu
	+ \mleft[ \Delta\sigma(X), \nu \mright]_{\mathcal{g}}
\eas
for all $\nu \in \Gamma(\mathcal{g})$ and $X \in \mathfrak{X}(M)$.
If $\mathrm{H}\mathcal{G}$ is an infinitesimal Yang-Mills connection, $\nabla^{\mathcal{G}} = \nabla^{\mathrm{YM}}$, especially $R_{\nabla^{\mathrm{YM}}} = \mathrm{ad}_\zeta$ for $\zeta \in \Omega^2(M; \mathcal{g})$, then $\widetilde{\nabla}^\sigma \coloneqq \mathrm{Ad}_{\sigma^{-1}} \circ \nabla^{\mathrm{YM}} \circ \mathrm{Ad}_\sigma$ is also an infinitesimal Yang-Mills connection with
\bas
R_{\widetilde{\nabla}^\sigma}
&=
\mathrm{ad}_{\mathrm{Ad}_{\sigma^{-1}}(\zeta)},
\eas
in particular we get
\ba\label{AlmostGeneralizedMaurerCartanEquation}
\mleft[ \mathrm{d}^{\nabla^{\mathrm{YM}}} \Delta \sigma + \frac{1}{2} \mleft[ \Delta \sigma \stackrel{\wedge}{,} \Delta\sigma \mright]_{\mathcal{g}} + \zeta - \mathrm{Ad}_{\sigma^{-1}} \circ \zeta, ~ \nu \mright]_{\mathcal{g}}
&=
0
\ea
for all $\nu \in \Gamma(\mathcal{g})$.
\end{lemmata}

\begin{proof}
\leavevmode\newline
By Prop.\ \ref{prop:DarbouxLeibnizRule} we get
\bas
\Delta\mleft( \e^{\mathrm{Ad}_\sigma (\nu)} \mright)
&=
\Delta\mleft( \sigma \e^{\nu} \sigma^{-1} \mright)
\\
&=
\mathrm{Ad}_{\sigma} \circ \Delta \mleft( \sigma \e^\nu \mright)
	+ \Delta \mleft( \sigma^{-1} \mright)
\\
&=
\mathrm{Ad}_{\sigma} \circ \mleft( \mathrm{Ad}_{\e^{-\nu}} \circ \Delta \sigma + \Delta e^\nu \mright)
	- \mathrm{Ad}_\sigma \circ \Delta \sigma
\\
&=
\mathrm{Ad}_{\sigma} \circ \mleft( \mathrm{Ad}_{\e^{-\nu}} \circ \Delta \sigma
	- \Delta \sigma + \Delta \e^\nu \mright)
\eas
for all $\sigma \in \Gamma(\mathcal{G})$ and $\nu \in \Gamma(\mathcal{g})$, where the first equality is a well-known fact, see \textit{e.g.}\ \cite[\S 1.7, Thm.\ 1.7.16, page 59]{Hamilton}. Then
\bas
\nabla^{\mathcal{G}}\mleft( \mathrm{Ad}_\sigma(\nu) \mright)
&=
\mleft. \frac{\mathrm{d}}{\mathrm{d}t} \mright|_{t=0} \mleft(
	\Delta\mleft( \e^{\mathrm{Ad}_\sigma (t\nu)} \mright)
\mright)
\\
&=
\mleft. \frac{\mathrm{d}}{\mathrm{d}t} \mright|_{t=0} \Bigl(
	\mathrm{Ad}_{\sigma} \circ \mleft( \mathrm{Ad}_{\e^{-t\nu}} \circ \Delta \sigma
	- \Delta \sigma + \Delta \e^{t\nu} \mright)
\Bigr)
\\
&=
\mathrm{Ad}_{\sigma} \circ \mleft( \mathrm{ad}_{-\nu} \circ \Delta \sigma
	+ \nabla^{\mathcal{G}} \nu \mright)
\\
&=
\mathrm{Ad}_{\sigma} \circ \mleft( \mathrm{ad}_{\Delta \sigma}(\nu)
	+ \nabla^{\mathcal{G}} \nu \mright)
\eas
making use of that this calculation is pointwise at $x \in M$ just a standard derivative in the vector space $\mathcal{g}_x$. Thus,
\bas
\mathrm{Ad}_{\sigma^{-1}} \circ \nabla^{\mathcal{G}} \circ \mathrm{Ad}_\sigma
&=
\nabla^{\mathcal{G}}
	+ \mathrm{ad}_{\Delta \sigma}.
\eas
In order to show $R_{\widetilde{\nabla}^\sigma} = \mathrm{ad}_{\mathrm{Ad}_{\sigma^{-1}}(\zeta)}$ one uses the definition $\widetilde{\nabla}^\sigma \coloneqq \mathrm{Ad}_{\sigma^{-1}} \circ \nabla^{\mathrm{YM}} \circ \mathrm{Ad}_\sigma$ to calculate the curvature, 
\bas
R_{\widetilde{\nabla}^\sigma} (X, Y) \nu
&=
\widetilde{\nabla}^\sigma_X \widetilde{\nabla}^\sigma_Y \nu
	- \widetilde{\nabla}^\sigma_Y \widetilde{\nabla}^\sigma_X \nu
	- \widetilde{\nabla}^\sigma_{[X, Y]} \nu
\\
&=
\mleft(\mathrm{Ad}_{\sigma^{-1}} \circ \mleft( \nabla^{\mathrm{YM}}_X \circ \nabla^{\mathrm{YM}}_Y 
	- \nabla^{\mathrm{YM}}_Y \circ \nabla^{\mathrm{YM}}_X
	- \nabla^{\mathrm{YM}}_{[X, Y]} \mright) \circ \mathrm{Ad}_\sigma\mright) (\nu)
\\
&=
\mleft( \mathrm{Ad}_{\sigma^{-1}} \circ
	R_{\nabla^{\mathcal{G}}}(X, Y) \circ \mathrm{Ad}_{\sigma}
\mright) (\nu)
\\
&=
\mathrm{Ad}_{\sigma^{-1}}\mleft( \mleft[ \zeta(X, Y), \mathrm{Ad}_{\sigma}(\nu) \mright]_{\mathcal{g}} \mright)
\\
&=
\mleft[ \mathrm{Ad}_{\sigma^{-1}}\bigl(\zeta(X, Y)\bigr), \nu \mright]_{\mathcal{g}}
\eas
for all $X, Y \in \mathfrak{X}(M)$ and $\nu \in \Gamma(\mathcal{g})$. The remaining parts of the statement are trivial and straight-forward calculations, making use of Def.\ \ref{def:YangMillsConnection}; in fact, we have proven all the remaining statements already in \cite[\S 3, part of Thm.\ 3.6]{My1stpaper} and in a more general manner in \cite[\S 4.6, Thm.\ 4.6.9]{MyThesis}: That $\widetilde{\nabla}^\sigma$ is a Lie bracket derivation is trivial to show, and one uses identity \eqref{SpecialFieldRedef} to show another identity of its curvature,
\bas
R_{\widetilde{\nabla}^\sigma} 
&=
\mleft[ \mathrm{d}^{\nabla^{\mathrm{YM}}} \Delta \sigma + \frac{1}{2} \mleft[ \Delta \sigma \stackrel{\wedge}{,} \Delta\sigma \mright]_{\mathcal{g}} + \zeta, \cdot \mright]_{\mathcal{g}}.
\eas
Combining both identities for $R_{\widetilde{\nabla}^\sigma}$ we conclude with a proof of Eq.\ \eqref{AlmostGeneralizedMaurerCartanEquation}. 
\end{proof}

We also get a generalization of statements like \cite[\S 5.5, Lemma 5.5.5, page 276]{Hamilton}:

\begin{lemmata}{Lie bracket of horizontal and vertical vector}{BracketVertHor}
Let $\mathcal{G} \to M$ be an LGB over a smooth manifold $M$ and $\mathcal{P} \stackrel{\pi}{\to} M$ a principal $\mathcal{G}$-bundle, also let $\mathrm{H}\mathcal{G}$ be a horizontal distribution on $\mathcal{G}$ and $A \in \Omega^1(\mathcal{P}; \pi^*\mathcal{g})$ be a connection 1-form on $\mathcal{P}$. Then we have
\bas
A\mleft( \mleft[ X, \widetilde{\nu} \mright] \mright)
&=
\mleft.\frac{\mathrm{d}}{\mathrm{d}t}\mright|_{t=0} \Bigl(
		\mleft( \pi^! \Delta \e^{t\nu} \mright)\mleft(X\mright) \circ r_{\e^{-t\nu}}
	\Bigr)
\eas
for all $\nu \in \Gamma(\mathcal{g})$ and horizontal vector fields $X \in \mathfrak{X}(\mathcal{P})$. If there is an $\omega \in \mathfrak{X}(M)$ such that $\mathrm{D}\pi(X) = \pi^*\omega$, then
\bas
A\bigl( \mleft[ X, \widetilde{\nu} \mright] \bigr)
&=
\mleft( \pi^*\nabla^{\mathcal{G}} \mright)_{X}(\pi^*\nu)
=
\pi^*\mleft( \nabla^{\mathcal{G}}_\omega \nu \mright).
\eas

If $\mathrm{H}\mathcal{G}$ is an Ehresmann connection, then we can conclude for all horizontal vector fields $X$ that
\bas
A\bigl( \mleft[ X, \widetilde{\nu} \mright] \bigr)
&=
\mleft( \pi^*\nabla^{\mathcal{G}} \mright)_{X}(\pi^*\nu).
\eas
\end{lemmata}

\begin{remark}
\leavevmode\newline
The last statement got also stated in \cite[Eq.\ (2.8)]{FernandesMarcutMultiplicativeForms} in the case of the LGB as principal bundle; we were unaware of this proof, which is why our proof is a bit different.
\end{remark}

\begin{proof}[Proof of Lemma \ref{lem:BracketVertHor}]
\leavevmode\newline
First of all recall the very basic fact (see \textit{e.g.}\ \cite[\S A.1, Thm.\ A.1.46, page 615]{Hamilton}) that for two vector fields $X, Y \in \mathfrak{X}(\mathcal{P})$ we have\footnote{Rigorously, similarly to Prop.\ \ref{prop:FinallyTheNablaInduction} one has a projection involved, but as explained in Remark \ref{RemarkAboutPrTwoInDarbouxDerivative} we will omit notating this projection; in the context of this proof we can allow ourselves to take the typical "less rigorous" approach for simplicity.}
\bas
\mleft[ X, Y \mright]_p
&=
-\mleft.\frac{\mathrm{d}}{\mathrm{d}t}\mright|_{t=0}
	\mathrm{D}_{\phi_t(p)}\phi_{-t} \mleft(X_{\phi_t(p)}\mright)
\eas
for all $p \in \mathcal{P}$, where $\phi_t$ is the local flow of $Y$ (with parameter $t$ in an open interval of $\mathbb{R}$ containing 0). Now let $X$ be a horizontal vector field and $Y = \widetilde{\nu}$. For such a $Y$ we know by Def.\ \ref{def:FundVecs} that
\bas
\widetilde{\nu}_p
&=
\mathrm{D}_{e_{x}}\Phi_p\mleft(\nu_{x}\mright)
=
\mleft. \frac{\mathrm{d}}{\mathrm{d}t} \mright|_{t=0}
\mleft( p \cdot \e^{t\nu_x} \mright)
\eas
for all $p \in \mathcal{P}$, where $\Phi_p$ is the orbit map through $p$, $x \coloneqq \pi(p)$, and $e_{x}$ the neutral element of $\mathcal{G}_x$. Thus, 
\bas
\phi_t(p)
&=
p \cdot \e^{t\nu_x}
=
r_{\e^{t\nu}}(p),
\eas
and by rewriting in sense of Def.\ \ref{def:WeModTheRIghtPush} we achieve
\bas
\mleft[ X, \widetilde{\nu} \mright]_p
&=
-\mleft.\frac{\mathrm{d}}{\mathrm{d}t}\mright|_{t=0}
	\mathrm{D}_{p \cdot \e^{t\nu_x}}r_{\e^{-t\nu}} \mleft(X_{p \cdot \e^{t\nu_x}}\mright)
\\
&=
- \mleft.\frac{\mathrm{d}}{\mathrm{d}t}\mright|_{t=0} \mleft(
	\mathrm{D}_{p \cdot \e^{t\nu_x}}r_{\e^{-t\nu}} \mleft(X_{p \cdot \e^{t\nu_x}}\mright)
	- \mleft.{\oversortoftilde{
		\mleft. \mleft( \pi^!\Delta \e^{-t\nu} \mright) \mright|_{p \cdot \e^{t\nu_x}}\mleft(X_{p \cdot \e^{t\nu_x}}\mright)
	}}\mright|_{p}
\mright)
\\
&\hspace{1cm}
- \mleft.\frac{\mathrm{d}}{\mathrm{d}t}\mright|_{t=0} \mleft(
	\mleft.{\oversortoftilde{
		\mleft. \mleft( \pi^!\Delta \e^{-t\nu} \mright) \mright|_{p \cdot \e^{t\nu_x}}\mleft(X_{p \cdot \e^{t\nu_x}}\mright)
	}}\mright|_{p}
\mright)
\\
&=
- \mleft.\frac{\mathrm{d}}{\mathrm{d}t}\mright|_{t=0} \Bigl(
	\mathcal{r}_{\e^{-t\nu_x}*}\mleft(X_{p \cdot \e^{t\nu_x}}\mright)
\Bigr)
\\
&\hspace{1cm}
- \mleft.\frac{\mathrm{d}}{\mathrm{d}t}\mright|_{t=0} \mleft(
	\mleft.{\oversortoftilde{
		\mleft. \mleft( \pi^!\Delta \e^{-t\nu} \mright) \mright|_{p \cdot \e^{t\nu_x}}\mleft(X_{p \cdot \e^{t\nu_x}}\mright)
	}}\mright|_{p}
\mright).
\eas
The first summand, $t \mapsto \mathcal{r}_{\e^{-t\nu_x}*}\mleft(X_{p \cdot \e^{t\nu_x}}\mright)$, is a curve with values in $\mathrm{H}_{p}\mathcal{P}$ due to the fact that $X$ is a horizontal vector field. Thus, the velocity of this curve at $t=0$ will also be an element of $\mathrm{H}_p\mathcal{P}$, and the first summand is therefore in the kernel of $A$. Similarly,
\bas
t &\mapsto \eta(t) \coloneqq
\mleft. \mleft( \pi^!\Delta \e^{-t\nu} \mright) \mright|_{p \cdot \e^{t\nu_x}}\mleft(X_{p \cdot \e^{t\nu_x}}\mright)
=
\mleft. \mleft( \Delta \e^{-t\nu} \mright) \mright|_{x}\mleft(\mathrm{D}_{p \cdot \e^{t\nu_x}}\pi \mleft(X_{p \cdot \e^{t\nu_x}}\mright)\mright)
\eas
is a curve with values in $\mathcal{g}_x$, and therefore its fundamental vector field at $p$ is a curve with values in $\mathrm{V}_p\mathcal{P}$. Hence, we can calculate in the sense of vector spaces to derive
\bas
- \mleft.\frac{\mathrm{d}}{\mathrm{d}t}\mright|_{t=0} \widetilde{\eta(t)}_p
&=
\mleft.\frac{\mathrm{d}}{\mathrm{d}t}\mright|_{t=0} \Bigl( \mathrm{D}_{e_x}\Phi_p \bigl(\eta(-t)\bigr) \Bigr)
=
\mathrm{D}_{e_x}\Phi_p \mleft( \mleft.\frac{\mathrm{d}}{\mathrm{d}t}\mright|_{t=0} \eta(-t)\mright)
=
\mleft.\oversortoftilde{\mleft.\frac{\mathrm{d}}{\mathrm{d}t}\mright|_{t=0} \eta(-t)}\mright|_p
\eas
where we substituted $t \leftrightarrow -t$ after the first equality sign. In total we achieve
\bas
A_p\mleft( \mleft[ X, \widetilde{\nu} \mright]_p \mright)
&=
A_p\mleft( \mleft.\oversortoftilde{\mleft.\frac{\mathrm{d}}{\mathrm{d}t}\mright|_{t=0} \eta(-t)}\mright|_p \mright)
=
\mleft( 
	p,
	\mleft.\frac{\mathrm{d}}{\mathrm{d}t}\mright|_{t=0} \Bigl(
		\mleft. \mleft( \Delta \e^{t\nu} \mright) \mright|_{x}\mleft(\mathrm{D}_{p \cdot \e^{-t\nu_x}}\pi \mleft(X_{p \cdot \e^{-t\nu_x}}\mright)\mright)
	\Bigr)
\mright),
\eas
and thus
\bas
A\mleft( \mleft[ X, \widetilde{\nu} \mright] \mright)
&=
\mleft.\frac{\mathrm{d}}{\mathrm{d}t}\mright|_{t=0} \Bigl(
		\mleft( \pi^! \Delta \e^{t\nu} \mright)\mleft(X\mright) \circ r_{\e^{-t\nu}}
	\Bigr).
\eas
If there is an $\omega \in \mathfrak{X}(M)$ such that $\mathrm{D}\pi(X) = \pi^*\omega$, then
\bas
A_p\mleft( \mleft[ X, \widetilde{\nu} \mright]_p \mright)
&=
\mleft( 
	p,
	\mleft.\frac{\mathrm{d}}{\mathrm{d}t}\mright|_{t=0} \Bigl(
		\mleft. \mleft( \Delta \e^{t\nu} \mright) \mright|_{x}\mleft(\omega_{x}\mright)
	\Bigr)
\mright),
\eas
and therefore by Def.\ \ref{def:ConnectionOnLAB}
\bas
A_p\mleft( \mleft[ X, \widetilde{\nu} \mright]_p \mright)
&=
\mleft( 
	p,
	\mleft.\nabla^{\mathcal{G}}_\omega \nu \mright|_x
\mright)
=
\mleft. \pi^*\mleft( \nabla^{\mathcal{G}}_\omega \nu \mright) \mright|_p.
\eas
By the definition of pullback connections we could also write
\bas
A\bigl( \mleft[ X, \widetilde{\nu} \mright] \bigr)
&=
\mleft( \pi^*\nabla^{\mathcal{G}} \mright)_{X}(\pi^*\nu).
\eas

If $\mathrm{H}\mathcal{G}$ is an Ehresmann connection, then recall Remark \ref{rem:baseManifoldHoriz}; this implies for the neutral element section $e$ that
\bas
\Delta e
&\equiv
0,
\eas
and, again, $t \mapsto \mleft( \Delta \e^{t\nu} \mright)_x$ and $t \mapsto \mathrm{D}_{p \cdot \e^{-t\nu_x}}\pi \mleft(X_{p \cdot \e^{-t\nu_x}}\mright)$ are curves with values in the vector spaces $\mathrm{T}^*_xM \otimes \mathcal{g}_x$ and $\mathrm{T}_xM$, respectively. Thus, we can just apply typical Leibniz rules to calculate the derivative of the contraction
\bas
&\mleft.\frac{\mathrm{d}}{\mathrm{d}t}\mright|_{t=0} \Bigl(
		\mleft. \mleft( \Delta \e^{t\nu} \mright) \mright|_{x}\mleft(\mathrm{D}_{p \cdot \e^{-t\nu_x}}\pi \mleft(X_{p \cdot \e^{-t\nu_x}}\mright)\mright)
	\Bigr)
\\
&\hspace{1cm}=
\mleft. \mleft( \mleft.\frac{\mathrm{d}}{\mathrm{d}t}\mright|_{t=0} \Delta \e^{t\nu} \mright) \mright|_{x}\bigl(\mathrm{D}_{p}\pi \mleft(X_{p}\mright)\bigr)
	+ {\underbrace{\mleft. \mleft( \Delta e \mright) \mright|_{x}}_{= 0}}\mleft(\mleft.\frac{\mathrm{d}}{\mathrm{d}t}\mright|_{t=0}\mathrm{D}_{p \cdot \e^{-t\nu_x}}\pi \mleft(X_{p \cdot \e^{-t\nu_x}}\mright)\mright)
\\
&\hspace{1cm}=
\nabla^{\mathcal{G}}_{\mathrm{D}_{p}\pi \mleft(X_{p}\mright)} \nu,
\eas
and as before we conclude
\bas
A\bigl( \mleft[ X, \widetilde{\nu} \mright] \bigr)
&=
\mleft( \pi^*\nabla^{\mathcal{G}} \mright)_{X}(\pi^*\nu).
\eas
\end{proof}

This concludes our discussion about the infinitesimal compatibility condition \eqref{CondSGleichNullLAB} and its integrated version, "$\mathrm{H}\mathcal{G} =$ Ehresmann connection"'. Via Thm.\ \ref{thm:TotMaurerIsMultiplicativeIfSEqual0} and Lemma \ref{lem:YangMillsConnAnEhresmann} we therefore concluded that compatibility condition \eqref{CondSGleichNullLAB} holds if the total Maurer-Cartan form is multiplicative. Let us therefore now turn to compatibility condition \eqref{CondKruemmungmitBLAB}, which we can actually integrate to an equivalent statement.

\begin{theorems}{Generalized Maurer-Cartan equation}{GenMCEq}
Let $\mathcal{G} \stackrel{\pi_{\mathcal{G}}}{\to} M$ be an LGB over a smooth manifold $M$, $\mathrm{H}\mathcal{G}$ an Ehresmann connection on $\mathcal{G}$, and $\zeta \in \Omega^2\mleft( M; \mathcal{g} \mright)$. Then $\mathrm{H}\mathcal{G}$ satisfies the infinitesimal compatibility condition \eqref{CondKruemmungmitBLAB} w.r.t.\ $\zeta$ if and only if the total Maurer-Cartan form $\mu_{\mathcal{G}}^{\mathrm{tot}}$ satisfies
\ba\label{THEGeneralizedMCEq}
\mleft.\mleft(\mathrm{d}^{\pi_{\mathcal{G}}^*\nabla^{\mathcal{G}}} \mu_{\mathcal{G}}^{\mathrm{tot}}
	+ \frac{1}{2} \mleft[ \mu_{\mathcal{G}}^{\mathrm{tot}} \stackrel{\wedge}{,} \mu_{\mathcal{G}}^{\mathrm{tot}} \mright]_{\pi_{\mathcal{G}}^*\mathcal{g}}
	+ \pi_{\mathcal{G}}^! \zeta\mright)\mright|_g
&=
%\sAd_{g^{-1}} \circ \mleft.\mleft(\pi_{\mathcal{G}}^! \zeta\mright)\mright|_g
\mleft( g, \mathrm{Ad}_{g^{-1}} \circ \zeta_x \circ \mleft(\mathrm{D}_g \pi_{\mathcal{G}}, \mathrm{D}_g \pi_{\mathcal{G}}\mright) \mright)
\ea
for all $g \in \mathcal{G}_x$ ($x \in M$). We then say that $\mu_{\mathcal{G}}^{\mathrm{tot}}$ solves the \textbf{generalized Maurer-Cartan equation (w.r.t.\ $\zeta$)}.
\end{theorems}

\begin{remark}\label{BasePointDifficultiesinGenMCEq}
\leavevmode\newline
The structure on the right hand side is again due to denoting the base point in pullback bundles; one may decide to drop the base point for simplicity in the notation. In that case, since the base point is clear by context, one may also just write
\bas
\sAd_{g^{-1}} \circ \mleft.\pi_{\mathcal{G}}^!\zeta\mright|_g
~\text{ or }~
\mathrm{Ad}_{g^{-1}} \circ \mleft.\pi_{\mathcal{G}}^!\zeta\mright|_g
\eas
on the right hand side.
%
%W.r.t.\ a section $\sigma \in \Gamma(\mathcal{G})$ the notation simplifies again to
%\bas
%\mleft.\mleft(\mathrm{d}^{\pi_{\mathcal{G}}^*\nabla^{\mathcal{G}}} \mu_{\mathcal{G}}^{\mathrm{tot}}
	%+ \frac{1}{2} \mleft[ \mu_{\mathcal{G}}^{\mathrm{tot}} \stackrel{\wedge}{,} \mu_{\mathcal{G}}^{\mathrm{tot}} \mright]_{\pi_{\mathcal{G}}^*\mathcal{g}}
	%+ \pi_{\mathcal{G}}^! \zeta\mright)\mright|_\sigma
%&=
%%\sAd_{g^{-1}} \circ \mleft.\mleft(\pi_{\mathcal{G}}^! \zeta\mright)\mright|_g
%\pi_{\mathcal{G}}^!\mleft(\mathrm{Ad}_{\sigma^{-1}} \circ \zeta \mright).
%\eas
\end{remark}

\begin{remarks}{Classical Maurer-Cartan equation recovered}{ClassicalMCEqinGeneralizedOne}
Recall Rem.\ \ref{rem:DarbouxOnCanonFlat}. If $\mathcal{G}$ is a trivial LGB, $\mathrm{H}\mathcal{G}$ its canonical flat connection and $\zeta \equiv 0$, then $\nabla^{\mathcal{G}}$ is the canonical flat connection on the trivial LAB $\mathcal{g}$ (recall Ex.\ \ref{ex:CanonicalFlatGConnection}) and it clearly satisfies the compatibility conditions in Def.\ \ref{def:YangMillsConnection} w.r.t.\ $\zeta \equiv 0$. Furthermore, the generalized Maurer-Cartan equation then reduces to the classical Maurer-Cartan equation.
\end{remarks}

\begin{proof}[Proof of Thm.\ \ref{thm:GenMCEq}]
\leavevmode\newline
The idea of the proof is similar to the proof of a statement about when two multiplicative 2-forms\footnote{See Rem.\ \ref{rem:SimplicialDifferentialStuff} later; multiplicativity is related to closedness w.r.t.\ a differential. However, we will not need this general notion.} are equal, see \textit{e.g.}\ \cite[\S 3, Cor.\ 3.4]{bursztyn2004integration}. That is, we want to discuss under what conditions 
\bas
\mathcal{G} \ni g &\mapsto \omega_g \coloneqq
\mleft.\mleft(\mathrm{d}^{\pi_{\mathcal{G}}^*\nabla^{\mathcal{G}}} \mu_{\mathcal{G}}^{\mathrm{tot}}
	+ \frac{1}{2} \mleft[ \mu_{\mathcal{G}}^{\mathrm{tot}} \stackrel{\wedge}{,} \mu_{\mathcal{G}}^{\mathrm{tot}} \mright]_{\pi_{\mathcal{G}}^*\mathcal{g}}
	+ \pi_{\mathcal{G}}^! \zeta\mright)\mright|_g
	- \mleft( g, \mathrm{Ad}_{g^{-1}} \circ \zeta_x \circ \mleft(\mathrm{D}_g \pi_{\mathcal{G}}, \mathrm{D}_g \pi_{\mathcal{G}}\mright) \mright)
\eas
is constant along the fibres of $\mathcal{G}$, that is, constant along the flows of vertical tangent vectors, the left-invariant vector fields (recall Cor.\ \ref{cor:TLGBAsLGB}). 

$\bullet$ First of all observe that $\omega$ vanishes if at least one tangent vector is vertical. Let $Y \in \mathrm{V}_g\mathcal{G}$ ($g \in \mathcal{G}_x$, $x \in M$), and write $Y = \mleft.\widetilde{\nu_x}\mright|_g$ for a $\nu_x \in \mathcal{g}_x$. Due to $\mathrm{D}\pi_{\mathcal{G}} \mleft( Y \mright) = 0$ it is clear that the contraction of $\pi_{\mathcal{G}}^! \zeta$ with $Y$ is 0. Let $Z \in \mathrm{T}_g\mathcal{G}$ for which we write
\bas
Z 
&\coloneqq
\mleft.\widetilde{\mu_x}\mright|_g
	+ X_g
\eas
for a $\mu_x \in \mathcal{g}_x$ and $X_g \in \mathrm{H}_g\mathcal{G}$;
then by Cor.\ \ref{cor:TotMCFormIsConnectionForm}
\bas
\frac{1}{2} \mleft.\mleft[ \mu_{\mathcal{G}}^{\mathrm{tot}} \stackrel{\wedge}{,} \mu_{\mathcal{G}}^{\mathrm{tot}} \mright]_{\pi_{\mathcal{G}}^*\mathcal{g}}\mright|_g (Y, Z)
&=
\mleft[ \mleft(\mu_{\mathcal{G}}^{\mathrm{tot}}\mright)_g(Y), \mleft(\mu_{\mathcal{G}}^{\mathrm{tot}}\mright)_g(Z) \mright]_{\pi_{\mathcal{G}}^*\mathcal{g}}
=
\mleft( g,
\mleft[ \nu_x, \mu_x \mright]_{\mathcal{g}_x}
\mright).
\eas
Extend $Y$ and $Z$ naturally to vector fields on $\mathcal{G}$ (for simplicity also denoted by $Y$ and $Z$, respectively) by viewing $\nu_x$ and $\mu_x$ as values of sections $\nu$ and $\mu$, respectively, of $\mathcal{g}$ at $x$, and extending $X_g$ (locally) to a horizontal vector field $X$; thence,
\bas
\mleft(\mathrm{d}^{\pi_{\mathcal{G}}^*\nabla^{\mathcal{G}}} \mu_{\mathcal{G}}^{\mathrm{tot}}\mright)_g(Y, Z)
&=
\mleft( \pi_{\mathcal{G}}^*\nabla^{\mathcal{G}} \mright)_{Y_g} \mleft( \mu_{\mathcal{G}}^{\mathrm{tot}}(Z) \mright)
	- \mleft( \pi_{\mathcal{G}}^*\nabla^{\mathcal{G}} \mright)_{Z_g} \mleft( \mu_{\mathcal{G}}^{\mathrm{tot}}(Y) \mright)
	- \mleft(\mu_{\mathcal{G}}^{\mathrm{tot}}\mright)_g\mleft( [Y, Z] \mright)
\\
&=
{\underbrace{\mleft( \pi_{\mathcal{G}}^*\nabla^{\mathcal{G}} \mright)_{\widetilde{\nu}_g} \mleft( \pi^*_{\mathcal{G}}\mu \mright)}
	_{\mathclap{ = \mleft. \pi^*_{\mathcal{G}}\mleft( \nabla^{\mathcal{G}}_{\mathrm{D} \pi_{\mathcal{G}} \mleft( \widetilde{\nu} \mright)} \mu \mright) \mright|_g = 0 }}}
	- \mleft( \pi_{\mathcal{G}}^*\nabla^{\mathcal{G}} \mright)_{\widetilde{\mu}_g + X_g} \mleft( \pi^*_{\mathcal{G}}\nu \mright)
\\
&\hspace{1cm}
	- \mleft(\mu_{\mathcal{G}}^{\mathrm{tot}}\mright)_g\mleft( [\widetilde{\nu}, \widetilde{\mu}] \mright)
	- \mleft(\mu_{\mathcal{G}}^{\mathrm{tot}}\mright)_g\mleft( [\widetilde{\nu}, X] \mright)
\\
&=
- \mleft( \pi_{\mathcal{G}}^*\nabla^{\mathcal{G}} \mright)_{X_g} \mleft( \pi^*_{\mathcal{G}}\nu \mright)
	- \mleft( g, \mleft[ \nu_x, \mu_x \mright]_{\mathcal{g}_x} \mright)
	+ \mleft( \pi^*_{\mathcal{G}}\nabla^{\mathcal{G}} \mright)_{X_g}\mleft( \pi^*_{\mathcal{G}}\nu \mright)
\\
&=
- \mleft( g, \mleft[ \nu_x, \mu_x \mright]_{\mathcal{g}_x} \mright)
\eas
using Rem.\ \ref{rem:FundVecsAreLABActions} (bracket of fundamental/left-invariant vector fields is fundamental/left-invariant) and Lemma \ref{lem:BracketVertHor}. Thus,
\bas
\mleft.\mleft(\mathrm{d}^{\pi_{\mathcal{G}}^*\nabla^{\mathcal{G}}} \mu_{\mathcal{G}}^{\mathrm{tot}}
	+ \frac{1}{2} \mleft[ \mu_{\mathcal{G}}^{\mathrm{tot}} \stackrel{\wedge}{,} \mu_{\mathcal{G}}^{\mathrm{tot}} \mright]_{\pi_{\mathcal{G}}^*\mathcal{g}}\mright)\mright|_g(Y, Z)
&=
0.
\eas
It follows that $\omega$ vanishes if contracted with a vertical tangent vector.

$\bullet$ Therefore we can study $\omega$ just w.r.t.\ tangent vectors complementary to the vertical structure, these do not necessarily need to be elements of $\mathrm{H}\mathcal{G}$. For a given $g \in \mathcal{G}_x$ ($x \in M$) define
\bas
\mathbb{R} &\to \Gamma(\mathcal{G}),\\
t &\mapsto \gamma(t) \coloneqq \sigma \e^{t\nu},
\eas
where $\nu$ is any section of $\mathcal{g}$ and $\sigma \in \Gamma(\mathcal{G})$ (local) with $\sigma_x = g$; pointwise we denote $\gamma(t)$ as a map $M \ni x \mapsto \gamma_x(t)$. By Cor.\ \ref{cor:TLGBAsLGB} the flows of elements of $\mathrm{V}_g\mathcal{G}$ can be precisely described by curves like $\gamma$, so that the derivative w.r.t.\ $t$ of pullbacks of tensors with $\gamma$ characterizes any derivative along a fibre of $\mathcal{G}$. Furthermore, $\gamma(t)$ is by definition a section of $\mathcal{G}$ for all $t$ and thence describes an embedding of $M$ into $\mathcal{G}$. Thus, the $t$-derivatives of terms like $\gamma(t)^!\omega$ describe the derivatives of $\omega$ along fibres of $\mathcal{G}$ contracted with tangent vectors complementary to the vertical structure. Therefore let us calculate $\gamma(t)^!\omega$; we have
\bas
\gamma(t)^!\pi_{\mathcal{G}}^!\zeta
&=
{\underbrace{\bigl( \pi_{\mathcal{G}} \circ \gamma(t) \bigr)^!}_{\equiv \mathds{1}_M^!}}\zeta
=
\zeta
\eas
and under the identification $\gamma^*\pi_{\mathcal{G}}^*\mathcal{g} \cong (\pi_{\mathcal{G}} \circ \gamma)^*\mathcal{g} = \mathds{1}_M^*\mathcal{g} \cong \mathcal{g}$ we get similarly
\bas
\mleft( \gamma(t)^!\mleft( g, \mathrm{Ad}_{g^{-1}}\circ \zeta_{\pi_{\mathcal{G}}(g)} \circ \mathrm{D}_g\pi_{\mathcal{G}} \mright) \mright)_x
&\cong
\mathrm{Ad}_{\gamma^{-1}_x(t)} \circ \zeta_{\pi_{\mathcal{G}}\mleft(\gamma_x(t)\mright)} \circ \mathrm{D}_x\bigl( \pi_{\mathcal{G}} \circ \gamma(t) \bigr)
=
\mathrm{Ad}_{\gamma^{-1}_x(t)} \circ \zeta_{x}
\eas
for all $x \in M$, where $\gamma^{-1}(t)$ is the $\mathcal{G}$-inverse of $\gamma(t)$. It is also a straight-forward calculation to show that 
\bas
\gamma(t)^!\mathrm{d}^{\pi_{\mathcal{G}}^*\nabla^{\mathcal{G}}} \mu_{\mathcal{G}}^{\mathrm{tot}}
&=
\mathrm{d}^{\gamma(t)^*\pi_{\mathcal{G}}^*\nabla^{\mathcal{G}}} \mleft( \gamma(t)^!\mu_{\mathcal{G}}^{\mathrm{tot}}\mright)
=
\mathrm{d}^{\mleft(\pi_{\mathcal{G}}\circ \gamma(t)\mright)^*\nabla^{\mathcal{G}}} \mleft( \gamma(t)^!\mu_{\mathcal{G}}^{\mathrm{tot}}\mright)
=
\mathrm{d}^{\nabla^{\mathcal{G}}} \mleft( \gamma(t)^!\mu_{\mathcal{G}}^{\mathrm{tot}}\mright),
\eas
using the definition of pullback vector bundle connections; alternatively see the explicit calulations in \cite[Appendix, Prop.\ A.1, Eq.\ (A.1)]{My1stpaper} or \cite[Appendix, Prop.\ A.1.1, Eq.\ (A.2)]{MyThesis}, and we have
\bas
&\mleft(\gamma(t)^!\mleft( \frac{1}{2} \mleft[ \mu_{\mathcal{G}}^{\mathrm{tot}} \stackrel{\wedge}{,} \mu_{\mathcal{G}}^{\mathrm{tot}} \mright]_{\pi_{\mathcal{G}}^*\mathcal{g}} \mright)\mright)_x(X,Y)
\\
&\hspace{1cm}=
\mleft[ \mleft(\mu_{\mathcal{G}}^{\mathrm{tot}}\mright)_{\gamma_x(t)}\mleft( \mathrm{D}_x\bigl(\gamma(t)\bigr)(X) \mright), 
\mleft(\mu_{\mathcal{G}}^{\mathrm{tot}}\mright)_{\gamma_x(t)}\mleft( \mathrm{D}_x\bigl(\gamma(t)\bigr)(Y) \mright) \mright]_{\mathcal{g}_{x}}
\\
&\hspace{1cm}=
\mleft[ \mleft(\gamma(t)^!\mu_{\mathcal{G}}^{\mathrm{tot}}\mright)_{x}(X), 
\mleft(\gamma(t)^!\mu_{\mathcal{G}}^{\mathrm{tot}}\mright)_{x}(Y) \mright]_{\mathcal{g}_{x}}
\\
&\hspace{1cm}=
\mleft( \frac{1}{2}
	\mleft[ \gamma(t)^!\mu_{\mathcal{G}}^{\mathrm{tot}} \stackrel{\wedge}{,}
	\gamma(t)^!\mu_{\mathcal{G}}^{\mathrm{tot}} \mright]_{\mathcal{g}}
\mright)_x(X,Y)
\eas
for all $x \in M$ and $X, Y \in \mathrm{T}_xM$. Thus, we have so far
\ba\label{PullbackofMCGeneralCurvPlusExtra}
\gamma(t)^!\omega
&=
\mathrm{d}^{\nabla^{\mathcal{G}}} \mleft( \gamma(t)^!\mu_{\mathcal{G}}^{\mathrm{tot}}\mright)
	+ \frac{1}{2}
	\mleft[ \gamma(t)^!\mu_{\mathcal{G}}^{\mathrm{tot}} \stackrel{\wedge}{,}
	\gamma(t)^!\mu_{\mathcal{G}}^{\mathrm{tot}} \mright]_{\mathcal{g}}
	+ \zeta
	- \mathrm{Ad}_{\gamma^{-1}(t)} \circ \zeta
\nonumber
\\
&=
\mathrm{d}^{\nabla^{\mathcal{G}}} \bigl( \Delta\gamma(t) \bigr)
	+ \frac{1}{2} \mleft[ \Delta\gamma(t) \stackrel{\wedge}{,} \Delta\gamma(t) \mright]_{\mathcal{g}}
	+ \zeta
	- \mathrm{Ad}_{\gamma^{-1}(t)} \circ \zeta.
\ea
By Prop.\ \ref{prop:DarbouxLeibnizRule} we get
\bas
\Delta\gamma(t)
&=
\mathrm{Ad}_{\e^{-t\nu}} \circ \Delta \sigma
	+ \Delta \e^{t\nu},
\eas
and via Lemma \ref{lem:FieldRedefViaLGBSection} we can derive
\bas
&\mleft(\mathrm{d}^{\nabla^{\mathcal{G}}} \mleft(\mathrm{Ad}_{\e^{-t\nu}} \circ \Delta \sigma \mright)\mright)(X, Y)
\\
&\hspace{1cm}=
\nabla^{\mathcal{G}}_X \bigl( \mathrm{Ad}_{\e^{-t\nu}} \mleft( \Delta \sigma(Y)\mright) \bigr)
	- \nabla^{\mathcal{G}}_Y \bigl( \mathrm{Ad}_{\e^{-t\nu}} \mleft( \Delta \sigma(X)\mright) \bigr)
%\\
%&\hspace{2cm}
	- \mathrm{Ad}_{\e^{-t\nu}} \bigl( \Delta \sigma([X, Y])\bigr)
\\
&\hspace{1cm}=
\mathrm{Ad}_{\e^{-t\nu}} \Bigl(
	\nabla^{\mathcal{G}}_X \bigl( \Delta \sigma(Y) \bigr)
	+ \mleft[\mleft(\Delta \e^{-t\nu}\mright)(X), \Delta \sigma(Y) \mright]_{\mathcal{g}}
\\
&\hspace{3.5cm}
	- \nabla^{\mathcal{G}}_Y \bigl( \Delta \sigma(X) \bigr)
	- \mleft[\mleft(\Delta \e^{-t\nu}\mright)(Y), \Delta \sigma(X) \mright]_{\mathcal{g}}
\\
&\hspace{3.5cm}
	-\Delta \sigma([X, Y])
\Bigr)
\\
&\hspace{1cm}=
\mathrm{Ad}_{\e^{-t\nu}} \mleft(
	\mleft(\mathrm{d}^{\nabla^{\mathcal{G}}} \Delta \sigma\mright)(X, Y)
	+ \mleft[\mleft(\Delta \e^{-t\nu}\mright)(X), \Delta \sigma(Y) \mright]_{\mathcal{g}}
	- \mleft[\mleft(\Delta \e^{-t\nu}\mright)(Y), \Delta \sigma(X) \mright]_{\mathcal{g}}
\mright)
\\
&\hspace{1cm}=
\mleft(\mathrm{Ad}_{\e^{-t\nu}} \circ \mleft(
	\mathrm{d}^{\nabla^{\mathcal{G}}} \Delta \sigma
	+ \mleft[\Delta \e^{-t\nu} \stackrel{\wedge}{,} \Delta \sigma \mright]_{\mathcal{g}}
\mright)\mright)(X, Y)
\eas
for all $X, Y \in \mathfrak{X}(M)$, hence,
\bas
\mleft. \frac{\mathrm{d}}{\mathrm{d}t} \mright|_{t=0}\mleft(
	\mathrm{d}^{\nabla^{\mathcal{G}}} \Delta \gamma(t)
\mright)
&=
\mleft[ \mathrm{d}^{\nabla^{\mathcal{G}}} \Delta \sigma, \nu \mright]_{\mathcal{g}}
	- \mleft[\Delta \sigma \stackrel{\wedge}{,} \nabla^{\mathcal{G}}\nu \mright]_{\mathcal{g}}
	+ \mathrm{d}^{\nabla^{\mathcal{G}}} \nabla^{\mathcal{G}}\nu
\\
&=
\mleft[ \mathrm{d}^{\nabla^{\mathcal{G}}} \Delta \sigma, \nu \mright]_{\mathcal{g}}
	- \mleft[\Delta \sigma \stackrel{\wedge}{,} \nabla^{\mathcal{G}}\nu \mright]_{\mathcal{g}}
	+ R_{\nabla^{\mathcal{G}}} (\cdot, \cdot) \nu
\eas
making use of that this calculation is pointwise at $x \in M$ just a standard derivative in the vector space $\mathcal{g}_x$ and that $\Delta e \equiv 0$ by Remark \ref{rem:baseManifoldHoriz}; we also used Prop.\ \ref{prop:GradedExtensionPlusAntiSymm}. In a similar fashion,
\bas
\mleft.\frac{\mathrm{d}}{\mathrm{d}t}\mright|_{t=0}\mleft[ \Delta\gamma(t) \stackrel{\wedge}{,} \Delta\gamma(t) \mright]_{\mathcal{g}}
&=
2\mleft[ \mleft[\Delta\sigma, \nu \mright]_{\mathcal{g}} \stackrel{\wedge}{,} \Delta\sigma \mright]_{\mathcal{g}}
	+ 2\mleft[ \Delta \sigma \stackrel{\wedge}{,} \nabla^{\mathcal{G}}\nu \mright]_{\mathcal{g}},
\eas
which we can rewrite by the Jacobi identity
\bas
\mleft[ \mleft[\Delta\sigma, \nu \mright]_{\mathcal{g}} \stackrel{\wedge}{,} \Delta\sigma \mright]_{\mathcal{g}}(X, Y)
&=
\mleft[ \mleft[\Delta\sigma(X), \nu \mright]_{\mathcal{g}} , \Delta\sigma(Y) \mright]_{\mathcal{g}}
	- \mleft[ \mleft[\Delta\sigma(Y), \nu \mright]_{\mathcal{g}} , \Delta\sigma(X) \mright]_{\mathcal{g}}
\\
&=
\mleft[ \mleft[\Delta\sigma(X), \Delta\sigma(Y) \mright]_{\mathcal{g}} , \nu \mright]_{\mathcal{g}}
\\
&=
\mleft[ \frac{1}{2} \mleft[\Delta\sigma \stackrel{\wedge}{,} \Delta\sigma \mright]_{\mathcal{g}} , \nu \mright]_{\mathcal{g}}(X, Y)
\eas
for all $X, Y \in \mathfrak{X}(M)$, and we also trivially have
\bas
\mleft.\frac{\mathrm{d}}{\mathrm{d}t}\mright|_{t=0} \mathrm{Ad}_{\gamma^{-1}(t)} \circ \zeta
&=
\mleft.\frac{\mathrm{d}}{\mathrm{d}t}\mright|_{t=0} \mathrm{Ad}_{\e^{-t\nu}} \circ \mathrm{Ad}_{\sigma^{-1}} \circ \zeta
=
\mleft[ \mathrm{Ad}_{\sigma^{-1}} \circ \zeta, \nu \mright]_{\mathcal{g}}.
\eas
Collecting everything we derive
\ba\label{PullbackofMCEqGivesCurv}
\mleft.\frac{\mathrm{d}}{\mathrm{d}t}\mright|_{t=0} \gamma(t)^!\omega
&=
\mleft[ \mathrm{d}^{\nabla^{\mathcal{G}}} \Delta \sigma, \nu \mright]_{\mathcal{g}}
	- \mleft[\Delta \sigma \stackrel{\wedge}{,} \nabla^{\mathcal{G}}\nu \mright]_{\mathcal{g}}
	+ R_{\nabla^{\mathcal{G}}} (\cdot, \cdot) \nu
\nonumber
\\
&\hspace{1cm}
	+ \mleft[ \frac{1}{2} \mleft[\Delta\sigma \stackrel{\wedge}{,} \Delta\sigma \mright]_{\mathcal{g}} , \nu \mright]_{\mathcal{g}}
	+ \mleft[ \Delta \sigma \stackrel{\wedge}{,} \nabla^{\mathcal{G}}\nu \mright]_{\mathcal{g}}
	- \mleft[ \mathrm{Ad}_{\sigma^{-1}} \circ \zeta, \nu \mright]_{\mathcal{g}}
\nonumber
\\
&=
R_{\nabla^{\mathcal{G}}} (\cdot, \cdot) \nu
	+ \mleft[ 
		\mathrm{d}^{\nabla^{\mathcal{G}}} \Delta \sigma
		+ \frac{1}{2} \mleft[\Delta\sigma \stackrel{\wedge}{,} \Delta\sigma \mright]_{\mathcal{g}}
		- \mathrm{Ad}_{\sigma^{-1}} \circ \zeta
		, \nu \mright]_{\mathcal{g}}
\ea
for all $\nu \in \Gamma(\mathcal{g})$ and $\sigma \in \Gamma(\mathcal{G})$.

$\bullet$ On the one hand, if $\mu_{\mathcal{G}}^{\mathrm{tot}}$ satisfies Eq.\ \eqref{THEGeneralizedMCEq}, then $\omega \equiv 0$. As previously mentioned, we have $\Delta e \equiv 0$, so that we then get in total for Eq.\ \eqref{PullbackofMCEqGivesCurv} w.r.t.\ $\sigma \coloneqq e$ that
\bas
R_{\nabla^{\mathcal{G}}} (\cdot, \cdot) \nu
&=
\mleft[ \zeta, \nu \mright]_{\mathcal{g}}
\eas
for all $\nu \in \Gamma(\mathcal{g})$, which is the infinitesimal compatibility condition \eqref{CondKruemmungmitBLAB}.

$\bullet$ On the other hand, if $\mathrm{H}\mathcal{G}$ satisfies the infinitesimal compatibility condition \eqref{CondKruemmungmitBLAB}, $R_{\nabla^{\mathcal{G}}} (\cdot, \cdot) \nu = \mleft[ \zeta, \cdot \mright]_{\mathcal{g}}$, then $\mathrm{H}\mathcal{G}$ is an infinitesimal Yang-Mills connection by Lemma \ref{lem:YangMillsConnAnEhresmann}, $\nabla^{\mathcal{G}} = \nabla^{\mathrm{YM}}$, and we know by Eq.\ \eqref{AlmostGeneralizedMaurerCartanEquation} (Lemma \ref{lem:FieldRedefViaLGBSection}) that 
\bas
\mleft[ \mathrm{d}^{\nabla^{\mathrm{YM}}} \Delta \sigma + \frac{1}{2} \mleft[ \Delta \sigma \stackrel{\wedge}{,} \Delta\sigma \mright]_{\mathcal{g}} - \mathrm{Ad}_{\sigma^{-1}} \circ \zeta, ~ \nu \mright]_{\mathcal{g}}
&=
- \mleft[ \zeta, \nu \mright]_{\mathcal{g}},
\eas
so that Eq.\ \eqref{PullbackofMCEqGivesCurv} then has the form
\bas
\mleft.\frac{\mathrm{d}}{\mathrm{d}t}\mright|_{t=0} \gamma(t)^!\omega
&=
R_{\nabla^{\mathcal{G}}} (\cdot, \cdot) \nu
	- \mleft[ \zeta, \nu \mright]_{\mathcal{g}}
\stackrel{\eqref{CondKruemmungmitBLAB}}{=}
0
\eas
for all $\nu \in \Gamma(\mathcal{g})$ and $\sigma \in \Gamma(\mathcal{G})$.
As aforementioned in the discussion around the definition of $\gamma$, also recall that $\omega$ vanishes if contracted with a vertical vector, we can conclude that $\omega$ is constant along the fibres of $\mathcal{G}$ which implies
\bas
\omega
=
\pi_{\mathcal{G}}^!\xi
\eas
for a $\xi \in \Omega^2(M;\mathcal{g})$, especially 
\bas
e^!\omega 
&=
\mleft(\pi_{\mathcal{G}}\circ e\mright)^!\xi
=
\xi.
\eas
But we already have calculated this in Eq.\ \eqref{PullbackofMCGeneralCurvPlusExtra}, set $t=0$ and $\sigma \coloneqq e$, then also again by $\Delta e = 0$ we have
\bas
\xi
&=
e^!\omega
\stackrel{\eqref{PullbackofMCGeneralCurvPlusExtra}}{=}
\zeta
	- \zeta
=
0.
\eas
Finally we get
\bas
\omega
&=
\pi_{\mathcal{G}}^!\xi
=
0,
\eas
which finishes the proof.
\end{proof}

Trivially, this theorem is an extension of Eq.\ \eqref{AlmostGeneralizedMaurerCartanEquation} (Lemma \ref{lem:FieldRedefViaLGBSection}).

\begin{corollaries}{Pullback of generalized Maurer-Cartan equation}{PullbackOfMCSupperEquation}
Let $\mathcal{G} \stackrel{\pi_{\mathcal{G}}}{\to} M$ be an LGB over a smooth manifold $M$, $\mathrm{H}\mathcal{G}$ an Ehresmann connection on $\mathcal{G}$ so that the total Maurer-Cartan form $\mu_{\mathcal{G}}^{\mathrm{tot}}$ satisfies the generalized Maurer-Cartan equation \eqref{THEGeneralizedMCEq} w.r.t.\ a $\zeta \in \Omega^2\mleft( M; \mathcal{g} \mright)$. Then
\bas
\mathrm{d}^{\nabla^{\mathrm{YM}}} \Delta \sigma 
	+ \frac{1}{2} \mleft[ \Delta \sigma \stackrel{\wedge}{,} \Delta\sigma \mright]_{\mathcal{g}} 
	+ \zeta 
&=
\mathrm{Ad}_{\sigma^{-1}} \circ \zeta
\eas
for all $\sigma \in \Gamma(\mathcal{G})$.
\end{corollaries}

\begin{proof}
\leavevmode\newline
Keeping the same notation as in the proof of Thm.\ \ref{thm:GenMCEq}, we have derived Eq.\ \eqref{PullbackofMCGeneralCurvPlusExtra}, and due to Thm.\ \ref{thm:GenMCEq} we get $\omega = 0$; altogether Eq.\ \eqref{PullbackofMCGeneralCurvPlusExtra} at $t=0$ gives then
\bas
0
&=
\mathrm{d}^{\nabla^{\mathrm{YM}}} \Delta\sigma
	+ \frac{1}{2} \mleft[ \Delta\sigma \stackrel{\wedge}{,} \Delta\sigma \mright]_{\mathcal{g}}
	+ \zeta
	- \mathrm{Ad}_{\sigma^{-1}} \circ \zeta.
\eas
\end{proof}

Thus, we can finally define the connection we need on the LGB.

\begin{definitions}{Yang-Mills connection}{NowReallyYangMillsConnectio}
Let $\mathcal{G} \to M$ be an LGB over a smooth manifold $M$, and $\mathrm{H}\mathcal{G}$ be a horizontal distribution of $\mathcal{G}$. Then we say that $\mathrm{H}\mathcal{G}$ is a \textbf{Yang-Mills connection (w.r.t.\ a $\zeta \in \Omega^2(M; \mathcal{g})$)}, if it satisfies the \textbf{compatibility conditions}:
\begin{enumerate}
	\item $\mathrm{H}\mathcal{G}$ is an Ehresmann connection,
	\item the total Maurer-Cartan form $\mu_{\mathcal{G}}^{\mathrm{tot}}$ satisfies the generalized Maurer-Cartan equation (w.r.t.\ a $\zeta \in \Omega^2(M; \mathcal{g})$).
\end{enumerate}
By Thm.\ \ref{thm:TotMaurerIsMultiplicativeIfSEqual0} we can equivalently define $\mathrm{H}\mathcal{G}$ as a Yang-Mills connection if
\begin{enumerate}
	\item $\mu_{\mathcal{G}}^{\mathrm{tot}}$ is multiplicative,
	\item $\mu_{\mathcal{G}}^{\mathrm{tot}}$ satisfies the generalized Maurer-Cartan equation (w.r.t.\ a $\zeta \in \Omega^2(M; \mathcal{g})$).
\end{enumerate}

We then label $\mu_{\mathcal{G}}^{\mathrm{tot}}$ as a Yang-Mills connection, too.
\end{definitions}

\begin{remarks}{Yang-Mills connections integrate infinitesimal Yang-Mills connections}{YangMillsEqualsInfYM}
By Lemma \ref{lem:YangMillsConnAnEhresmann} and Thm.\ \ref{thm:GenMCEq}, every Yang-Mills connection is also an infinitesimal Yang-Mills connections, \textit{i.e.}\ the infinitesimal compatibility conditions are also satisfied.
\end{remarks}

\begin{remarks}{Relation to simplicial differential}{SimplicialDifferentialStuff}
In fact, Def.\ \ref{def:MultiplicativeFormsDef} comes from a simplicial differential (w.r.t.\ points on $\mathcal{G}$) defined in \cite[beginning of \S 1.2]{crainic2003differentiable}. We do not have time to introduce it completely, but we give a short sketch about its definitions, also following the style of \cite[appendix]{FernandesMarcutMultiplicativeForms}. On degree one the cited differential is a map
\bas
\Omega^\bullet\mleft(\mathcal{G}; \pi_{\mathcal{G}}^*\mathcal{g}\mright)
&\to
\Omega^\bullet\mleft(\mathcal{G} * \mathcal{G}; \pi_{\mathcal{G}}^*\mathcal{g}\mright),
\\
\omega
&\mapsto
\delta\omega,
\eas
with
\bas
\mleft(\delta\omega\mright)_{(g,q)}
&\coloneqq
\sAd_{q^{-1}}\circ\mleft(\mathrm{pr}_1^!\omega\mright)_{(g,q)} 
	+ \mleft(\mathrm{pr}_2^!\omega\mright)_{(g, q)}
	- \mleft( \Phi^!\omega \mright)_{(g,q)}
\eas
for all $(g, q) \in \mathcal{G} * \mathcal{G}$, where $\Phi: \mathcal{G}*\mathcal{G}\to \mathcal{G}$ is the $\mathcal{G}$-multiplication, and $\mathrm{pr}_i: \mathcal{G}*\mathcal{G} \to \mathcal{G}$ are the canonical projections onto the $i$-th component ($i \in \{1,2\}$); $\pi_{\mathcal{G}}$ is canonically extended to $\mathcal{G} \times \mathcal{G}$. Recall Def.\ \ref{def:MultiplicativeFormsDef} and its Rem.\ \ref{sloppynotationformultiplicativity}, $\omega$ can be defined to be multiplicative if $\delta\omega = 0$, \textit{i.e.}\ $\omega$ is closed w.r.t.\ $\delta$. 

On degree 0 $\delta$ is a map $\Omega^\bullet(M; \mathcal{g}) \to \Omega^\bullet\mleft(\mathcal{G}; \pi^{*}_{\mathcal{g}}\mright)$ given for $\xi \in \Omega^\bullet(M; \mathcal{g})$ by
\bas
\mleft(\delta\xi\mright)_g
\coloneqq
\bigl( g, \mathrm{Ad}_{g^{-1}} \circ \xi_x \circ {\underbrace{\mleft(\mathrm{D}_g \pi_{\mathcal{G}}, \dotsc, \mathrm{D}_g \pi_{\mathcal{G}}\mright)}_{ \text{times the degree of $\xi$} }} \bigr)
	- \mleft.\mleft(\pi_{\mathcal{G}}^! \xi\mright)\mright|_g
\eas
for all $g\in \mathcal{G}$. Then observe that we can rewrite the generalized Mauer-Cartan equation to
\bas
F_{\mathcal{G}}
&=
\delta \zeta,
\eas
where
\bas
F_{\mathcal{G}}
&=
\mathrm{d}^{\pi_{\mathcal{G}}^*\nabla^{\mathcal{G}}} \mu_{\mathcal{G}}^{\mathrm{tot}}
	+ \frac{1}{2} \mleft[ \mu_{\mathcal{G}}^{\mathrm{tot}} \stackrel{\wedge}{,} \mu_{\mathcal{G}}^{\mathrm{tot}} \mright]_{\pi_{\mathcal{G}}^*\mathcal{g}}
\in \Omega^2\mleft( \mathcal{G}; \pi_{\mathcal{G}}^*\mathcal{g} \mright).
\eas
It is straight-forward to check that $\delta F_{\mathcal{G}} = 0$ (see \textit{e.g.}\ \cite[\S 4.6, Thm.\ 4.27, Eq.\ (53)]{LAURENTGENGOUXStienonXuMultiplicativeForms} or \cite[\S 2.5, Prop.\ 2.22]{FernandesMarcutMultiplicativeForms}) by making use of that $\mathrm{H}\mathcal{G}$ is an Ehresmann connection. In total we concluded that compatibility condition \eqref{CondSGleichNullLAB} is implied, if $\mu_{\mathcal{G}}^{\mathrm{tot}}$ is multiplicative, which also implies that $F_{\mathcal{G}}$ is multiplicative and thus $\delta$-closed; additionally with compatibility condition \eqref{CondKruemmungmitBLAB} we achieve $\delta$-exactness of $F_{\mathcal{G}}$. This is no wonder, the compatibility conditions already tell us by definition that the connection-1-form of $\nabla$ is closed w.r.t.\ the Chevalley-Eilenberg complex, which also implies the same for the curvature, however the curvature has to be exact by \eqref{CondKruemmungmitBLAB}. We depict that as $\delta^{\mathrm{CE}}\nabla^{\mathcal{G}} = 0$ and $R_{\nabla^{\mathcal{G}}} = \delta^{\mathrm{CE}}\zeta$; a summary is shown in Table \ref{tab:CompatibilityConditionsOnLGBAndItsLAB}.
\end{remarks}

\begin{table}[htbp]
	\centering
		\caption{Infinitesimal compatibility conditions on the LAB $\mathcal{g}$} and integrated compatibility conditions on the overlying LGB $\mathcal{G}$
		\begin{tabular}{|llll|}
		\hline
			%\multicolumn{2}{l}{Infinitesimal compatibility condition} 
			& \multicolumn{1}{c}{$\mathcal{g}$} && \multicolumn{1}{c}{$\mathcal{G}$} \\
			\hline
			\textbf{Closedness}
			%$\nabla^{\mathcal{G}}\mleft( \mleft[ \cdot, \cdot \mright]_{\mathcal{g}} \mright) = 0$
	& $\delta^{\mathrm{CE}}\nabla^{\mathcal{G}} = 0$ $\mleft(\Rightarrow \delta^{\mathrm{CE}}R_{\nabla^{\mathcal{G}}} = 0\mright)$
	&& $\delta\mu_{\mathcal{G}}^{\mathrm{tot}} = 0$ $\mleft(\Rightarrow \delta F_{\mathcal{G}} = 0\mright)$
	%\\
	%& \hspace{0.2cm} $\Rightarrow \delta^{\mathrm{CE}}R_{\nabla^{\mathcal{G}}} = 0$
	%&& \hspace{0.2cm} $\Rightarrow \delta F_{\mathcal{G}} = 0$
	\\
	\textbf{Exactness}
			%\hspace{0.2cm} \& $R_{\nabla^{\mathcal{G}}} = \mathrm{ad} \circ \zeta$
		& $R_{\nabla^{\mathcal{G}}} = \delta^{\mathrm{CE}}\zeta$
		&& $F_{\mathcal{G}} = \delta \zeta$
		\\
		\hline
		\end{tabular}
	\label{tab:CompatibilityConditionsOnLGBAndItsLAB}
\end{table}

To conclude this subsubsection, let us provide canonical examples of Yang-Mills connections; we will make use of Thm.\ \ref{thm:GenMCEq} in order to prove that the generalized Maurer-Cartan equation is satisfied.

\begin{examples}{(Pre-)Classical Yang-Mills connection}{ClassicalYangMillsConnection}
Let us look at a classical principal bundle $P$ with trivial LGB $\mathcal{G} = M \times G$, $G$ a Lie group; recall Ex.\ \ref{ex:TheCLassicalPrincAsEx} and \ref{ex:OurConnectionIsReallyMoreGeneral}. The canonical flat connection $\mathrm{H}\mathcal{G}$ on $\mathcal{G}$ induces the canonical flat connection $\nabla^{\mathcal{G}}$ on its LAB $\mathcal{g} = M \times \mathfrak{g}$, recall Ex.\ \ref{ex:CanonicalFlatGConnection}. Thus,
\bas
R_{\nabla^{\mathcal{G}}} &= 0
\eas
and so we can set $\zeta \equiv 0$. By Ex.\ \ref{ex:OurConnectionIsReallyMoreGeneral} (viewing $\mathcal{G}$ also as a principal bundle equipped with the same horizontal distribution) we know that $\mathrm{H}\mathcal{G}$ is an Ehresmann connection, so that its total Maurer-Cartan form is multiplicative.

We therefore call flat $\nabla^{\mathrm{cYM}} \coloneqq \nabla^{\mathcal{G}}$ a \textbf{classical Yang-Mills connection}, regardless of whether or not $\mathcal{G}$ is trivial.

However, observe that $\zeta$ can be centre-valued and non-zero while $\nabla^{\mathcal{G}}$ is still flat. In such an occasion we speak of a \textbf{pre-classical Yang-Mills connection}, denoted by $\nabla^{\mathrm{pYM}}$.
\end{examples}

\begin{examples}{Associated connections of inner group bundles are Yang-Mills connections}{OurVeryImportantExample}
Recall Ex.\ \ref{ex:AssociatedLGBsAndTheirCanonicalConnection}, its setup and notation. We have discussed a canonical connection on associated LGBs $\mathcal{H} = P \times_\psi H$, called associated connection which is an Ehresmann connection. 

As it is well-known, the curvature measures the holonomy, that is, the parallel transport over closed (contractible) curves. Together with Eq.\ \eqref{CondKruemmungmitBLAB}, the parallel transport over a closed (contractible) curve $\alpha$ should act as conjugation with some group element. Let us see whether this holds here; there is a unique $g \in G$ for $p \in P$ such that
\bas
\mathrm{PT}^{P}_\alpha(p)
&=
p \cdot g,
\eas
and thus for $[p, h] \in \mathcal{H}$
\bas
\mathrm{PT}_\alpha^\mathcal{H}\bigl( [p, h] \bigr)
&=
\mleft[ p \cdot g, h \mright]
=
\mleft[ p, \psi_g(h) \mright].
\eas
Thus, if $\mathcal{H} = c_G(P)$, the inner group bundle of Ex.\ \ref{ex:InnerLGBs}, we get
\bas
\mathrm{PT}_\alpha^\mathcal{H}\bigl( [p, h] \bigr)
&=
\mleft[ p, ghg^{-1} \mright]
=
\mleft[p, g\mright] \cdot
\mleft[p, h\mright] \cdot
\mleft[p, g^{-1}\mright],
\eas
the conjugation of $[p, h]$ with $[p, g]$ in $c_G(P)$. It follows that there is a $\zeta \in \Omega^2(M; \mathcal{g})$ such that Eq.\ \eqref{CondKruemmungmitBLAB} is satisfied. This also implies that the associated connection of an inner group bundle is a Yang-Mills connection. We will not prove this approach, instead the next approach provides an alternative proof with an explicit expression for $\zeta$.
\end{examples}

\begin{examples}{Yang-Mills connections on inner group bundles: Alternative point of view and explanation}{NablaYMAsAdjointConnection}
Alternatively one can prove that the total Maurer-Cartan form solves the generalized Maurer-Cartan equation in Ex.\ \ref{ex:OurVeryImportantExample} by using the aforementioned relationship of the compatibility conditions with Mackenzie's study about extending Lie algebroids via LABs; recall Remark \ref{MackenziesStuffRelation}. The following is an information for the experienced reader: The LAB of $c_G(P)$ is given by the adjoint bundle $\mathrm{Ad}(P)$; recall Ex.\ \ref{ex:AssociatedLABsFromAssocLGBs}. $\mathrm{Ad}(P)$ is the kernel of the Atiyah sequence (see \textit{e.g.}\ \cite[\S 3.2, page 90ff.]{mackenzieGeneralTheory}), a short exact sequence of Lie algebroids over $M$
\begin{center}
	\begin{tikzcd}
		0 \arrow{r} & \mathrm{Ad}(P) \arrow{r} & \mathrm{At}(P) \arrow{r} & \mathrm{T}M \arrow{r} & 0,
	\end{tikzcd}
\end{center}
where $\mathrm{At}(P)$ is the Atiyah bundle of $P$. A connection on $P$ in the "classical" sense (that is, related to a canonical flat connection on $M \times G$ in our sense) has a 1:1 correspondence to a splitting $\chi: \mathrm{T}M \to \mathrm{AT}(P)$ to that sequence, \textit{i.e.}\ a section of the right non-trivial arrow. We can then construct a vector bundle connection $\nabla^{\mathrm{YM}}$ on the adjoint bundle by using the Lie algebroid bracket $\mleft[ \cdot, \cdot \mright]_{\mathrm{At}(P)}$ on $\mathrm{At}(P)$, that is,
\bas
\nabla^{\mathrm{YM}}_X \nu
&=
\mleft[ \chi(X), \nu \mright]_{\mathrm{At}(P)}
\eas
for all $X \in \mathfrak{X}(M)$ and $\nu \in \Gamma(\mathrm{Ad}(P))$. It follows by straight-forward calculations as in \cite[\S 7.3, Prop.\ 7.3.2 and Lemma 7.3.3, page 278]{mackenzieGeneralTheory} that $\nabla^{\mathrm{YM}}$ is indeed a Yang-Mills connection with $\zeta \in \Omega^2(M;\mathrm{Ad}(P))$ for example given by 
\bas
\zeta(Y, Z)
&\coloneqq
\mleft[ \chi(Y), \chi(Z) \mright]_{\mathrm{At}(P)} - \chi([Y, Z])
\eas
for all $Y, Z \in \mathfrak{X}(M)$. In other words, $\nabla^{\mathrm{YM}}$ is the \textbf{adjoint connection} induced by a connection on $P$ (see for example \cite[\S 5.3, especially Prop.\ 5.3.13, page 199]{mackenzieGeneralTheory}). Thence, by the discussion in \cite[\S 5.9, page 289ff.]{Hamilton}\ $\nabla^{\mathrm{YM}}$ corresponds to the natural parallel transport $\mathrm{PT}^{\mathrm{Ad}(P)}$ on $\mathrm{Ad}(P)$ (omitting the notation of the corresponding curve) given by
\bas
\mathrm{PT}^{\mathrm{Ad}(P)}\bigl( [p, v] \bigr)
&\coloneqq
\mleft[ \mathrm{PT}^{P}(p), v \mright]
\eas
for all $[p, v] \in \mathrm{Ad}(P)$. This is clearly just the infinitesimal parallel transport inherited by $\mathrm{PT}^{c_G(P)}$, so that we can conclude that $\nabla^{\mathrm{YM}}$ corresponds to the Yang-Mills connection on $c_G(P)$ in sense of Def.\ \ref{def:ConnectionOnLAB} (also recall Rem.\ \ref{PTLABComingFromPTLGB}).

This proof gives a certain interpretation of $\nabla^{\mathrm{YM}}$ and $\zeta$: In the case of $\mathcal{G}$ being the inner group bundle of a typical gauge theory related to a principal $G$-bundle $P$, one can view $\nabla^{\mathrm{YM}}$ as the adjoint connection on the adjoint bundle, induced by $\chi$ a classical connection on $P$, this implies by Eq.\ \eqref{CondKruemmungmitBLAB} that $\zeta$ is the field strength related to $\chi$ (as a field strength with values in the adjoint bundle). Thus one may say in this case that the generalised sense of connection as in Def.\ \ref{def:GaugeBosonsOnLGBPrincies} is modified by a classical connection from a classical principal bundle, thence also the label as Yang-Mills connection to reflect its relation to the classical theory.
\end{examples}

\subsubsection{Field strength related to Yang-Mills connections}

Let us now introduce and discuss the field strength.

\begin{definitions}{(Generalized) Field strength}{NewFieldStrength}
Let $\mathcal{G} \to M$ be an LGB over a smooth manifold $M$ and $\mathcal{P} \stackrel{\pi}{\to} M$ a principal $\mathcal{G}$-bundle, also let $\mathrm{H}\mathcal{G}$ be a Yang-Mills connection on $\mathcal{G}$ (w.r.t.\ a $\zeta \in \Omega^2(M; \mathcal{g})$) and $A \in \Omega^1(\mathcal{P}; \pi^*\mathcal{g})$ be a connection 1-form on $\mathcal{P}$. Then we define the \textbf{(generalized) curvature} or \textbf{(generalized) field strength $F$ (of $A$)} as an element of $\Omega^2\mleft( \mathcal{P}; \pi^*\mathcal{g} \mright)$ by
\bas
F
&\coloneqq
\mathrm{d}^{\pi^*\nabla^{\mathrm{YM}}} A \circ \mleft( \pi^{\mathrm{H}\mathcal{P}}, \pi^{\mathrm{H}\mathcal{P}} \mright)
	+ \pi^!\zeta,
\eas
where $\pi^{\mathrm{H}\mathcal{P}}: \mathrm{T}\mathcal{P} \to \mathrm{H}\mathcal{P}$ is the canonical projection onto the associated Ehresmann connection $\mathrm{H}\mathcal{P}$ on $\mathcal{P}$; that is,
\bas
F(X, Y)
&=
\mathrm{d}^{\pi^*\nabla^{\mathrm{YM}}} A \mleft( \pi^{\mathrm{H}\mathcal{P}}(X), \pi^{\mathrm{H}\mathcal{P}}(Y) \mright)
	+ \mleft(\pi^*\zeta\mright) \bigl( \mathrm{D}\pi(X), \mathrm{D}\pi(Y) \bigr)
\eas
for all $X, Y \in \mathfrak{X}(\mathcal{P})$.
\end{definitions}

Of course, we expect certain properties, naturally generalized by our previous discussions:

\begin{propositions}{Properties of the generalized field strength}{NewFieldStrengthWithCoolProps}
Let $\mathcal{G} \to M$ be an LGB over a smooth manifold $M$ and $\mathcal{P} \stackrel{\pi}{\to} M$ a principal $\mathcal{G}$-bundle, also let $\mathrm{H}\mathcal{G}$ be a Yang-Mills connection on $\mathcal{G}$ (w.r.t.\ a $\zeta \in \Omega^2(M; \mathcal{g})$) and $A \in \Omega^1(\mathcal{P}; \pi^*\mathcal{g})$ be a connection 1-form on $\mathcal{P}$. Then we have the following properties of the field strength:
\begin{itemize}
	\item \textbf{(Form of type $\sAd$)}
	\bas
	\mathcal{r}_{\sigma}^!F
	&=
	\sAd_{\sigma^{-1}} \circ F
	\eas
	for all (local) $\sigma \in \Gamma(\mathcal{G})$.
	\item \textbf{(Horizontal form)}
	\newline
	For $X, Y \in \mathrm{T}_p\mathcal{P}$ ($p \in \mathcal{P}$) we have
	\bas
	F(X, Y)
	&=
	0
	\eas
	if either of $X$ and $Y$ is vertical.
\end{itemize}
\end{propositions}

However, despite the easy proof in the classical theory, the proof of the first property in Prop.\ \ref{prop:NewFieldStrengthWithCoolProps} will be rather involved; most of the proof is straight-forward calculation, and then use our discussion about Yang-Mills connections; the quintessence will be about using Lemma \ref{lem:BracketVertHor} (for first order terms w.r.t.\ $\sigma$) and Cor.\ \ref{cor:PullbackOfMCSupperEquation} (for second order terms w.r.t.\ $\sigma$).
%Now we can finally prove Prop.\ \ref{prop:NewFieldStrengthWithCoolProps}.

\begin{proof}[Proof of Prop.\ \ref{prop:NewFieldStrengthWithCoolProps}]
\leavevmode\newline
For this proof we will neglect caring for the base point information in the involved pullback bundles since the base point information is clear by context; recall Subsection \ref{BasicNotations}. In that sense we also have $\sAd_\sigma \cong \pi^*\mathrm{Ad}_\sigma \cong \mathrm{Ad}_\sigma$ for all $\sigma \in \Gamma(\mathcal{G})$ \textit{etc.}; this will simplify the following notation a bit.

$\bullet$ The second bullet point quickly follows by construction: The first summand of $F$ (recall Def.\ \ref{def:NewFieldStrength}) is clearly zero if either of $X$ and $Y$ is vertical; similar holds for the second summand because the vertical bundle is the kernel of $\mathrm{D}\pi$.

$\bullet$ Thence, let us focus on proving the first bullet point, and denote with $\pi_h: \mathrm{T}\mathcal{P} \to \mathrm{H}\mathcal{P}$ the canonical projection onto the associated Ehresmann connection $\mathrm{H}\mathcal{P}$ on $\mathcal{P}$, so that we write
\bas
F
&=
\mathrm{d}^{\pi^*\nabla^{\mathrm{YM}}} A \circ \mleft( \pi_h, \pi_h \mright)
	+ \pi^!\zeta.
\eas
We have for $\sigma \in \Gamma(\mathcal{G})$
\bas
\mleft(\mathcal{r}_\sigma^!\pi^!\zeta\mright)_p(X, Y)
&=
\zeta_{x}\bigl(
	\mathrm{D}_{p \cdot \sigma_x}\pi\mleft( \mathcal{r}_{\sigma_x*}(X) \mright),
	\mathrm{D}_{p \cdot \sigma_x}\pi\mleft( \mathcal{r}_{\sigma_x*}(Y) \mright)
\bigr),
\\
&=
\zeta_{x}\bigl(
	{\underbrace{\mathrm{D}_{p \cdot \sigma_x}\pi\mleft( \mathrm{D}_pr_\sigma\mleft( X \mright) \mright)}_{\mathclap{ = \mathrm{D}_p\mleft( \pi \circ r_\sigma \mright) = \mathrm{D}_p\pi }}},
	\mathrm{D}_{p \cdot \sigma_x}\pi\mleft( \mathrm{D}_pr_\sigma\mleft( Y \mright) \mright)
\bigr)
\\
&=
\mleft(\pi^!\zeta\mright)_p(X, Y)
\eas
for all $p \in \mathcal{P}_x$ ($x \in M$) and $X, Y \in \mathrm{T}_p\mathcal{P}$, making use of that fundamental vector fields are in the kernel of $\mathrm{D}\pi$ so that $\mathrm{D}\pi \circ \mathcal{r}_{\sigma_x*} = \mathrm{D}\pi \circ r_{\sigma_x*}$. For the first term in the field strength use Cor.\ \ref{cor:ModifiedRightPushyCommutesWithProj} and Remark \ref{RemOohThesePullbacksConfusOrNotToConfus} so that
\bas
\mathcal{r}_\sigma^!\mleft(\mathrm{d}^{\pi^*\nabla^{\mathrm{YM}}} A \circ \mleft( \pi_h, \pi_h \mright)\mright)
&=
\mleft(r_\sigma^*\mathrm{d}^{\pi^*\nabla^{\mathrm{YM}}} A\mright) \circ \mleft( r_\sigma^*\pi_h \circ \mathcal{r}_{\sigma*}, r_\sigma^*\pi_h \circ \mathcal{r}_{\sigma*} \mright)
\\
&=
\mleft(r_\sigma^*\mathrm{d}^{\pi^*\nabla^{\mathrm{YM}}} A\mright) \circ \mleft( \mathcal{r}_{\sigma*} \circ \pi_h, \mathcal{r}_{\sigma*} \circ \pi_h \mright)
\\
&=
\mathcal{r}_\sigma^!\mleft(\mathrm{d}^{\pi^*\nabla^{\mathrm{YM}}} A\mright) \circ (\pi_h, \pi_h).
%\mleft(\mathcal{r}_\sigma^!\mleft(\mathrm{d}^{\pi^*\nabla^{\mathrm{YM}}} A \circ \mleft( \pi_h, \pi_h \mright)\mright)\mright)_p(X, Y)
%&=
%\mleft(\mleft(r_\sigma^*\mathrm{d}^{\pi^*\nabla^{\mathrm{YM}}} A\mright) \circ \mleft( r_\sigma^*\pi_h, r_\sigma^*\pi_h \mright)\mright)_p\bigl(\mathcal{r}_{\sigma*}(X), \mathcal{r}_{\sigma*}Y\bigr)
%\\
%&=
%\mleft(\mleft(r_\sigma^*\mathrm{d}^{\pi^*\nabla^{\mathrm{YM}}} A\mright) \circ \mleft( r_\sigma^*\pi_h \circ \mathcal{r}_{\sigma*}, r_\sigma^*\pi_h \circ \mathcal{r}_{\sigma*} \mright)\mright)_p(X, Y)
%\\
%&=
%\mleft(\mleft(r_\sigma^*\mathrm{d}^{\pi^*\nabla^{\mathrm{YM}}} A\mright) \circ \mleft( \mathcal{r}_{\sigma*} \circ \pi_h, \mathcal{r}_{\sigma*} \circ \pi_h \mright)\mright)_p(X, Y)
%\\
%&=
%\mleft(\mathcal{r}_\sigma^!\mleft(\mathrm{d}^{\pi^*\nabla^{\mathrm{YM}}} A\mright)\mright)_p \bigl( \pi_h(X), \pi_h(Y) \bigr)
%\\
%&=
%\mleft(\mleft(r_\sigma^*\mathrm{d}^{\pi^*\nabla^{\mathrm{YM}}} A\mright) \circ \mleft( r_\sigma^*\pi_h, r_\sigma^*\pi_h \mright)\mright)
%\mleft(
%\mathrm{D}_pr_\sigma\mleft( X \mright)
	%- \mleft.{\oversortoftilde{
		%\mleft. \mleft( \pi^!\Delta\sigma \mright) \mright|_p(X)
	%}}\mright|_{p \cdot \sigma_{x}}, 
	%\mathcal{r}_{\sigma*}Y
%\mright)
\eas
Let us denote $X^h \coloneqq \pi_h(X)$ and $Y^h \coloneqq \pi_h(Y)$, where $X$ and $Y$ are now elements of $\mathfrak{X}(\mathcal{P})$; since we are looking at tensorial equations we can w.l.o.g.\ assume that $X^h$ and $Y^h$ are the horizontal lifts of vector fields $\omega^X$ and $\omega^Y$, respectively, on $M$, especially $\mathrm{D}\pi\mleft( X^h \mright) = \mathrm{D}\pi\mleft( X \mright) = \pi^*\omega^X$ and $\mathrm{D}\pi\mleft( Y^h \mright) = \mathrm{D}\pi\mleft( Y \mright) = \pi^*\omega^Y$.
We also rewrite
\bas
\mathcal{r}_{\sigma*}(X)
&=
\mathrm{D}r_{\sigma}(X)
	- r_{\sigma}^*\mleft({\oversortoftilde{ \mleft(\pi^! \Delta \sigma \mright) (X)}}\mright)
\in
\Gamma\mleft( r_\sigma^* \mathrm{T}\mathcal{P} \mright) \cong \mathfrak{X}(\mathcal{P}).
\eas
In the following we will make use of the canonical identification $\mathfrak{X}(\mathcal{P}) \cong \Gamma\mleft( r_\sigma^* \mathrm{T}\mathcal{P} \mright)$, $X \mapsto r_\sigma^*X$, especially in order to calculate Lie brackets.
%
%Also observe that we have 
%\bas
%r_{\sigma}^!\mleft( \mathrm{d}^{\pi^*\nabla^{\mathrm{YM}}} A \mright)
%&=
%\mathrm{d}^{r_{\sigma}^*\pi^*\nabla^{\mathrm{YM}}} \mleft( r_{\sigma}^!A \mright)
%=
%\mathrm{d}^{(\pi \circ r_{\sigma})^*\nabla^{\mathrm{YM}}} \mleft( r_{\sigma}^!A \mright)
%=
%\mathrm{d}^{\pi^*\nabla^{\mathrm{YM}}} \mleft( r_{\sigma}^!A \mright),
%\eas
%whose first equality is a very straight-forward generalization of the commutation of pullbacks and de-Rham differentials; this is easy to check, but in case the reader is interested in an explicit proof, then see either \cite[Appendix, Prop.\ A.1, Eq.\ (A.1)]{My1stpaper} or \cite[Appendix, Prop.\ A.1.1, Eq.\ (A.2)]{MyThesis}. 

Then make use of Def.\ \ref{def:FinallyTheConnection} and Thm.\ \ref{thm:OurConnectionHasAUniqueoneForm} to show
\ba\label{TransformedClassicStrengthTerm}
&
r_\sigma^*\mleft(
\mleft(\mathrm{d}^{\pi^*\nabla^{\mathrm{YM}}} A\mright) \mleft(\mathcal{r}_{\sigma*}\mleft(X^h\mright), \mathcal{r}_{\sigma*}\mleft(Y^h\mright)\mright)
\mright)
\nonumber
\\
&\hspace{1cm}=
r_\sigma^*\biggl(
\mleft( \pi^*\nabla^{\mathrm{YM}} \mright)_{\mathcal{r}_{\sigma*}\mleft(X^h\mright)} {\underbrace{\mleft( (A \circ \mathcal{r}_{\sigma*})\mleft(Y^h\mright) \mright)}_{= 0}}
\nonumber
\\
&\hspace{2.5cm}
	- \mleft( \pi^*\nabla^{\mathrm{YM}} \mright)_{\mathcal{r}_{\sigma*}\mleft(Y^h\mright)} \mleft( (A \circ \mathcal{r}_{\sigma*})\mleft(X^h\mright) \mright)
\nonumber
\\
&\hspace{2.5cm}
	- A\mleft( \mleft[ \mathcal{r}_{\sigma*}\mleft(X^h\mright), \mathcal{r}_{\sigma*}\mleft(Y^h\mright) \mright] \mright)
\biggr)
\nonumber
\\
&\hspace{1cm}=r_\sigma^*\biggl(
	- A\mleft( \mleft[ \mathcal{r}_{\sigma*}\mleft(X^h\mright), \mathcal{r}_{\sigma*}\mleft(Y^h\mright) \mright] \mright)
\biggr)
\nonumber
\\
&\hspace{1cm}=r_\sigma^*\Biggl(
- A\mleft(
	\mleft[\mathrm{D}r_\sigma\mleft( X^h \mright), \mathrm{D}r_\sigma\mleft(Y^h\mright)\mright]
\mright)
\nonumber
\\
&\hspace{2.5cm}
	+ A\mleft(
	\mleft[\mathrm{D}r_\sigma\mleft(X^h\mright), r_{\sigma}^*\mleft({\oversortoftilde{ \mleft(\pi^! \Delta \sigma \mright) \mleft(Y^h\mright) }}\mright) \mright]
	\mright)
\nonumber
\\
&\hspace{2.5cm}
	+ A\mleft(
	\mleft[r_{\sigma}^*\mleft({\oversortoftilde{ \mleft(\pi^! \Delta \sigma \mright) \mleft(X^h\mright) }}\mright), \mathrm{D}r_\sigma\mleft(Y^h\mright) \mright]
	\mright)
\nonumber
\\
&\hspace{2.5cm}
	- A\mleft(
	\mleft[r_{\sigma}^*\mleft({\oversortoftilde{ \mleft(\pi^! \Delta \sigma \mright) \mleft(X^h \mright) }}\mright), r_{\sigma}^*\mleft({\oversortoftilde{ \mleft(\pi^! \Delta \sigma \mright)\mleft(Y^h \mright) }}\mright) \mright]
	\mright)
\Biggr).
\ea
Let us look at each summand individually; for the first summand we use the very well-known fact (see \textit{e.g.}\ \cite[\S A.1.11, Cor.\ A.1.51, page 615]{Hamilton}) that
\bas
\mleft[\mathrm{D}r_\sigma\mleft( X^h \mright), \mathrm{D}r_\sigma\mleft(Y^h\mright)\mright]
&=
\mathrm{D}r_\sigma\mleft(\mleft[ X^h, Y^h\mright]\mright),
\eas
making use of that $r_\sigma$ is an isomorphism, and thus we get for the first summand by using Def.\ \ref{def:GaugeBosonsOnLGBPrincies}
\bas
&r_\sigma^*\biggl(
- A\mleft(
	\mleft[\mathrm{D}r_\sigma\mleft( X^h \mright), \mathrm{D}r_\sigma\mleft(Y^h\mright)\mright]
\mright)
\biggr)
\\
&\hspace{1cm}=
-\mleft( r_\sigma^*A \mright)\mleft( \mathcal{r}_{\sigma*} \mleft( \mleft[ X^h, Y^h\mright] \mright)\mright)
	- \mleft( r_\sigma^* A \mright) \mleft( r_{\sigma}^*\mleft({\oversortoftilde{ \mleft(\pi^! \Delta \sigma \mright) \mleft(\mleft[ X^h, Y^h\mright]\mright)}}\mright) \mright)
\\
&\hspace{1cm}=
-\mleft( \mathcal{r}_\sigma^!A \mright)\mleft( \mleft[ X^h, Y^h\mright] \mright)
	- r_{\sigma}^*\mleft(\mleft(\pi^! \Delta \sigma \mright) \mleft(\mleft[ X^h, Y^h\mright]\mright)\mright)
\\
&\hspace{1cm}=
-\mleft( \sAd_{\sigma^{-1}} \circ A \mright)\mleft( \mleft[ X^h, Y^h\mright] \mright)
	- r_{\sigma}^*\mleft(\mleft(\pi^! \Delta \sigma \mright) \mleft(\mleft[ X^h, Y^h\mright]\mright)\mright)
\\
&\hspace{1cm}=
\sAd_{\sigma^{-1}}\mleft(\mleft( \mathrm{d}^{\pi^*\nabla^{\mathrm{YM}}}A \mright)\mleft( X^h, Y^h \mright) \mright)
	- r_{\sigma}^*\mleft(\mleft(\pi^! \Delta \sigma \mright) \mleft(\mleft[ X^h, Y^h\mright]\mright)\mright),
\eas
where we used a similar argument for the last equality as for the beginning of the calculation for Eq.\ \eqref{TransformedClassicStrengthTerm}.
We know by Remark \ref{rem:FundVecsAreLABActions} that the map to fundamental vector fields is a homomorphism of Lie algebras, so that
%For the fourth summand recall that fundamental vector fields are fundamental vector fields along the fibres of $\mathcal{P}$; if we fix a base point $x \in M$, then each fibre $\mathcal{P}_x$ is an embedded submanifold of $\mathcal{P}$ onto which the Lie group $\mathcal{G}_x$ acts on the right. Thus, we know from Lie group action theory (see \textit{e.g.}\ \cite[\S 3.4, Prop.\ 3.4.4, page 144]{Hamilton}) that the Lie bracket of two fundamental vector fields over $\mathcal{P}_x$ is again a fundamental vector field over $\mathcal{P}_x$ given by
%\bas
%\mleft[ \widetilde{\mu}, \widetilde{\nu} \mright]
%&=
%\widetilde{\mleft[ \mu, \nu \mright]}_{\mathcal{g}_x}
%\eas
%for all $\mu, \nu \in \mathcal{g}_x$. This naturally extends to fundamental vector fields over $\mathcal{P}$, so that
\bas
\mleft[r_{\sigma}^*\mleft({\oversortoftilde{ \mleft(\pi^! \Delta \sigma \mright) \mleft(X^h \mright) }}\mright), r_{\sigma}^*\mleft({\oversortoftilde{ \mleft(\pi^! \Delta \sigma \mright)\mleft(Y^h \mright) }}\mright) \mright]
&=
r_\sigma^*\mleft(
{\oversortoftilde{
	\mleft[ \mleft(\pi^! \Delta \sigma \mright) \mleft(X^h \mright), \mleft(\pi^! \Delta \sigma \mright)\mleft(Y^h \mright) \mright]_{\mathcal{\pi^*\mathcal{g}}}
}}
\mright),
\eas
and therefore we can derive for the fourth term in Eq.\ \eqref{TransformedClassicStrengthTerm} that
\bas
\mleft( r_\sigma^*A \mright)\mleft(
	\mleft[r_{\sigma}^*\mleft({\oversortoftilde{ \mleft(\pi^! \Delta \sigma \mright) \mleft(X^h \mright) }}\mright), r_{\sigma}^*\mleft({\oversortoftilde{ \mleft(\pi^! \Delta \sigma \mright)\mleft(Y^h \mright) }}\mright) \mright]
	\mright)
&=
r_\sigma^*\mleft(
	\mleft[ \mleft(\pi^! \Delta \sigma \mright) \mleft(X^h \mright), \mleft(\pi^! \Delta \sigma \mright)\mleft(Y^h \mright) \mright]_{\mathcal{\pi^*\mathcal{g}}}
\mright).
\eas
For the second and third summands in Eq.\ \eqref{TransformedClassicStrengthTerm} we want to use Lemma \ref{lem:BracketVertHor} and our result of the fourth summand to show
\bas
&\mleft( r_\sigma^*A \mright)\mleft(
	\mleft[\mathrm{D}r_\sigma\mleft(X^h\mright), r_{\sigma}^*\mleft({\oversortoftilde{ \mleft(\pi^! \Delta \sigma \mright) \mleft(Y^h\mright) }}\mright) \mright]
	\mright)
\\ 
&\hspace{0.5cm}
=
\mleft( r_\sigma^*A \mright)\mleft(
	\mleft[\mathcal{r}_{\sigma*}\mleft(X^h\mright), r_{\sigma}^*\mleft({\oversortoftilde{ \mleft(\pi^! \Delta \sigma \mright) \mleft(Y^h\mright) }}\mright) \mright]
	\mright)
\\&\hspace{1.5cm}
	+ \mleft( r_\sigma^*A \mright)\mleft(\mleft[
	r_{\sigma}^*\mleft({\oversortoftilde{ \mleft(\pi^! \Delta \sigma \mright) \mleft(X^h\mright) }}\mright), r_{\sigma}^*\mleft({\oversortoftilde{ \mleft(\pi^! \Delta \sigma \mright) \mleft(Y^h\mright) }}\mright) \mright]
	\mright)
\\ 
&\hspace{0.5cm}
=
r_\sigma^*\Biggl(
\mleft( \pi^*\nabla^{\mathrm{YM}} \mright)_{\mathcal{r}_{\sigma*}\mleft(X^h\mright)}\biggl( \mleft(\pi^! \Delta \sigma \mright) \mleft(Y^h\mright)\biggr)
\Biggr)
	+ r_\sigma^*\mleft(
	\mleft[ \mleft(\pi^! \Delta \sigma \mright) \mleft(X^h \mright), \mleft(\pi^! \Delta \sigma \mright)\mleft(Y^h \mright) \mright]_{\mathcal{\pi^*\mathcal{g}}}
\mright)
\eas
using that $\mathcal{r}_{\sigma*}\mleft(X^h\mright)$ is horizontal due to the fact that $X^h$ is horizontal, and, last but not least, we can write for terms like
\bas
\mleft(\pi^! \Delta \sigma \mright) \mleft(Y^h\mright)
&=
\mleft(\pi^* \Delta \sigma \mright) \mleft(\mathrm{D}\pi\mleft(Y^h\mright)\mright)
=
\mleft(\pi^* \Delta \sigma \mright) \mleft(\pi^*\omega^Y\mright)
=
\pi^*\Bigl( \mleft(\Delta \sigma \mright) \mleft(\omega^Y\mright) \Bigr),
\eas
and
\bas
\mathrm{D}\pi\mleft( \mathcal{r}_{\sigma*}\mleft(X^h\mright) \mright)
&=
\mathrm{D}\pi\mleft( \mathrm{D}r_{\sigma}\mleft(X^h\mright)
	- r_{\sigma}^*\mleft({\oversortoftilde{ \mleft(\pi^! \Delta \sigma \mright) \mleft(X^h\mright)}}\mright) \mright)
=
\mathrm{D}{\underbrace{\mleft( \pi \circ r_\sigma \mright)}_{= \pi}}\mleft(X^h\mright)
=
\pi^*\omega^X,
\eas
using that fundamental vector fields are in the kernel of $\mathrm{D}\pi$, 
and therefore
\bas
%r_\sigma^*\Biggl(
\mleft( \pi^*\nabla^{\mathrm{YM}} \mright)_{\mathcal{r}_{\sigma*}\mleft(X^h\mright)}\biggl( \mleft(\pi^! \Delta \sigma \mright) \mleft(Y^h\mright)\biggr)
%\Biggr)
&=
\pi^*\mleft(
	\nabla^{\mathrm{YM}}_{\omega^X}\Bigl( \mleft(\Delta \sigma \mright) \mleft(\omega^Y\mright)\Bigr)
\mright).
%\\
%&=
%\mleft( \pi^*\nabla^{\mathrm{YM}} \mright)_{X^h}\biggl( \mleft(\pi^! \Delta \sigma \mright) \mleft(Y^h\mright)\biggr).
%\mleft.\mleft(\pi^! \Delta \sigma \mright)\mleft(Y^h \mright)\mright|_p
%&=
%\mleft(\Delta \sigma \mright)_x\mleft(\mathrm{D}_p\pi\mleft(Y^h_p\mright) \mright)
%=
%\mleft(\pi^* \Delta \sigma \mright)\mleft(\mathrm{D}\pi\mleft( \mathcal{r}_{\sigma*}\mleft(Y^h\mright) \mright) \mright)
%=
%\mleft(\pi^! \Delta \sigma \mright)\mleft( \mathcal{r}_{\sigma*}\mleft(Y^h\mright) \mright)
\eas
%for all $p \in \mathcal{P}_x$ ($x \in M$).

Collecting all these results and using Cor.\ \ref{cor:PullbackOfMCSupperEquation} we can write Eq.\ \eqref{TransformedClassicStrengthTerm} as
\bas
&
r_\sigma^*\mleft(
\mleft(\mathrm{d}^{\pi^*\nabla^{\mathrm{YM}}} A\mright) \mleft(\mathcal{r}_{\sigma*}\mleft(X^h\mright), \mathcal{r}_{\sigma*}\mleft(Y^h\mright)\mright)
\mright)
\\&\hspace{1cm}
=
\sAd_{\sigma^{-1}}\mleft(\mleft( \mathrm{d}^{\pi^*\nabla^{\mathrm{YM}}}A \mright)\mleft( X^h, Y^h \mright) \mright)
\\&\hspace{1.5cm}
	+ r_{\sigma}^*\Biggl(
		\pi^*\mleft(
	\nabla^{\mathrm{YM}}_{\omega^X}\Bigl( \mleft(\Delta \sigma \mright) \mleft(\omega^Y\mright)\Bigr)
\mright)
		- \pi^*\mleft(
	\nabla^{\mathrm{YM}}_{\omega^Y}\Bigl( \mleft(\Delta \sigma \mright) \mleft(\omega^X\mright)\Bigr)
\mright)
\\&\hspace{3cm}
		- {\underbrace{\mleft(\pi^! \Delta \sigma \mright) \mleft(\mleft[ X^h, Y^h\mright]\mright)}_{= \pi^* \mleft(\Delta \sigma \mleft(\mleft[ \omega^X, \omega^Y\mright]\mright)\mright)}}
		+ \mleft[ \pi^*\Bigl( \mleft(\Delta \sigma \mright) \mleft(\omega^X\mright) \Bigr), \pi^*\Bigl( \mleft(\Delta \sigma \mright) \mleft(\omega^Y\mright) \Bigr) \mright]_{\mathcal{\pi^*\mathcal{g}}}
\Biggr)
\\&\hspace{1cm}
=
\sAd_{\sigma^{-1}}\mleft(\mleft( \mathrm{d}^{\pi^*\nabla^{\mathrm{YM}}}A \mright)\mleft( X^h, Y^h \mright) \mright)
%\\&\hspace{1.5cm}
	+ {\underbrace{r_{\sigma}^* \pi^*}_{= \pi^*}}\mleft( 
		\mleft(
			\mathrm{d}^{\nabla^{\mathrm{YM}}} \Delta\sigma
			+ \frac{1}{2} \mleft[ \Delta\sigma \stackrel{\wedge}{,} \Delta\sigma \mright]_{\mathcal{g}}
		\mright)
	\mleft(\omega^X, \omega^Y\mright) \mright)
\\&\hspace{1cm}
=
\sAd_{\sigma^{-1}}\mleft(\mleft( \mathrm{d}^{\pi^*\nabla^{\mathrm{YM}}}A \mright)\mleft( X^h, Y^h \mright) \mright)
%\\&\hspace{1.5cm}
	+ \bigl( \pi^*
		\mleft( \mathrm{Ad}_{\sigma^{-1}} \circ \zeta 
		- \zeta \mright) \bigr)
	\bigl(\mathrm{D}\pi(X), \mathrm{D}\pi(Y)\bigr)
\\&\hspace{1cm}
=
\sAd_{\sigma^{-1}}\mleft(\mleft( \mathrm{d}^{\pi^*\nabla^{\mathrm{YM}}}A \mright)\mleft( X^h, Y^h \mright) \mright)
%\\&\hspace{1.5cm}
	+ \mleft(\mleft(\pi^!\mleft(  
		\mathrm{Ad}_{\sigma^{-1}} \circ \zeta 
		- \zeta 
	\mright)\mright)(X, Y) \mright)
\\&\hspace{1cm}
=
\mleft(
	\sAd_{\sigma^{-1}}\circ \mleft( \mathrm{d}^{\pi^*\nabla^{\mathrm{YM}}}A \circ ( \pi_h, \pi_h) + \pi^!\zeta \mright) 
		- \pi^!\zeta 
\mright)(X, Y)
%%%%%%%%%%%%%%%%%%%%%%%%%%%%%%%%%%%%%%%%%%%%%%%%%%%%%%%%%%%%%%%%%%%%%%%%%%%%%%%%%%%%%%%
%\\&\hspace{1cm}
%=
%\sAd_{\sigma^{-1}}\mleft(\mleft( \mathrm{d}^{\pi^*\nabla^{\mathrm{YM}}}A \mright)\mleft( X^h, Y^h \mright) \mright)
%\\&\hspace{1.5cm}
	%+ r_{\sigma}^*\Biggl( \mleft(
		%\mathrm{d}^{\pi^*\nabla^{\mathrm{YM}}}\mleft( \pi^!\Delta\sigma \mright)
		%+ \frac{1}{2} \mleft[ \pi^!\Delta\sigma \stackrel{\wedge}{,} \pi^!\Delta\sigma \mright]_{\mathcal{\pi^*\mathcal{g}}} \mright)\mleft( X^h, Y^h \mright)
%\Biggr)
\eas
where we used that $\mathrm{D}\pi (X) = \pi^*\omega^X$ and $\mathrm{D}\pi (Y) = \pi^*\omega^Y$, which immediately implies
\bas
\mathrm{D}\pi\mleft( [X, Y] \mright)
&=
\pi^*\mleft( \mleft[\omega^X, \omega^Y\mright] \mright),
\eas
which is a well-known fact, as also given in \cite[Proposition A.1.49; page 615]{Hamilton}, but also straight-forward to check. 

Finally, we can conclude this proof by combining the previous result with our first calculation of this proof, that is,
\bas
\mathcal{r}_{\sigma}^!F
&=
\sAd_{\sigma^{-1}}\circ \mleft( \mathrm{d}^{\pi^*\nabla^{\mathrm{YM}}}A \circ ( \pi_h, \pi_h) + \pi^!\zeta \mright) 
		- \pi^!\zeta
		+ \pi^!\zeta
=
\sAd_{\sigma^{-1}} \circ F.
\eas
\end{proof}

We achieve of course also a structure equation.

\begin{theorems}{Structure equation of the generalized field strength}{StructureEq}
Let $\mathcal{G} \to M$ be an LGB over a smooth manifold $M$ and $\mathcal{P} \stackrel{\pi}{\to} M$ a principal $\mathcal{G}$-bundle, also let $\mathrm{H}\mathcal{G}$ be a Yang-Mills connection on $\mathcal{G}$ (w.r.t.\ a $\zeta \in \Omega^2(M; \mathcal{g})$) and $A \in \Omega^1(\mathcal{P}; \pi^*\mathcal{g})$ be a connection 1-form on $\mathcal{P}$. Then we have the \textbf{structure equation}
\bas
F
=
\mathrm{d}^{\pi^*\nabla^{\mathrm{YM}}} A
	+ \frac{1}{2} \mleft[ A \stackrel{\wedge}{,} A \mright]_{\pi^*\mathcal{g}}
	+ \pi^!\zeta.
\eas
\end{theorems}

\begin{remarks}{The generalized field strength of the total Maurer-Cartan form}{CurvoftotMCForm}
Again viewing $\mathcal{G} \stackrel{\pi_{\mathcal{G}}}{\to} M$ as the principal bundle itself with $\mathrm{H}\mathcal{P} = \mathrm{H}\mathcal{G}$, we know by Cor.\ \ref{cor:TotMCFormIsConnectionForm} that the total Maurer-Cartan form $\mu_{\mathcal{G}}^{\mathrm{tot}}$ is the connection 1-form corresponding to $\mathrm{H}\mathcal{G}$. Denoting the associated generalized field strength by $F\mleft( \mu_{\mathcal{G}}^{\mathrm{tot}} \mright)$, the structure equation and the generalized Maurer-Cartan equation, Thm.\ \ref{thm:GenMCEq} (also recall Rem.\ \ref{BasePointDifficultiesinGenMCEq}), imply
\bas
F_g\mleft( \mu_{\mathcal{G}}^{\mathrm{tot}} \mright)
&=
\sAd_{g^{-1}} \circ \mleft. \pi_{\mathcal{G}}^!\zeta \mright|_g
\eas
for all $g \in \mathcal{G}_x$ ($x \in M$). This implies that $F\mleft( \mu_{\mathcal{G}}^{\mathrm{tot}} \mright)$ is fully encoded by $\zeta$, especially we have
\ba\label{ZetaIsTheCurvatureOnG}
e^!\Bigl(F\mleft( \mu_{\mathcal{G}}^{\mathrm{tot}} \mright)\Bigr)
&=
\zeta,
\ea
where $e$ is the neutral section of $\mathcal{G}$. Thus, $\zeta$ is the curvature of $\mu_{\mathcal{G}}^{\mathrm{tot}}$ restricted on $M$, and 
\bas
F
&=
\mathrm{d}^{\pi^*\nabla^{\mathrm{YM}}} A
	+ \frac{1}{2} \mleft[ A \stackrel{\wedge}{,} A \mright]_{\pi^*\mathcal{g}}
	+ (e \circ \pi)^!\Bigl(F\mleft( \mu_{\mathcal{G}}^{\mathrm{tot}} \mright)\Bigr).
\eas
Following similar calculations as in the proof of Prop.\ \ref{prop:NewFieldStrengthWithCoolProps}, it is trivial to show that the first two summands measure the failure of $\mathrm{H}\mathcal{P}$ being a foliation, hence this meaning carries over from the classical theory as also argued in \cite[\S 2.5, Cor.\ 2.23]{FernandesMarcutMultiplicativeForms} for LGBs. Due to the third summand, $\mathrm{H}\mathcal{P}$ is involutive if and only if
\bas
F
&=
(e \circ \pi)^!\Bigl(F\mleft( \mu_{\mathcal{G}}^{\mathrm{tot}} \mright)\Bigr).
\eas
In that sense Eq.\ \eqref{ZetaIsTheCurvatureOnG} reflects the fact that $M$ is a horizontal leave of $\mathrm{H}\mathcal{G}$. Furthermore, $F$ can be non-zero while $\mathrm{H}\mathcal{P}$ is involutive, if $F\mleft( \mu_{\mathcal{G}}^{\mathrm{tot}} \mright)$ is non-zero along $M$.
\end{remarks}

\begin{proof}[Proof of Thm.\ \ref{thm:StructureEq}]
\leavevmode\newline
The idea of the proof is similar to the proof of the "classical" statement; following the structure of \cite[\S 5.5, Lemma 5.5.5, page 276]{Hamilton}, see also works like \cite[\S 4.6, Lemma 4.25]{LAURENTGENGOUXStienonXuMultiplicativeForms} and \cite[\S 2.5, Prop.\ 2.24]{FernandesMarcutMultiplicativeForms} where something similar is shown on Lie groupoids instead of principal bundles, but with no $\zeta$. That is, we will look at vertical and horizontal tangent vectors and their mixed terms; recall Def.\ \ref{def:GaugeBosonsOnLGBPrincies} and Thm.\ \ref{thm:OurConnectionHasAUniqueoneForm}.

$\bullet$ We have by Prop.\ \ref{prop:NewFieldStrengthWithCoolProps}
\bas
F\mleft( \widetilde{\nu}, \widetilde{\mu} \mright)
&=
0
\eas
for all $\mu, \nu \in \Gamma(\mathcal{g})$. Regarding the right hand side of the structure equation, this trivially also holds for the third summand $\mleft(\pi^!\zeta\mright)\mleft( \widetilde{\nu}, \widetilde{\mu} \mright) = 0$. We also have
\bas
\mleft(\frac{1}{2} \mleft[ A \stackrel{\wedge}{,} A \mright]_{\pi^*\mathcal{g}}\mright)\mleft( \widetilde{\nu}, \widetilde{\mu} \mright)
&=
\mleft[ A\mleft( \widetilde{\nu} \mright), A\mleft( \widetilde{\mu} \mright) \mright]_{\pi^*\mathcal{g}}
=
\mleft[ \pi^*\nu, \pi^*\mu \mright]_{\pi^*\mathcal{g}}
=
\pi^*\mleft( \mleft[ \nu, \mu \mright]_{\mathcal{g}} \mright),
\eas
and
\bas
\mleft(\mathrm{d}^{\pi^*\nabla^{\mathrm{YM}}} A\mright)\mleft( \widetilde{\nu}, \widetilde{\mu} \mright)
&=
{\underbrace{\mleft(\pi^*\nabla^{\mathrm{YM}}\mright)_{\widetilde{\nu}} \bigl(A\mleft( \widetilde{\mu} \mright) \bigr)}_{\mathclap{ = \pi^*\mleft( \nabla^{\mathrm{YM}}_{\mathrm{D}\pi\mleft( \widetilde{\nu} \mright)} \mu \mright) = 0 }}}
	- \mleft(\pi^*\nabla^{\mathrm{YM}}\mright)_{\widetilde{\mu}} \bigl(A\mleft( \widetilde{\nu} \mright) \bigr)
	- A {\underbrace{\bigl( \mleft[ \widetilde{\nu}, \widetilde{\mu} \mright] \bigr)}_{= \widetilde{[\nu, \mu]}}}
=
- \pi^*\mleft( \mleft[ \nu, \mu \mright]_{\mathcal{g}} \mright).
\eas
Thus,
\bas
\mleft(
	\mathrm{d}^{\pi^*\nabla^{\mathrm{YM}}} A
	+ \frac{1}{2} \mleft[ A \stackrel{\wedge}{,} A \mright]_{\pi^*\mathcal{g}}
	+ \pi^!\zeta
\mright)\mleft( \widetilde{\nu}, \widetilde{\mu} \mright)
&=
0
=
F\mleft( \widetilde{\nu}, \widetilde{\mu} \mright).
\eas

$\bullet$ Let $X$ and $Y$ be horizontal vector fields of $\mathcal{P}$, then clearly
\bas
\mleft(\frac{1}{2} \mleft[ A \stackrel{\wedge}{,} A \mright]_{\pi^*\mathcal{g}}\mright)(X, Y)
&=
\bigl[ A\mleft( X \mright), A\mleft( Y \mright) \bigr]_{\pi^*\mathcal{g}}
=
0,
\eas
then clearly
\bas
\mleft(
	\mathrm{d}^{\pi^*\nabla^{\mathrm{YM}}} A
	+ \frac{1}{2} \mleft[ A \stackrel{\wedge}{,} A \mright]_{\pi^*\mathcal{g}}
	+ \pi^!\zeta
\mright)(X, Y)
&=
\mleft(
	\mathrm{d}^{\pi^*\nabla^{\mathrm{YM}}} A \circ \mleft( \pi_h, \pi_h \mright)
	+ \pi^!\zeta
\mright)(X, Y)
=
F(X, Y),
\eas
where $\pi_h: \mathrm{T}\mathcal{P} \to \mathrm{H}\mathcal{P}$ denotes the canonical projection onto the horizontal bundle $\mathrm{H}\mathcal{P}$ (especially, $\pi_h(X) = X$, $\pi_h(Y)= Y$).

$\bullet$ Now let $X \in \mathfrak{X}(\mathcal{P})$ be again horizontal and $\nu \in \Gamma(\mathcal{g})$; then again by Prop.\ \ref{prop:NewFieldStrengthWithCoolProps}
\bas
F\mleft(X, \widetilde{\nu}\mright)
&=
0,
\eas
as also clearly $\mleft( \pi^!\zeta \mright)\mleft(X, \widetilde{\nu}\mright) = 0$,
furthermore
\bas
\mleft(\frac{1}{2} \mleft[ A \stackrel{\wedge}{,} A \mright]_{\pi^*\mathcal{g}}\mright)\mleft( X, \widetilde{\nu} \mright)
&=
\mleft[ A(X), A\mleft( \widetilde{\nu} \mright) \mright]_{\pi^*\mathcal{g}}
=
0,
\eas
and by Lemma \ref{lem:BracketVertHor}
\bas
\mleft(\mathrm{d}^{\pi^*\nabla^{\mathrm{YM}}} A\mright)\mleft( X, \widetilde{\nu} \mright)
&=
\mleft( \pi^*\nabla^{\mathrm{YM}} \mright)_X \mleft( \pi^*\nu \mright)
	- A\bigl( \mleft[ X, \widetilde{\nu} \mright] \bigr)
\stackrel{\ref{lem:BracketVertHor}}{=}
0,
\eas
which concludes the proof with
\bas
F\mleft(X, \widetilde{\nu}\mright)
&=
\mleft(
	\mathrm{d}^{\pi^*\nabla^{\mathrm{YM}}} A
	+ \frac{1}{2} \mleft[ A \stackrel{\wedge}{,} A \mright]_{\pi^*\mathcal{g}}
	+ \pi^!\zeta
\mright)\mleft(X, \widetilde{\nu}\mright).
\eas
\end{proof}

Concluding this subsubsection, we also achieve a generalized Bianchi identity, which we already have proven in \cite[\S 7, Thm.\ 7.3]{My1stpaper} and \cite[\S 5, Thm.\ 5.1.42]{MyThesis}; the proof is straightforward to show, making use of the infinitesimal compatibility conditions, Def.\ \ref{def:YangMillsConnection}.

\begin{theorems}{Generalized Bianchi identity, \cite[\S 5, Thm.\ 5.1.42]{MyThesis}}{GenBianchi}
Let $\mathcal{G} \to M$ be an LGB over a smooth manifold $M$ and $\mathcal{P} \stackrel{\pi}{\to} M$ a principal $\mathcal{G}$-bundle, also let $\mathrm{H}\mathcal{G}$ be a Yang-Mills connection on $\mathcal{G}$ (w.r.t.\ a $\zeta \in \Omega^2(M; \mathcal{g})$) and $A \in \Omega^1(\mathcal{P}; \pi^*\mathcal{g})$ be a connection 1-form on $\mathcal{P}$. Then we have the \textbf{(generalized) Bianchi identity}
\bas
\mathrm{d}^{\pi^*\nabla^{\mathrm{YM}}}F
	+ \mleft[ A \stackrel{\wedge}{,} F \mright]_{\pi^*\mathcal{g}}
&=
\pi^! \mathrm{d}^{\nabla^{\mathrm{YM}}} \zeta.
\eas
\end{theorems}

\subsubsection{Gauge transformation of the generalized field strength}

Similar to Subsection \ref{GaugeTrafoForA} we will now discuss the gauge transformation of the generalized field strength $F$; recall Thm.\ \ref{thm:GaugeTrafoOfGaugeBoson}.

\begin{theorems}{Gauge transformation of the generalized field strength}{GaugeTrafoOfCurv}
Let $\mathcal{G} \to M$ be an LGB over a smooth manifold $M$ and $\mathcal{P} \stackrel{\pi}{\to} M$ a principal $\mathcal{G}$-bundle, also let $\mathrm{H}\mathcal{G}$ be a Yang-Mills connection on $\mathcal{G}$ and $A \in \Omega^1(\mathcal{P}; \pi^*\mathcal{g})$ be a connection 1-form on $\mathcal{P}$. Furthermore, let $H \in \sAut(\mathcal{P})$. We then have that $H^!F$ is the field strength related to $H^!A$ and
\bas
H^!F
&=
{\sAd_{\mathrm{pr}_2\circ\mleft(\sigma^H\mright)^{-1}}} \circ F,
\eas
where $\sigma^H \in C^\infty(\mathcal{P}; \mathcal{G})^{\mathcal{G}}$ is defined as in Prop.\ \ref{prop:GaugeTrafoAsBundleIsomIsASectionOfConjugationMaps} and $\mathrm{pr}_2: \pi^*\mathcal{G} \to \mathcal{G}$ is the projection onto the second component.

Similar to Def.\ \eqref{MultiWithPulli} we may shortly just write 
\bas
H^!F
&=
{\sAd_{\mleft(\sigma^H\mright)^{-1}}} \circ F.
\eas
\end{theorems}

\begin{proof}
\leavevmode\newline
That $H^!F$ is the field strength related to $H^!A$ follows quickly by the same calculation as for proving Eq.\ \eqref{PullbackofMCGeneralCurvPlusExtra}, using $\pi \circ H = \pi$,
\bas
H^!F
&=
\mathrm{d}^{(\pi \circ H)^*\nabla^{\mathrm{YM}}} \mleft(H^!A\mright)
	+ \frac{1}{2} \mleft[ H^!A \stackrel{\wedge}{,} H^!A \mright]_{\pi^*\mathcal{g}}
	+ (\pi \circ H)^!\zeta
\\
&=
\mathrm{d}^{\pi^*\nabla^{\mathrm{YM}}} \mleft(H^!A\mright)
	+ \frac{1}{2} \mleft[ H^!A \stackrel{\wedge}{,} H^!A \mright]_{\pi^*\mathcal{g}}
	+ \pi^!\zeta,
\eas
which is the field strength of $H^!A$. Now recall Eq.\ \eqref{DiffOfPrincAutom}, that is,
\bas
\mathrm{D}_p H(X)
&=
\mathcal{r}_{\widetilde{\sigma}_p*}(X)
	+ \mleft.{\oversortoftilde{ \mleft(\mleft(\pi^*\Delta\mright)\sigma^H\mright)_p(X) }}\mright|_{H(p)}
\eas
for all $X \in \mathrm{T}_p\mathcal{P}$ ($p \in \mathcal{P}$), where $\widetilde{\sigma}_p \coloneqq \mathrm{pr}_2\mleft( \sigma^H_p \mright)$. Thus, by Prop.\ \ref{prop:NewFieldStrengthWithCoolProps},
\bas
\mleft( H^!F \mright)_p(X, Y)
&=
F_{H(p)}\bigl( \mathrm{D}_pH(X), \mathrm{D}_pH(Y) \bigr)
\\
&=
F_{p \cdot \widetilde{\sigma}_p}\bigl( \mathcal{r}_{\widetilde{\sigma}_p*}(X), \mathcal{r}_{\widetilde{\sigma}_p*}(Y) \bigr)
\\
&=
\mleft(\mathcal{r}_{\widetilde{\sigma}}^!F\mright)_p(X, Y)
\\
&=
\mleft(\sAd_{\widetilde{\sigma}_p^{-1}} \circ F\mright)_p(X, Y)
\eas
for all $X, Y \in \mathrm{T}_p \mathcal{P}$; this finishes the proof.
\end{proof}

We will now extend this again to a local change of gauges (sections of $\mathcal{P}$).

\begin{definitions}{Local field strength}{LocalFieldStrength}
Let $\mathcal{G} \to M$ be an LGB over a smooth manifold $M$ and $\mathcal{P} \stackrel{\pi}{\to} M$ a principal $\mathcal{G}$-bundle, also let $\mathrm{H}\mathcal{G}$ be a Yang-Mills connection on $\mathcal{G}$ and $A \in \Omega^1(\mathcal{P}; \pi^*\mathcal{g})$ be a connection 1-form on $\mathcal{P}$. Furthermore, let $s \in \Gamma(\mathcal{P}|_U)$ be a (local) gauge over an open subset $U \subset M$. Then we define the \textbf{local curvature} of \textbf{local field strength $F_s \in \Omega^2\mleft(U; \mleft.\mathcal{g}\mright|_U\mright)$ (w.r.t.\ $s$)} by
\bas
F_s
&\coloneqq
s^!F.
\eas
\end{definitions}

In previous calculations for the total Maurer-Cartan form and the generalized Maurer-Cartan equation we have basically already shown the pullback of the structure equation.

\begin{corollaries}{Pullback of the structure equation}{PullbackOfStructureEq}
Let $\mathcal{G} \to M$ be an LGB over a smooth manifold $M$ and $\mathcal{P} \stackrel{\pi}{\to} M$ a principal $\mathcal{G}$-bundle, also let $\mathrm{H}\mathcal{G}$ be a Yang-Mills connection on $\mathcal{G}$ and $A \in \Omega^1(\mathcal{P}; \pi^*\mathcal{g})$ be a connection 1-form on $\mathcal{P}$. Furthermore, let $s \in \Gamma(\mathcal{P}|_U)$ be a (local) gauge over an open subset $U \subset M$. Then we can express the local field strength as the pullback of the structure equation, that is,
\bas
F_s
&=
\mathrm{d}^{\nabla^{\mathrm{YM}}}A_s
	+ \frac{1}{2} \mleft[ A_s \stackrel{\wedge}{,} A_s \mright]_{\mathcal{g}}
	+ \zeta.
\eas
\end{corollaries}

\begin{proof}
\leavevmode\newline
This follows quickly by the same calculation as for proving Eq.\ \eqref{PullbackofMCGeneralCurvPlusExtra}.
\end{proof}

The gauge transformation of the field strength $F$ describes the behaviour of $F$ under a change of gauge.

\begin{theorems}{Gauge transformations again as a change of gauge}{LocalGaugeTrafoChangeGaugeFieldStrength}
Let $\mathcal{G} \to M$ be an LGB over a smooth manifold $M$ and $\mathcal{P} \stackrel{\pi}{\to} M$ a principal $\mathcal{G}$-bundle, also let $\mathrm{H}\mathcal{G}$ be a Yang-Mills connection on $\mathcal{G}$ and $A \in \Omega^1(\mathcal{P}; \pi^*\mathcal{g})$ be a connection 1-form on $\mathcal{P}$. Also let $U_i$ and $U_j$ be two open subsets of $M$ so that $U_i \cap U_j \neq \emptyset$, two gauges $s_i \in \Gamma\mleft(\mathcal{P}|_{U_i}\mright)$ and $s_j \in \Gamma\mleft(\mathcal{P}|_{U_j}\mright)$, and the unique $\sigma_{ji} \in \Gamma\mleft( \mleft.\mathcal{G}\mright|_{U_i \cap U_j} \mright)$ with $s_i = s_j \cdot \sigma_{ji}$ on $U_i \cap U_j$.

Then we have for the fields strength of $A$ over $U_i \cap U_j$ that
\bas
F_{s_i}
&=
\mathrm{Ad}_{\sigma_{ji}^{-1}}\circ F_{s_j}.
\eas
\end{theorems}

\begin{proof}
\leavevmode\newline
The proof is precisely as for Thm.\ \ref{thm:LocalGaugeTrafoChangeGauge}, just without catering for the Darboux derivative in the gauge transformation due to that one uses Thm.\ \ref{thm:GaugeTrafoOfCurv} instead of Thm.\ \ref{thm:GaugeTrafoOfGaugeBoson}.
\end{proof}

\begin{remarks}{Integrating curved Yang-Mills gauge theories, part II}{IntegratingKotovStroblFieldStrength}
Similar to Rem.\ \ref{rem:IntegratingKotovStrobl}, Cor.\ \ref{cor:PullbackOfStructureEq} and Thm.\ \ref{thm:LocalGaugeTrafoChangeGaugeFieldStrength} show that we recover the field strength and its gauge transformation as developed by Alexei Kotov and Thomas Strobl, see \cite{CurvedYMH} for a concise summary or \cite{MyThesis} for an extended introduction. These references are for the very general situation using Lie algebroids, hence see alternatively \cite{My1stpaper} for this type of gauge theory restricted to Lie algebra bundles.
\end{remarks}

%\subsection{Generalized covariant derivative/minimal coupling}

\section{Curved Yang-Mills gauge theory}\label{CYMSection}

\subsection{Definition and gauge invariance}\label{CYMDefGaugeInv}

We eventually are able to conclude this paper with the definition of what we will call \textbf{curved Yang-Mills gauge theory}, highlighting that this theory is an integral of the infinitesimal gauge theory originally developed by Alexei Kotov and Thomas Strobl. For the following we say that a fibre metric $\kappa$ of $\mathcal{g}$ is \textbf{$\mathrm{Ad}$-invariant} if
\bas
\kappa\bigl( \mathrm{Ad}_g(\nu), \mathrm{Ad}_g(\mu) \bigr)
&=
\kappa (\nu, \mu)
\eas
for all $g \in \mathcal{G}_x$ ($x \in M$) and $\mu, \nu \in \mathcal{g}_x$.

\begin{corollaries}{Contraction of local field strength with Ad-invariant fibre metric is well-defined}{ContractionOfLocalFIeldWellDefined}
Let $\mathcal{G} \to M$ be an LGB over a spacetime $M$ so that its LAB $\mathcal{g}$ admits an $\mathrm{Ad}$-invariant fibre metric $\kappa$, and let $\mathcal{P} \stackrel{\pi}{\to} M$ be a principal $\mathcal{G}$-bundle; also let $\mathrm{H}\mathcal{G}$ be a Yang-Mills connection on $\mathcal{G}$ and $A \in \Omega^1(\mathcal{P}; \pi^*\mathcal{g})$ be a connection 1-form on $\mathcal{P}$. Furthermore, let $\mleft( U_i \mright)_i$ be an open covering of $M$ so that there are subordinate gauges $s_i \in \Gamma\mleft(\mathcal{P}|_{U_i}\mright)$.

Then the top-degree form $\mathfrak{L}_{\mathrm{CYM}}[A] \in \Omega^{\mathrm{dim}(M)}(M; \mathbb{R})$, defined locally by
\bas
\mleft.\bigl(\mathfrak{L}_{\mathrm{CYM}}[A]\bigr)\mright|_{U_i}
&\coloneqq 
- \frac{1}{2} \kappa \mleft( F_{s_i} \stackrel{\wedge}{,} *F_{s_i} \mright),
\eas
is well-defined and independent of the choice of gauge, where $*$ is the Hodge star operator w.r.t.\ the spacetime metric of $M$.
\end{corollaries}

\begin{proof}
\leavevmode\newline
Let $s_j$ be another gauge corresponding to $U_j$ so that $U_i \cap U_j \neq \emptyset$, and we have a unique $\sigma_{ji} \in \Gamma\mleft( \mleft.\mathcal{G}\mright|_{U_i \cap U_j} \mright)$ with $s_i = s_j \cdot \sigma_{ji}$ on $U_i \cap U_j$. Then it follows by Thm.\ \ref{thm:LocalGaugeTrafoChangeGaugeFieldStrength} and the definition of the Hodge star operator (which is just a certain contraction w.r.t.\ $M$) that
\bas
\kappa \mleft( F_{s_i} \stackrel{\wedge}{,} *F_{s_i} \mright)
&=
\kappa \mleft( \mleft(\mathrm{Ad}_{\sigma_{ji}^{-1}}\circ F_{s_j} \mright) \stackrel{\wedge}{,} *\mleft( \mathrm{Ad}_{\sigma_{ji}^{-1}}\circ F_{s_j} \mright) \mright)
=
\kappa \mleft( F_{s_j} \stackrel{\wedge}{,} *F_{s_j} \mright),
\eas
using the $\mathrm{Ad}$-invariance of $\kappa$. This concludes the proof.
\end{proof}

\begin{definitions}{Curved Yang-Mills gauge theory}{CYMGTFinally}
Let $\mathcal{G} \to M$ be an LGB over a spacetime $M$ so that its LAB $\mathcal{g}$ admits an $\mathrm{Ad}$-invariant fibre metric $\kappa$, and let $\mathcal{P} \stackrel{\pi}{\to} M$ be a principal $\mathcal{G}$-bundle; furthermore let $\mathrm{H}\mathcal{G}$ be a Yang-Mills connection on $\mathcal{G}$ and $A \in \Omega^1(\mathcal{P}; \pi^*\mathcal{g})$ be a connection 1-form on $\mathcal{P}$.

Then the functional $A \mapsto \mathfrak{L}_{\mathrm{CYM}}[A]$ is called the \textbf{curved Yang-Mills Lagrangian}, where $\mathfrak{L}_{\mathrm{CYM}}[A]$ is defined in Cor.\ \ref{cor:ContractionOfLocalFIeldWellDefined}.
\end{definitions}

By construction, this Lagrangian is gauge-invariant.

\begin{theorems}{Gauge invariance of the curved Yang-Mills Lagrangian}{GaugeInvarianceOfLagrangian}
Let $\mathcal{G} \to M$ be an LGB over a spacetime $M$ so that its LAB $\mathcal{g}$ admits an $\mathrm{Ad}$-invariant fibre metric $\kappa$, and let $\mathcal{P} \stackrel{\pi}{\to} M$ be a principal $\mathcal{G}$-bundle; also let $\mathrm{H}\mathcal{G}$ be a Yang-Mills connection on $\mathcal{G}$ and $A \in \Omega^1(\mathcal{P}; \pi^*\mathcal{g})$ be a connection 1-form on $\mathcal{P}$.

Then we have
\bas
\mathfrak{L}_{\mathrm{CYM}}\mleft[ H^!A \mright]
&=
\mathfrak{L}_{\mathrm{CYM}}[A]
\eas
for all $H \in \sAut(\mathcal{P})$.
\end{theorems}

\begin{proof}
\leavevmode\newline
By Thm.\ \ref{thm:GaugeTrafoOfCurv} we have 
\bas
H^!F
&=
{\sAd_{\mathrm{pr}_2\circ\mleft(\sigma^H\mright)^{-1}}} \circ F,
\eas
where $\sigma^H \in C^\infty(\mathcal{P}; \mathcal{G})^{\mathcal{G}}$ is defined as in Prop.\ \ref{prop:GaugeTrafoAsBundleIsomIsASectionOfConjugationMaps} and $\mathrm{pr}_2: \pi^*\mathcal{G} \to \mathcal{G}$ is the projection onto the second component. Similar to the argument in the proof of Thm.\ \ref{thm:LocalGaugeTrafoChangeGauge} we have
\bas
\mleft( H^!F \mright)_s
&=
\mathrm{Ad}_{\mathrm{pr}_2\circ\mleft(\sigma^H\mright)^{-1} \circ s} \circ F_s
\eas
for a given gauge $s \in \Gamma(\mathcal{P}|_U)$, where $U$ is an open subset of $M$. With the same argument as in Cor.\ \ref{cor:ContractionOfLocalFIeldWellDefined} it follows that
\bas
\mathfrak{L}_{\mathrm{CYM}}\mleft[ H^!A \mright]
&=
\mathfrak{L}_{\mathrm{CYM}}[A].
\eas
\end{proof}

\begin{remarks}{Infinitesimal gauge invariance}{InfinitesimalGaugeInv}
As we have seen previously, the compatibility conditions (Def.\ \ref{def:NowReallyYangMillsConnectio}) were important to derive how $F$ behaves under modified right push-forwards. Recall Rem.\ \ref{rem:IntegratingKotovStrobl}, we have a corresponding formulation of infinitesimal gauge transformation and it is straight-forward to show that we achieve \textbf{infinitesimal gauge invariance} as in the classical theory. However, as argued by Alexei Kotov and Thomas Strobl in \cite{CurvedYMH} (for a concise summary; \cite{MyThesis} for an extended introduction, or alternatively \cite{My1stpaper} for a simplified language) the corresponding infinitesimal gauge invariance just needs the infinitesimal compatibility conditions (Def.\ \ref{def:YangMillsConnection}).
\end{remarks}

\subsection{Examples}\label{CYMExamples}

Let us provide some examples of this theory, especially also recall Ex.\ \ref{ex:ClassicalYangMillsConnection} and the labels introduced there. However, before we do so, it is natural to ask whether or not there is a transformation keeping the Lagrangian invariant, but non-trivially transforming data like $\nabla^{\mathrm{YM}}$ so that it may become flat, and we may end up at a classical theory describing the same physics. In fact, there is such a transformation which we have studied in \cite{MyThesis} (and \cite{My1stpaper} which is just formulated for LABs so that it may be easier for the reader to follow given the context here). These transformations are simply called \textbf{field redefinitions}, but given the time of their development, they just focus on infinitesimal gauge invariance and the underlying infinitesimal compatibility conditions.

\begin{theorems}{Invariance w.r.t.\ field redefinitions, \newline \cite[\S 4, Def.\ 4.5.1 and 4.7.10, Thm.\ 4.7.13]{MyThesis}, \cite[\S 3, Thm.\ 3.6]{My1stpaper}}{FieldRedefStuff}
Let $\mathcal{G} \to M$ be an LGB over a spacetime $M$ so that its LAB $\mathcal{g}$ admits an $\mathrm{Ad}$-invariant fibre metric $\kappa$, and let $\mathcal{P} \stackrel{\pi}{\to} M$ be a principal $\mathcal{G}$-bundle; furthermore let $\mathrm{H}\mathcal{G}$ be a Yang-Mills connection on $\mathcal{G}$ and $A \in \Omega^1(\mathcal{P}; \pi^*\mathcal{g})$ be a connection 1-form on $\mathcal{P}$. We also define \textbf{field redefinitions w.r.t.\ $\lambda \in \Omega^1(M; \mathcal{g})$} by
\bas
\widetilde{A}^\lambda
&\coloneqq
A + \pi^!\lambda,
\\
\widetilde{\nabla^{\mathrm{YM}}}^\lambda
&\coloneqq
\nabla^{\mathrm{YM}} - \mathrm{ad}_\lambda,
\\
\widetilde{\zeta}^\lambda
&\coloneqq
\zeta
	- \mathrm{d}^{\nabla^{\mathrm{YM}}}\lambda + \frac{1}{2} \mleft[ \lambda \stackrel{\wedge}{,} \lambda \mright]_{\mathcal{g}}.
\eas

Then we have that $\widetilde{\nabla^{\mathrm{YM}}}^\lambda$ and $\widetilde{\zeta}^\lambda$ satisfy the infinitesimal compatibility conditions of Def.\ \ref{def:YangMillsConnection} and
\bas
\widetilde{F}^\lambda
&\coloneqq
\mathrm{d}^{\pi^*\widetilde{\nabla^{\mathrm{YM}}}^\lambda} \widetilde{A}^\lambda
	+ \frac{1}{2} \mleft[ \widetilde{A}^\lambda \stackrel{\wedge}{,} \widetilde{A}^\lambda \mright]_{\pi^*\mathcal{g}}
	+ \pi^!\widetilde{\zeta}^\lambda
\equiv
F
\eas
for all $\lambda \in \Omega^1(M; \mathcal{G})$. That is, the Lagrangian $\mathfrak{L}_{\mathrm{CYM}}$ is invariant under these field redefinitions, and infinitesimal gauge invariance is preserved.
\end{theorems}

In fact, these field redefinitions are not just any transformation preserving the physics. As pointed out in \cite{MyThesis}, curved Yang-Mills gauge theory is a reformulation of classical gauge theory in such a way that gauge theory is form-invariant w.r.t.\ the field redefinitions, similar to how one reformulates Classical Mechanics so that it is form-invariant w.r.t.\ Galilei or Lorentz transformations. This reformulation introduces extra terms, $\zeta$ and connection 1-forms of $\nabla$, that is, every classical gauge theory is equivalent to a possibly curved theory with a non-zero $\zeta$ contributing to the field strength.

As a next step one can then investigate whether there is a curved Yang-Mills gauge theory which cannot be transformed to a classical infinitesimal gauge theory via a field redefinition, this has be done in the mentioned references whose results will be used in the following examples.

\begin{examples}{Classical examples}{ClassicalYMTheoriesRecovered}
By Ex.\ \ref{ex:ClassicalYangMillsConnection}, we just recover the classical formalism of gauge theory if having a classical principal bundle $P$ on which a trivial LGB $\mathcal{G} = M \times G$ acts on the right, where $G$ is a Lie group with Lie algebra $\mathfrak{g}$. $\mathrm{H}\mathcal{G}$ is the canonical flat Yang-Mills connection, $\nabla^{\mathrm{YM}}=\nabla^{\mathrm{cYM}}$ is the canonical flat connection on $\mathcal{g} = M \times \mathfrak{g}$, and $\zeta \equiv 0$. Then every example known in the classical formalism carries over.

Furthermore, starting with such a classical example, by \cite[\S 6, Cor.\ 6.2]{My1stpaper} we know that if $\mathfrak{g}$ has a non-zero centre (and if $M$ is at least three-dimensional), then one can always add a centre-valued $\zeta$ to this gauge theory, transforming it into a flat pre-classical gauge theory, $\nabla^{\mathrm{YM}} = \nabla^{\mathrm{pYM}}$, with non-trivial $\zeta$. If $\mathrm{d}^{\nabla^{\mathrm{pYM}}} \zeta \neq 0$, then there is no $\lambda \in \Omega^1(M; \mathcal{g})$ with $\widetilde{\zeta}^\lambda = 0$.
\end{examples}

This shows that there are examples with non-trivial $\zeta$, stable w.r.t.\ field redefinitions. However, as shown in \cite[\S 5.2, Thm.\ 5.16]{My1stpaper} (or alternatively \cite[\S 5.1.5, Thm.\ 5.1.33]{MyThesis}), if $M$ is contractible, then there is $\lambda \in \Omega^1(M; \mathcal{g})$ so that $\widetilde{\nabla^{\mathrm{YM}}}^\lambda$ is flat. Thus, if $M$ is contractible, then every curved Yang-Mills Lagrangian is equivalent to a Lagrangian of a pre-classical gauge theory ($\zeta$ may be non-vanishing). This may be due to that $\mathcal{G}$ is a trivial fibre bundle, and then the LGB action breaks down to Lie group action; recall all our discussions related to classical principal bundles equipped with an LGB action of a trivial LGB as in the previous example. 
Hence we conjecture:

\begin{conjectures}{Trivial LGBs are associated with pre-classical theories}{TrivialLGBIsPreclassical}
Let $\mathcal{G} \to M$ be a trivial LGB over a spacetime $M$ so that its LAB $\mathcal{g}$ admits an $\mathrm{Ad}$-invariant fibre metric $\kappa$, and let $\mathcal{P} \stackrel{\pi}{\to} M$ be a principal $\mathcal{G}$-bundle; furthermore let $\mathrm{H}\mathcal{G}$ be a Yang-Mills connection on $\mathcal{G}$ and $A \in \Omega^1(\mathcal{P}; \pi^*\mathcal{g})$ be a connection 1-form on $\mathcal{P}$.

Then there is a $\lambda \in \Omega^1(M; \mathcal{g})$ so that $\widetilde{\nabla^{\mathrm{YM}}}^\lambda$ is flat.
\end{conjectures}

However, we will not have time to attempt proving this, but as a first step one may integrate the field redefinitions to a redefinition of the horizontal bundle on the LGB $\mathcal{G}$ itself, not just of the connection on its LAB.

Obviously, non-trivial LGBs are not used in the classical formalism, and by the previous discussion one may ask if there is a (global) example with a curved Yang-Mills connection starting with a non-trivial LGB. In fact, there is one related to a Hopf fibration, as already argued in \cite[\S 5.3, Ex.\ 5.18]{My1stpaper} and \cite[\S 5.1.5, Ex.\ 5.1.35]{MyThesis}.

\begin{examples}{Hopf fibration $\mathds{S}^7 \to \mathds{S}^4$ giving rise to a curved Yang-Mills gauge theory}{HopfEx}
By Ex.\ \ref{ex:NablaYMAsAdjointConnection} (recall also \ref{ex:OurVeryImportantExample}) we know that $c_G(P)$ is an LGB admitting a Yang-Mills connection $\nabla^{\mathrm{YM}}$, where $P$ is a classical principal $G$-bundle, and $G$ a Lie group with Lie algebra $\mathfrak{g}$. We define $\mathcal{P}$ to be $c_G(P)$ itself now, but $A$ is arbitrary and does not need to be related to the horizontal distribution of $\nabla^{\mathrm{YM}}$.

Now define $P$ as the Hopf bundle
\begin{center}
	\begin{tikzcd}
		\mathrm{SU}(2) \cong \mathds{S}^3 \arrow{r}	& \mathds{S}^7 \arrow{d} \\
			& \mathds{S}^4
	\end{tikzcd}
\end{center}
where $\mathds{S}^n$ ($n \in \mathbb{N}$) denotes the $n$-dimensional sphere.
For the inner bundle
\bas
\mathcal{P}
\coloneqq
\mathcal{G}
&\coloneqq
c_{\mathrm{SU}(2)}(P)
\eas
we then have that its associated Yang-Mills connection $\nabla^{\mathrm{YM}}$, as constructed previously in Ex.\ \ref{ex:NablaYMAsAdjointConnection}, is non-flat, and there is also no $\lambda \in \Omega^1(M; \mathrm{ad}(P))$ such that $\widetilde{\nabla^{\mathrm{YM}}}^\lambda$ is flat. As argued in the mentioned references, this is due to a discussion in \cite[Example 7.3.20; page 287]{mackenzieGeneralTheory}: $\mathds{S}^4$ is not contractible, but simply connected. Due to the fact that $\mathrm{SU}(2)$ is semisimple and following Ex.\ \ref{ex:NablaYMAsAdjointConnection}, a flat $\widetilde{\nabla^{\mathrm{YM}}}^\lambda$ would be equivalent to a flat connection on $P$, inducing a trivialization for $P$ since $\mathds{S}^4$ is simply connected. But $P$ cannot be trivial due to it being a Hopf fibration, so that we arrive at a contradiction, proving the statement. This argument is independent of the field redefinitions, and it shows that all possible Yang-Mills connections are curved here (globally).

An $\mathrm{Ad}$-invariant fibre metric exists clearly, so that we have all ingredients to define a truly curved Yang-Mills gauge theory with Lagrangian as given in Def.\ \ref{def:CYMGTFinally}.
\end{examples}

\section{Future prospects}\label{conclusions}

We already summarized the results given here in the introduction. So, let us discuss the future prospects of this theory. Since a curved connection $\nabla^{\mathrm{YM}}$ only exists globally, one might say that this theory may be just a toy model for physics, but thinking of symplectic geometry, the global aspects may be at least interesting in mathematics; as already mentioned here several times, there are papers from other authors (than the ones directly working with curved gauge theories) with somewhat similar subjects, \cite{LAURENTGENGOUXStienonXuMultiplicativeForms}, \cite{OtherPreprintAboutConnection}, and \cite{FernandesMarcutMultiplicativeForms}. Alternatively, gauge theory is also a theory about connections, so one could investigate what the sense of connection and curvature given here implies for the plethora of other theories based on connections.

As a next step one could now define minimal couplings. Given a vector bundle $V \to M$, equipped with a LGB representation $\rho: \mathcal{G} \to \mathrm{Aut}(V)$, we should be able to define associated vector bundles $\mathcal{V} \coloneqq ( \mathcal{P} \times_M V ) \Big/ \mathcal{G}$, as we already mentioned before. W.r.t.\ a local gauge $s$ of $\mathcal{P}$ every section $\Psi \in \Gamma(\mathcal{V})$ has locally the form
\bas
[s, \psi_s],
\eas
where $\psi$ is a local section of $V$ and $[ \cdot, \cdot ]$ denotes the equivalence class related to the construction in $\mathcal{V}$. Then one can define the \textbf{minimal coupling $\nabla^A \Psi$} locally by
\bas
\mleft[  s,
	\nabla \psi_s + \rho_*(A_s)(\psi_s)
\mright]
\eas 
where $\rho_*$ is the LAB representation inherited by $\rho$, and $\nabla$ is a vector bundle connection with
\bas
\nabla \circ \rho_*
&=
\rho_* \circ \nabla^{\mathrm{YM}},
\\
R_\nabla
&=
\rho_* \circ \zeta.
\eas
Observe the similarity with the infinitesimal compatibility conditions (Def.\ \ref{def:YangMillsConnection}); these conditions should precisely lead to that that $\nabla^A \Psi$ is well-defined and independent of the choice of $s$, also making use of an analogue of Lemma \ref{lem:FieldRedefViaLGBSection}. After showing this one could try to construct all the other obvious analogues of the classical formalism.

However, as already highlighted in the introduction, Alexei Kotov and Thomas Strobl have constructed an infinitesimal gauge theory based on Lie algebroids, and the anchor of the involved Lie algebroid is describing the minimal coupling. Thus, as a next step one may actually attempt to integrate this infinitesimal gauge theory. In the context of this work, this gauge theory should probably be based on principal $\mathcal{G}$-bundles where $\mathcal{G}$ is now a Lie groupoid and not a Lie group. Such principal bundles were already constructed in \cite[beginning of \S 5.7, page 144f.]{GroupoidBasedPrincipalBundles}. The corresponding Lie groupoid should have two "projections", one over $M$ and one other over the target manifold $N$, the target of the Higgs field $\Xi$.

\begin{center}
	\begin{tikzcd}
		& \mathcal{G} \arrow{dr}\arrow{dl} & \\
		M \arrow{rr}{\Xi} & & N
	\end{tikzcd}
\end{center}

In other words, $\mathcal{G}$ is a Lie groupoid over the graph of $\Xi$, or to be more precise, over the graph of an evaluation map $(p, \Xi, A) \to \Xi(p)$, where $p \in M$; a formulation using such an evaluation map is already formulated in \cite{MyThesis}, but only regarding the infinitesimal gauge theory. I may start with such a research project now, and if successful, such a type of gauge theory should actually lead to a notion of a principal bundle for Yang-Mills-Higgs gauge theories. So, the involved principal bundle is not only describing the Yang-Mills part, we would achieve a principal bundle analogue for the whole Yang-Mills-Higgs Lagrangian. Last, such a Lie groupoid based gauge theory may help to answer whether or not Alexei Kotov's and Thomas Strobl's theory provide new physical examples, especially examples "stable" w.r.t.\ the field redefinitions constructed in \cite{MyThesis}.

\textbf{Acknowledgements:} I want to thank Siye Wu, Camille Laurent-Gengoux, Mark John David Hamilton and Alessandra Frabetti for their great help and support in making this paper. Also big thanks to my mother, father, Dennis, Gregor, Marco, Nico, Jakob, Kathi, Konstantin, Lukas, Locki, Gareth, Philipp, Ramona, Annerose, Michael, Maxim, Anna, and my girlfriend Rachelle for all their love.

\textbf{Funding:} The paper is part of my post-doc fellowship at the National Center for Theoretical Sciences (NCTS), which is why I also want to thank the NCTS.

%%%%%%%%%%%%%%%%%%%%%%%%%%%% Hier beginnt der Anhang %%%%%%%%%%%%%%%%%%%%%%%%%%%%

%\newpage

%\listoftables % Tabellenverzeichnis

%\listoffigures %Abbildungsverzeichnis

\appendix
\setcounter{equation}{0}
\renewcommand{\theequation}{\Alph{section}.\arabic{equation}} %Reset first and then add section to number

\renewcommand\refname{List of References}

%\begin{thebibliography}{99}
%\bibitem[I]{Anl01} \url{http://adsabs.harvard.edu/abs/1971ApJ...170..319D}, Datum: 01.11.2014
%\end{thebibliography}

%\printbibliography 
\bibliography{Literatur}
\bibliographystyle{unsrt}

%\newpage\thispagestyle{empty}\hspace{1em}\newpage

\section{Double tangent bundle and its canonical flip map}\label{DoubleTangentFlip}

We will follow \cite[\S 9.6, page 363]{mackenzieGeneralTheory}. For a smooth manifold $M$ we denote the projection of its tangent bundle by $\pi_{\mathrm{T}M}: \mathrm{T}M \to M$; similarly we have the projection of the \textbf{double tangent bundle $\pi_{\mathrm{TT}M}: \mathrm{TT}M \to \mathrm{T}M$}, the tangent bundle of $\mathrm{T}M$. However, there is also $\mathrm{D}\pi_{\mathrm{T}M}: \mathrm{TT}M \to \mathrm{T}M$, and in fact there is another vector bundle structure on $\mathrm{TT}M$ rendering $\mathrm{D}\pi_{\mathrm{T}M}$ a projection; see \textit{e.g.}\ \cite[\S 3.4 \textit{et seq.}; page 110ff.]{mackenzieGeneralTheory}. Let us give a very rough sketch:

Let $\xi, \eta \in \mathrm{TT}M$ with 
\bas
\mathrm{D}_{X_0}\pi_{\mathrm{T}M}(\xi)
&=
\mathrm{D}_{Y_0}\pi_{\mathrm{T}M}(\eta)
\eqqcolon
\omega,
\eas
where $X_0 \coloneqq \pi_{\mathrm{TT}M}(\xi)$ and $Y_0 \coloneqq \pi_{\mathrm{TT}M}(\eta)$,
and due to the fact that $\mathrm{D}\pi_{\mathrm{T}M}$ is a vector bundle morphism over $\pi_{\mathrm{T}M}$ we get
\bas
p
&= 
\pi_{\mathrm{T}M} (X_0)
=
\pi_{\mathrm{T}M}(Y_0),
\eas
where $\pi_{\mathrm{T}M}(\omega) \eqqcolon p$.
Thus, one can take curves $X,Y: I \to \mathrm{T}M$ ($I \subset \mathbb{R}$ an open interval around 0) with
\bas
X(0)
&=
X_0,
&
\mleft.\frac{\mathrm{d}}{\mathrm{d}t}\mright|_{t=0} X
&=
\xi,
\\
Y(0)
&=
Y_0,
&
\mleft.\frac{\mathrm{d}}{\mathrm{d}t}\mright|_{t=0} Y
&=
\eta,
\eas
such that
\bas
\pi_{\mathrm{T}M} \circ X = \pi_{\mathrm{T}M} \circ Y,
\eas
because the condition on $\xi$ and $\eta$ imply on the base paths $\pi_{\mathrm{T}M} \circ X, \pi_{\mathrm{T}M} \circ Y: I \to M$ that
\bas
(\pi_{\mathrm{T}M}\circ X)(0)
&=
p
=
(\pi_{\mathrm{T}M} \circ Y)(0),
\\
\mleft.\frac{\mathrm{d}}{\mathrm{d}t}\mright|_{t=0} \bigl( \pi_{\mathrm{T}M} \circ X \bigr)
&=
\mathrm{D}_{X_0}\pi_{\mathrm{T}M}(\xi)
=
\omega
=
\mathrm{D}_{Y_0}\pi_{\mathrm{T}M}(\eta)
=
\mleft.\frac{\mathrm{d}}{\mathrm{d}t}\mright|_{t=0} \bigl( \pi_{\mathrm{T}M} \circ Y \bigr).
\eas
%So, one takes one base path satisfying those, and then $f$ and $h$ are the lifts this base path, which is possible because $\xi$ and $\eta$ just differ along the vertical structures (base points in same fibre, and sharing the same projection under $\mathrm{D}\pi$).
Then the addition and scalar multiplication with $\lambda \in \mathbb{R}$ for $\mathrm{TT}M \stackrel{\mathrm{D}\pi_{\mathrm{T}M}}{\to} \mathrm{T}M$ is defined by
\bas
\xi \RPlus \eta
&\coloneqq
\mleft.\frac{\mathrm{d}}{\mathrm{d}t}\mright|_{t=0} (X + Y),
\\
\lambda \boldsymbol{\cdot} \xi
&\coloneqq
\mleft.\frac{\mathrm{d}}{\mathrm{d}t}\mright|_{t=0} (\lambda X),
\eas
where the addition of curves is well-defined because of $\pi_{\mathrm{T}M} \circ X = \pi_{\mathrm{T}M} \circ Y$ which implies $\pi_{\mathrm{T}M}\circ (X+Y)= \pi_{\mathrm{T}M} \circ X = \pi_{\mathrm{T}M} \circ Y$; so, one can take the sum of the curves and
\bas
\mathrm{D}\pi_{\mathrm{T}M}(\xi \RPlus \eta)
&=
\mleft.\frac{\mathrm{d}}{\mathrm{d}t}\mright|_{t=0} \bigl( \underbrace{\pi_{\mathrm{T}M} \circ (X+Y)}_{= \pi_{\mathrm{T}M}\circ X} \bigr)
=
\mathrm{D}_{X_0}\pi_{\mathrm{T}M}(\xi)
=
\omega.
\eas
The operations of the linear structure in $\mathrm{TT}M \stackrel{\pi_{\mathrm{TT}M}}{\to} \mathrm{T}M$ is still denoted in the same manner as usual, and by definition one also gets
\bas
\pi_{\mathrm{TT}M}(\xi \RPlus \eta)
&=
\pi_{\mathrm{TT}M}(\xi)
	+ \pi_{\mathrm{TT}M}(\eta),
\\
\pi_{\mathrm{TT}M}(\lambda \boldsymbol{\cdot} \xi)
&=
\lambda ~ \pi_{\mathrm{TT}M}(\xi).
\eas

%As we already did frequently, a tangent vector $\xi \in \mathrm{T}_X\mathrm{T}M$ at $X \in \mathrm{T}_pM$ ($p \in M$) can be written as
%\bas
%\xi
%&=
%\underbrace{\mathrm{D}_X\pi_{\mathrm{T}M}(\xi)}_{\eqqcolon \omega}
	%+ \underbrace{\xi - \mathrm{D}_X\pi_{\mathrm{T}M}(\xi)}_{\eqqcolon \xi^v}
%=
%\omega
	%+ \xi^v,
%\eas
%where $\xi^v$ is a vertical vector of $\mathrm{T}M \stackrel{\pi_{\mathrm{T}M}}{\to} M$, and by definition $\omega \in \mathrm{T}_pM$. For another tangent vector $\eta \in \mathrm{T}_Y\mathrm{T}M$ ($Y \in \mathrm{T}_pM$) with $\mathrm{D}_Y\pi_{\mathrm{T}M}(\eta) = \omega$, \textit{i.e.}\ $\eta \in \mleft(\mathrm{D}\pi_{\mathrm{T}M}\mright)^{-1}(\{\omega\})$, we introduce a similar notation, and we define the addition\ $\RPlus$\ and scalar multiplication $\boldsymbol{\cdot}$ of $\mathrm{D}\pi_{\mathrm{T}M}: \mathrm{TT}M \to \mathrm{T}M$ by
%\bas
%\xi \RPlus \eta
%&\coloneqq
%\omega
	%+ \xi^v + \eta^v,\\
%\alpha \boldsymbol{\cdot} \xi
%&\coloneqq
%\omega
	%+ \alpha \xi^v
%\eas
%for all $\alpha \in \mathbb{R}$, that is, an affine (and prolonged along the fibres of $\mathrm{T}M$) vertical structure. 
In total, we have a double vector bundle given by the following commuting diagram
\be\label{DoubleTangentAsDiagram}
	\begin{tikzcd}
		 \mathrm{TT}M \arrow{r}{\mathrm{D}\pi_{\mathrm{T}M}} \arrow{d}{\pi_{\mathrm{TT}M}} & \mathrm{T}M \arrow{d}{\pi_{\mathrm{T}M}} \\
		\mathrm{T}M \arrow[r, "\pi_{\mathrm{T}M}"]& M
	\end{tikzcd}
\ee
\textit{i.e.}\ each horizontal and vertical line is a vector bundle so that the horizontal and vertical scalar multiplications on $\mathrm{TT}M$ commute; see \textit{e.g.}~\cite[\S 3ff.]{Highervectorbundles}\ or \cite[\S 9.1, page 340ff.]{mackenzieGeneralTheory}\ for a definition on double vector bundles in general.

Now observe that the flow $X$ of $\xi$ has values in $\mathrm{T}M$, that is, for all $t \in I$ we have a curve $\alpha_t: J \to M$ ($J$ another open interval containing 0), $s \mapsto \alpha_t(s)$, so that 
\bas
\alpha_t(0) &= \pi_{\mathrm{T}M}\bigl( X(t) \bigr),\\
\mleft.\frac{\mathrm{d}}{\mathrm{d}s}\mright|_{s=0} \alpha_t &= X(t),
\eas
and the first equation implies
\bas
\mleft.\frac{\mathrm{d}}{\mathrm{d}t}\mright|_{t=0} \alpha_t(0)
&=
\mathrm{D}_{X_0}\pi_{\mathrm{T}M}(\xi)
=
\omega.
\eas
So, in total we have for all $\xi \in \mathrm{T}\mathrm{T}M$ a smooth map $\alpha: I \times J \to M$, $(t, s) \mapsto \alpha(t,s) = \alpha_t(s)$, such that
\bas
\alpha(0,0) &= p = \mleft(\pi_{\mathrm{T}M} \circ \pi_{\mathrm{TT}M}\mright)(\xi) 
= \mleft(\pi_{\mathrm{T}M} \circ \mathrm{D}\pi_{\mathrm{T}M}\mright)(\xi),
\eas
\bas
\mleft.\frac{\mathrm{d}}{\mathrm{d}t}\mright|_{t=0} \alpha(t,0)
&=
\omega
=
\mathrm{D}_{X_0}\pi_{\mathrm{T}M}(\xi),&
\mleft.\frac{\mathrm{d}}{\mathrm{d}s}\mright|_{s=0} \alpha(0,s) 
&= 
X_0
=
\pi_{\mathrm{TT}M}(\xi),
\eas
\bas
\mleft.\frac{\mathrm{d}}{\mathrm{d}t}\mright|_{t=0} \mleft.\frac{\mathrm{d}}{\mathrm{d}s}\mright|_{s=0} \alpha &= \xi.
\eas
As for tangent vectors, the class $[\alpha]$ of all such $\alpha$ uniquely describes $\xi$, in fact giving rise to an equivalence relation so that $[\alpha]$ is an equivalence class.

In the context of Schwarz's Theorem one may find it natural to define the \textbf{canonical involution (or flip) on $\mathrm{TT}M$} as a map $S: \mathrm{TT}M \to \mathrm{TT}M$ by
\ba\label{definitionOfFlipMap}
S(\xi)
&\coloneqq
\mleft.\frac{\mathrm{d}}{\mathrm{d}s}\mright|_{s=0} \mleft.\frac{\mathrm{d}}{\mathrm{d}t}\mright|_{t=0} \alpha.
\ea
By \cite[\S 9.6, Thm.\ 9.6.1, page 363; but without proof]{mackenzieGeneralTheory} one has that $S$ is an isomorphism of double vector bundles with certain "special" properties; we will not need the general definition of these. What we need of this, is that $S$ is a base-preserving vector bundle isomorphism as both maps, from $\mathrm{TT}M \stackrel{\mathrm{D}\pi_{\mathrm{T}M}}{\to} \mathrm{T}M$ to $\mathrm{TT}M \stackrel{\pi_{\mathrm{TT}M}}{\to} \mathrm{T}M$ and vice versa. Let us prove this, by starting with showing the well-definedness of $S$:

W.l.o.g.\ we can put $J = I$ for simplicity. By construction, the equivalence class $[\beta]$ of $S(\xi)$ is represented by $\beta: I^2 \to M$ given by $\beta(t, s) \coloneqq \alpha(s,t)$ with 
\bas
\beta(0,0) &= p,
\\
\mathrm{D}\pi_{\mathrm{T}M}\bigl( S(\xi) \bigr) 
&=
\mleft.\frac{\mathrm{d}}{\mathrm{d}t}\mright|_{t=0} \beta(t, 0)
=
\mleft.\frac{\mathrm{d}}{\mathrm{d}t}\mright|_{t=0} \alpha(0, t)
=
\pi_{\mathrm{TT}M}(\xi)
= 
X_0,
\\
\pi_{\mathrm{TT}M}\bigl(S(\xi)\bigr)
&=
\mleft.\frac{\mathrm{d}}{\mathrm{d}s}\mright|_{s=0} \beta(0, s)
=
\mleft.\frac{\mathrm{d}}{\mathrm{d}s}\mright|_{s=0} \alpha(s, 0)
=
\mathrm{D}\pi_{\mathrm{T}M}(\xi)
=
\omega,
\\
S(\xi)
&=
\mleft.\frac{\mathrm{d}}{\mathrm{d}t}\mright|_{t=0} \mleft.\frac{\mathrm{d}}{\mathrm{d}s}\mright|_{s=0} \beta.
\eas
With this, we can express $S(\xi)$ in coordinates: Fix local coordinates $\mleft(x^i\mright)_i$ on $M$, then $\mleft(\pi_{\mathrm{T}M}^*x^i\mright)_i$ and $\mleft(\mathrm{d}x^i\mright)_i$ are local coordinates on $\mathrm{T}M$. Then by chain rule
\bas
\xi\mleft(\pi_{\mathrm{T}M}^*x^i\mright)
&=
\xi\mleft(x^i \circ \pi_{\mathrm{T}M} \mright)
=
\mathrm{d}_p x^i\bigl( \mathrm{D}\pi_{\mathrm{T}M}(\xi) \bigr)
=
\omega^i,
\eas
and
\bas
\xi\mleft( \mathrm{d}x^i \mright)
&=
\mleft( \mleft.\frac{\mathrm{d}}{\mathrm{d}t}\mright|_{t=0} \mleft.\frac{\mathrm{d}}{\mathrm{d}s}\mright|_{s=0} \alpha \mright) \mleft( \mathrm{d}x^i \mright)
=
\mleft.\frac{\mathrm{d}}{\mathrm{d}t}\mright|_{t=0} \underbrace{\mleft(
	\mathrm{d}x^i \circ \mleft.\frac{\mathrm{d}}{\mathrm{d}s}\mright|_{s=0} \alpha
\mright)}
	_{= \mleft(\mleft.\frac{\mathrm{d}}{\mathrm{d}s}\mright|_{s=0} \alpha \mright)\mleft( x^i \mright)}
=
\mleft.\frac{\mathrm{d}}{\mathrm{d}t}\mright|_{t=0} \mleft.\frac{\mathrm{d}}{\mathrm{d}s}\mright|_{s=0} \alpha^i,
\eas
thus we have in total
\ba\label{CoordinateExpressionOfTTM}
\xi
&=
\omega^i \mleft.\frac{\partial}{\partial \mleft(\pi_{\mathrm{T}M}^*x^i\mright)}\mright|_{X_0}
	+ \mleft(\mleft.\frac{\mathrm{d}}{\mathrm{d}t}\mright|_{t=0} \mleft.\frac{\mathrm{d}}{\mathrm{d}s}\mright|_{s=0} \alpha^i \mright) \mleft.\frac{\partial}{\partial \mleft(\mathrm{d}x^i\mright)}\mright|_{X_0}.
\ea
Similarly we can proceed with $S(\xi)$ and then apply the "classical" Schwarz' Theorem on the derivatives of $\alpha^i$ (recall the properties of $\beta$), so that we get 
\ba\label{CoordinateExprOfFlippy}
S(\xi)
&=
X_0^i \mleft.\frac{\partial}{\partial \mleft(\pi_{\mathrm{T}M}^*x^i\mright)}\mright|_{\omega}
	+ \mleft(\mleft.\frac{\mathrm{d}}{\mathrm{d}t}\mright|_{t=0} \mleft.\frac{\mathrm{d}}{\mathrm{d}s}\mright|_{s=0} \beta^i \mright) \mleft.\frac{\partial}{\partial \mleft(\mathrm{d}x^i\mright)}\mright|_{\omega}
\nonumber
\\
&=
X_0^i \mleft.\frac{\partial}{\partial \mleft(\pi_{\mathrm{T}M}^*x^i\mright)}\mright|_{\omega}
	+ \mleft(\mleft.\frac{\mathrm{d}}{\mathrm{d}s}\mright|_{s=0} \mleft.\frac{\mathrm{d}}{\mathrm{d}t}\mright|_{t=0} \alpha^i \mright) \mleft.\frac{\partial}{\partial \mleft(\mathrm{d}x^i\mright)}\mright|_{\omega}
\nonumber
\\
&=
X_0^i \mleft.\frac{\partial}{\partial \mleft(\pi_{\mathrm{T}M}^*x^i\mright)}\mright|_{\omega}
	+ \mleft(\mleft.\frac{\mathrm{d}}{\mathrm{d}t}\mright|_{t=0} \mleft.\frac{\mathrm{d}}{\mathrm{d}s}\mright|_{s=0} \alpha^i \mright) \mleft.\frac{\partial}{\partial \mleft(\mathrm{d}x^i\mright)}\mright|_{\omega}.
\ea
It is now clear that $S(\xi)$ is independent of the choice of $\alpha$ and just depends on $[\alpha]$. Furthermore, due to what we have shown for $\beta$, it is also clear that $S$ is a base-preserving map either from $\mathrm{TT}M \stackrel{\mathrm{D}\pi_{\mathrm{T}M}}{\to} \mathrm{T}M$ to $\mathrm{TT}M \stackrel{\pi_{\mathrm{TT}M}}{\to} \mathrm{T}M$ or vice versa. 
%Let $[\gamma] \in \mathrm{TT}M$, then by the same calculation as for the properties of $\beta$ we have $S([\tau]) = [\gamma]$, where $\tau(t, s) \coloneqq \gamma(s, t)$; thence, surjectivity of $S$ follows. Similarly, again by the properties of $\beta(t, s)$ above, if we have $S([\alpha]) = S\mleft( \mleft[ \alpha^\prime \mright] \mright)$, then we clearly get $[\alpha] = [\alpha^\prime]$, and hence injectivity. Alternatively, 
Bijectivity follows by $S\circ S = \mathds{1}_{\mathrm{TT}M}$.

It is only left to show that we have linearity for $S$ as a map from $\mathrm{TT}M \stackrel{\mathrm{D}\pi_{\mathrm{T}M}}{\to} \mathrm{T}M$ to $\mathrm{TT}M \stackrel{\pi_{\mathrm{TT}M}}{\to} \mathrm{T}M$ and vice versa. For this we use the derived coordinate expressions. Let $\eta \in \mathrm{TT}M$ be defined as before with associated area function $\gamma$ such that $\eta = [\gamma]$. Then one can derive by using what we have shown in the discussion about $\mathrm{TT}M \stackrel{\mathrm{D}\pi_{\mathrm{T}M}}{\to} \mathrm{T}M$ as vector bundle that
\ba\label{LinearStructureOfProlongInCoordinates}
\lambda \boldsymbol{\cdot} \xi \RPlus \kappa \boldsymbol{\cdot} \eta
&=
\omega^i \mleft.\frac{\partial}{\partial \mleft(\pi_{\mathrm{T}M}^*x^i\mright)}\mright|_{\lambda X_0+\kappa Y_0}
	+ \mleft(\mleft.\frac{\mathrm{d}}{\mathrm{d}t}\mright|_{t=0} \mleft.\frac{\mathrm{d}}{\mathrm{d}s}\mright|_{s=0} \mleft(\lambda \alpha^i + \kappa \gamma^i\mright)\mright) \mleft.\frac{\partial}{\partial \mleft(\mathrm{d}x^i\mright)}\mright|_{\lambda X_0+ \kappa Y_0}
\ea
for all $\lambda, \kappa \in \mathbb{R}$,
and so
\bas
S(\lambda \boldsymbol{\cdot} \xi \RPlus \kappa \boldsymbol{\cdot} \eta)
&=
\mleft( \lambda X_0^i + \kappa Y_0^i \mright) \mleft.\frac{\partial}{\partial \mleft(\pi_{\mathrm{T}M}^*x^i\mright)}\mright|_{\omega}
	+ \mleft(\mleft.\frac{\mathrm{d}}{\mathrm{d}t}\mright|_{t=0} \mleft.\frac{\mathrm{d}}{\mathrm{d}s}\mright|_{s=0} \mleft(\lambda \alpha^i + \kappa \gamma^i\mright) \mright) \mleft.\frac{\partial}{\partial \mleft(\mathrm{d}x^i\mright)}\mright|_{\omega}
\\
&=
\lambda S(\xi) + \kappa S(\eta).
\eas
In the same fashion, let $\zeta = [\delta] \in \mathrm{T}_{X_0}\mathrm{T}M$ with $\mathrm{D}_{X_0}\pi_{\mathrm{T}M}(\zeta) \eqqcolon \varphi$, then $\lambda \xi + \kappa \zeta$ is just the typical sum of tangent vectors and we get
\bas
S(\lambda \xi + \kappa \zeta)
&=
X_0^i \mleft.\frac{\partial}{\partial \mleft(\pi_{\mathrm{T}M}^*x^i\mright)}\mright|_{\lambda \omega + \kappa \varphi}
	+ \mleft(\mleft.\frac{\mathrm{d}}{\mathrm{d}t}\mright|_{t=0} \mleft.\frac{\mathrm{d}}{\mathrm{d}s}\mright|_{s=0} \mleft(\lambda\alpha^i + \kappa\delta^i \mright) \mright) \mleft.\frac{\partial}{\partial \mleft(\mathrm{d}x^i\mright)}\mright|_{\lambda \omega + \kappa \varphi}
\\
&=
\lambda \boldsymbol{\cdot} \mleft(
X_0^i \mleft.\frac{\partial}{\partial \mleft(\pi_{\mathrm{T}M}^*x^i\mright)}\mright|_{\omega}
	+ \mleft(\mleft.\frac{\mathrm{d}}{\mathrm{d}t}\mright|_{t=0} \mleft.\frac{\mathrm{d}}{\mathrm{d}s}\mright|_{s=0} \alpha^i \mright) \mleft.\frac{\partial}{\partial \mleft(\mathrm{d}x^i\mright)}\mright|_{\omega}
\mright)
\\
&\hspace{1cm}\RPlus
\kappa \boldsymbol{\cdot} \mleft(
X_0^i \mleft.\frac{\partial}{\partial \mleft(\pi_{\mathrm{T}M}^*x^i\mright)}\mright|_{\varphi}
	+ \mleft(\mleft.\frac{\mathrm{d}}{\mathrm{d}t}\mright|_{t=0} \mleft.\frac{\mathrm{d}}{\mathrm{d}s}\mright|_{s=0} \delta^i \mright) \mleft.\frac{\partial}{\partial \mleft(\mathrm{d}x^i\mright)}\mright|_{\varphi}
\mright)
\\
&=
\lambda \boldsymbol{\cdot} S(\xi) \RPlus \kappa \boldsymbol{\cdot} S(\zeta).
\eas

This finishes the proof, so we have:

\begin{remarks}{The canonical involution/flip map an isomorphism, \newline\cite[\S 9.6, Thm.\ 9.6.1, page 363; but without proof]{mackenzieGeneralTheory}}{CanonicalInvolutionAnIsom}
The canonical involution/flip $S$ is a base-preserving vector bundle isomorphism from $\mathrm{TT}M \stackrel{\mathrm{D}\pi_{\mathrm{T}M}}{\to} \mathrm{T}M$ to $\mathrm{TT}M \stackrel{\pi_{\mathrm{TT}M}}{\to} \mathrm{T}M$, and similarly vice versa. We also write $S_M \coloneqq S$ to give an accentuation on $M$.
\end{remarks}

This isomorphism is basically now Schwarz's Theorem:

\begin{remarks}{Revisit: Schwarz's Theorem}{SchwarzThmInDiffGeo}
By definition we have 
\bas
\mleft.\frac{\mathrm{d}}{\mathrm{d}s}\mright|_{s=0} \mleft.\frac{\mathrm{d}}{\mathrm{d}t}\mright|_{t=0} \alpha
&=
S_M\mleft(
	\mleft.\frac{\mathrm{d}}{\mathrm{d}t}\mright|_{t=0} \mleft.\frac{\mathrm{d}}{\mathrm{d}s}\mright|_{s=0} \alpha 
\mright)
\eas
for all smooth $\alpha: I \times J \to M$, where $I$ and $J$ are open intervals containing 0. Similarly this extends to smooths maps $F: M \to N$, where $N$ is another smooth manifold. So let $\xi = [\alpha] \in \mathrm{TT}M$, and then we calculate for $\mathrm{DD}F: \mathrm{TT}M \to \mathrm{TT}N$ that
\bas
\mathrm{DD}F(\xi)
&=
\mleft.\frac{\mathrm{d}}{\mathrm{d}t}\mright|_{t=0} \mleft.\frac{\mathrm{d}}{\mathrm{d}s}\mright|_{s=0} ( F \circ \alpha )
\\
&=
S_N \mleft(
	\mleft.\frac{\mathrm{d}}{\mathrm{d}s}\mright|_{s=0} \mleft.\frac{\mathrm{d}}{\mathrm{d}t}\mright|_{t=0} ( F \circ \alpha )
\mright)
\\
&=
S_N \mleft(
	\mathrm{DD}F \bigl(
	S_M(\xi)
	\bigr)
\mright)
\\
&=
\mleft( S_N \circ \mathrm{DD}F \circ S_M \mright)(\xi).
\eas
Since the canonical involutions are clearly self-inverse, we can also write
\bas
S_N \circ \mathrm{DD}F
&=
\mathrm{DD}F \circ S_M.
\eas
For a notation with base points in $M$, consider a special case with $M = M_1 \times M_2$, where $M_i$ ($i \in \{1, 2\}$) are smooth manifolds. Thus, $\mathrm{T}M \cong \pi^*_1\mathrm{T}M_1 \oplus \pi^*_2\mathrm{T}M_2$, where $\pi_i: M_1 \times M_2 \to M_i$ are the projections onto the $i$-th component. Then let $p_i \in M_i$, $Y_i \in \mathrm{T}_{p_i}M_i$ and $\xi \coloneqq \mleft( Y_1, Y_2 \mright)$. Denoting with $\gamma_1: I \to M_1$ and $\gamma_2: J \to M_2$ the curves with velocities $Y_1$ and $Y_2$ at 0, respectively, and so $\alpha(t, s) = \mleft( \gamma_1(s), \gamma_2(t) \mright)$; then we define $\mathrm{D}_{p_1} F(Y_1)$ as a map
\bas
M_2 &\to \mathrm{T}N,\\
p_2 &\mapsto
\mleft.\mathrm{D}_{p_1} F(Y_1)\mright|_{p_2}
\coloneqq 
\mathrm{D}_{(p_1, p_2)} F\mleft(Y_1, 0_{p_2}\mright),
\eas
where $0_{p_2}$ is the zero tangent vector at $\mathrm{T}_{p_2}M_2$,
in a similar fashion for $\mathrm{D}_{p_2} F(Y_2)$. Then
\bas
\mathrm{D}_{p_2} \bigl(\mathrm{D}_{p_1} F(Y_1) \bigr) (Y_2)
&=
\mleft.\frac{\mathrm{d}}{\mathrm{d}t}\mright|_{t=0} 
	\mathrm{D}_{\mleft(p_1, \gamma_2(t) \mright)} F\mleft(Y_1, 0_{\gamma_2(t)}\mright) 
\\
&=
\mleft.\frac{\mathrm{d}}{\mathrm{d}t}\mright|_{t=0} 
\mleft.\frac{\mathrm{d}}{\mathrm{d}s}\mright|_{s=0} \bigl( F\circ \alpha(t, s) \bigr)
\\
&=
\mathrm{DD}F(\xi),
\eas
similarly
\bas
\mathrm{D}_{p_1} \bigl(\mathrm{D}_{p_2} F(Y_2) \bigr) (Y_1)
&=
\mleft.\frac{\mathrm{d}}{\mathrm{d}s}\mright|_{s=0} 
	\mathrm{D}_{\mleft(\gamma_1(s), p_2 \mright)} F\mleft( 0_{\gamma_1(s)}, Y_2 \mright) 
\\
&=
\mleft.\frac{\mathrm{d}}{\mathrm{d}s}\mright|_{s=0} 
\mleft.\frac{\mathrm{d}}{\mathrm{d}t}\mright|_{t=0} \bigl( F\circ \alpha(t, s) \bigr)
\\
&=
S_N\bigl(\mathrm{DD}F(\xi)\bigr),
\eas
in total
\bas
\mathrm{D}_{p_1} \bigl(\mathrm{D}_{p_2} F(Y_2) \bigr) (Y_1)
&=
S_N\Bigl(\mathrm{D}_{p_2} \bigl(\mathrm{D}_{p_1} F(Y_1) \bigr) (Y_2)\Bigr).
\eas
\end{remarks}

In fact, using the canonical flip/involution, we can construct and state other properties which can be useful for calculations related to second derivatives. 

\begin{remarks}{Total derivatives of tangent bundle morphisms linear with respect to prolonged vertical structure}{TotalDerivativesAreLinearWithRTOtherLinearStructure}
Consider a vector bundle morphism $L: \mathrm{T}M \to \mathrm{T}N$ (over some map $M \to N$), where $N$ is another smooth manifold. Then for $\xi, \eta \in \mathrm{TT}M$ with their approximating curves $X$ and $Y$, respectively, as previously,
\bas
\mathrm{D}L\mleft(\lambda \boldsymbol{\cdot} \xi \RPlus \kappa \boldsymbol{\cdot} \eta\mright)
&=
\mleft. \frac{\mathrm{d}}{\mathrm{d}t} \mright|_{t=0} \bigl( 
	L(\lambda X +\kappa Y)
\bigr)
\\
&=
\mleft. \frac{\mathrm{d}}{\mathrm{d}t} \mright|_{t=0} \bigl( 
	\lambda L(X) +\kappa L(Y)
\bigr)
\\
&=
\lambda \boldsymbol{\cdot} \mathrm{D}L(\xi) \RPlus \kappa \boldsymbol{\cdot} \mathrm{D}L(\eta)
\eas
for all $\lambda, \kappa \in \mathbb{R}$. Hence we achieve linearity with respect to both vector bundle structures of $\mathrm{TT}M$. If we denote the base points as before, then this reads
\bas
\mathrm{D}_{\lambda X_0 + \kappa Y_0}L\mleft(\lambda \boldsymbol{\cdot} \xi \RPlus \kappa \boldsymbol{\cdot} \eta\mright)
&=
\lambda \boldsymbol{\cdot} \mathrm{D}_{X_0}L(\xi) \RPlus \kappa \boldsymbol{\cdot} \mathrm{D}_{Y_0}L(\eta).
\eas
\end{remarks}

\begin{remarks}{Alignment of both vector bundle structures on $\mathrm{TT}M$ on the restricted vertical bundle}{BothLinearStructuresTheSameOnTheVerticalBundle}
By definition, the vertical bundle $\mathrm{VT}M$ of $\mathrm{T}M$ is a subbundle of $\mathrm{TT}M$. The zero section of $\mathrm{T}M$ is a natural embedding of $M$ into $\mathrm{T}M$; this embedded submanifold will be denoted by $\widetilde{M}$. Then Diagram \eqref{DoubleTangentAsDiagram} restricts onto
\begin{center}
	\begin{tikzcd}
		 \mathrm{VT}M|_{\widetilde{M}} \arrow{r}{\mathrm{D}\pi_{\mathrm{T}M}} \arrow{d}{\pi_{\mathrm{TT}M}} & \widetilde{M} \arrow{d}{\cong} \\
		\widetilde{M} \arrow[r, "\cong"]& M
	\end{tikzcd}
\end{center} 
and $S=S_M$ does not only restrict onto that, $S$ is actually the identity on $\mathrm{VT}M|_{\widetilde{M}}$, as also stated in \cite[\S 9.6, Thm.\ 9.6.1, page 363; but without proof]{mackenzieGeneralTheory}. This follows simply by the coordinate expressions Eq.\ \eqref{CoordinateExpressionOfTTM} and \eqref{CoordinateExprOfFlippy} (recall the involved notation). $\xi \in \mathrm{VT}M|_{\widetilde{M}}$ is in the kernel of both projections, $\mathrm{D}\pi_{\mathrm{T}M}$ and $\pi_{\mathrm{TT}M}$, thus $\omega = X_0 = 0 \in \mathrm{T}_pM$, therefore $S(\xi) = \xi$ by Eq.\ \eqref{CoordinateExpressionOfTTM} and \eqref{CoordinateExprOfFlippy}. By Remark \ref{rem:CanonicalInvolutionAnIsom} the addition $\RPlus$ and scalar multiplication $\boldsymbol{\cdot}$ then also align with the typical addition $+$ and scalar multiplication $\cdot$, respectively, of $\mathrm{TT}M$ as tangent bundle.

Motivated by this, we can actually recover something similar for $\mathrm{VT}M$. In this case 
\begin{center}
	\begin{tikzcd}
		 \mathrm{VT}M \arrow{r}{\mathrm{D}\pi_{\mathrm{T}M}} \arrow{d}{\pi_{\mathrm{TT}M}} & \widetilde{M} \arrow{d}{\cong} \\
		\mathrm{T}M \arrow[r, "\pi_{\mathrm{T}M}"]& M
	\end{tikzcd}
\end{center} 
and, as already mentioned earlier, the vertical bundle of vector bundles is trivially the pullback of the bundle along itself, here $\mathrm{VT}M \cong \pi_{\mathrm{T}M}^*\mathrm{T}M$. Thus, we have a canonical projection $\mathrm{pr}_2: \mathrm{VT}M \to \mathrm{T}M$ onto the second component, and we can write for $\xi = [\alpha] \in \mathrm{V}_{X_0}\mathrm{T}M$
\bas
\xi
&=
\mleft(\mleft.\frac{\mathrm{d}}{\mathrm{d}t}\mright|_{t=0} \mleft.\frac{\mathrm{d}}{\mathrm{d}s}\mright|_{s=0} \alpha^i \mright) \mleft.\frac{\partial}{\partial \mleft(\mathrm{d}x^i\mright)}\mright|_{X_0}
\cong
\mleft(
	X_0, 
	\mleft(\mleft.\frac{\mathrm{d}}{\mathrm{d}t}\mright|_{t=0} \mleft.\frac{\mathrm{d}}{\mathrm{d}s}\mright|_{s=0} \alpha^i \mright) \mleft.\frac{\partial}{\partial x^i}\mright|_{p}
\mright)
\eas
using the notation of Eq.\ \eqref{CoordinateExpressionOfTTM} (especially $X_0 \in \mathrm{T}_p M$, $p \in M$). Similarly for $\eta= [\beta] \in \mathrm{V}_{Y_0}\mathrm{T}M$ (same notation as previously in this appendix, \textit{i.e.}\ $Y_0 \in \mathrm{T}_pM$). Then by Eq.\ \eqref{LinearStructureOfProlongInCoordinates}
\bas
\lambda \boldsymbol{\cdot} \xi
	\RPlus \kappa \boldsymbol{\cdot} \eta
&\cong
\mleft( 
	\lambda X_0 + \kappa Y_0,
	\mleft(\mleft.\frac{\mathrm{d}}{\mathrm{d}t}\mright|_{t=0} \mleft.\frac{\mathrm{d}}{\mathrm{d}s}\mright|_{s=0} \mleft(\lambda \alpha^i + \kappa \beta^i \mright) \mright) \mleft.\frac{\partial}{\partial x^i}\mright|_{p}
\mright)
\eas
for all $\lambda, \kappa \in \mathbb{R}$. Therefore
\bas
\mathrm{pr}_2\mleft(\lambda \boldsymbol{\cdot} \xi
	\RPlus \kappa \boldsymbol{\cdot} \eta\mright)
&=
\lambda ~ \mathrm{pr}_2(\xi) + \kappa ~ \mathrm{pr}_2(\eta).
\eas
For $\zeta \in \mathrm{V}_{X_0}\mathrm{T}M$ we clearly get
$
\mathrm{pr}_2\mleft(\lambda \xi
	+ \kappa \zeta\mright)
=
\lambda ~ \mathrm{pr}_2(\xi) + \kappa ~ \mathrm{pr}_2(\zeta),
$
and so both linear structures on $\mathrm{VT}M$ align under $\mathrm{pr}_2$.
\end{remarks}

\begin{remarks}{Tangent lift, \cite[\S 2.2, last parapgraph in Subsection 2.2]{meinrenkensplitting}}{TangentLifts}
Let $X \in \mathfrak{X}(M)$, then its total derivative is a map $\mathrm{D}X: \mathrm{T}M \to \mathrm{TT}M$ satisfying
\bas
\mathrm{D}\pi_{\mathrm{T}M} \circ \mathrm{D}X
&=
\mathrm{D}\underbrace{\mleft( \pi_{\mathrm{T}M} \circ X \mright)}_{= \mathds{1}_M}
=
\mathds{1}_{\mathrm{T}M},
\eas
so that $\mathrm{D}X$ is a section of $\mathrm{TT}M \stackrel{\mathrm{D}\pi_{\mathrm{T}M}}{\to} \mathrm{T}M$. Due to the fact that $S = S_M$ is a base-preserving vector bundle isomorphism from $\mathrm{TT}M \stackrel{\mathrm{D}\pi_{\mathrm{T}M}}{\to} \mathrm{T}M$ to $\mathrm{TT}M \stackrel{\pi_{\mathrm{TT}M}}{\to} \mathrm{T}M$ (and vice versa), we have a vector field $X_T \in \mathfrak{X}(\mathrm{T}M)$ given by
\bas
X_T &\coloneqq S \circ \mathrm{D}X,
\eas
called the \textbf{tangent lift of $X$}. We then also have
\bas
\mathrm{D}\pi_{\mathrm{T}M} \circ X_T
&=
\pi_{\mathrm{TT}M}\circ \mathrm{D}X
=
X,
\eas
thus the label as tangent lift. We also have linearity properties (sort of) and a Leibniz rule, that is, we clearly get
\bas
X_T\mleft( \lambda Y + \kappa Z \mright)
&=
\lambda \boldsymbol{\cdot} X_T\mleft(Y\mright)
	\RPlus \kappa \boldsymbol{\cdot} X_T\mleft(Z\mright).
\eas
for all $Y, Z \in \mathrm{T}_p M$ and $\kappa, \lambda \in \mathbb{R}$, and we also have something similar w.r.t.\ $X$: Recall Eq.\ \eqref{CoordinateExpressionOfTTM} and \eqref{CoordinateExprOfFlippy}, including their notation w.r.t.\ to local coordinates $\mleft( x^i \mright)_{i}$. That is, by definition of $\mathrm{D}X$ we can derive
\bas
\mathrm{D}_pX(Y)
&=
Y^i \mleft.\frac{\partial}{\partial \mleft(\pi_{\mathrm{T}M}^*x^i\mright)}\mright|_{X_p}
	+ Y^j \mleft.\frac{\partial X^i}{\partial x^j}\mright|_p \mleft.\frac{\partial}{\partial \mleft(\mathrm{d}x^i\mright)}\mright|_{X_p},
\eas
thus,
\bas
\mathrm{D}_p(\kappa X + \lambda W)(Y)
&=
Y^i \mleft.\frac{\partial}{\partial \mleft(\pi_{\mathrm{T}M}^*x^i\mright)}\mright|_{\lambda X_p + \kappa W_p}
	+ Y^j \mleft.\frac{\partial \mleft( \lambda X^i + \kappa W^i \mright)}{\partial x^j}\mright|_p \mleft.\frac{\partial}{\partial \mleft(\mathrm{d}x^i\mright)}\mright|_{\lambda X_p + \kappa W_p}
\\
&=
\lambda \boldsymbol{\cdot} \mathrm{D}_p X(Y) \RPlus \kappa \boldsymbol{\cdot} \mathrm{D}_p W(Y)
\eas
for all $X, W \in \mathfrak{X}(M)$ and $\lambda, \kappa \in \mathbb{R}$, and 
\ba\label{TotalDerivativeOfTMWithLeibniz}
&\mathrm{D}_p(f X)(Y)
\nonumber
\\
&=
Y^i \mleft.\frac{\partial}{\partial \mleft(\pi_{\mathrm{T}M}^*x^i\mright)}\mright|_{f(p)X_p}
	+ f(p) ~ Y^j \mleft.\frac{\partial X^i}{\partial x^j}\mright|_p \mleft.\frac{\partial}{\partial \mleft(\mathrm{d}x^i\mright)}\mright|_{f(p)X_p}
	+ Y^j X^i(p) \mleft.\frac{\partial f}{\partial x^j}\mright|_p \mleft.\frac{\partial}{\partial \mleft(\mathrm{d}x^i\mright)}\mright|_{f(p)X_p}
%\nonumber
%\\
%&=
%f(p) \boldsymbol{\cdot} \mathrm{D}_p(X)(Y)
	%+ Y(f) ~ X^i(p) \mleft.\frac{\partial}{\partial \mleft(\mathrm{d}x^i\mright)}\mright|_{f(p)X_p}
\ea
for all $f \in C^\infty(M)$. Therefore
\bas
(\lambda X + \kappa W)_T
&=
\lambda X_T + \kappa W_T,
\eas
and by Eq.\ \eqref{CoordinateExprOfFlippy}
\bas
(fX)_T(Y)
&=
f(p)~ X^i_p \mleft.\frac{\partial}{\partial \mleft(\pi_{\mathrm{T}M}^*x^i\mright)}\mright|_{Y}
	+ f(p) ~ Y^j \mleft.\frac{\partial X^i}{\partial x^j}\mright|_p \mleft.\frac{\partial}{\partial \mleft(\mathrm{d}x^i\mright)}\mright|_{Y}
	+ Y(f) ~ X^i(p) \mleft.\frac{\partial}{\partial \mleft(\mathrm{d}x^i\mright)}\mright|_{Y}
\eas
so
\bas
(fX)_T
&=
f ~ X_T
	+ \mathrm{d}f \otimes X^i \frac{\partial}{\partial (\mathrm{d}x^i)}.
	%\RPlus \mathrm{d}f \boldsymbol{\cdot} \mleft(
		%f X^i \mleft.\frac{\partial}{\partial \mleft(\pi_{\mathrm{T}M}^*x^i\mright)}\mright|_{0}
		%+ X^i \mleft.\frac{\partial}{\partial \mleft(\mathrm{d}x^i\mright)}\mright|_{0}
	%\mright)
\eas
%where we viewed the last summand in Eq.\ \eqref{TotalDerivativeOfTMWithLeibniz} as a separate vector which is clearly vertical, so that its image under $S$ is a tangent vector along the zero vector field $0$. 
Similar to Remark \ref{rem:BothLinearStructuresTheSameOnTheVerticalBundle}, the vertical bundle of $\mathrm{T}M$ is canonically isomorphic to $\pi_{\mathrm{T}M}^*\mathrm{T}M$, one can think of the second term as $\mathrm{d}f \otimes \pi^*_{\mathrm{T}M}X$.
\end{remarks}

%\section{Other Interpretations of the compatibility conditions}
%
%\subsection{Minimal coupling}
%
%Notation as in \cite{Hamilton}
%
%\begin{itemize}
	%\item $\widetilde{G}$ Lie group with Lie algebra $\mathfrak{g}$
	%\item $M$ smooth manifold (usually also a spacetime). An open subset of $M$ is usually denoted by $U$; typically small enough that "everything works out" (especially without further mentioning intersections of given open sets and so on)
	%\item $P \to M$ a principal bundle, a (local) gauge is usually denoted by $s$, an element of $\Gamma(P)$, sections of $P$
	%\item $V$ a vector space
	%\item $\rho$ a Lie group representation on $V$, $\rho_*$ the induced Lie algebra representation on $V$
	%\item $K \coloneqq P \times_\rho V$ the associated vector bundle induced by $P$ and $\rho$ on $V$. An element $\Phi$ of $K$ is denoted by by $[p, \phi]$ for $p \in P$ and $\phi \in V$, where $[ \cdot, \cdot]$ denotes the equivalence class with respect to the equivalence
	%\bas
		%(p, \phi) &\sim \mleft(p g, \rho\mleft( g^{-1} \mright) \cdot \phi \mright)
	%\eas
	%for all $g \in \widetilde{G}$; $pg$ denotes the canonical group action (from the right) $P \times \widetilde{G} \to P$ and $\cdot$ the action of $\mathrm{Aut}(V) \subset \mathrm{End}(V)$ on $V$.
	%\item Especially if fixing a local gauge $s: U \to P$ we can write for sections $\Phi \in \Gamma(K)$ locally 
	%\bas
		%\Phi|_U
		%&=
		%[s, \phi],
	%\eas
	%where $\phi: U \to V$, \textit{i.e.}\ a local section of the trivial vector bundle $M \times V \to M$.
	%\item We especially focus on $V = \mathfrak{g}$ and $\rho = \mathrm{Ad}$ the adjoint representation of $\widetilde{G}$ on $\mathfrak{g}$.
%\end{itemize}
%
%The field of gauge bosons $A$ is a connection on the principal bundle, \textit{i.e.}\ an element of $\Omega^1(P; \mathfrak{g})$ with
%
%\bas
%r_g^!A &= \mathrm{Ad}_{g^{-1}} (A) \coloneqq \mathrm{Ad}_{g^{-1}} \circ A, \\
%A\mleft( \widetilde{X} \mright) &= X
%\eas
%for all $g \in \widetilde{G}$ and $X \in \mathfrak{g}$, where $r_g^!$ is the pullback of forms via the right $\widetilde{G}$-multiplication on $P$, and $\widetilde{X}$ the fundamental vector field of $X$ on $P$. 
%
%Typically, a lot of the formalism of gauge theory comes from how to define the minimal coupling. So, let us look at this and reinvent it a bit. Usually the covariant derivative/minimal coupling $\nabla^A$ of $A$ and $\Phi \in \Gamma(K)$ is locally (w.r.t.\ to a gauge $s$) defined by
%\bas
%\nabla^A \Phi
%&\coloneqq
%\mleft[ s, \nabla^A \phi \mright],
%\eas
%where
%\ba\label{ClassicalMiniCoupling}
%\nabla^A \phi
%&\coloneqq
%\mathrm{d}\phi
	%+ \rho_*\mleft( A_s \mright) \cdot \phi,
%\ea
%where $A_s \coloneqq s^! A \in \Omega^1(U; \mathfrak{g})$ (local pullback as a form of $A$ via $s$) and $\mathrm{d} \phi \coloneqq \nabla^0 \phi$, $\nabla^0$ the canonical flat connection on $M \times V$.
%
%The explicit definition of the field strength $F$ of $A$ is then usually motivated by looking at the curvature $R_{\nabla^A}$ of $\nabla^A$, that is
%\bas
%\mleft.R_{\nabla^A}(\cdot, \cdot) \Phi\mright|_U
%&=
%\mleft[ s,
	%\rho_*(F_s) \cdot \phi
%\mright],
%\eas
%where 
%\bas
%F_s
%&\coloneqq
%\mathrm{d}A_s
	%+ \frac{1}{2} \mleft[ A_s \stackrel{\wedge}{,} A_s \mright]_{\mathfrak{g}}
%\eas
%is the typical local definition of $F_s \in \Omega^2(U; \mathfrak{g})$ with
%\bas
%\mleft[ A_s \stackrel{\wedge}{,} A_s \mright]_{\mathfrak{g}}(X, Y)
%&=
%2 ~ \mleft[ A_s(X) , A_s(Y) \mright]_{\mathfrak{g}}
%\eas
%for all $X, Y \in \mathfrak{X}(U)$. (The notation $F_s$ is of course due to the fact that $F_s = s^!F$, where $F$ is the curvature of $A$. But I want to avoid that for now because of what we are about doing to do.) We shortly could denote this also as
%\ba\label{MotivOfFieldStrength}
%R_{\nabla^A} \phi
%&=
%\rho_*(F_s) \cdot \phi
%\ea
%
%%\section{Motivation}
%
%\textbf{Now:} One could question why using $\mathrm{d}\phi = \nabla^0 \phi$ in Eq.\ \eqref{ClassicalMiniCoupling}. Thence, let us assume that we have a general vector bundle connection $\widehat{\nabla}$ on the trivial vector bundle $M \times V \to M$. We are going to redefine $\nabla^A$ and $F$ locally w.r.t.\ a gauge $s$, then discuss how the gauge transformations have to look like to receive definitions independent of the chosen gauge $s$. This also means that the following discussion is now often local by fixing a gauge without further mentioning it.
%
%Let us first locally redefine $\nabla^A \phi$:
%\ba\label{NewMinimalCoupling}
%\nabla^A \phi
%&\coloneqq
%\widehat{\nabla} \phi
	%+ \rho_*(A_s) \cdot \phi.
%\ea
%Motivated by Eq.\ \eqref{MotivOfFieldStrength}, we want to identify the field strength with the curvature of $\nabla^A$. One can check that we have
%\ba\label{FirstStepTowardsNewFieldStrength}
%R_{\nabla^A}
%&=
%R_{\widehat{\nabla}}
	%+ \mathrm{d}^{\widehat{\nabla}} \bigl( \rho_*(A_s) \bigr)
	%+ \rho_*(A_s) \wedge \rho_*(A_s),
%\ea
%where $\mathrm{d}^{\widehat{\nabla}}$ is the exterior covariant derivative of $\widehat{\nabla}$ canonically extended to $\mathrm{End}(V)$, viewing $\rho_*(A_s)$ as an element of $\Omega^1(U; \mathrm{End}(V))$, and where $\rho_*(A_s) \wedge \rho_*(A_s)$ is an element of $\Omega^2(U; \mathrm{End}(V))$ given by
%\bas
%\bigl(\rho_*(A_s) \wedge \rho_*(A_s)\bigr)(X, Y)
%&\coloneqq
%\rho_*\bigl( A_s(X) \bigr) \circ \rho_*\bigl(A_s(Y)\bigr)
	%- \rho_*\bigl( A_s(Y) \bigr) \circ \rho_*\bigl(A_s(X)\bigr)
%\\
%&=
%\mleft[ \rho_*\bigl( A_s(X) \bigr), \rho_*\bigl( A_s(Y) \bigr) \mright]_{\mathrm{End}(V)}
%\\
%&=\rho_* \mleft(
	%\mleft[ A_s(X), A_s(Y) \mright]_{\mathfrak{g}}
%\mright)
%\\
%&=\rho_* \mleft(
	%\frac{1}{2}\mleft[ A_s \stackrel{\wedge}{,} A_s \mright]_{\mathfrak{g}}
%\mright)(X,Y)
%\eas
%for all $X, Y \in \mathfrak{X}(U)$.
%
%In order to have a similar shape as in Eq.\ \eqref{MotivOfFieldStrength}, we now assume that $\widehat{\nabla}$ satisfies the following \textbf{compatibility conditions}:
%\begin{remarks}{Comaptibility conditions}{CompCondsSimple}
%\ba
%R_{\widehat{\nabla}} 
%&=
%\rho_*(\zeta),\label{ZetaCondition}\\
%\widehat{\nabla} \circ \rho_*
%&=
%\rho_* \circ \nabla\label{VanishingBasicCurvature}
%\ea
%for some $\zeta \in \Omega^2(M; \mathfrak{g})$ and $\nabla$ a vector bundle connection on the trivial vector bundle $M \times \mathfrak{g} \to M$.
%\end{remarks}
%
%If we want that Eq.\ \eqref{FirstStepTowardsNewFieldStrength} has a shape like Eq.\ \eqref{MotivOfFieldStrength}, it is obvious why we require \eqref{ZetaCondition}; \eqref{VanishingBasicCurvature} is needed for the second summand in Eq.\ \eqref{FirstStepTowardsNewFieldStrength}. Hence, let us study \eqref{VanishingBasicCurvature}, that is
%\bas
%\widehat{\nabla}\bigl( \rho_*(\nu) \bigr)
%&=
%\rho_*(\nabla \nu)
%\eas
%for all $\nu \in \Gamma(M \times \mathfrak{g})$,\footnote{Elements of $\mathfrak{g}$ are viewed as constant sections of $M \times \mathfrak{g}$.} especially $\widehat{\nabla}$ is again extended to $\mathrm{End}(V)$ on the left hand side. With this we get
%\bas
%\mathrm{d}^{\widehat{\nabla}} \bigl( \rho_*(A_s) \bigr)(X,Y)
%&=
%\widehat{\nabla}_X\bigl( \rho_*(A_s(Y)) \bigr)
	%- \widehat{\nabla}_Y\bigl( \rho_*(A_s(X)) \bigr)
	%- \rho_*\mleft( A_s\bigl([X, Y]\bigr) \mright)
%\\
%&=
%\rho_*\mleft(\nabla_X \bigl( A_s(Y) \bigr) \mright)
	%- \rho_*\mleft( \nabla_Y \bigl(A_s(X)\bigr) \mright)
	%- \rho_*\mleft( A_s\bigl([X, Y]\bigr) \mright)
%\\
%&=
%\rho_* \mleft(
	%\nabla_X \bigl( A_s(Y) \bigr)
	%- \nabla_Y \bigl(A_s(X)\bigr)
	%- A_s\bigl([X, Y]\bigr)
%\mright)
%\\
%&=
%\rho_*\mleft( \mathrm{d}^\nabla A_s \mright)(X,Y)
%\eas
%for all $X, Y \in \mathfrak{X}(U)$. Collecting everything, Eq.\ \eqref{FirstStepTowardsNewFieldStrength} has now the following form
%\bas
%R_{\nabla^A}
%&=
%\rho_*\mleft( \mathrm{d}^\nabla A_s + \frac{1}{2}\mleft[ A_s \stackrel{\wedge}{,} A_s \mright]_{\mathfrak{g}}+ \zeta \mright).
%\eas
%So, we have a new form of the field strength, assuming that $\nabla$ and $\zeta$ satisfy the compatibility conditions in Remark \ref{rem:CompCondsSimple}. This is precisely the definition of the field strength as in the gauge theory of Thomas and Alexei, that is, we have a new field strength
%\bas
%G
%&\coloneqq
%\mathrm{d}^\nabla A_s + \frac{1}{2}\mleft[ A_s \stackrel{\wedge}{,} A_s \mright]_{\mathfrak{g}}+ \zeta.
%\eas
%Furthermore, if we are interested into Yang-Mills gauge theories, then we'd have $K = P \times_{\mathrm{Ad}} \mathfrak{g}$ (the adjoint bundle), and so also $\rho_* = \mathrm{ad}$. In this case we can put $\widehat{\nabla} = \nabla$ and then the compatibility conditions in Remark \ref{rem:CompCondsSimple} read
%\bas
%R_\nabla
%&=
%\mathrm{ad}(\zeta),
%\\
%\nabla \circ \mathrm{ad}
%&=
%\mathrm{ad} \circ \nabla.
%\eas
%The second condition precisely gives after a short calculation
%\bas
%\nabla\mleft( \mleft[ \mu, \nu \mright]_{\mathfrak{g}} \mright)
%&=
%\mleft[ \nabla\mu, \nu \mright]_{\mathfrak{g}}
	%+ \mleft[ \mu, \nabla\nu \mright]_{\mathfrak{g}}
%\eas
%for all $\mu, \nu \in \Gamma(M \times \mathfrak{g})$, so, $\nabla$ has to be a Lie bracket derivation. So, in this case the compatibility conditions in Remark \ref{rem:CompCondsSimple} precisely reduce to the compatibility conditions of Alexei's and Thomas's theory! (in the case of Lie algebra bundles; the general theory is more general, formulated on general Lie algebroids)
%
%As a summary:
%
%\begin{remarks}{Summary}{FirstSummary}
%We have
%\ba
%R_\nabla
%&=
%\mathrm{ad}(\zeta),
%\\
%\nabla \circ \mathrm{ad}
%&=
%\mathrm{ad} \circ \nabla,
%\\
%G
%&=
%\mathrm{d}^\nabla A_s + \frac{1}{2}\mleft[ A_s \stackrel{\wedge}{,} A_s \mright]_{\mathfrak{g}}+ \zeta.
%\ea
%\end{remarks}
%
%In fact, the compatibility conditions lead to a gauge invariant theory: Fix an $\mathrm{ad}$-invariant scalar product $\kappa$ on $\mathfrak{g}$; then define the Lagrangian by
%\ba
%\mathfrak{L}_{\mathrm{YM}}
%&\coloneqq
%- \frac{1}{2} \kappa\mleft( G \stackrel{\wedge}{,} *G \mright)
%\ea
%where $*$ is the Hodge star operator w.r.t.\ some spacetime metric. (In short, the typical definition, but replace $F$ with $G$) It is easier to look at the infinitesimal version of the gauge transformations, hence everything with respect to a gauge $s$ now.
%
%In order to derive a formula for these, let us again look at $\nabla^A$. Fix an $\varepsilon \in \Gamma(M \times \mathfrak{g})$, then the infinitesimal gauge transformation $\delta_\varepsilon \phi$ of $\phi \in \Gamma(M \times \mathfrak{g})$ is usually defined by
%\bas
%\delta_\varepsilon \phi
%&\coloneqq
%\rho_*(\varepsilon) \cdot \phi.
%\eas
%We fix the infinitesimal gauge trafo $\delta_\varepsilon A$ of $A$ by looking at the gauge trafo of $\nabla^A \phi$ via
%\bas
%\delta_\varepsilon \nabla^A \phi
%&=
%\mleft.\frac{\mathrm{d}}{\mathrm{dt}}\mright|_{t=0}\mleft( \nabla^{A + t\delta_\varepsilon A} \mleft( \phi + t \delta_\varepsilon \phi \mright) \mright)
%\\
%&=
%\underbrace{\widehat{\nabla} \mleft( \delta_\varepsilon \phi \mright)}
	%_{\mathclap{ = \mleft(\widehat{\nabla} \mleft( \rho_*(\varepsilon) \mright)\mright) \cdot \phi + \rho_*(\varepsilon) \cdot \widehat{\nabla} \phi }}
	%+ \rho_*\mleft( \delta_\varepsilon A_s \mright) \cdot \phi
	%+ \rho_*(A_s) \cdot \delta_\varepsilon \phi
%\\
%&=
%\bigl( \rho_*\mleft( \nabla \varepsilon + \delta_\varepsilon A_s \mright) + \rho_*(A_s) \cdot \rho_*(\varepsilon) \bigr) \cdot \phi
	%+ \rho_*(\varepsilon) \cdot \widehat{\nabla} \phi
%\eas
%using Remark \ref{rem:CompCondsSimple}. We want $\delta_\varepsilon \nabla^A \phi = \rho_*(\varepsilon) \cdot \nabla^A \phi$ which gives
%\bas
%\rho_*(\varepsilon) \cdot \nabla^A \phi
%&=
%\rho_*(\varepsilon) \cdot \widehat{\nabla} \phi
	%+ \rho_*(\varepsilon) \cdot \rho_*(A_s) \cdot \phi.
%\eas
%Imposing $\delta_\varepsilon \nabla^A \phi = \rho_*(\varepsilon) \cdot \nabla^A \phi$ we get
%\bas
%\rho_*\mleft(\delta_\varepsilon A_s + \nabla \varepsilon + \mleft[ A_s, \varepsilon \mright]_{\mathfrak{g}} \mright)
%&=
%0
%\eas
%using again that $\rho_*$ is a Lie algebra representation. If we require that this shall work for all $\rho_*$, we may say
%\ba\label{GaugeTrafoOfANew}
%\delta_\varepsilon A_s
%&\coloneqq
%-\nabla \varepsilon
	%+ \mleft[ \varepsilon, A_s \mright]_{\mathfrak{g}}.
%\ea
%This is precisely the infinitesimal gauge trafo of $A$ as in the theory of Thomas and Alexei! Hence, we achieve infinitesimal gauge invariance of $\mathfrak{L}_{\mathrm{YM}}$. For completeness, let us check the gauge trafo of $G$ using Def.\ \eqref{GaugeTrafoOfANew} and Remark \ref{rem:FirstSummary}, it is very similar to the "classical" calculation due to Remark \ref{rem:FirstSummary} which is why I skip some straightforward calculations to keep it short,
%\bas
%\delta_\varepsilon G
%&=
%\mleft.\frac{\mathrm{d}}{\mathrm{d}t}\mright|_{t=0}
%\mleft(
	%\mathrm{d}^\nabla (A_s + t\delta_\varepsilon A_s) + \frac{1}{2}\mleft[ A_s + t\delta_\varepsilon A_s \stackrel{\wedge}{,} A_s + t\delta_\varepsilon A_s \mright]_{\mathfrak{g}}+ \zeta
%\mright)
%\\
%&=
%\mathrm{d}^\nabla \mleft( -\nabla \varepsilon
	%+ \mleft[ \varepsilon, A_s \mright]_{\mathfrak{g}} \mright)
		%+ \mleft[ A_s \stackrel{\wedge}{,} -\nabla \varepsilon + \mleft[ \varepsilon, A_s \mright]_{\mathfrak{g}} \mright]_{\mathfrak{g}}
%\\
%&=
%\underbrace{- \mleft(\mathrm{d}^\nabla\mright)^2 \varepsilon}_{= - R_\nabla \varepsilon = \mleft[ \varepsilon, \zeta \mright]_{\mathfrak{g}}}
	%+ \mleft[ \nabla \varepsilon \stackrel{\wedge}{,} A_s \mright]_{\mathfrak{g}}
	%+ \mleft[ \varepsilon, \mathrm{d}^\nabla A_s \mright]_{\mathfrak{g}}
	%+ \underbrace{\mleft[ A_s \stackrel{\wedge}{,} -\nabla \varepsilon \mright]_{\mathfrak{g}}}_{= - \mleft[ \nabla \varepsilon \stackrel{\wedge}{,} A_s \mright]_{\mathfrak{g}}}
	%+ \mleft[ A_s \stackrel{\wedge}{,} \mleft[ \varepsilon, A_s \mright]_{\mathfrak{g}} \mright]_{\mathfrak{g}}
%\\
%&=
%\mleft[ \varepsilon,
	%\mathrm{d}^\nabla A_s + \zeta
%\mright]_{\mathfrak{g}}
	%+ \mleft[ A_s \stackrel{\wedge}{,} \mleft[ \varepsilon, A_s \mright]_{\mathfrak{g}} \mright]_{\mathfrak{g}}
%\eas
%and, using the Jacobi identity,
%\bas
%\mleft[ A_s \stackrel{\wedge}{,} \mleft[ \varepsilon, A_s \mright]_{\mathfrak{g}} \mright]_{\mathfrak{g}} (X,Y)
%&=
%\mleft[ A_s(X) , \mleft[ \varepsilon, A_s(Y) \mright]_{\mathfrak{g}} \mright]_{\mathfrak{g}}
	%- \mleft[ A_s(Y) , \mleft[ \varepsilon, A_s(X) \mright]_{\mathfrak{g}} \mright]_{\mathfrak{g}}
%\\
%&=
%\mleft[ \varepsilon, \mleft[ A_s(X), A_s(Y) \mright]_{\mathfrak{g}} \mright]_{\mathfrak{g}}
%\\
%&=
%\mleft[ \varepsilon, \frac{1}{2} \mleft[ A_s \stackrel{\wedge}{,} A_s \mright]_{\mathfrak{g}} \mright]_{\mathfrak{g}} (X, Y)
%\eas
%for all $X, Y \in \mathfrak{X}(U)$. Altogether
%\bas
%\delta_\varepsilon G
%&=
%\mleft[ \varepsilon,
	%\mathrm{d}^\nabla A_s + \frac{1}{2} \mleft[ A_s \stackrel{\wedge}{,} A_s \mright]_{\mathfrak{g}} + \zeta
%\mright]_{\mathfrak{g}}
%=
%\mleft[ \varepsilon, G
%\mright]_{\mathfrak{g}}.
%\eas
%Hence, the field strength transforms with the adjoin of $\varepsilon$; since $\kappa$ is $\mathrm{ad}$-invariant, we can derive that $\mathfrak{L}_{\mathrm{YM}}$ is invariant under the infinitesimal gauge trafo in Def.\ \eqref{GaugeTrafoOfANew}!
%
%Observe that by Remark \ref{rem:FirstSummary} that $\zeta$ can be non-trivial even if we still use $\nabla = \nabla^0$, the canonical flat connection on $M\times \mathfrak{g}$, even though this whole discussion started with allowing more general connections.
%
%If we minimise $\mathfrak{L}_{\mathrm{YM}}$, then one obvious way would be to search solutions with $G \equiv 0$ for an absolute minimum/maximum (because of the sign), doing so would result into that the classical Yang-Mills energy would have a bound which is non-zero. May this be an explanation for the mass gap? As shown in my thesis, every classical theory has a $\zeta$ after a field redefinition. Even though field redefinitions are an equivalence for the classical theories, one may argue that it does not describe an equivalence for the quantised theory, leading to a possible explanation of the mass gap? But that is just high hope right now :) 
%
%%\subsection{Integration}
%\textbf{Integration (probably no need for the following)}
%
%For an integrated version of Def.\ \eqref{GaugeTrafoOfANew} we need to discuss when the new "minimal coupling" of Def.\ \eqref{NewMinimalCoupling} behaves nicely under a change of the gauge $s$. That is, we now want to extend the new definition of $\nabla^A$ to a well-defined connection on $K = P \times_{\rho} V$, especially on the adjoint bundle $K = P \times_{\mathrm{Ad}} \mathfrak{g}$ in our case. (and later maybe generalise this to a $\widetilde{G}$-quotient of a general Lie algebra bundle over $P$)
%
%Let $s^\prime$ be another (local) gauge such that we have a unique smooth map $g: U \to G$ such that
%\bas
%s^\prime
%&=
%s g,
%\eas
%then we want for well-definedness
%\ba\label{WellDef}
%\nabla^A \Phi
%&=
%\mleft[ s, \nabla \phi + \mathrm{ad}(A_s) \cdot \phi \mright]
%\stackrel{!}{=}
%\mleft[ s^\prime, \nabla \phi^\prime + \mathrm{ad}(A_{s^\prime}) \cdot \phi^\prime \mright],
%\ea
%where we have $\Phi = [s, \phi] = [s^\prime, \phi^\prime]$, especially
%\bas
%\phi^\prime
%&=
%\mathrm{Ad}\mleft( g^{-1} \mright) \cdot \phi.
%\eas
%Since the new field strength $G$ still transforms via the adjoin under $\delta_\varepsilon$ (see above), we make the following ansatz
%\ba
%A_{s^\prime} 
%&=
%\mathrm{Ad}\mleft( g^{-1} \mright) \cdot A_s
	%+ \mu,
%\ea
%where $\mu \in \Omega^1(U; \mathfrak{g})$. Usually, $\mu = g^!\mu_{\widetilde{G}}$, the pullback as a form of the Maurer-Cartan-Form $\mu_{\widetilde{G}}$ on $\widetilde{G}$. One can then check with some short calculation that Eq.\ \eqref{WellDef} is equivalent to
%\bas
%\nabla\mleft( \mathrm{Ad}\mleft( g^{-1} \mright) \cdot \phi \mright)
	%+ \mathrm{ad}(\mu) \cdot \mathrm{Ad}\mleft( g^{-1} \mright) \cdot \phi
%&\stackrel{!}{=}
%\mathrm{Ad}\mleft( g^{-1} \mright) \cdot \nabla \phi
%\eas
%using the definition of $P \times_{\mathrm{Ad}} \mathfrak{g}$. Equivalently,
%\bas
%\mathrm{ad}(\mu)
%&=
%\mathrm{Ad}\mleft( g \mright) \circ \mleft(
%\mright)
%\eas

%\subsection{Axiomatic Yang-Mills gauge theories}
%
%Let us discuss where the compatibility conditions may arise from a certain axiomatic point of view.

\end{document}