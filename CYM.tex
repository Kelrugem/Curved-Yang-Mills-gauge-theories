\documentclass[a4paper,oneside,11pt,bibliography=totoc]{scrartcl}
%Use bibliography=totoc or bibliography=totocnumbered to show the List of References in the table of contents, numbered if using the latter
\usepackage[utf8]{inputenc}
\usepackage[T1]{fontenc}
%\usepackage{exscale}
%\newcommand\hmmax{0}
%\newcommand\bmmax{1}
%\newcommand{\Plus}{\mathord{\tikz\draw[line width=0.3ex, x=1ex, y=1ex] (0.75,0) -- (0.75,1.5)(0,0.75) -- (1.5,0.75);}} 
\def\RPlus{\ensuremath{\mathbin{\rule[.13em]{.66em}{.22em}\hspace{-.44em}\rule[-.08em]{.22em}{.66em}\,}}} %Fat plus symbol
\usepackage[english]{babel}
\usepackage{graphicx} %extended \includegraphics options
%\usepackage{CJKutf8} %Kanji signs
\usepackage{dsfont}%even more mathematical symbols
%\usepackage{bm} %use \bm to get bold math symbols
\usepackage{amssymb} %even more mathematical symbols
\usepackage{amsthm} %theorem style usually needed for AMS (american mathematical society)
\usepackage{amsmath} %important stuff like equation structures and so on
\usepackage{amsfonts} %new mathematical symbols like fractals
%\usepackage{amscd}
%\usepackage{amstext}
%\usepackage{mathabx}
\usepackage{booktabs} %enhanced table stuff
%\usepackage{amsbsy} %bold mathematical symbols, also loaded in amsmath
\usepackage{fancybox} %for theorem boxes
%\usepackage[hypcap=false]{caption} %new captions for floating options
%\captionsetup{font={small,sf},labelfont=bf,textfont=rm} 
%\usepackage{float} %provides H for images-here
%\usepackage{subfigure} %citing subfigures in a figure
\usepackage{lmodern} %latin modern font
\usepackage{fixcmex} %Fixing the issue with scaling brackets etc. in lmodern
%\renewcommand\familydefault{lmr}% uses lm only for text
%\usepackage[numbib,nottoc]{tocbibind} %Options for including the bibliography to content list etc.; nottoc prevents listing of Contents in the list of contents, numbib also numbers the section of the biblio
%\usepackage[pdfstartview={FitH},linkbordercolor={0 1 0}]{hyperref}
\usepackage[pdfstartview={FitH},linkbordercolor={0 1 0},colorlinks]{hyperref}
\usepackage{esint} %für \oiint und andere Integralformen
%\usepackage{fancyhdr} %Different handling for footers and headers
\usepackage[headsepline=.5pt,footsepline=.5pt,automark]{scrlayer-scrpage}
\usepackage{pdfpages} %For \includepdf etc
%\usepackage{trfsigns} % Fourier and Laplacesymbols for engineering und für \e für \exp
\newcommand{\e}{\ensuremath{\mathrm{e\;\!}}}
%\usepackage{wasysym} %Smiley
%\usepackage{siunitx} %SI Einheiten
%\usepackage{eqnarray} %Gleichungen in Tabellenform
\usepackage{mathtools} %für mathclap
\usepackage{relsize} %Integralgröße
\usepackage{scrhack} %Warnung bei book bezüglich toc
\usepackage{mleftright} %Distanz zu \left \right weg
\usepackage{tikz} %Diagramme
\usepackage{stmaryrd} % fuer llbracket usw.
\usepackage[many]{tcolorbox} %Coloured boxes around theorems etc
\tcbuselibrary{theorems} % siehe oben
%\usepackage{xfrac}    %for \sfrac quotient of sets

\usetikzlibrary{cd}

\makeatletter
\DeclareFontEncoding{LS1}{}{}
\DeclareFontSubstitution{LS1}{stix}{m}{n}
\DeclareMathAlphabet{\mathKel}{LS1}{stixscr}{m}{n}
\DeclareMathAlphabet{\mathcal}{LS1}{stixscr}{m}{n}
\makeatother
\DeclareMathOperator{\sAut}{\mathKel{A\mkern-5.5mu u\mkern-4mu t\mkern-1.5mu}}
\DeclareMathOperator{\saut}{\mathKel{a\mkern-4.5mu u\mkern-4mu t\mkern-1.5mu}}
\DeclareMathOperator{\sEnd}{\mathKel{E\mkern-4mu n\mkern-4.5mu d\mkern-1mu}}

%\tikzset{every node/.style={align=center}}
%\usepackage{pst-node} %Diagramme
%\usepackage{auto-pst-pdf}
%\usepackage{stackengine} %Kommawedge
%\usepackage{mathabx}
%\usepackage{here}
%\usepackage[style=authortitle-icomp]{biblatex}
%\usepackage[babel,german=guillemets]{csquotes}

\setcounter{tocdepth}{4}
\setcounter{tocdepth}{5}
\setcounter{secnumdepth}{4}
\setcounter{secnumdepth}{5}
\setlength{\jot}{12pt} % Zeilenabstand in der Align-Umgebung
\allowdisplaybreaks % mit \displaybreak einen Seitenumbruch in align-Umgebung erzeugen

%\pagestyle{headings}
\pagestyle{scrheadings}
%\pagestyle{fancy} %eigener Seitenstil
%\fancyhf{}
%\lhead{\leftmark} %\nouppercase, um auch Kleinbuchstaben zu bekommen
%\renewcommand{\sectionmark}[1]{\markright{#1}{}}
%  \markboth{\thesection{} #1}} 
%\fancyhead[L]{Titel} %Kopfzeile links
\clearpairofpagestyles
\ihead{\headmark} %Kopfzeile links, automark has to be loaded in the package scrlayer-scrpage
\ohead{Simon-Raphael Fischer} %Kopfzeile rechts
%\renewcommand{\headrulewidth}{0.4pt} %obere Trennlinie
%\fancyfoot[C]{\thepage} %Seitennummer
\cfoot{\pagemark}
%\renewcommand{\footrulewidth}{0.4pt} %untere Trennlinie

\renewcommand{\theequation}{\arabic{equation}}


%\renewcommand*{\chapterpagestyle}{scrheadings}


%Folgender Befehl dafür da, dass sections auf ungeraden Seiten anfangen
%\let\Section\section
%\renewcommand\section[2][]{%
  %\cleardoublepage
  %\def\myTEMP{#1}\ifx\myTEMP\empty\Section{#2}\else\Section[#1]{#2}\fi}
	
	
%\usepackage[automark]{scrlayer-scrpage}
%\ohead{\pagemark}
%\ihead{\leftmark}


\renewcommand{\listoffigures}{\begingroup
\tocsection
\tocfile{\listfigurename}{lof}
\endgroup} 

\renewcommand{\listoftables}{\begingroup
\tocsection
\tocfile{\listtablename}{lot}
\endgroup} 

\oddsidemargin -0.0 cm
\evensidemargin -0.5 cm
\topmargin -1.5 cm
\headheight 1.2 cm
\headsep 1.3 cm
\topskip 0.5 cm
\textheight 23.0 cm
\textwidth 16.0 cm
\parindent 0.0 cm
\renewcommand{\baselinestretch}{1.2}
%\renewcommand{\arraystretch}{1.6}
\setcounter{totalnumber}{2}

\def\tp{^{^{^{\leftrightarrow}}}\!\!\!\!\!\!\!}

%%%%%%%%%%%%%%%%%%%%%%%%%%%% Definitionen %%%%%%%%%%%%%%%%%%%%%%%%%%%%

%\renewcommand{\chapterheadstartvskip}{\vspace{0cm}} %für bookklasse

\def\be{\begin{equation}}
\def\ee{\end{equation}}
\def\bs{\begin{subequations}}
\def\es{\end{subequations}}
\def\ba#1\ea{\begin{align}#1\end{align}}
\def\bes{\begin{equation*}}
\def\ees{\end{equation*}}
\def\bas#1\eas{\begin{align*}#1\end{align*}}


%um die Theoreme etc. in das Inhaltsverzeichnis zu packen
%\let\amsthmhead\thmhead
%\let\amsswappedhead\swappedhead
%\makeatletter
%\renewcommand*\thmhead[3]{\amsthmhead{#1}{#2}{#3}%
%%\@ifnotempty{#2}{\addcontentsline{toc}{subsection}{#2 ~ #3}}{}}
%\renewcommand*\swappedhead[3]{\amsswappedhead{#1}{#2}{#3}%
%%\@ifnotempty{#2}{\addcontentsline{toc}{subsection}{#2 ~ #3}}{}}
%\makeatother

\renewcommand{\qedsymbol}{$\blacksquare$}

\theoremstyle{plain}
\newtheorem{theorem}{Theorem}[section]
%\newtcbtheorem[number within=section]{theorem}{Theorem}
%{colback=green!5,colframe=green!35!black,fonttitle=\bfseries}{th}
\newtheorem{corollary}[theorem]{Corollary}
\newtheorem{lemma}[theorem]{Lemma}
\newtcbtheorem
  [use counter*=theorem,number within=section]% init options
  {lemmata}% name
  {Lemma}% title
  {%
		fontupper=\itshape,
		breakable,
		enhanced,
    colback=gray!25,
    colframe=gray!0!black,
    fonttitle=\bfseries,
  }% options
  {lem}% prefix
\newtheorem{proposition}[theorem]{Proposition}
\newtcbtheorem
  [use counter*=theorem,number within=section]% init options
  {propositions}% name
  {Proposition}% title
  {%
		fontupper=\itshape,
		breakable,
		enhanced,
    colback=gray!25,
    colframe=gray!0!black,
    fonttitle=\bfseries,
  }% options
  {prop}% prefix
\newtcbtheorem
  [use counter*=theorem,number within=section]% init options
  {theorems}% name
  {Theorem}% title
  {%
		fontupper=\itshape,
		breakable,
		enhanced,
    colback=gray!25,
    colframe=gray!0!black,
    fonttitle=\bfseries,
  }% options
  {thm}% prefix
\newtcbtheorem
  [use counter*=theorem,number within=section]% init options
  {corollaries}% name
  {Corollary}% title
  {%
		fontupper=\itshape,
		breakable,
		enhanced,
    colback=gray!25,
    colframe=gray!0!black,
    fonttitle=\bfseries,
  }% options
  {cor}% prefix
\newtheorem{conjecture}[theorem]{Conjecture}


\theoremstyle{remark}
\newtcbtheorem
  [use counter*=theorem,number within=section]% init options
  {remarks}% name
  {Remark}% title
  {%
		breakable,
		enhanced,
    colback=gray!5,
    colframe=gray!50!black,
    fonttitle=\bfseries,
  }% options
  {rem}% prefix
\newtheorem*{note}{Note}
\newtheorem{remark}[theorem]{Remarks}
\newtheorem{motivation}[theorem]{Motivation} 


\theoremstyle{definition}
\newtheorem{definition}[theorem]{Definition}
\newtcbtheorem
  [use counter*=theorem,number within=section]% init options
  {definitions}% name
  {Definition}% title
  {%
		breakable,
		enhanced,
    colback=gray!25,
    colframe=gray!0!black,
    fonttitle=\bfseries,
  }% options
  {def}% prefix
\newtcbtheorem
  [use counter*=theorem,number within=section]% init options
  {examples}% name
  {Example}% title
  {%colback=black!5,colframe=red!35!black
		breakable,
		enhanced,
    colback=gray!5,
    colframe=gray!50!black,
    fonttitle=\bfseries,
  }% options
  {ex}% prefix
\newtcbtheorem
  [use counter*=theorem,number within=section]% init options
  {situations}% name
  {Situation}% title
  {%colback=black!5,colframe=red!35!black
		breakable,
		enhanced,
    colback=gray!5,
    colframe=gray!50!black,
    fonttitle=\bfseries,
  }% options
  {sit}% prefix
\newtcbtheorem
  [use counter*=theorem,number within=section]% init options
  {fieldredefinitions}% name
  {Theorem}% title
  {%colback=black!5,colframe=red!35!black
		breakable,
		enhanced,
    colback=gray!25,
    colframe=gray!0!black,
    fonttitle=\bfseries,
  }% options
  {fieldredef}% prefix
\newtheorem{example}[theorem]{Example}


% hier Namen etc. einsetzen
%\newcommand{\fullname}{Simon-Raphael Fischer}
%\newcommand{\email}{sfischer@ncts.tw}
%\newcommand{\titel}{Examples and No-Gos of curves Yang-Mill-Higgs gauge theories}
%\newcommand{\titel}{Titel der Arbeit}
%\newcommand{\jahr}{2020}
%\newcommand{\matnr}{11424951}
%\newcommand{\gutachterA}{Prof. Dr. Mark John David Hamilton}
%\newcommand{\betreuer}{Professor Dr. Anna Dall'Acqua}
% hier richtige Fakultät auswählen
%\newcommand{\fakultaet}{Fakultät noch ergänzen}
%\newcommand{\fakultaet}{Mathematik und\\Wirtschaftswissenschaften}
%\newcommand{\fakultaet}{Naturwissenschaften}
%\newcommand{\fakultaet}{Medizin}
% nun noch unten das Institut einsetzen
%\newcommand{\institut}{Institut noch ergänzen}

% Für \widecheck: (umgekehrter Zirkumflex)

\DeclareFontFamily{U}{mathx}{\hyphenchar\font45}
\DeclareFontShape{U}{mathx}{m}{n}{
      <5> <6> <7> <8> <9> <10>
      <10.95> <12> <14.4> <17.28> <20.74> <24.88>
      mathx10
      }{}
\DeclareSymbolFont{mathx}{U}{mathx}{m}{n}
\DeclareFontSubstitution{U}{mathx}{m}{n}
\DeclareMathAccent{\widecheck}{0}{mathx}{"71}
\DeclareMathAccent{\wideparen}{0}{mathx}{"75}

%\newtheorem{cor}{Corollary}
%\newtheorem{ad}{Theorem}
%\newtheorem{theorem}{Theorem}
%\newtheorem{theorem}{Theorem}
%\newtheorem{theorem}{Theorem}

%\includeonly{PhysicalBasics/f(R)gravity} %nur das kompilieren

%\let\endtitlepage\relax %No page break after titlepage

\begin{document}
\pagenumbering{Roman}
\renewcommand{\thefootnote}{\fnsymbol{footnote}}

\begin{titlepage}
%\thispagestyle{empty}

\author{Simon-Raphael Fischer\footnote{Email: \href{mailto:sfischer@ncts.tw}{sfischer@ncts.tw}; ORCID iD: \href{https://orcid.org/0000-0002-5859-2825}{0000-0002-5859-2825}} }
\title{Curved Yang-Mills gauge theories} 
\subtitle{Infinitesimal and integrated gauge theory}
\date{\today} 
\maketitle
\thispagestyle{empty}

\begin{center}
National Center for Theoretical Sciences, Mathematics Division, National Taiwan University\\
No. 1, Sec. 4, Roosevelt Rd., Taipei City 106, Taiwan Room 503, Cosmology Building, Taiwan
\ \\
\ \\
\ \\
\textbf{Abstract}\footnote[2]{Abbreviations used in this paper: \textbf{LGB} for Lie group bundle, \textbf{LAB} for Lie algebra bundle.}
%\footnote[2]{Abbreviations used in this paper: \textbf{(C)YMH GT} for (curved) Yang-Mills-Higgs gauge theory.}
\begin{abstract}
  \small{
}
 \end{abstract}
\end{center}

\textit{2020 MSC:} Primary 53D17; Secondary 81T13, 17B99.

\textit{Keywords:} \texttt{Mathematical Gauge Theory}, Differential Geometry, High Energy Physics - Theory, Mathematical Physics

\end{titlepage}

%%%%%%%%%%%%%%%%%%%%%%%%%%%% Inhaltsverzeichnis %%%%%%%%%%%%%%%%%%%%%%%%%%%%



\tableofcontents



%\thispagestyle{empty}\hspace{1em}\newpage
% \thispagestyle{empty}\hspace{1em}\newpage


\renewcommand{\thefootnote}{\arabic{footnote}}
%
\setlength{\parindent}{12 pt}
%%%%%%%%%%%%%%%%%%%%%%%%%%%% Hier beginnt der Hauptteil %%%%%%%%%%%%%%%%%%%%%%%%%%%%

%\cleardoublepage
% für die leere(n) Seite(n)


\pagenumbering{arabic}

\section{Introduction}

\subsection{Basic notations}

\subsection{Assumed background knowledge}

It is highly recommended to have basic knowledge about differential geometry and gauge theory as presented in \cite[especially Chapter 1 to 5]{Hamilton}; however, sometimes we will still give explicit references to help with more technical details. It can be useful to have knowledge about Lie algebra and Lie group bundles, and even Lie algebroids and Lie groupoids, but we will introduce their basic notions such that it is not necessarily needed to have knowledge about these upfront.

We also often give references about Lie group bundles (LGBs), but the given references are often about Lie groupoids. If the reader has no knowledge about Lie groupoids, then it is important to know that LGBs are a special example of Lie groupoids; Lie groupoids carry "two projections", called \textbf{source} and \textbf{target}. An LGB is a special example of a Lie groupoid whose source equals the target.\footnote{But not every Lie groupoid with equal source and target are LGBs, they're in general bundles of Lie groups which is not completely the same; this nuance will not be important here.} If you look into such a reference, then the source and target are often denoted by $\alpha$ and $\beta$, or by $s$ and $t$; simply put both to be the same and identify these with our bundle projection which we often denote by $\pi$. In that way it should be possible to read the references without the need to know Lie groupoids. However, we try to re-prove the needed statements such that these types of references could be avoided by the reader.

See also the previous subsection about notions we assume to be known.

%\twocolumn
\section{Basic definitions}\label{BasicDefinitions}
%
%In the following, we denote with $V^*$ the dual of a vector bundle $V \to N$ over a smooth manifold $N$, and $\Phi^*V$ denotes the pull-back of $V$ by $\Phi: M \to N$, a smooth map from a smooth manifold $M$ to $N$. We have a similar notation for the pull-back of sections, especially we will have sections $F$ as an element of $\Gamma\left( \left(\bigotimes_{m=1}^{l} E_m^*\right) \otimes E_{l+1} \right)$, where $E_1, \dots, E_{l+1} \to N$ ($l \in \mathbb{N}$) are real vector bundles of finite rank over a smooth manifold $N$, and $\Gamma(\cdot)$ denotes the space of smooth sections. Then we view the pull-back $\Phi^*F$ as an element of $\Gamma\left( \mleft(\bigotimes_{m=1}^{l} \mleft(\Phi^*E_m\mright)^*\mright) \otimes \Phi^*E_{l+1} \right)$, and it is essentially given by
%\bas
	%(\Phi^*F)(\Phi^*\nu_1, \dotsc , \Phi^*\nu_l)
	%&=
	%\Phi^*\mleft( F\mleft( \nu_1, \dotsc, \nu_l \mright) \mright)
%\eas
%for all $\nu_1 \in \Gamma(E_1), \dotsc, \nu_l \in \Gamma(E_l)$. In general we also make use of that sections of $\Phi^*E$ can be viewed as sections of $E$ along $\Phi$, where $E \stackrel{\pi}{\to} N$ is any vector bundle over $N$. Let $\mu \in \Gamma(\Phi^*E)$, then it has the form $\mu_p = (p, u_p)$ for all $p \in M$, where $u_p \in E_{\Phi(p)}$, the fibre of $E$ at $\Phi(p)$; and a section $\nu$ of $E$ along $\Phi$ is a smooth map $M \to E$ such that $\pi \circ \nu = \Phi$. Then on one hand $\mathrm{pr}_2 \circ \mu$ is a section along $\Phi$, where $\mathrm{pr}_2$ is the projection onto the second component, and on the other hand $M \ni p \mapsto (p, \nu_p)$ defines an element of $\Gamma(\Phi^*E)$. With that one can show that there is a 1:1 correspondence of $\Gamma(\Phi^*E)$ with sections along $\Phi$. Similarly, vector bundle morphisms $L: G \to E$ over $\Phi$ have 1:1 correspondences to base-preserving vector bundle morpishms $G \to \Phi^*E$, where $G \to M$ is a vector bundle over $M$. We do not necessarily mention it when we make use of such trivial identifications, it should be clear by the context. For example $\mathrm{D}\Phi$ denotes the total differential of $\Phi$ (also called tangent map). It can be viewed as a vector bundle morphism $\mathrm{T}M \to \mathrm{T}N$ over $\Phi$, and we often view it as an element of $\Omega^1(M; \Phi^*\mathrm{T}N)$ by $\mathfrak{X}(M) \ni Y \mapsto \mathrm{D}\Phi(Y)$, where $\mathrm{D}\Phi(Y) \in \Gamma(\Phi^*\mathrm{T}N), M \ni p \mapsto \mathrm{D}_p\Phi(Y_p)$.
%
%Additionally, with $\Omega^k(N; E)$ ($k \in \mathbb{N}_0$) we denote $k$-forms on $N$ with values in a vector bundle $E \to N$, and we always use the Einstein's sum convention. If one has a connection $\nabla$ on a vector bundle $V \to N$, then one has the notion of the exterior covariant derivative on $\Omega^p(M;E)$, denoted by $\mathrm{d}^\nabla$. In the case of a trivial vector bundle $V=N \times W \to N$, where $W$ is some vector space, we will often use the \textbf{canonical flat connection} for $\nabla$, defined by $\nabla \nu = 0$, where $\nu$ is a constant section of $N \times W$, see \textit{e.g.}~\cite[Example 5.1.7; page 260f.]{hamilton} for a geometric interpretation as horizontal distribution. The canonical flat connection is clearly uniquely defined (if a trivialization is given) because constant sections generate all sections and due to the Leibniz rule and linearity of $\nabla$. Let $\mleft( e_a \mright)_a$ be a constant global frame of $N \times W$, thence,
%\bas
%\mathrm{d}^\nabla \omega
%&=
%\mathrm{d} \omega^a \otimes e_a
%\eas
%for all $\omega \in \Omega^p(M; W)$, where we write $\omega= \omega^a \otimes e_a$. Hence, we define
%\ba
%\mathrm{d}\omega
%&\coloneqq
%\mathrm{d}^\nabla \omega,
%\ea
%when $\nabla$ is the canonical flat connection. $\mathrm{d}$ is clearly a differential.
%
%As usual, there will be definitions of certain objects depending on other elements, and for keeping notations simple we will not always explicitly denote all dependencies. It will be clear by context on which it is based on, that is, when we define an object $A$ using the notion of Lie algebra actions $\gamma$ and we write "Let $A$ be [as defined before]", then it will be clear by context which Lie algebra action is going to be used, for example given in a previous sentence writing "Let $\gamma$ be a Lie algebra action".
%%, and recall the following wedge product\footnote{As also defined in \cite[\S 5, third part of Exercise 5.15.12; page 316]{hamilton}.} of forms with values in a vector bundle $E$ and values in its space of endomorphisms $\mathrm{End}(E)$,
%%\bas
%%\wedge: \Omega^k(N; \mathrm{End}(E)) \times \Omega^l(N; E)
%%&\mapsto
%%\Omega^{k+l}(N; E) \\
%%(T, \omega) &\mapsto T \wedge \omega
%%\eas
%%for all $k, l \in \mathbb{N}_0$, given by
%%\ba\label{DefVonWedgedemitEnd}
%%\mleft( T \wedge \omega \mright) \mleft( Y_1, \dotsc, Y_{k+l} \mright)
%%&\coloneqq
%%\frac{1}{k! l!} \sum_{\sigma \in S_{k+l}} \mathrm{sgn}(\sigma) ~
	%%T \mleft( Y_{\sigma(1)}, \dotsc, Y_{\sigma(k)} \mright)
		%%\mleft( \omega\mleft( Y_{\sigma(k+1)}, \dotsc, Y_{\sigma(k+l)} \mright) \mright),
%%\ea
%%where $S_{k+l}$ is the group of permutations $\{1, \dotsc, k+l\}$. This is then locally given by, with respect to a frame $\mleft( e_a \mright)_a$ of $E$,
%%\bas
%%T \wedge \omega &= T(e_a) \wedge w^a,
%%\eas
%%where $T$ acts as an endomorphism on $e_a$, \textit{i.e.}~$T(e_a) \in \Omega^k(N; E)$, and $\omega = \omega^a \otimes e_a$. Also recall that there is the canonical extension of $\nabla$ on $\mathrm{End}(E)$ by forcing the Leibniz rule. We still denote this connection by $\nabla$, too.
%
%We also need the following definitions.
%
%\begin{definitions}{Graded extension of products, \newline \cite[generalization of Definition 5.5.3; page 275]{hamilton}}{GradingOfProducts}
%Let $l \in \mathbb{N}$ and $E_1, \dots E_{l+1} \to N$ be vector bundles over a smooth manifold $N$, and $F \in \Gamma\left( \left(\bigotimes_{m=1}^{l} E_m^*\right) \otimes E_{l+1} \right)$. Then we define the \textbf{graded extension of $F$} as
	%\bas
%\Omega^{k_1}(N; E_1) \times \dots \times \Omega^{k_l}(N; E_l)
%&\to \Omega^{k}(N; E_{l+1}), \\
%(A_1, \dots, A_l)
%&\mapsto
%F\mleft(A_1\stackrel{\wedge}{,} \dotsc \stackrel{\wedge}{,} A_l\mright),
%\eas
%where $k := k_1+\dots k_l$ and $k_i \in \mathbb{N}_0$ for all $i\in \{1, \dots, l\}$. $F\mleft(A_1\stackrel{\wedge}{,} \dotsc \stackrel{\wedge}{,} A_l\mright)$ is defined as an element of $\Omega^{k}(N; E_{l+1})$ by
%\bas
%&F\mleft(A_1\stackrel{\wedge}{,} \dotsc \stackrel{\wedge}{,} A_l\mright)\mleft(Y_1, \dots, Y_{k}\mright)
%\coloneqq \\
%&\frac{1}{k_1! \cdot \dots \cdot k_l!} \sum_{\sigma \in S_{k}} \mathrm{sgn}(\sigma) ~ F\left( A_1\left( Y_{\sigma(1)}, \dots, Y_{\sigma(k_1)} \right), \dots, A_l\left( Y_{\sigma(k-k_l+1)}, \dots, Y_{\sigma(k)} \right) \right)
%\eas
%for all $Y_1, \dots, Y_{k} \in \mathfrak{X}(N)$, where $S_{k}$ is the group of permutations of $\{1, \dots, k\}$ and $\mathrm{sgn}(\sigma)$ the signature of a given permutation $\sigma$. 
%
%$\stackrel{\wedge}{,}$ may be written just as a comma when a zero-form is involved.
%
%Locally, with respect to given frames $\mleft( e^{(i)}_{a_i} \mright)_{a_i}$ of $E_i$, this definition has the form
%\ba\label{CoordExprOfGradedExtension}
%F\mleft(A_1\stackrel{\wedge}{,} \dotsc \stackrel{\wedge}{,} A_l\mright)
%&=
%F\mleft(e^{(1)}_{a_1}, \dotsc, e^{(l)}_{a_l}\mright) \otimes A_1^{a_1} \wedge \dotsc \wedge A_l^{a_l}
%\ea
%for all $A_i = A_i^{a_i} \otimes e^{(i)}_{a_i}$, where $A_i^{a_i}$ are $k_i$-forms on $N$.
%\end{definitions}
%
%\begin{remark}
%\leavevmode\newline
%Assume $F \in \Gamma\left( \mleft(\bigwedge_{m=1}^{l} \mathrm{T}^*N \mright) \otimes E \right) \cong \Omega^l(N; E)$ for some vector bundle $E$, \textit{i.e.}~$F$ is an $l$-form on $N$ with values in $E$. The pull-back $\Phi^*F$ by $\Phi$ can be then viewed as an element of $\Gamma\left( \bigwedge_{m=1}^{l} \mleft(\Phi^*\mathrm{T}N\mright)^* \otimes \Phi^*E \right)$.
%
%Do not confuse this pull-back with the pull-back of forms, here denoted by $\Phi^!F$, which is an element of $\Gamma\left( \mleft(\bigwedge_{m=1}^{l} \mathrm{T}^*M \mright) \otimes \Phi^*E \right) \cong \Omega^l(M; \Phi^*E)$ defined by
%\ba
%\mleft.\mleft(\Phi^!F\mright)(Y_1, \dots, Y_l)\mright|_p
%&\coloneqq
%F_{\Phi(p)}\mleft(\mathrm{D}_p\Phi\mleft(\mleft.Y_1\mright|_p\mright), \dots, \mathrm{D}_p\Phi\mleft(\mleft.Y_l\mright|_p\mright)\mright)
%\ea
%for all $p \in M$ and $Y_1, \dots, Y_l \in \mathfrak{X}(M)$. Then
%\ba\label{EqPullBackFormelFuerVerschiedeneDefinitionen}
%\Phi^!F 
%&=
%\frac{1}{l!}~
%\mleft(\Phi^*F\mright) ( \underbrace{\mathrm{D}\Phi \stackrel{\wedge}{,} \dotsc \stackrel{\wedge}{,} \mathrm{D}\Phi}_{l \text{ times}} )
%\ea
%by using the anti-symmetry of $F$ and Def.~\ref{def:GradingOfProducts}, \textit{i.e.}
%\bas
%&\mleft.\frac{1}{l!}~
%\Big(\mleft(\Phi^*F\mright) ( \mathrm{D}\Phi \stackrel{\wedge}{,} \dotsc \stackrel{\wedge}{,} \mathrm{D}\Phi ) \Big) (Y_1, \dots, Y_l)\mright|_p \\
%&\hspace{1cm}
%=
%\frac{1}{l!}~
%\sum_{\sigma \in S_{l}} \mathrm{sgn}(\sigma) ~ \underbrace{(\Phi^*F)\mleft(\mathrm{D}\Phi\mleft(Y_{\sigma(1)}\mright), \dots, \mathrm{D}\Phi\mleft(Y_{\sigma(l)}\mright)\mright)}_{\mathclap{= \mathrm{sgn}(\sigma) ~ (\Phi^*F)\mleft(\mathrm{D}\Phi\mleft(Y_{1}\mright), \dots, \mathrm{D}\Phi\mleft(Y_{l}\mright)\mright)}}\Big|_p \\
%&\hspace{1cm}
%=
%\frac{1}{l!}~ \underbrace{\mleft( \sum_{\sigma \in S_{l}} 1 \mright)}_{= l!} ~
%F_{\Phi(p)}\mleft(\mathrm{D}_p\Phi\mleft(\mleft.Y_{1}\mright|_p\mright), \dots, \mathrm{D}_p\Phi\mleft(\mleft.Y_{l}\mright|_{p}\mright)\mright) \\
%&\hspace{1cm}
%= \mleft.\mleft(\Phi^!F\mright)(Y_1, \dots, Y_l)\mright|_p
%\eas
%for all $p \in M$ and $Y_1, \dots, Y_l \in \mathfrak{X}(M)$.
%\end{remark}
%
%In case of antisymmetric tensors we of course preserve that.
%
%\begin{propositions}{Graded extensions of antisymmetric tensors}{GradedExtensionPlusAntiSymm}
%Let $E_1, E_2 \to N$ be real vector bundles of finite rank over a smooth manifold $N$, $F \in \Omega^2(E_1; E_2)$. Then
%\ba
%F \mleft( A \stackrel{\wedge}{,} B \mright)
%&=
%-\mleft( -1 \mright)^{km}
%F \mleft( B \stackrel{\wedge}{,} A \mright)
%\ea 
%for all $A \in \Omega^k(N; E_1)$ and $B \in \Omega^m(N; E_2)$ ($k,m \in \mathbb{N}_0$). Similarly extended to all $F \in \Omega^l(E_1; E_2)$.
%\end{propositions}
%
%\begin{remark}
%\leavevmode\newline
%This is a generalization of similar relations just using the Lie algebra bracket $\mleft[ \cdot, \cdot\mright]_{\mathfrak{g}}$ of a Lie algebra $\mathfrak{g}$, see \cite[\S 5, first statement of Exercise 5.15.14; page 316]{hamilton}.
%\end{remark}
%
%\begin{proof}
%\leavevmode\newline
%Trivial by using Eq.~\eqref{CoordExprOfGradedExtension}.
%\end{proof}
%
%We also need to know what a Lie algebroid is, a generalization of both, tangent bundles and Lie algebras; this concept will just be defined, refer to the references for thorough discussions of these definitions, especially \cite{mackenzieGeneralTheory} and \cite[\S VII; page 113ff.]{DaSilva}.
%
%\begin{definitions}{Lie algebroid, \newline \cite[\S 3.3, first part of Definition 3.3.1; page 100]{mackenzieGeneralTheory}}{test}
%%\leavevmode\newline
%Let $E \to N$ be a real vector bundle of finite rank. Then $E$ is a smooth Lie algebroid if there is a bundle map $\rho: E \to \mathrm{T}N$, called the \textbf{anchor}, and a Lie algebra structure on $\Gamma(E)$ with Lie bracket $\mleft[ \cdot, \cdot \mright]_E$ satisfying
%\ba
  %\mleft[\mu, f \nu\mright]_E = f \mleft[\mu, \nu\mright]_E + \mathcal{L}_{\rho(\mu)}(f) ~ \nu
%\label{eq:E-Leibniz}
%\ea
%for all $f \in C^\infty(N)$ and $\mu, \nu \in \Gamma(E)$, where $\mathcal{L}_{\rho(\mu)}(f)$ is the action of the vector field $\rho(\mu)$ on the function $f$ by derivation. We will sometimes denote a Lie algebroid by $\mleft( E, \rho, \mleft[ \cdot, \cdot \mright]_E \mright)$.
%%We will sometimes denote a Lie algebroid by $\mleft( E, \rho, \mleft[ \cdot, \cdot \mright]_E \mright)$.
%\end{definitions}
%
%%Tangent bundles are a canonical example of Lie algebroids, their anchor is the identity with which we also equip them; another canonical example with zero anchor are the Lie algebra bundles:
%Tangent bundles and bundles of Lie algebras are canonical examples of Lie algebroids, their anchor is the identity and zero, respectively. The important example for us is a mixture of those examples:
%
%\begin{propositions}{Action Lie algebroids, \cite[\S 16.2, Example 5; page 114]{DaSilva}}{ActionLieoidsAreOids}
%Let $\mleft(\mathfrak{g}, \mleft[\cdot, \cdot \mright]_{\mathfrak{g}}\mright)$ be some Lie algebra equipped with a Lie algebra action $\gamma: \mathfrak{g} \to \mathfrak{X}(N)$ on a smooth manifold $N$. Then there is a unique Lie algebroid structure on $E = N \times \mathfrak{g}$ such that we have
%\ba
%\rho(\nu)
%&=
%\gamma(\nu),
%\\
%\mleft[\mu, \nu\mright]_E
%&=
%\mleft[\mu, \nu\mright]_{\mathfrak{g}}
%\ea
%for all constant sections $\mu, \nu \in \Gamma(E)$. We call this structure \textbf{action Lie algebroid}.
%\end{propositions}

\section{Curved Yang-Mills gauge theory}

Notation as in \cite{Hamilton}

\begin{itemize}
	\item $\widetilde{G}$ Lie group with Lie algebra $\mathfrak{g}$
	\item $M$ smooth manifold (usually also a spacetime). An open subset of $M$ is usually denoted by $U$; typically small enough that "everything works out" (especially without further mentioning intersections of given open sets and so on)
	\item $P \to M$ a principal bundle, a (local) gauge is usually denoted by $s$, an element of $\Gamma(P)$, sections of $P$
	\item $V$ a vector space
	\item $\rho$ a Lie group representation on $V$, $\rho_*$ the induced Lie algebra representation on $V$
	\item $K \coloneqq P \times_\rho V$ the associated vector bundle induced by $P$ and $\rho$ on $V$. An element $\Phi$ of $K$ is denoted by by $[p, \phi]$ for $p \in P$ and $\phi \in V$, where $[ \cdot, \cdot]$ denotes the equivalence class with respect to the equivalence
	\bas
		(p, \phi) &\sim \mleft(p g, \rho\mleft( g^{-1} \mright) \cdot \phi \mright)
	\eas
	for all $g \in \widetilde{G}$; $pg$ denotes the canonical group action (from the right) $P \times \widetilde{G} \to P$ and $\cdot$ the action of $\mathrm{Aut}(V) \subset \mathrm{End}(V)$ on $V$.
	\item Especially if fixing a local gauge $s: U \to P$ we can write for sections $\Phi \in \Gamma(K)$ locally 
	\bas
		\Phi|_U
		&=
		[s, \phi],
	\eas
	where $\phi: U \to V$, \textit{i.e.}\ a local section of the trivial vector bundle $M \times V \to M$.
	\item We especially focus on $V = \mathfrak{g}$ and $\rho = \mathrm{Ad}$ the adjoint representation of $\widetilde{G}$ on $\mathfrak{g}$.
\end{itemize}

The field of gauge bosons $A$ is a connection on the principal bundle, \textit{i.e.}\ an element of $\Omega^1(P; \mathfrak{g})$ with

\bas
r_g^!A &= \mathrm{Ad}_{g^{-1}} (A) \coloneqq \mathrm{Ad}_{g^{-1}} \circ A, \\
A\mleft( \widetilde{X} \mright) &= X
\eas
for all $g \in \widetilde{G}$ and $X \in \mathfrak{g}$, where $r_g^!$ is the pullback of forms via the right $\widetilde{G}$-multiplication on $P$, and $\widetilde{X}$ the fundamental vector field of $X$ on $P$. 

Typically, a lot of the formalism of gauge theory comes from how to define the minimal coupling. So, let us look at this and reinvent it a bit. Usually the covariant derivative/minimal coupling $\nabla^A$ of $A$ and $\Phi \in \Gamma(K)$ is locally (w.r.t.\ to a gauge $s$) defined by
\bas
\nabla^A \Phi
&\coloneqq
\mleft[ s, \nabla^A \phi \mright],
\eas
where
\ba\label{ClassicalMiniCoupling}
\nabla^A \phi
&\coloneqq
\mathrm{d}\phi
	+ \rho_*\mleft( A_s \mright) \cdot \phi,
\ea
where $A_s \coloneqq s^! A \in \Omega^1(U; \mathfrak{g})$ (local pullback as a form of $A$ via $s$) and $\mathrm{d} \phi \coloneqq \nabla^0 \phi$, $\nabla^0$ the canonical flat connection on $M \times V$.

The explicit definition of the field strength $F$ of $A$ is then usually motivated by looking at the curvature $R_{\nabla^A}$ of $\nabla^A$, that is
\bas
\mleft.R_{\nabla^A}(\cdot, \cdot) \Phi\mright|_U
&=
\mleft[ s,
	\rho_*(F_s) \cdot \phi
\mright],
\eas
where 
\bas
F_s
&\coloneqq
\mathrm{d}A_s
	+ \frac{1}{2} \mleft[ A_s \stackrel{\wedge}{,} A_s \mright]_{\mathfrak{g}}
\eas
is the typical local definition of $F_s \in \Omega^2(U; \mathfrak{g})$ with
\bas
\mleft[ A_s \stackrel{\wedge}{,} A_s \mright]_{\mathfrak{g}}(X, Y)
&=
2 ~ \mleft[ A_s(X) , A_s(Y) \mright]_{\mathfrak{g}}
\eas
for all $X, Y \in \mathfrak{X}(U)$. (The notation $F_s$ is of course due to the fact that $F_s = s^!F$, where $F$ is the curvature of $A$. But I want to avoid that for now because of what we are about doing to do.) We shortly could denote this also as
\ba\label{MotivOfFieldStrength}
R_{\nabla^A} \phi
&=
\rho_*(F_s) \cdot \phi
\ea

\textbf{Now:} One could question why using $\mathrm{d}\phi = \nabla^0 \phi$ in Eq.\ \eqref{ClassicalMiniCoupling}. Thence, let us assume that we have a general vector bundle connection $\widehat{\nabla}$ on the trivial vector bundle $M \times V \to M$. We are going to redefine $\nabla^A$ and $F$ locally w.r.t.\ a gauge $s$, then discuss how the gauge transformations have to look like to receive definitions independent of the chosen gauge $s$. This also means that the following discussion is now often local by fixing a gauge without further mentioning it.

Let us first locally redefine $\nabla^A \phi$:
\ba\label{NewMinimalCoupling}
\nabla^A \phi
&\coloneqq
\widehat{\nabla} \phi
	+ \rho_*(A_s) \cdot \phi.
\ea
Motivated by Eq.\ \eqref{MotivOfFieldStrength}, we want to identify the field strength with the curvature of $\nabla^A$. One can check that we have
\ba\label{FirstStepTowardsNewFieldStrength}
R_{\nabla^A}
&=
R_{\widehat{\nabla}}
	+ \mathrm{d}^{\widehat{\nabla}} \bigl( \rho_*(A_s) \bigr)
	+ \rho_*(A_s) \wedge \rho_*(A_s),
\ea
where $\mathrm{d}^{\widehat{\nabla}}$ is the exterior covariant derivative of $\widehat{\nabla}$ canonically extended to $\mathrm{End}(V)$, viewing $\rho_*(A_s)$ as an element of $\Omega^1(U; \mathrm{End}(V))$, and where $\rho_*(A_s) \wedge \rho_*(A_s)$ is an element of $\Omega^2(U; \mathrm{End}(V))$ given by
\bas
\bigl(\rho_*(A_s) \wedge \rho_*(A_s)\bigr)(X, Y)
&\coloneqq
\rho_*\bigl( A_s(X) \bigr) \circ \rho_*\bigl(A_s(Y)\bigr)
	- \rho_*\bigl( A_s(Y) \bigr) \circ \rho_*\bigl(A_s(X)\bigr)
\\
&=
\mleft[ \rho_*\bigl( A_s(X) \bigr), \rho_*\bigl( A_s(Y) \bigr) \mright]_{\mathrm{End}(V)}
\\
&=\rho_* \mleft(
	\mleft[ A_s(X), A_s(Y) \mright]_{\mathfrak{g}}
\mright)
\\
&=\rho_* \mleft(
	\frac{1}{2}\mleft[ A_s \stackrel{\wedge}{,} A_s \mright]_{\mathfrak{g}}
\mright)(X,Y)
\eas
for all $X, Y \in \mathfrak{X}(U)$.

In order to have a similar shape as in Eq.\ \eqref{MotivOfFieldStrength}, we now assume that $\widehat{\nabla}$ satisfies the following \textbf{compatibility conditions}:
\begin{remarks}{Comaptibility conditions}{CompCondsSimple}
\ba
R_{\widehat{\nabla}} 
&=
\rho_*(\zeta),\label{ZetaCondition}\\
\widehat{\nabla} \circ \rho_*
&=
\rho_* \circ \nabla\label{VanishingBasicCurvature}
\ea
for some $\zeta \in \Omega^2(M; \mathfrak{g})$ and $\nabla$ a vector bundle connection on the trivial vector bundle $M \times \mathfrak{g} \to M$.
\end{remarks}

If we want that Eq.\ \eqref{FirstStepTowardsNewFieldStrength} has a shape like Eq.\ \eqref{MotivOfFieldStrength}, it is obvious why we require \eqref{ZetaCondition}; \eqref{VanishingBasicCurvature} is needed for the second summand in Eq.\ \eqref{FirstStepTowardsNewFieldStrength}. Hence, let us study \eqref{VanishingBasicCurvature}, that is
\bas
\widehat{\nabla}\bigl( \rho_*(\nu) \bigr)
&=
\rho_*(\nabla \nu)
\eas
for all $\nu \in \Gamma(M \times \mathfrak{g})$,\footnote{Elements of $\mathfrak{g}$ are viewed as constant sections of $M \times \mathfrak{g}$.} especially $\widehat{\nabla}$ is again extended to $\mathrm{End}(V)$ on the left hand side. With this we get
\bas
\mathrm{d}^{\widehat{\nabla}} \bigl( \rho_*(A_s) \bigr)(X,Y)
&=
\widehat{\nabla}_X\bigl( \rho_*(A_s(Y)) \bigr)
	- \widehat{\nabla}_Y\bigl( \rho_*(A_s(X)) \bigr)
	- \rho_*\mleft( A_s\bigl([X, Y]\bigr) \mright)
\\
&=
\rho_*\mleft(\nabla_X \bigl( A_s(Y) \bigr) \mright)
	- \rho_*\mleft( \nabla_Y \bigl(A_s(X)\bigr) \mright)
	- \rho_*\mleft( A_s\bigl([X, Y]\bigr) \mright)
\\
&=
\rho_* \mleft(
	\nabla_X \bigl( A_s(Y) \bigr)
	- \nabla_Y \bigl(A_s(X)\bigr)
	- A_s\bigl([X, Y]\bigr)
\mright)
\\
&=
\rho_*\mleft( \mathrm{d}^\nabla A_s \mright)(X,Y)
\eas
for all $X, Y \in \mathfrak{X}(U)$. Collecting everything, Eq.\ \eqref{FirstStepTowardsNewFieldStrength} has now the following form
\bas
R_{\nabla^A}
&=
\rho_*\mleft( \mathrm{d}^\nabla A_s + \frac{1}{2}\mleft[ A_s \stackrel{\wedge}{,} A_s \mright]_{\mathfrak{g}}+ \zeta \mright).
\eas
So, we have a new form of the field strength, assuming that $\nabla$ and $\zeta$ satisfy the compatibility conditions in Remark \ref{rem:CompCondsSimple}. This is precisely the definition of the field strength as in the gauge theory of Thomas and Alexei, that is, we have a new field strength
\bas
G
&\coloneqq
\mathrm{d}^\nabla A_s + \frac{1}{2}\mleft[ A_s \stackrel{\wedge}{,} A_s \mright]_{\mathfrak{g}}+ \zeta.
\eas
Furthermore, if we are interested into Yang-Mills gauge theories, then we'd have $K = P \times_{\mathrm{Ad}} \mathfrak{g}$ (the adjoint bundle), and so also $\rho_* = \mathrm{ad}$. In this case we can put $\widehat{\nabla} = \nabla$ and then the compatibility conditions in Remark \ref{rem:CompCondsSimple} read
\bas
R_\nabla
&=
\mathrm{ad}(\zeta),
\\
\nabla \circ \mathrm{ad}
&=
\mathrm{ad} \circ \nabla.
\eas
The second condition precisely gives after a short calculation
\bas
\nabla\mleft( \mleft[ \mu, \nu \mright]_{\mathfrak{g}} \mright)
&=
\mleft[ \nabla\mu, \nu \mright]_{\mathfrak{g}}
	+ \mleft[ \mu, \nabla\nu \mright]_{\mathfrak{g}}
\eas
for all $\mu, \nu \in \Gamma(M \times \mathfrak{g})$, so, $\nabla$ has to be a Lie bracket derivation. So, in this case the compatibility conditions in Remark \ref{rem:CompCondsSimple} precisely reduce to the compatibility conditions of Alexei's and Thomas's theory! (in the case of Lie algebra bundles; the general theory is more general, formulated on general Lie algebroids)

As a summary:

\begin{remarks}{Summary}{FirstSummary}
We have
\ba
R_\nabla
&=
\mathrm{ad}(\zeta),
\\
\nabla \circ \mathrm{ad}
&=
\mathrm{ad} \circ \nabla,
\\
G
&=
\mathrm{d}^\nabla A_s + \frac{1}{2}\mleft[ A_s \stackrel{\wedge}{,} A_s \mright]_{\mathfrak{g}}+ \zeta.
\ea
\end{remarks}

In fact, the compatibility conditions lead to a gauge invariant theory: Fix an $\mathrm{ad}$-invariant scalar product $\kappa$ on $\mathfrak{g}$; then define the Lagrangian by
\ba
\mathfrak{L}_{\mathrm{YM}}
&\coloneqq
- \frac{1}{2} \kappa\mleft( G \stackrel{\wedge}{,} *G \mright)
\ea
where $*$ is the Hodge star operator w.r.t.\ some spacetime metric. (In short, the typical definition, but replace $F$ with $G$) It is easier to look at the infinitesimal version of the gauge transformations, hence everything with respect to a gauge $s$ now.

In order to derive a formula for these, let us again look at $\nabla^A$. Fix an $\varepsilon \in \Gamma(M \times \mathfrak{g})$, then the infinitesimal gauge transformation $\delta_\varepsilon \phi$ of $\phi \in \Gamma(M \times \mathfrak{g})$ is usually defined by
\bas
\delta_\varepsilon \phi
&\coloneqq
\rho_*(\varepsilon) \cdot \phi.
\eas
We fix the infinitesimal gauge trafo $\delta_\varepsilon A$ of $A$ by looking at the gauge trafo of $\nabla^A \phi$ via
\bas
\delta_\varepsilon \nabla^A \phi
&=
\mleft.\frac{\mathrm{d}}{\mathrm{dt}}\mright|_{t=0}\mleft( \nabla^{A + t\delta_\varepsilon A} \mleft( \phi + t \delta_\varepsilon \phi \mright) \mright)
\\
&=
\underbrace{\widehat{\nabla} \mleft( \delta_\varepsilon \phi \mright)}
	_{\mathclap{ = \mleft(\widehat{\nabla} \mleft( \rho_*(\varepsilon) \mright)\mright) \cdot \phi + \rho_*(\varepsilon) \cdot \widehat{\nabla} \phi }}
	+ \rho_*\mleft( \delta_\varepsilon A_s \mright) \cdot \phi
	+ \rho_*(A_s) \cdot \delta_\varepsilon \phi
\\
&=
\bigl( \rho_*\mleft( \nabla \varepsilon + \delta_\varepsilon A_s \mright) + \rho_*(A_s) \cdot \rho_*(\varepsilon) \bigr) \cdot \phi
	+ \rho_*(\varepsilon) \cdot \widehat{\nabla} \phi
\eas
using Remark \ref{rem:CompCondsSimple}. We want $\delta_\varepsilon \nabla^A \phi = \rho_*(\varepsilon) \cdot \nabla^A \phi$ which gives
\bas
\rho_*(\varepsilon) \cdot \nabla^A \phi
&=
\rho_*(\varepsilon) \cdot \widehat{\nabla} \phi
	+ \rho_*(\varepsilon) \cdot \rho_*(A_s) \cdot \phi.
\eas
Imposing $\delta_\varepsilon \nabla^A \phi = \rho_*(\varepsilon) \cdot \nabla^A \phi$ we get
\bas
\rho_*\mleft(\delta_\varepsilon A_s + \nabla \varepsilon + \mleft[ A_s, \varepsilon \mright]_{\mathfrak{g}} \mright)
&=
0
\eas
using again that $\rho_*$ is a Lie algebra representation. If we require that this shall work for all $\rho_*$, we may say
\ba\label{GaugeTrafoOfANew}
\delta_\varepsilon A_s
&\coloneqq
-\nabla \varepsilon
	+ \mleft[ \varepsilon, A_s \mright]_{\mathfrak{g}}.
\ea
This is precisely the infinitesimal gauge trafo of $A$ as in the theory of Thomas and Alexei! Hence, we achieve infinitesimal gauge invariance of $\mathfrak{L}_{\mathrm{YM}}$. For completeness, let us check the gauge trafo of $G$ using Def.\ \eqref{GaugeTrafoOfANew} and Remark \ref{rem:FirstSummary}, it is very similar to the "classical" calculation due to Remark \ref{rem:FirstSummary} which is why I skip some straightforward calculations to keep it short,
\bas
\delta_\varepsilon G
&=
\mleft.\frac{\mathrm{d}}{\mathrm{d}t}\mright|_{t=0}
\mleft(
	\mathrm{d}^\nabla (A_s + t\delta_\varepsilon A_s) + \frac{1}{2}\mleft[ A_s + t\delta_\varepsilon A_s \stackrel{\wedge}{,} A_s + t\delta_\varepsilon A_s \mright]_{\mathfrak{g}}+ \zeta
\mright)
\\
&=
\mathrm{d}^\nabla \mleft( -\nabla \varepsilon
	+ \mleft[ \varepsilon, A_s \mright]_{\mathfrak{g}} \mright)
		+ \mleft[ A_s \stackrel{\wedge}{,} -\nabla \varepsilon + \mleft[ \varepsilon, A_s \mright]_{\mathfrak{g}} \mright]_{\mathfrak{g}}
\\
&=
\underbrace{- \mleft(\mathrm{d}^\nabla\mright)^2 \varepsilon}_{= - R_\nabla \varepsilon = \mleft[ \varepsilon, \zeta \mright]_{\mathfrak{g}}}
	+ \mleft[ \nabla \varepsilon \stackrel{\wedge}{,} A_s \mright]_{\mathfrak{g}}
	+ \mleft[ \varepsilon, \mathrm{d}^\nabla A_s \mright]_{\mathfrak{g}}
	+ \underbrace{\mleft[ A_s \stackrel{\wedge}{,} -\nabla \varepsilon \mright]_{\mathfrak{g}}}_{= - \mleft[ \nabla \varepsilon \stackrel{\wedge}{,} A_s \mright]_{\mathfrak{g}}}
	+ \mleft[ A_s \stackrel{\wedge}{,} \mleft[ \varepsilon, A_s \mright]_{\mathfrak{g}} \mright]_{\mathfrak{g}}
\\
&=
\mleft[ \varepsilon,
	\mathrm{d}^\nabla A_s + \zeta
\mright]_{\mathfrak{g}}
	+ \mleft[ A_s \stackrel{\wedge}{,} \mleft[ \varepsilon, A_s \mright]_{\mathfrak{g}} \mright]_{\mathfrak{g}}
\eas
and, using the Jacobi identity,
\bas
\mleft[ A_s \stackrel{\wedge}{,} \mleft[ \varepsilon, A_s \mright]_{\mathfrak{g}} \mright]_{\mathfrak{g}} (X,Y)
&=
\mleft[ A_s(X) , \mleft[ \varepsilon, A_s(Y) \mright]_{\mathfrak{g}} \mright]_{\mathfrak{g}}
	- \mleft[ A_s(Y) , \mleft[ \varepsilon, A_s(X) \mright]_{\mathfrak{g}} \mright]_{\mathfrak{g}}
\\
&=
\mleft[ \varepsilon, \mleft[ A_s(X), A_s(Y) \mright]_{\mathfrak{g}} \mright]_{\mathfrak{g}}
\\
&=
\mleft[ \varepsilon, \frac{1}{2} \mleft[ A_s \stackrel{\wedge}{,} A_s \mright]_{\mathfrak{g}} \mright]_{\mathfrak{g}} (X, Y)
\eas
for all $X, Y \in \mathfrak{X}(U)$. Altogether
\bas
\delta_\varepsilon G
&=
\mleft[ \varepsilon,
	\mathrm{d}^\nabla A_s + \frac{1}{2} \mleft[ A_s \stackrel{\wedge}{,} A_s \mright]_{\mathfrak{g}} + \zeta
\mright]_{\mathfrak{g}}
=
\mleft[ \varepsilon, G
\mright]_{\mathfrak{g}}.
\eas
Hence, the field strength transforms with the adjoin of $\varepsilon$; since $\kappa$ is $\mathrm{ad}$-invariant, we can derive that $\mathfrak{L}_{\mathrm{YM}}$ is invariant under the infinitesimal gauge trafo in Def.\ \eqref{GaugeTrafoOfANew}!

Observe that by Remark \ref{rem:FirstSummary} that $\zeta$ can be non-trivial even if we still use $\nabla = \nabla^0$, the canonical flat connection on $M\times \mathfrak{g}$, even though this whole discussion started with allowing more general connections.

If we minimise $\mathfrak{L}_{\mathrm{YM}}$, then one obvious way would be to search solutions with $G \equiv 0$ for an absolute minimum/maximum (because of the sign), doing so would result into that the classical Yang-Mills energy would have a bound which is non-zero. May this be an explanation for the mass gap? As shown in my thesis, every classical theory has a $\zeta$ after a field redefinition. Even though field redefinitions are an equivalence for the classical theories, one may argue that it does not describe an equivalence for the quantised theory, leading to a possible explanation of the mass gap? But that is just high hope right now :) 

\subsection{Integration}

For an integrated version of Def.\ \eqref{GaugeTrafoOfANew} we need to discuss when the new "minimal coupling" of Def.\ \eqref{NewMinimalCoupling} behaves nicely under a change of the gauge $s$. That is, we now want to extend the new definition of $\nabla^A$ to a well-defined connection on $K = P \times_{\rho} V$, especially on the adjoint bundle $K = P \times_{\mathrm{Ad}} \mathfrak{g}$ in our case. (and later maybe generalise this to a $\widetilde{G}$-quotient of a general Lie algebra bundle over $P$)

Let $s^\prime$ be another (local) gauge such that we have a unique smooth map $g: U \to G$ such that
\bas
s^\prime
&=
s g,
\eas
then we want for well-definedness
\ba\label{WellDef}
\nabla^A \Phi
&=
\mleft[ s, \nabla \phi + \mathrm{ad}(A_s) \cdot \phi \mright]
\stackrel{!}{=}
\mleft[ s^\prime, \nabla \phi^\prime + \mathrm{ad}(A_{s^\prime}) \cdot \phi^\prime \mright],
\ea
where we have $\Phi = [s, \phi] = [s^\prime, \phi^\prime]$, especially
\bas
\phi^\prime
&=
\mathrm{Ad}\mleft( g^{-1} \mright) \cdot \phi.
\eas
Since the new field strength $G$ still transforms via the adjoin under $\delta_\varepsilon$ (see above), we make the following ansatz
\ba
A_{s^\prime} 
&=
\mathrm{Ad}\mleft( g^{-1} \mright) \cdot A_s
	+ \mu,
\ea
where $\mu \in \Omega^1(U; \mathfrak{g})$. Usually, $\mu = g^!\mu_{\widetilde{G}}$, the pullback as a form of the Maurer-Cartan-Form $\mu_{\widetilde{G}}$ on $\widetilde{G}$. One can then check with some short calculation that Eq.\ \eqref{WellDef} is equivalent to
\bas
\nabla\mleft( \mathrm{Ad}\mleft( g^{-1} \mright) \cdot \phi \mright)
	+ \mathrm{ad}(\mu) \cdot \mathrm{Ad}\mleft( g^{-1} \mright) \cdot \phi
&\stackrel{!}{=}
\mathrm{Ad}\mleft( g^{-1} \mright) \cdot \nabla \phi
\eas
using the definition of $P \times_{\mathrm{Ad}} \mathfrak{g}$. Equivalently,
\bas
\mathrm{ad}(\mu)
&=
\mathrm{Ad}\mleft( g \mright) \circ \mleft(
\mright)
\eas

\section{Lie group bundles (LGBs)}

\subsection{Definition and examples}
%\pagebreak
\begin{definitions}{Lie group bundle, \cite[\S 1.1, Def.\ 1.1.19; p. 11]{mackenzieGeneralTheory}}{LieGroupBundle}
Let $G, \mathcal{G}, M$ be smooth manifolds. A fibre bundle
\begin{center}
	\begin{tikzcd}
		G \arrow{r} & \mathcal{G} \arrow{d}{\pi} \\
		& M
	\end{tikzcd}
\end{center}
is called a \textbf{Lie group bundle} if:
\begin{enumerate}
	\item $G$ and each fibre $\mathcal{G}_x \coloneqq \pi^{-1}\mleft( \{x\} \mright)$, $x\in M$, are Lie groups;
	\item there exists a bundle atlas $\mleft\{ \mleft( U_i, \phi_i \mright) \mright\}_{i \in I}$ such that the induced maps
	\bas
	\phi_{ix}
	&\coloneqq
	\mathrm{pr}_2 \circ \mleft. \phi_i\mright|_{\mathcal{G}_x}: \mathcal{G}_x \to G
	\eas
	are Lie group isomorphisms, where $I$ is an (index) set, $U_i$ are open sets covering $M$, $\phi_i: \mathcal{G}|_U \to U \times G$ subordinate trivializations, and $\mathrm{pr}_2$ the projection onto the second factor. This atlas will be called \textbf{Lie group bundle atlas} or \textbf{LGB atlas}.
\end{enumerate}

We also often say that \textbf{$\mathcal{G}$ is an LGB (over $M$)}, whose structural Lie group is either clear by context or not explicitly needed; and we may also denote LGBs by $G \to \mathcal{G} \stackrel{\pi}{\to} M$.
\end{definitions}

\begin{remarks}{Principal and Lie group bundles}{LiegroupbundlesNotPrincipalBundles}
Beware, a Lie group bundle is \textbf{not} the same as a principal bundle $P \to M$ with the same fibre type $G$. First of all, the fibres of $P$ are just diffeomorphic to a Lie group, a priori they carry no Lie group structure, while the fibres of $\mathcal{G}$ carry a Lie group structure.
\newline

Second, on $P$ we have a multiplication given as an action of $G$ on $P$
\bas
P \times G \to P,
\eas
preserving the fibres $P_x$ ($x\in M$) and simply transitive on them. Restricted on $P_x$ we have
\bas
P_x \times G \to P_x.
\eas
For $\mathcal{G}$ we have canonically a multiplication over $x$ given by
\bas
\mathcal{G}_x \times \mathcal{G}_x \to \mathcal{G}_x,
\eas
also clearly simply transitive. Observe, the second factor is not "constant", \textit{i.e.}\ we do not have $\mathcal{G}_x \times G \to \mathcal{G}_x$ in general. Hence, there is in general no well-defined product $\mathcal{G} \times \mathcal{G} \to \mathcal{G}$.
\newline

All of that is also resembled in the existence of sections. The existence of a section of $P$ has a 1:1 correspondence to trivializations of $P$, which is why $P$ in general only admits sections locally; see \textit{e.g.}\ \cite[\S 4.2, Thm.\ 4.2.19; page 219f.]{Hamilton}. $\mathcal{G}$ clearly admits always a global section, even if $\mathcal{G}$ is non-trivial; just take the section which assigns each base point the neutral element of its fibre.
\end{remarks}

If $M$ is a point we recover the notion of Lie groups, and, as usual, we have the notion of trivial LGBs:

\begin{examples}{Trivial examples}{TrivialLGBundle}
The \textbf{trivial LGB} is given as the trivial bundle $M \times G \to M$ with canonical multiplication $(x, g) \cdot (x, q) \coloneqq (x, gq)$, and we recover the notion of a Lie group in the case of $M = \{*\}$.
\end{examples}

We are of course also interested into LGB bundle morphisms:

\begin{definitions}{LGB morphism, \newline \cite[\S 1.2, special situation of Def.\ 1.2.1 \& 1.2.3, page 12]{mackenzieGeneralTheory}}{LGB morphism}
Let $\mathcal{G} \stackrel{\pi_{\mathcal{G}}}{\to} M$ and $\mathcal{H} \stackrel{\pi_{\mathcal{H}}}{\to} N$ be two LGBs over two smooth manifolds $M$ and $N$. An \textbf{LGB morphism} is a pair of smooth maps $F: \mathcal{H} \to \mathcal{G}$ and $f: N \to M$ such that
\ba\label{FibreRelationOverf}
\pi_{\mathcal{G}} \circ F &= f \circ \pi_{\mathcal{H}},\\
F(gq) &= F(g) ~ F(q)\label{LGBHomomorph}
\ea
for all $g, q \in \mathcal{H}$ with $\pi_{\mathcal{H}}(g) = \pi_{\mathcal{H}}(q)$. We then say that \textbf{$F$ is an LGB morphism over $f$}. If $N = M$ and $f = \mathrm{id}_M$, then we often omit mentioning $f$ explicitly and either just write that \textbf{$F$ is a (base-preserving) LGB morphism}.

We speak of an \textbf{LGB isomorphism (over $f$)} if $F$ is a diffeomorphism.
\end{definitions}

\begin{remark}
\leavevmode\newline
\indent $\bullet$ The right hand side of Eq.\ \eqref{LGBHomomorph} is well-defined because of Eq.\ \eqref{FibreRelationOverf}.

$\bullet$ It is clear that condition 2 in Def.\ \ref{def:LieGroupBundle} is equivalent to say that $\mathcal{G}$ is locally isomorphic to a trivial LGB; as one may have expected already.

$\bullet$ If $F$ is a diffeomorphism, then also $f$: By Eq.\ \eqref{FibreRelationOverf} surjectivity of $f$ is clear; for $y \in M$ just take any $g \in \mathcal{G}_y$, and since $F$ is a bijective, we have a $q \in \mathcal{H}_x$ for some $x\in N$ with $F(q)=g$. By Eq.\ \eqref{FibreRelationOverf} we have $y = \pi_{\mathcal{G}}(F(q)) \stackrel{\eqref{FibreRelationOverf}}{=} f(x)$, thence, surjectivity follows. For injectivity we know by Eq.\ \eqref{LGBHomomorph} and \eqref{FibreRelationOverf} that $F\mleft(e^{\mathcal{H}}_x\mright) = e^{\mathcal{G}}_{f(x)}$, where $e^{\mathcal{H}}_x$ and $e^{\mathcal{G}}_{f(x)}$ denote the unique neutral elements of $\mathcal{H}_x$ and $\mathcal{G}_{f(x)}$, respectively. Assume that there are $x, x^\prime \in N$ with $f(x) = f(x^\prime)$, then we can derive
\bas
F\mleft(e^{\mathcal{H}}_x\mright)
&=
e^{\mathcal{G}}_{f(x)}
=
e^{\mathcal{G}}_{f(x^\prime)}
=
F\mleft(e^{\mathcal{H}}_{x^\prime}\mright).
\eas
Then we have $e^{\mathcal{H}}_x = e^{\mathcal{H}}_{x^\prime}$ due to that $F$ is bijective, and hence $x = x^\prime$. Therefore $f$ is bijective. Finally, $F^{-1}$ is by assumption also a diffeomorphism, Eq.\ \eqref{LGBHomomorph} clearly carries over, and Eq.\ \eqref{FibreRelationOverf} is clearly w.r.t.\ $f^{-1}$, that is
\bas
\pi_{\mathcal{H}} \circ F^{-1} &= f^{-1} \circ \pi_{\mathcal{G}}.
\eas
Since $\pi_{\mathcal{H}} \circ F^{-1}$ is smooth and $\pi_{\mathcal{G}}$ is a smooth surjective submersion, it follows that $f^{-1}$ is smooth; this is a well-known fact for right-compositions with surjective submersions, see \textit{e.g.}\ \cite[\S 3.7.2, Lemma 3.7.5, page 153]{Hamilton}. We can conclude that $f$ is a diffeomorphism. Observe that we also concluded that $F^{-1}$ is an LGB isomorphism, too.
\end{remark}

For another important example recall that there is the notion of associated fibre bundles; following and stating the results of \cite[\S1, Construction 1.3.8, page 20]{mackenzieGeneralTheory} and \cite[\S 4.7, page 237ff.; see also Rem.\ 4.7.8, page 242f.]{Hamilton}: Let $P \stackrel{\pi_P}{\to} M$ be a principal bundle with structural Lie group $G$, a smooth manifold $F$ and a smooth left $G$-action $\Psi$ given by
\bas
G \times F &\to F,\\
(g, v) &\mapsto \Psi(g, v) \coloneqq g \cdot v.
\eas
Then we have a right $G$-action on $P \times F$ given by
\bas
(P \times F) \times G &\to P \times F,\\
(p,v,g) &\mapsto \mleft( p \cdot g, g^{-1} \cdot v \mright),
\eas
and one can show that the quotient under this action, $P\times_\Psi F \coloneqq ( P \times F) \Big/ G$, yields the structure of a fibre bundle
\begin{center}
	\begin{tikzcd}
		F \arrow{r} & P\times_\Psi F \arrow{d}{\pi_{P\times_\Psi F}} \\
		& M
	\end{tikzcd}
\end{center}
such that the projection $P \times F \to P \times_\Psi F$ is a smooth surjective submersion,
where the projection $\pi_{P\times_\Psi F}: P\times_\Psi F \to M$ is given by 
\bas
\pi_{P\times_\Psi F}\mleft( [p, v] \mright)
&\coloneqq
\pi_P(p)
\eas
for all $[p, v] \in P\times_\Psi F$, denoting equivalence classes of $(p, v)$ by square brackets. For $x \in M$, the fibre $\mleft(P\times_\Psi F\mright)_x$ is given by $\mleft( P_x \times F  \mright) \Big/ G = P_x \times_\Psi F$, and the fibre is diffeomorphic to $F$ by $F \ni v \mapsto [p, v] \in \mleft(P\times_\Psi F\mright)_x$ for a fixed $p \in P_x$. We will frequently use this diffeomorphism in the following without further notice.

A very important example are of course associated vector bundles, related to $F$ being a vetor space. We need a similar concept for Lie groups.

\begin{definitions}{Lie group representation on Lie groups, \newline \cite[special situation of the comment after Ex.\ 1.7.14, page 47]{mackenzieGeneralTheory}}{LieGroupActingOnLieGroup}
Let $G, H$ be Lie groups. Then a \textbf{Lie group representation of $G$ on $H$} is a smooth left action $\psi$ of $G$ on $H$
\bas
G \times H
&\to H,\\
(g,h)
&\mapsto
\psi_g(h)
\coloneqq
\psi(g, h)
\eas
such that
\ba
\psi_g(hq)
&=
\psi_g(h)
~ \psi_g(q)
\ea
for all $g \in G$ and $h,q \in H$.
\end{definitions}

\begin{remarks}{Note about labeling}{WhyRepresentation}
Observe that we have by the definition of group actions
\bas
\psi_{gg^\prime}
&=
\psi_g \circ \psi_{g^\prime}
\eas
for all $g, g^\prime \in G$, viewing $\psi_g$ as a map $H \to H$. Therefore we can view the action $\psi$ as a homomorphism
\bas
G &\to \mathrm{Aut}(H),
\eas
where $\mathrm{Aut}(H)$ is the set of Lie group automorphisms. The similarity to Lie group representations on vector spaces is obvious, thence the name.
\newline

This definition is of course also motivated by various references pointing out that Lie group representations define Lie group actions with extra properties; see for example \cite[\S 3, Ex.\ 3.4.2, page 143f.]{Hamilton}. In \cite[comments after Ex.\ 1.7.14, page 47]{mackenzieGeneralTheory} this definition is also called \textit{action by Lie group isomorphisms}.
\end{remarks}

With this we can discuss and define associated Lie group bundles.

\begin{theorems}{Associated Lie group bundle as quotient, \newline\cite[motivated by vector spaces as in \S 4, Thm.\ 4.7.2, page 239f.]{Hamilton}}{AssociatedGroupBundlesHaveGroupStructure}
Let $G, H$ be Lie groups, $P \stackrel{\pi_P}{\to} M$ a principal $G$-bundle over a smooth manifold $M$, and $\psi$ a $G$-representation on $H$. Then $\mathcal{H} \coloneqq P \times_\psi H$ is an LGB 
\begin{center}
	\begin{tikzcd}
		H \arrow{r} & \mathcal{H} \arrow{d}{\pi} \\
		& M
	\end{tikzcd}
\end{center}
with projection $\pi$ given by
\ba
\mathcal{H} &\to M,\nonumber\\
[p, h] &\mapsto \pi_P(p),
\ea
and fibres
\ba
\mathcal{H}_x
&=
P_x \times_\psi H
\ea
for all $x \in M$, which are isomorphic to $H$ as Lie groups. The Lie group structure on each fibre $\mathcal{H}_x$ is defined by
\ba\label{LiegroupStructureOnFibresofAssociated}
[p, h] \cdot \mleft[p, q\mright]
&\coloneqq
\mleft[ p, hq \mright]
\ea
for all $h, q \in H$ and $p_x \in P_x$, where $\pi_P(p) = x$.
\end{theorems}

\begin{remarks}{Neutral and inverse elements}{NeutralAndInverseInAssocLGB}
The neutral element for $\mathcal{H}_x$ ($x \in M$) is given by
\bas
e_x
&=
[p, e],
\eas
where $p \in P_x$ is arbitrary and $e$ is the neutral element of $H$. This is clearly independent of the choice of $p$ due to
\bas
\mleft[ p, e \mright]
&=
\mleft[ p \cdot g, \psi_{g^{-1}}(e) \mright]
=
\mleft[ p \cdot g, e \mright]
\eas
for all $g \in G$. Thence, the fact that $e_x$ is the neutral element follows immediately.

The inverse of $[p, h] \in \mathcal{H}_x$ is clearly given by
\bas
\mleft( [p, h] \mright)^{-1}
&=
\mleft[ p, h^{-1} \mright].
\eas
\end{remarks}

\begin{proof}
\leavevmode\newline
\indent $\bullet$ That $\pi$ is the well-defined projection and that the fibres are precisely $P_x \times_\psi H$ for all $x \in M$ is well-known, see our discussion before Def.\ \ref{def:LieGroupActingOnLieGroup} and the references therein; it is also very straightforward to check. We also discussed that $\mathcal{H}$ is a fibre bundle with structural fibre $H$. Hence, if one knows that the proposed group structure in Def.\ \eqref{LiegroupStructureOnFibresofAssociated} is well-defined, then the smoothness of the group structure is implied by the smoothness structures of $H$ and $\mathcal{H}$. Thence, let us check whether Def.\ \eqref{LiegroupStructureOnFibresofAssociated} is well-defined. Let $x \in M$, $p \in P_x$ and $p^\prime \coloneqq p \cdot g^\prime$ be another element of $P_x$, where $g^\prime \in G$. Also let $[p_1,h_1], [p_2, h_2] \in P_x \times_\psi H$; then we have unique elements $q_i, q_i^\prime$ of $G$ such that ($i \in \{1,2\}$)
\bas
p_i &= p \cdot q_i,&
p_i &= p^\prime \cdot q_i^\prime,
\eas
especially, it follows $q_i = g^\prime q_i^\prime$.
On the one hand, if we use $p$ as fixed element of $P_x$ to calculate the multiplication, we get
\ba\label{MultiPlicationInAssocGroup}
[p_1,h_1] \cdot [p_2,h_2]
&=
\mleft[ p, \psi_{q_1}(h_1) \mright]
\cdot \mleft[ p, \psi_{q_2}(h_2) \mright]
=
\mleft[ p, \psi_{q_1}(h_1) ~ \psi_{q_2}(h_2) \mright],
\ea
on the other hand, using Def.\ \ref{def:LieGroupActingOnLieGroup} and $p^\prime = p \cdot g^\prime$ instead of $p$,
\bas
[p_1,h_1] \cdot [p_2,h_2]
&=
\mleft[ p \cdot g^\prime, \psi_{q_1^\prime}(h_1) ~ \psi_{q_2^\prime}(h_2) \mright]
\\
&=
\Bigl[ p, \underbrace{\psi_{g^\prime} \mleft( \psi_{q_1^\prime}(h_1) ~ \psi_{q_2^\prime}(h_2) \mright)}_{= \psi_{g^\prime} \mleft( \psi_{q_1^\prime}(h_1) \mright) ~ \psi_{g^\prime} \mleft( \psi_{q_2^\prime}(h_2) \mright)} \Bigr]
\\
&=
\mleft[ p, \psi_{g^\prime q_1^\prime}(h_1) ~ \psi_{g^\prime q_2^\prime}(h_2) \mright]
\\
&=
\mleft[ p, \psi_{q_1}(h_1) ~ \psi_{q_2}(h_2) \mright],
\eas
which implies that Def.\ \eqref{LiegroupStructureOnFibresofAssociated} is well-defined, and thus defines a Lie group structure on each fibre of $\mathcal{H}$.

$\bullet$ That the fibres $\mathcal{H}_x$ are isomorphic to $H$ as Lie groups for all $x \in M$ also quickly follows. Recall by our discussion before Def.\ \ref{def:LieGroupActingOnLieGroup} that the fibres are diffeormorphic to $H$ by $H \ni h \mapsto [p, h] \in \mathcal{H}_x$ for a fixed $p \in P_x$. By Def.\ \eqref{LiegroupStructureOnFibresofAssociated} it is clear that this map is a Lie group homomorphism and hence a Lie group isomorphism.

$\bullet$ Let us now construct an LGB atlas for $\mathcal{H}$ by using a principal bundle atlas for $P$. That is, for some $U \subset M$ open and a trivialization $\varphi_U: P|_U \to U \times G$ we write
\bas
\varphi_U(p)
&=
\bigl( \pi_P(p), \beta_U(p) \bigr)
\eas
for all $p \in P$, where $\beta_U: P|_U \to G$ is an equivariant map, \textit{i.e.}\ $\beta_U(p \cdot g) = \beta_U(p) ~ g$ for all $g \in G$. Then define $\phi_U$ as a map by 
\bas
\mathcal{H}|_U
&\to
U \times H,\\
[p, h]
&\mapsto
\mleft(
	\pi_P(p), \psi_{\beta_U(p)} (h)
\mright).
\eas
$\phi_U$ is well-defined: Let $\mleft[p^\prime, h^\prime\mright] \in \mathcal{H}|_U$ with $\mleft[p^\prime, h^\prime\mright] = \mleft[p, h\mright]$. Then there is a $g \in G$ such that
\bas
\mleft(p^\prime, h^\prime\mright)
&=
\mleft( p \cdot g, \psi_{g^{-1}}(h) \mright),
\eas
hence, using the equivariance of $\beta_U$ and Def.\ \ref{def:LieGroupActingOnLieGroup},
\bas
\phi_U\mleft( \mleft[p^\prime, h^\prime\mright] \mright)
&=
\Bigl(
	\underbrace{\pi_P\mleft(p \cdot g\mright)}_{= \pi_P(p)}, \underbrace{\mleft(\psi_{\beta_U\mleft(p \cdot g\mright)} \circ \psi_{g^{-1}} \mright)}_{= \psi_{\beta_U(p)} \circ \psi_g \circ \psi_{g^{-1}} } (h)
\Bigr)
=
\mleft(
	\pi_P(p), \psi_{\beta_U(p)} (h)
\mright)
=
\phi_U\bigl( [p, h] \bigr),
\eas
which proves that $\phi_U$ is well-defined. Denote the projection onto equivalence classes $P \times H \to \mathcal{H}$ by $\varpi$, then observe
\bas
\phi_U \circ \varpi
&=
L,
\eas
where $L_U: P|_U \times H \to U \times H$ is given by $L_U(p,h) \coloneqq \mleft( \pi_P(p), \psi_{\beta_U(p)} (h) \mright)$ for all $(p, h) \in P|_U \times H$. $L_U$ is clearly smooth and recall that $\varpi$ is a smooth surjective submersion, therefore $\phi_U$ is smooth; this is a well-known fact for right-compositions with surjective submersions, see \textit{e.g.}\ \cite[\S 3.7.2, Lemma 3.7.5, page 153]{Hamilton}. We define a candidate of the inverse $\phi_U^{-1}: U \times H \to \mathcal{H}|_U$ by
\bas
\phi_U^{-1}(x, h)
&=
\mleft[ \varphi_U^{-1}\mleft(x, e\mright), h \mright]
\eas
for all $(x, h) \in U \times H$, where $e$ is the neutral element of $G$.
By the definition of $\varphi_U$ we immediately get
\bas
\mleft( \varphi_U \circ \varphi^{-1}_U \mright)(x, e)
&=
\Bigl(
	\pi_P\mleft( \varphi_U^{-1}(x, e) \mright), \beta_U\mleft( \varphi_U^{-1}(x, e) \mright)
\Bigr)
=
(x, e),
\eas
for all $x \in U$, and, also using again the equivariance of $\beta_U$,
\bas
\varphi^{-1}_U\mleft(\pi_P(p), e\mright)
&=
\varphi^{-1}_U\Bigl(\pi_P\mleft(p \cdot \beta_U^{-1}(p) \mright), \beta_U(p)~\beta_U^{-1}(p)\Bigr)
\\
&=
\varphi^{-1}_U\Bigl(\pi_P\mleft(p \cdot \beta_U^{-1}(p) \mright), \beta_U\mleft(p \cdot \beta_U^{-1}(p)\mright)\Bigr)
\\
&=
\mleft(\varphi^{-1}_U \circ \varphi_U\mright)\mleft( p \cdot \beta_U^{-1}(p) \mright)
\\
&=
p \cdot \beta_U^{-1}(p)
\eas
for all $p \in P|_U$.
Then
\bas
\mleft(\phi_U \circ \phi_U^{-1}\mright)(x, h)
&=
\mleft(
	\pi_P\mleft( \varphi^{-1}_U(x, e) \mright), \psi_{\beta_U\mleft( \varphi^{-1}_U(x, e) \mright)} (h)
\mright)
=
\bigl(
	x, \psi_e(h)
\bigr)
=
(x, h),
\eas
for all $(x, h) \in U \times H$, and
\bas
\mleft(\phi_U^{-1} \circ \phi_U\mright)([p, h])
&=
\bigl[
	\underbrace{\varphi_U^{-1}\mleft( \pi_P(p), e \mright)}_{= p \cdot \beta_U^{-1}(p) },
	\psi_{\beta_U(p)}(h)
\bigr]
\\
&=
\mleft[
	p, h
\mright]
\eas
for all $[p, h] \in \mathcal{H}|_U$. Thus, $\phi_U$ is bijective; additionally observe
\bas
\phi_U^{-1}(x, h)
&=
\varpi\mleft( \varphi_U^{-1}(x, e), h \mright)
\eas
such that $\phi_U^{-1}$ is clearly smooth as the composition of smooth maps, and we therefore conclude that $\phi_U$ is a diffeomorphism. Finally, derive with Def.\ \ref{def:LieGroupActingOnLieGroup} and Eq.\ \eqref{MultiPlicationInAssocGroup} that
\bas
\mleft(\mathrm{pr}_2 \circ \phi_U\mright)\bigl( [p_1, h_1] \cdot [p_2, h_2] \bigr)
&=
\mleft(\mathrm{pr}_2 \circ \phi_U\mright)\bigl( \mleft[ p, \psi_{q_1}(h_1) \cdot \psi_{q_2}(h_2) \mright] \bigr)
\\
&=
\psi_{\beta_U(p)} \bigl( \psi_{q_1}(h_1) \cdot \psi_{q_2}(h_2) \bigr)
\\
&=
\underbrace{\psi_{\beta_U(p)} \bigl( \psi_{q_1}(h_1) \bigr)}_{= \psi_{\beta_U(p) \cdot q_1} (h)}
	\cdot ~ \psi_{\beta_U(p)} \bigl( \psi_{q_2}(h_2) \bigr)
\\
&=
\psi_{\beta_U(p_1)} (h) \cdot \psi_{\beta_U(p_2)} (h)
\\
&=
\mleft( \mathrm{pr}_2 \circ \phi_U \mright)\bigl( [p_1, h_1] \bigr)
	\cdot \mleft( \mathrm{pr}_2 \circ \phi_U \mright)\bigl( [p_1, h_1] \bigr)
\eas
for all $[p_1, h_1], [p_2, h_2] \in \mathcal{H}_x$, where we used again the equivariance of $\beta_U$ and the same notation as introduced for Eq.\ \eqref{MultiPlicationInAssocGroup}, and $\mathrm{pr}_2$ denotes the projection onto the second factor. Thence, $\mathrm{pr}_2 \circ \phi_U$ induces Lie group isomorphisms $\mathcal{H}_x \to H$ for all $x \in U$; by Def.\ \ref{def:LieGroupBundle} we can finally conclude that $\mathcal{H}$ is an LGB.
\end{proof}

Hence, we define:

\begin{definitions}{Associated Lie group bundle, \newline labeling similar to \cite[\S 4.7, Def.\ 4.7.3, page 240]{Hamilton}}{AssociatedLGB}
Let $G, H$ be Lie groups, $P \stackrel{\pi_P}{\to} M$ a principal $G$-bundle over a smooth manifold $M$, and $\psi$ a $G$-representation on $H$. Then we call the LGB
\bas
\mathcal{H}
&\coloneqq
P \times_\psi H
=
\mleft( P \times H \mright) \Big/ G
\eas
the \textbf{Lie group bundle (LGB) associated} to the principal bundle $P$ and the representation $\psi$ on $H$:
\begin{center}
	\begin{tikzcd}
	H \arrow{r}& P \times_\psi H \arrow{d}{\pi_{\mathcal{H}}}\\
	& M
	\end{tikzcd}
\end{center}
\end{definitions}

The special situation of $H = G$ is already an important example:

\begin{examples}{Inner group bundle, \newline \cite[\S1, paragraph after Def.\ 1.1.19, page 11; comment after Construction 1.3.8, page 20]{mackenzieGeneralTheory}}{InnerLGBs}
The \textbf{inner group bundle} or \textbf{inner LGB} of a principal bundle $P \to M$, denoted by $c_G(P)$, is defined by
\ba
c_G(P)
&\coloneqq
P \times_{c_G} G,
\ea
where $c_G: G \times G \to G$ is the left action of $G$ on itself given by the very well-known \textbf{conjugation}
\ba
c_G(g,h)
&\coloneqq
c_g (h)
=
\mleft(L_g \circ R_{g^{-1}}\mright)(h)
=
ghg^{-1}
\ea
for all $g, h \in G$, where we also denote left- and right-multiplications (with $g$) by $L_g$ and $R_g$, respectively; see \textit{e.g.}\ \cite[beginning of \S 1.5.2, page 40f.]{Hamilton} for its common properties. It is well-known that $c_G$ satsfies the properties of a Lie group representation of $G$ on itself in the sense of Def.\ \ref{def:LieGroupActingOnLieGroup}.

$c_G(P)$ is an LGB by Thm.\ \ref{thm:AssociatedGroupBundlesHaveGroupStructure}.
\end{examples}

\subsection{Lie algebra bundles (LABs)}

Lie algebras are the infinitesimal version of Lie groups, hence, we expect something similar for LGBs, the Lie algebra bundles:

\begin{definitions}{Lie algebra bundle (LAB), \cite[\S 3.3, Definition 3.3.8, page 104]{mackenzieGeneralTheory}}{LAB}
Let $\mathfrak{g}$ be a vector space, and $\mathcal{g}, M$ be smooth manifolds. A vector bundle
\begin{center}
	\begin{tikzcd}
	\mathfrak{g} \arrow{r} & \mathcal{g} \arrow{d}{\pi} \\
	& M
	\end{tikzcd}
\end{center}
is called a \textbf{Lie algebra bundle} if:
\begin{enumerate}
	\item $\mathfrak{g}$ and each fibre $\mathcal{g}_x$, $x \in M$, are Lie algebras;
	\item there exists a bundle atlas $\mleft\{ \mleft( U_i, \phi_i \mright) \mright\}_{i \in I}$ such that the induced maps
	\bas
	\phi_{ix}
	&\coloneqq
	\mathrm{pr}_2 \circ \mleft. \phi_i\mright|_{\mathcal{g}_x}: \mathcal{g}_x \to \mathfrak{g}
	\eas
	are Lie algebra isomorphisms, where $I$ is an (index) set, $U_i$ are open sets covering $M$, $\phi_i: \mathcal{g}|_U \to U \times G$ subordinate trivializations, and $\mathrm{pr}_2$ the projection onto the second factor. This atlas will be called \textbf{Lie algebra bundle atlas} or \textbf{LAB atlas}.
\end{enumerate}
We often say that \textbf{$\mathcal{g}$ is an LAB (over $M$)}, whose structural Lie group is either clear by context or not explicitly needed; and we may also denote LABs by $\mathfrak{g} \to \mathcal{g} \stackrel{\pi}{\to} M$.
The corresponding field of Lie algebra brackets is denoted by $\mleft[ \cdot, \cdot \mright]_{\mathfrak{g}}: \Gamma(\mathcal{g}) \times \Gamma(\mathcal{g}) \to \Gamma(\mathcal{g})$, \textit{i.e.}\ $\mleft[ \cdot, \cdot \mright]_{\mathfrak{g}} \in \Gamma\mleft(\bigwedge^2 \mathcal{g}^* \otimes \mathcal{g} \mright)$ which restricts to a Lie algebra bracket on each fibre.
\end{definitions}

\subsection{From LGBs to LABs}

\section{LGB actions}

\subsection{Definition}

As for Lie groups, we are interested into their actions. The idea is the following, similar to \cite[\S 1.6, discussion around Def.\ 1.6.1, page 34]{mackenzieGeneralTheory}: We have an LGB $\mathcal{G} \to M$ over a smooth manifold $M$, and we want to construct an action of $\mathcal{G}$ on another smooth manifold $N$. Each fibre of $\mathcal{G}$ is a Lie group, and we have a notion of Lie groups actions on manifold $N$. Therefore one could define an LGB action as a collection of Lie group actions, that is, only sections of $\mathcal{G}$ act on $N$; however, one then expects that the general outcome of a product of $\Gamma(\mathcal{G})$ on $N$ would be smooth maps from $M$ to $N$. In order to recover a typical structure of action one could instead introduce a "multiplication rule", \textit{i.e.}~each point $p \in N$ can only be multiplied with elements of a specific fibre of $\mathcal{G}$. This "multiplication rule" will be described by a smooth map $f: N \to M$ in the sense of that the fibre over $f(p)$ will act on $p$.

For this recall that there is the notion of pullbacks of fibre bundles, see \textit{e.g.}\ \cite[\S 4.1.4, page 203ff.; especially Thm.\ 4.1.17, page 204f.]{Hamilton}. That is, if we additionally have a smooth manifold $N$ and a smooth map $f: M \to N$, then we have the pullback $f^*\mathcal{G}$ of $\mathcal{G}$ as a fibre bundle defined as usual by
\ba
f^*\mathcal{G}
&\coloneqq
\left\{
	(x, g) \in N \times \mathcal{G} ~\middle|~
	f(x) = \pi(g)
\right\}.
\ea
The structural fibre is the same Lie group as for $\mathcal{G}$.
That is, the following diagram commutes
\begin{center}
	\begin{tikzcd}
		f^*\mathcal{G} \arrow{d}{\pi_1} \arrow{r}{\pi_2}& \mathcal{G} \arrow{d}{\pi} \\
		N \arrow{r}{f}& M
	\end{tikzcd}
\end{center}
where $\pi_1$ and $\pi_2$ are the projections onto the first and second factor, respectively, of $N \times \mathcal{G}$. Actually, $f^*\mathcal{G}$ carries a natural structure as an LGB.

\begin{corollaries}{Pullbacks of LGBs are LGBs, \newline \cite[\S 2.3, simplified situation of the discussion around Prop.\ 2.3.1, page 63ff.]{mackenzieGeneralTheory}}{PullbackLGB}
Let $M, N$ be smooth manifolds, $\mathcal{G} \stackrel{\pi}{\to} M$ an LGB over $M$ and $f: N \to M$ a smooth map. Then $f^*\mathcal{G}$ has a unique (up to isomorphisms) LGB structure such that the projection $\pi_2: f^*\mathcal{G} \to \mathcal{G}$ onto the second factor is an LGB morphism over $f$ with $\pi_2|_x: (f^*\mathcal{G})_x \to \mathcal{G}_{f(x)}$ being a Lie group isomorphism for all $x \in N$.
\end{corollaries}

\begin{remark}
\leavevmode\newline
The mentioned reference, \cite[\S 2.3, discussion around Prop.\ 2.3.1, page 63ff.]{mackenzieGeneralTheory}, is rather general, formulated for Lie groupoids. If the reader is only interested into LGBs, then see \textit{e.g.}\ \cite[\S 3, Thm.\ 3.1]{PullbackLGBLAB}.
\end{remark}

\begin{proof}
\leavevmode\newline
By construction, the structural fibre of $f^*\mathcal{G}$ is the same Lie group $G$ as for $\mathcal{G}$, and for all $x \in N$ we have $\mleft(f^*\mathcal{G}\mright)_x \cong \mathcal{G}_{f(x)}$, thence, the fibres are Lie groups and the fibrewise group multiplication has the form
\ba\label{MultiplicationForPullbackLGBs}
(x, g) \cdot (x, q) &= (x, gq)
\ea
for all $x \in N$ and $g,q \in \mleft(f^*\mathcal{G}\mright)_x$.
We are left to show the existence of an LGB atlas. For this fix an LGB atlas $\{(U_i, \phi_i \}_{i \in I}$ of $\mathcal{G}$, where $I$ is an (index) set, $(U_i)_{i \in I}$ an open covering of $M$, and $\phi_i: \mathcal{G}|_{U_i} \to U_i \times G$ are LGB isomorphisms. Then $f^{-1}(U_i)$ gives rise to an open covering of $N$, and we get 
\bas
f^*\phi_i: \mleft.f^*\mathcal{G}\mright|_{f^{-1}(U_i)} &\to f^{-1}(U_i) \times G,\\
(x, g) &\mapsto \mleft( x, \phi_{i, f(x)}(g) \mright),
\eas
where $\phi_{i, f(x)}: \mathcal{G}_{f(x)} \to G$ are the Lie group isomorphisms as defined in Def.\ \ref{def:LieGroupBundle}. It is immediate by construction that this gives an LGB atlas.

That this is the unique (up to isomorphisms) LGB structure such that $\pi_2: f^*\mathcal{G} \to \mathcal{G}$ is an LGB morphism over $f$ inducing a Lie group isomorphism on each fibre simply follows by construction; observe for all $x \in N$ that $\pi_2|_x$ is clearly bijective. Furthermore, LGB morphisms need to be homomorphisms, which means here
\bas
\pi_2\bigl(
	(x, g) \cdot (x, q)
\bigr)
&\stackrel{!}{=}
\pi_2\mleft( (x, g) \mright) \cdot \pi_2\mleft( (x, q) \mright)
=
gq
=
\pi_2\bigl( (x, gq) \bigr)
\eas
for all $x \in N$ and $g,q \in \mleft(f^*\mathcal{G}\mright)_x$. By using the bijectivity of $\pi_2|_x$, the group structure leading to this is uniquely the one provided in Eq.\ \eqref{MultiplicationForPullbackLGBs}. Especially, $\pi_2$ is a homomorphism with the provided structure. Assume we have another LGB chart $\psi_i$ on (a subset of) $f^{-1}(U_i)$, then
\bas
\phi_i \circ \pi_2 \circ \psi_i^{-1}
&=
\underbrace{\phi_i \circ \pi_2 \circ \mleft( f^*\phi_i \mright)^{-1}}_{= (f, \mathds{1}_G)} \circ f^*\phi_i \circ \psi_i^{-1}
=
(f, \mathds{1}_G) \circ
f^*\phi_i \circ \psi_i^{-1},
\eas
If evaluating this at $x \in f^{-1}(U_i)$, then all parts are bijective, and thus the condition about $\pi_2|_x$ being a Lie group isomorphism enforces that $\psi_i$ is an LGB atlas compatible with $f^*\phi_i$. This concludes the proof.
\end{proof}

Let us now define $\mathcal{G}$-actions.

\begin{definitions}{Lie group bundle actions, \newline \cite[\S 1.6, special case of Def.\ 1.6.1, page 34]{mackenzieGeneralTheory}}{LiegroupACtion}
Let $M, N$ be smooth manifolds, $\mathcal{G} \stackrel{\pi}{\to} M$ an LGB over $M$ and $f: N \to M$ a smooth map. Then a \textbf{right-action of $\mathcal{G}$ on $N$} is a smooth map 
\bas\label{InvarianceOffUnderGAction}
f^*\mathcal{G} &\to N,\\
(p, g) &\mapsto p \cdot g,
\eas
satisfying the following properties:
\ba
f(p \cdot g) &= \pi(g),\\
(p \cdot g) \cdot h &= p \cdot (gh),\\
p \cdot e_{f(p)} &= p
\ea
for all $p \in N$ and $g, h \in \mathcal{G}_{f(p)}$, where $e_{f(p)}$ is the neutral element of $\mathcal{G}_{f(p)}$.
\newline

We similarly define left-actions, and we may sometimes write (left or right) \textbf{$\mathcal{G}$-action on $N$}. Furthermore, in order to increase readability as long as the dependency on $f$ is not important, we introduce the notation
\ba
N * \mathcal{G}
&\coloneqq
f^*\mathcal{G},
\ea
such that the action's notation has the typical shape $N * \mathcal{G} \to N$.
\end{definitions}

\begin{remarks}{Relation to the structure of the canonical pullback Lie group bundle over $N$}{ActionAndPullbackLGBs}
Observe that by the definition of $f^*\mathcal{G}$ we can also write
\bas
f(p \cdot g) &= f(p),
\eas
so, the $\mathcal{G}$-action is defined in such a way that $f$ is invariant under it. Moreover, the fibre-wise group structure on $\mathcal{G}$ naturally defines a $\mathcal{G}$-action on $\mathcal{G}$; in this situation $f$ would be $\pi$ itself. This is mainly a technical condition. On one hand, having $M = \{*\}$ already recovers the notion of a Lie group action and condition \eqref{InvarianceOffUnderGAction} is then trivial, and on the other hand the mentioned reference, \cite[\S 1.6, Def.\ 1.6.1, page 34]{mackenzieGeneralTheory}, actually generalizes this condition making use of the structure of groupoids.
\newline

Furthermore, the other conditions are the typical conditions for actions, especially such that we get a $\mathcal{G}$-action on $f^*\mathcal{G}$ by
\ba
(p, g) \cdot q
&\coloneqq
\mleft( p \cdot q, q^{-1} g \mright)
\ea
for all $p \in N$ and $g, q \in \mathcal{G}_{f(p)}$.\footnote{In alignment to Def.\ \ref{def:LiegroupACtion}, this action is a map $(f \circ \pi_1)^*\mathcal{G} \to f^*\mathcal{G}$, where $\pi_1$ is the projection onto the first factor in $f^*\mathcal{G}$.}
As usual, this gives rise to an equivalence relation, whose set of equivalence classes $f^*\mathcal{G} \Big/ G$ is isomorphic to $N$ (as a set) by $[p, g] \mapsto p \cdot g$, where we denote equivalence classes of $(p, g) \in f^*\mathcal{G}$ by $[p, g]$. All of this is straight-forward to check. Finally, observe the similarity to associated fibre bundles.
\end{remarks}

\begin{remarks}{Left- and right-actions}{LeftRightAction}
In the following we usually define everything with respect to right-actions; however, one can of course define the same for left actions in a similar manner. If we ever speak of a left action, then we assume precisely this. Some subtle changes like a sign change will be pointed out though. 
\end{remarks}

If $M$ is a point or $f$ a constant map, then we recover the typical notion of a Lie group action acting on $N$.

One can probably see that it is straightforward to extend a lot of the typical notions of Lie group actions to LGB actions; hence, we mainly focus on the definitions and properties which we need in this paper. 

\begin{definitions}{Left and right translations, \newline \cite[\S 3.2, notation similar to Def.\ 3.2.3, page 131]{Hamilton} \newline \cite[\S 1.4, special situation of Def.\ 1.4.1 and its discussion, page 22]{mackenzieGeneralTheory}}{LRTranslations}
Let $M, N$ be smooth manifolds, $\mathcal{G} \stackrel{\pi}{\to} M$ an LGB over $M$ and $f: N \to M$ a smooth map. Furthermore assume that we have a right action $N * \mathcal{G} \to N$. We define the \textbf{right translation} over $x \in M$ with $g \in \mathcal{G}$ as a map $R_g$ defined by
\bas
f^{-1}(\{x\}) &\to f^{-1}(\{x\}),\\
p &\mapsto p \cdot g,
\eas
and we define the \textbf{orbit map} through $p \in f^{-1}(\{x\})$ as a map $\Phi_p$ given by
\bas
\mathcal{G}_x &\to N,\\
g&\mapsto p \cdot g.
\eas
For $\sigma \in \Gamma(\mathcal{G})$ we define the \textbf{right translation} on $N$ as a map $R_\sigma$ by
\bas
N &\to N,\\
p &\mapsto p \cdot \sigma_p.
\eas
\end{definitions}

\begin{remarks}{Left action and translation}{LeftTranslation}
Similarly we define left translations for left actions, which we similarly denote by $L_g$ and $L_\sigma$.
\end{remarks}

\begin{remark}
\leavevmode\newline
Similar to the arguments in \cite[\S 3.2, discussion after Def.\ 3.2.3, page 131]{Hamilton}, $\Phi_p$ is given by the composition of smooth maps
\bas
\mathcal{G}_x &\to N * \mathcal{G} \to N,\\
g &\mapsto (p, g) \mapsto p \cdot g.
\eas
The second arrow/map is smooth due to the fact that we have a smooth action; the first one is smooth because $N * \mathcal{G}$ is the pullback LGB $f^*\mathcal{G}$ and the first arrow is precisely the embedding of $\mathcal{G}_x$ into $f^*\mathcal{G}$ as a fibre over $p$; recall \textit{e.g.}\ Cor.\ \ref{cor:PullbackLGB}.

For the right-translation $R_\sigma$ we have a similar argument, namely, $R_\sigma$ is a composition of smooth maps
\bas
N &\to N*\mathcal{G} \to N,\\
p &\mapsto (p, \sigma_p) \mapsto p \cdot \sigma_p.
\eas
The first map describes now a section of $N*\mathcal{G} = f^*\mathcal{G}$, and thus an embedding. Thus, smoothness follows again.

However, for $R_g$ smoothness can only be discussed if $f^{-1}(\{x\})$ is a smooth manifold. That is for example the case if $x$ is a regular value of $f$; recall the regular value theorem as cited in \cite[\S A.1, Thm.\ A.1.32, page 611]{Hamilton}. This would be the case if \textit{e.g.}\ $f$ is a submersion. If $x$ is a regular value, then $f^{-1}(\{x\})$ is an embedded submanifold of $N$, and $R_g$ is a similar composition of smooth maps as for $R_\sigma$ but restricted to $f^{-1}(\{x\})$
\bas
f^{-1}(\{x\}) &\to \mleft.N*\mathcal{G}\mright|_{f^{-1}(\{x\})} \to f^{-1}(\{x\}),\\
p &\mapsto (p, g) \mapsto p \cdot g.
\eas
Since $f^{-1}(\{x\})$ is an embedded submanifold, $\mleft.N*\mathcal{G}\mright|_{f^{-1}(\{x\})}$ is also a fibre bundle, see for example \cite[\S 4.1, Lemma 4.1.16, page 204]{Hamilton}, and trivially an embedded submanifold of $N*\mathcal{G}$. Altogether, the same arguments as for $R_\sigma$ apply.
\end{remark}

Motivated by the previous remark, it might be hence useful to require that $f$ is a submersion, or that $N$ is actually some bundle over $M$ and $f$ its projection. In fact, this will be later the case.

In order to generalize principal bundles we need the following terms.

\begin{definitions}{}{}

\end{definitions}

\subsection{Examples of LGB actions}

We have the following examples, the second of which will be important in this paper.

\begin{examples}{Trivial action, \cite[\S 1.6, special situation of Ex.\ 1.6.3, page 35]{mackenzieGeneralTheory}}{TrivialAction}
The projection $\pi_1$ onto the first factor of $f^*\mathcal{G} \stackrel{\pi_1}{\to} N$ satisfies the properties of a right $\mathcal{G}$-action on $N$, that is, the action is given by
\bas
N * \mathcal{G} &\to N,\\
(p, g) &\mapsto p \cdot g \coloneqq p.
\eas
That this action satisfies the properties of an action for all $f$ is trivial, hence we call it the \textbf{trivial action}.
\end{examples}

\begin{examples}{Inner group bundle acting on associated fibre bundles, \newline\cite[\S 1.6, simplified version of Ex.\ 1.6.4, page 35]{mackenzieGeneralTheory}}{AssocLGACtingOnAssocVec}
Let $P \stackrel{\pi_P}{\to} M$ be a principal bundle with structural Lie group $G$ over a smooth manifold $M$, and recall Ex.\ \ref{ex:InnerLGBs}. Furthermore, let $F$ be another smooth manifold, equipped with a smooth left $G$-action $\Psi: G \times F \to F$. In total we have two associated bundles over $M$: 
\begin{center}
	\begin{tikzcd}
		G \arrow{r} & c_G(P) \arrow{d}{\pi_{c_G(P)}} && F \arrow{r} & \mathcal{F} \coloneqq P \times_\Psi F \arrow{d}{\pi_{\mathcal{F}}} \\
		& M & && M
	\end{tikzcd}
\end{center}
the inner group bundle of $P$ and an associated $F$-bundle, respectively.

Then we have a right $c_G(P)$-action on $\mathcal{F}$ given by
\bas
\mathcal{F} * c_G(P)
\coloneqq
\pi_{\mathcal{F}}^*c_G(P) &\to \mathcal{F},\\
\bigl( \mleft[ p, v \mright], \mleft[ p, g \mright] \bigr)
&\mapsto
\mleft[ p, \Psi \mleft(g, v\mright) \mright]
=
\mleft[ p \cdot g, v \mright]
\eas
for all $p \in P_x$ ($x \in M$), $g \in G$ and $v \in F$.
\end{examples}

\begin{proof}
\leavevmode\newline
\indent $\bullet$ We first check again that the action is well-defined, that is, we are going to prove that the action is independent of the choice of fixed point in $P_x$.
Thence, let $x \in M$, $p \in P_x$ and $p^\prime \coloneqq p \cdot g^\prime$ be another element of $P_x$, where $g^\prime \in G$. Also let $[p_1,v] \in \mathcal{F}_x$ and $[p_2, g] \in c_G(P)_x$; then we have unique elements $q_i, q_i^\prime$ of $G$ such that ($i \in \{1,2\}$)
\bas
p_i &= p \cdot q_i,&
p_i &= p^\prime \cdot q_i^\prime,
\eas
especially, it follows $q_i = g^\prime q_i^\prime$.

On one hand, if we use $p$ as fixed element of $P_x$ to calculate the multiplication, we get 
\bas
[p_1, v] \cdot [p_2, g]
&=
[p, \Psi(q_1, v)] \cdot [p, c_{q_2}(g)]
=
\mleft[p \cdot c_{q_2}(g), \Psi(q_1, v) \mright]
=
\mleft[p \cdot c_{q_2}\mleft( g \mright) ~ q_1, v \mright].
\eas
On the other hand, using $p^\prime$ as a fixed element, we derive, using $q_i^\prime = \mleft( g^\prime \mright)^{-1} q_i$,
\bas
[p_1, v] \cdot [p_2, h]
&=
\mleft[p^\prime \cdot c_{q_2^\prime}\mleft( g \mright) ~ q_1^\prime, v \mright]
=
\mleft[p \cdot g^\prime q_2^\prime g \mleft( q_2^\prime \mright)^{-1} q_1^\prime, v \mright]
=
\mleft[p \cdot q_2 g q_2^{-1} q_1, v \mright]
=
\mleft[p \cdot c_{g_2}\mleft(g\mright) ~ q_1, v \mright],
\eas
which finalizes the argument needed to show that the action is well-defined.

$\bullet$ Let us now quickly check that the conditions in Def.\ \ref{def:LiegroupACtion} are satisfied. We have
\bas
\pi_{\mathcal{F}}\bigl( [p, v] \cdot [p, g] \bigr)
&=
\pi_{\mathcal{F}}\mleft( \mleft[p , \Psi \mleft(g, v \mright) \mright] \mright)
=
\pi_P(p)
=
\pi_{c_G(P)}\bigl( [p, g] \bigr)
\eas
for all $p \in P_x$ ($x \in M$), $v \in F$ and $g \in G$; similarly, having additionally $h \in G$,
\bas
\bigl([p, v] \cdot [p, g] \bigr) \cdot [p, h]
&=
\mleft[ p \cdot g, v \mright] \cdot [p, h]
=
\mleft[ p \cdot g h, v \mright]
=
[p, v] \cdot [p, gh]
=
[p, v] \cdot \bigl([p, g] ~ [p, h] \bigr),
\eas
and
\bas
[p, v] \cdot [p, e]
&=
[p \cdot e, v]
=
[p, v].
\eas
Therefore this describes an action.
\end{proof}

\begin{remarks}{Relation to automorphisms of principal bundles and gauge transformations}{ClassGaugeTrafosAndcgPMulti}
Recall that gauge transformations have a strong relation to principal bundle automorphisms $f$ of the principal bundle $P$; see \textit{e.g.}\ \cite[\S 5.3, Def.\ 5.3.1, page 256f.]{Hamilton} and \cite[\S 5.4, Thm.\ 5.4.4, page 273]{Hamilton}. That is, $f$ is a diffeomorphism $P \to P$ with
\bas
\pi_P \circ f &= \mathds{1}_M,\\
f(p \cdot g) &= f(p) \cdot g
\eas
for all $p \in P$ and $g \in G$. The group of such maps will be denoted by $\sAut(P)$. One can identify such automorphisms with certain $G$-valued maps on $P$, following \cite[\S 5.3, Def.\ 5.3.2 \& Prop.\ 5.3.3, page 266f.]{Hamilton}: We define the following set of smooth maps $P \to G$ by
\bas
C^\infty(P;G)^G
&\coloneqq
\left\{
	\sigma: P \to G \text{ smooth}
	~\middle|~
	\sigma(p \cdot g) = c_{g^{-1}}\bigl( \sigma(p) \bigr) \text{ for all } p \in P, g \in G
\right\}.
\eas
It is straightforward to check that this is a group w.r.t.\ pointwise multiplication. Furthermore, there is a group isomorphism
\bas
\sAut(P) &\to C^\infty(P; G)^G,\\
f &\mapsto \sigma_f,
\eas
where $\sigma_f$ is defined by
\bas
f(p)
&=
p \cdot \sigma_f(p)
\eas
for all $p \in P$; one can prove that this is well-defined.

As argued in \cite[\S 5.3, Thm.\ 5.3.8, page 269; formulated as left action there, which is why we have an inverse here]{Hamilton}, $\sAut(P)$ acts (on the right) on associated fibre bundles $\mathcal{F} = P \times_\Psi F$ by
\bas
[p, v] \cdot f
&\coloneqq
\mleft[ f^{-1}(p), v \mright]
=
\mleft[ p \cdot \sigma_{f}(p)^{-1}, v \mright]
\eas
for all $[p, v] \in \mathcal{F}_x$ ($x \in M$) and $f \in \sAut(P)$. $\sigma_f$ can also be just locally defined, therefore one could investigate whether there is also an action just with an element $g$ of $G$, basically the restriction of $\sigma_f$ onto the fibre $P_x$. However, the action given by $[p, v] \cdot g = \mleft[ p \cdot g^{-1}, v \mright]$ for $g \in G$ is in general clearly only well-defined w.r.t.\ a change of the representative of $[p, v] = \mleft[ p \cdot q, \Psi_{q^{-1}}(v) \mright]$ ($q \in G$), if $G$ is abelian. But one can resolve this by looking at it carefully: The rough idea is that $g$ basically comes from $\sigma_f(p)$ in this context, but
\bas
\sigma_f(p \cdot q)
&=
c_{q^{-1}}\bigl( \sigma_f(p) \bigr).
\eas
Roughly, while $p$ is multiplied with $g^{-1}$, $p \cdot q$ has to be multiplied with $q^{-1} g^{-1} q$. It is easy to check that this resolves that issue, and the result is precisely the action described in Ex.\ \ref{ex:AssocLGACtingOnAssocVec}. In fact, we have the following proposition:
\end{remarks}

For the following proposition observe that the (local) sections of an LGB have a group structure given by pointwise multiplication.

\begin{propositions}{Gauge transformations as sections of the inner LGB, \newline \cite[\S 1.4, (the last sentence of) Ex.\ 1.4.7, page 25]{mackenzieGeneralTheory}}{GaugeTrafoAndInnerLGB}
Let $P \stackrel{\pi_P}{\to} M$ be a principal bundle with structural Lie group $G$ over a smooth manifold $M$. Then there is a group isomorphism 
\bas
\sAut(P) &\to \Gamma\bigl( c_G(P) \bigr),\\
f &\mapsto q_f
\eas
where $g_f \in \Gamma\bigl( c_G(P) \bigr)$ is defined by
\bas
\mleft.q_f\mright|_x
&\coloneqq
\bigl[ p, \sigma_f(p) \bigr]
\eas
for all $x \in M$, where $p$ is any element of $P$ such that $\pi_P(p) = x$, and $\sigma_f$ is the element of $C^\infty(P; G)^G$ corresponding to $f$ as introduced in Rem.\ \ref{rem:ClassGaugeTrafosAndcgPMulti}.
\end{propositions}

\begin{remark}
\leavevmode\newline
As one may guess, $\Gamma\bigl( c_G(P) \bigr)$ is the analogue of $C^\infty(M; G)^G$ such that one could ask for a more direct analogue to $\sAut(P)$. Indeed, as argued in \cite[\S 1.3, Prop.\ 1.3.9, page 20]{mackenzieGeneralTheory}, $c_G(P)$ is actually isomorphic to $\mleft(P \times_M P\mright) \Big/ G$, where $P\times_M P \coloneqq \pi_P^*P$, and the $G$-action is the diagonal action on $P \times P$. One can prove that an isomorphism is given by 
\bas
c_G(P) &\to \mleft(P \times_M P\mright) \Big/ G,\\
[p, g] &\mapsto [p, p \cdot g].
\eas
It is also argued in \cite[\S 1.4, Ex.\ 1.4.7, page 25]{mackenzieGeneralTheory} that $\sAut(P)$ is then directly isomorphic to $\Gamma\mleft(\mleft(P \times_M P\mright) \Big/ G\mright)$ by 
\bas
\sAut(P) &\to \Gamma\mleft(\mleft(P \times_M P\mright) \Big/ G\mright),\\
f &\mapsto L_f,
\eas
where $L_f \in \Gamma\mleft(\mleft(P \times_M P\mright) \Big/ G\mright)$ is given by
\bas
\mleft.L_f\mright|_x
&\coloneqq
[p, f(p)]
=
\mleft[ p, p \cdot \sigma_f(p) \mright]
\eas
for all $x \in M$, where $p$ is any element of $P$ such that $\pi_P(p) = x$. This is clearly well-defined, and, so, while $c_G(P)$ is the bundle-analogue of $C^\infty(P; G)^G$ one can think of $\mleft(P \times_M P\mright) \Big/ G$ as the bundle-analogue of $\sAut(P)$.

However, this description often arises if one wants to use the formalism of groupoids and algebroids, here especially using the \textbf{gauge groupoid} and \textbf{Atiyah algebroid} induced by $P$. These would allow an even more elegant version of the gauge transformations, however, we intend to write this paper in such a way that there is no need that the reader has knowledge about those bundle structures. See the cited references for more details in that regard.
\end{remark}

\begin{proof}[Proof of Prop.\ \ref{prop:GaugeTrafoAndInnerLGB}]
\leavevmode\newline
\indent $\bullet$ Let us first quickly check whether $g_f \in \Gamma\bigl( c_G(P) \bigr)$ is well-defined for all $f \in \sAut(P)$. For $p \in P_x$ ($x \in M$) we have
\bas
\mleft. q_f \mright|_x
&=
\mleft[ p, \sigma_f(p) \mright],
\eas
If $p^\prime = p \cdot g$ ($g \in G$) is another element of $P_x$, then, using $p^\prime$ to define $q_f$,
\bas
\mleft. q_f \mright|_x
&=
\mleft[ p \cdot g, \sigma_f(p \cdot g) \mright]
=
\mleft[ p \cdot g, c_{g^{-1}}\bigl(\sigma_f(p)\bigr) \mright]
=
\mleft[ p, \sigma_f(p) \mright],
\eas
also using the definition of $c_G(P)$, recall Ex.\ \ref{ex:InnerLGBs}. It follows that $q_f$ is well-defined, and it is clear that $q_f$ is smooth.

$\bullet$ We want to show that $f \mapsto q_f$ is a group isomorphism by using that it is a composition of the group isomorphisms $\sAut \to C^\infty(P; G)^G$ as in Rem.\ \ref{rem:ClassGaugeTrafosAndcgPMulti} and 
\ba
C^\infty(P; G)^G &\to \Gamma\bigl( c_G(P) \bigr),\nonumber\\
\sigma &\mapsto q_\sigma,\label{IsomCPGGTocGPSec}
\ea
where $q_\sigma$ is effectively the same definition as $q_f$, that is $q_\sigma|_x = [p, \sigma(p)]$ which is well-defined by the very same reasons as before. It is only left to show that $C^\infty(P; G)^G \to \Gamma\bigl( c_G(P) \bigr)$ is a group isomorphism. For injectivity let $\sigma^\prime$ be another element of $C^\infty(P; G)^G$ and assume $[p, \sigma(p)] = \mleft[ p, \sigma^\prime(p) \mright]$. Then
\bas
e_x
&=
[p, e]
=
[p, \sigma(p)] \cdot \underbrace{\mleft( \mleft[ p, \sigma^\prime(p) \mright] \mright)^{-1}}
	_{= \mleft[ p, \mleft(\sigma^\prime(p)\mright)^{-1} \mright]}
=
\mleft[ p, \sigma(p) \mleft(\sigma^\prime(p)\mright)^{-1} \mright],
\eas
such that
\bas
\sigma(p) \mleft(\sigma^\prime(p)\mright)^{-1}
&=
e,
\eas
so $\sigma = \sigma^\prime$ and hence injectivity follows. For surjectivity observe that for a section $q \in \Gamma\bigl( c_G(P) \bigr)$ we can define a map $\sigma: P \to G$ by
\bas
q_x
&=
[p, \sigma(p)].
\eas
This map satisfies
\bas
[p, \sigma(p)]
&=
\mleft[p \cdot g, c_{g^{-1}}\bigl(\sigma(p)\bigr)\mright]
=
[p \cdot g, \sigma(p \cdot g)]
\eas
for all $g \in G$; the last equality implies $\sigma(p \cdot g) = c_{g^{-1}}\bigl(\sigma(p)\bigr)$, which is precisely what we need for $C^\infty(P; G)^G$. It is only left to show smoothness of $\sigma$. For an open neighbourhood $U \subset M$ of $x$ fix a trivialization $\varphi_U: P|_U \to U \times G$, and we denote
\bas
\varphi_U\mleft(p^\prime\mright)
&=
\mleft( \pi_P\mleft(p^\prime\mright), \beta_U\mleft(p^\prime\mright) \mright)
\eas
for all $p^\prime \in P$, where $\beta_U: P|_U \to G$ is an equivariant map, \textit{i.e.}\ $\beta_U\mleft(p^\prime \cdot g\mright) = \beta_U\mleft(p^\prime\mright) ~ g$ for all $g \in G$. As shown in the proof of Thm.\ \ref{thm:AssociatedGroupBundlesHaveGroupStructure}, we have a trivialization of $c_G(P)$ given by
\bas
c_G(P)|_U
&\to
U \times G,\\
\mleft[p^\prime, g\mright]
&\mapsto
\mleft(
	\pi_P\mleft(p^\prime\mright), \psi_{\beta_U\mleft(p^\prime\mright)} (g)
\mright).
\eas
Applying that trivialization to $q$ we derive that
\bas
\mleft[ p^\prime \mapsto \psi_{\beta_U\mleft(p^\prime\mright)}\mleft( \sigma\mleft(p^\prime\mright) \mright) \mright]
\eas
is smooth, because $q$ is smooth. Since $\psi_{\beta_U\mleft(p^\prime\mright)}$ is smooth and bijective, we conclude that $\sigma$ is smooth. Hence, $\sigma \in C^\infty(P; G)^G$, so, Def.\ \eqref{IsomCPGGTocGPSec} is also surjective and thence bijective.

Finally let us show that Def.\ \eqref{IsomCPGGTocGPSec} is a group isomorphism. Let $\sigma, \sigma^\prime$ be elements of $C^\infty(P; G)^G$, then use Def.\ \eqref{IsomCPGGTocGPSec} to derive
\bas
\sigma \sigma^\prime
&\mapsto
q_{\sigma \sigma^\prime}
\eas
with
\bas
\mleft.q_{\sigma \sigma^\prime}\mright|_x
&=
\mleft[ p, \sigma(p) ~ \sigma^\prime(p) \mright]
=
[p, \sigma(p)] \cdot \mleft[p, \sigma^\prime(p) \mright]
=
\mleft.q_{\sigma}\mright|_x \cdot \mleft.q_{\sigma^\prime}\mright|_x,
\eas
such that Def.\ \eqref{IsomCPGGTocGPSec} satisfies
\bas
\sigma\sigma^\prime &\mapsto q_\sigma \cdot q_{\sigma^\prime}.
\eas
This concludes the proof.
\end{proof}

Associated fibre bundles are motivated by making the invariance of gauge theory under local gauge transformations (that is, the change of gauge/local section of $P$) an inherent part of the bundle, similar to typical manifold coordinates; while the action of the global and other transformations "remain", similar to diffeomorphisms of a manifold. This procedure of "reducing" the action onto these is reflected in the quotient bundle $c_G(P)$.

\subsection{Fundamental vector fields}

\subsection{Toy model}

We want to use LGBs in the context in the context of gauge theory, somewhat as a replacement of the structural Lie group. 

\section{Conclusion}

\textbf{Acknowledgements:} I want to thank Mark John David Hamilton and Alessandra Frabetti for their great help and support in making this paper.

\textbf{Funding:} The paper was finalised as part of my post-doc fellowship at the National Center for Theoretical Sciences (NCTS), which is why I also want to thank the NCTS.

%%%%%%%%%%%%%%%%%%%%%%%%%%%% Hier beginnt der Anhang %%%%%%%%%%%%%%%%%%%%%%%%%%%%

%\newpage


%\listoftables % Tabellenverzeichnis

%\listoffigures %Abbildungsverzeichnis

\appendix
\setcounter{equation}{0}
\renewcommand{\theequation}{\Alph{chapter}.\arabic{equation}} %Reset first and then add section to number

\renewcommand\refname{List of References}

%\begin{thebibliography}{99}
%\bibitem[I]{Anl01} \url{http://adsabs.harvard.edu/abs/1971ApJ...170..319D}, Datum: 01.11.2014
%\end{thebibliography}

%\printbibliography 
\bibliography{Literatur}
\bibliographystyle{unsrt}

%\newpage\thispagestyle{empty}\hspace{1em}\newpage

\section{Axiomatic Yang-Mills gauge theories}

Let us discuss where the compatibility conditions may arise from a certain axiomatic point of view.

\end{document}